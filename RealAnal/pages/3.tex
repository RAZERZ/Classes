\subsection{Sup \& Inf}\hfill\\\par
\begin{itemize}
  \item Upper bound
  \item Lower bound
  \item Smallest upper bound (not always in set, see $[1,3)$)
  \item Biggest lower bound (not always in set, see $(1,3]$)
  \item Theorem 3.7
  \item Sats 1.20, 1.27 i Rudin, Sats 3.9 i Douglas
\end{itemize}
\par\bigskip
\noindent\textbf{Bevisskiss av sats 3.7:}\par
\noindent $M$ (eller $S$ i Douglas) $\neq \Q_+$. $\forall s\in T,\quad S\subseteq B(:=S\leq B)$\par
\noindent $\bigcup_{i=1}^{\infty}S_i\subseteq B$\par
\noindent $B\neq\Q_+\Rightarrow M\neq\Q_+$\par
\noindent Måste visa att $\forall S\in T\quad, S\subseteq M$.\par
\noindent Måste visa att Supremum är unik, en bra teknik för att visa unikhet är motsägelsebevis. Antag att det finns $M_1$. $\forall S\in T\quad S\subseteq M_1\Rightarrow\bigcup_{i=1}^{\infty}\subseteq M_1$, men då är $M\subseteq M_1$, alltså är $M$ den minsta, och därmed Supremum.
\par\bigskip
\subsection{Sequence}\hfill\\\par
\noindent Det som är av intresse är det som händer i $\infty$
\par\bigskip
\begin{itemize}
  \item Varje $a_i$ kan ses som en evaluering av en funktion i $i$, dvs $a_i=f(i)$
  \item Följd kallas för bounded $\cdots$
  \item Lokalt monoton, typ $\left\{\dfrac{\sin(x)}{x}\right\}_{x=0}^{\infty}$
  \item Cauchy sekvens (jämför med definitionen av konvergens)
  \item Konvergens implicerar Cauchy, men inte motsats 
  \item För att "döda" Cauchy, låt sekvensen vara bounded
  \item Bevis 3.6 i Douglas, där det står $\varepsilon+\varepsilon$ kan man tänka att det står $\varepsilon/2+\varepsilon/2$
\end{itemize}
\par\bigskip
\noindent Vi kom till Corollary 3.14
