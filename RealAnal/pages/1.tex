\section{Introduction}\par
\noindent Before proving injectivity \& surjectivity, first show it is a function 
\par\bigskip
\noindent Give an example of:\par
\begin{itemize}
  \item Abelian group (and non-abelian)
  \item Non-commutative ring
  \item Commutative ring which is not a field
\end{itemize}
\newpage
\subsection{Sammanfattning Kap2}\hfill\\
\subsection{Notation}\hfill\\\par
\noindent Vi använder ett "$+$" för att beteckna om mängden innehåller positiva element. Om det är känt att mängden innehåller negativa element behöver vi tydliggöra om 0 finns i mängden, det finns olika notation.
\par\bigskip
\noindent Vi skriver "upphöjt i +" för att visa att mängden är helt positivt (innehåller ingen 0), och "nedsänkt i +" för att visa att mängden är positiv men innehåller 0.
\par\bigskip
\noindent\textbf{Exempel:}\par
\noindent De positiva reella talen: $\R^+$\par
\noindent Icke-negativa rella talen: $\R_+$
\par\bigskip
\subsection{Relationer}\hfill\\\par
\noindent En relation $R$ på en mängd $M$ är:
\begin{equation*}
  \begin{gathered}
    R\subseteq M\times M
  \end{gathered}
\end{equation*}
\noindent Vi har 5 "adjektiv" att beskriva våra relationer med:\par
\begin{itemize}
  \item Reflexiv ($xRx\quad\forall x\in M$)
  \item Symmetrisk ($xRy\Rightarrow yRx\quad\forall x,y\in M$)
  \item Antisymetrisk ($xRy\;\&\;yRx\Rightarrow x=y$)
  \item Transitiv ($xRy\;\&\;yRz\Rightarrow xRz\quad\forall x,y,z\in M$)
  \item Connex ($xRy\text{ eller } yRx\quad\forall x,y\in M$)
\end{itemize}
\par\bigskip
\noindent Som exempel på dessa kan man betrakta relationen $=$ eller relationen $<$ över $R$
\par\bigskip
\noindent\textbf{Exempel:}\par
\noindent Låt $A = \left\{a,b\right\}$ och potensmängden till $A$ vara $\mathcal{P}(A) = \left\{\O,\left\{a\right\},\left\{b\right\},\left\{a,b\right\}\right\}$\par
\noindent Vilka av de 5 kraven uppfylls om relationen är $\subseteq$?
\par\bigskip
\noindent För att lösa detta så kommer vi ihåg att relationer är definierade på mängdens kartesiska produkt, så vi har i själva verket en relation $\subseteq$ över $\mathcal{P}(A)\times\mathcal{P}(A)$\par
\noindent Vi kollar reflexivitet. Om vi tar ett element från $\mathcal{P}(A)$, är det då en delmängd till sig själv? \textbf{Ja, en mängd är alltid delmängd till sig själv}. Viktigt att notera att den inte är en äkta delmängd!
\par\bigskip
\noindent Är relationen symmetrisk? Tag $\left\{a\right\}\in\mathcal{P}(A)$ och $\left\{a,b\right\}\in\mathcal{P}(A)$. Då gäller att $\left\{a\right\}\subseteq\left\{a,b\right\}$ men $\left\{a,b\right\}\not\subseteq\left\{a\right\}$. Eftersom relationen skulle gälla $\forall$ element i relationsmängden, så gäller \textit{inte} symmetri!
\par\bigskip
\noindent Är relationen antisymetrisk? Antisymetri gäller om $xRy\;\&\;yRx\Rightarrow x=y$, vi undersöker negationen, vilket är $xRy\;\&\; x\neq y\Rightarrow\neg yRx$. Detta säger i princip "om $x$ är delmängd till $y$ och $x$ inte är lika med $y$ så är $y$ inte en delmängd till $x$", vilket vi vet gäller.
\par\bigskip
\noindent Är relationen transitiv? Ja, om $x$ är en delmängd till $y$ och $y$ är en delmängd till $z$ så gäller att $x$ är en delmäng till $z$
\par\bigskip
\noindent Connex? Ja.
\par\bigskip
\subsubsection{Klassifikationer av relationer}\hfill\\\par
\noindent Vi har 3st Klassifikationer för relationer:\par
\begin{itemize}
  \item Ekvivalensrelation (reflexiv, symmetrisk, transitiv)
  \item Partiell ordning (reflexiv, antisymetrisk, transitiv)
  \item Total ordning (connex, reflexiv, antisymetrisk, transitiv)
\end{itemize}
\par\bigskip
\noindent\textbf{Anmärkning:}\par
\noindent En total ordning är en connex partiell ordning.
\par\bigskip
\noindent\textbf{Exempel:} (Ekvivalensrelation)\par
\noindent $=$ (likhet) är en ekvivalensrelation
\par\bigskip
\noindent\textbf{Exempel:} (Partiell ordning)\par
\noindent $\leq$  är en partiell ordning (men inte en ekvivalensrelation)
\par\bigskip
\noindent\textbf{Exempel:} (Total ordning)\par
\noindent $\leq$ är en total ordning 
\par\bigskip
\subsection{Funktioner}\hfill\\\par
\subsubsection{Domän, kodomän}\hfill\\
\noindent En funktion (eller även kallad en avbildning) definierad från en \textit{domän} $A$ till en \textit{kodomän} $B$, kan ses som en delmängd till $A\times B$.\par
\noindent Denna brukar även kallas för \textit{grafen} till funktionen.\par
\noindent Om $f:A\to B$ så är alltså grafen($f$) $\subseteq A\times B$ 
\par\bigskip
\noindent\textbf{Anmärkning:}\par
\noindent För varje $x\in A$ så finns det ett \textit{unikt} $y\in B$ så att $f(x) = y$ (alternativ beteckning: $(x,y)\in$ graf($f$))
\par\bigskip
\subsubsection{Bilden}\hfill\\\par
\noindent Något som verkar lite förvirrande först är att kodomänen inte är bilden av funktionen. Om vi exempelvis betraktar $f:\R\to\C$ $x\mapsto x$ så ser vi att de värden som faktiskt "träffas" är hela $\R$, vilket är en delmängd till $\C$ men inte hela $\C$!
\par\bigskip
\begin{theo}[Bilden till en avbildning]{thm:img}
  \begin{equation*}
    \begin{gathered}
      f:A\to B\\
      f(A) = \left\{y\in B\;|\;\exists x\in A\text{ s.t } y=f(x)\right\}
    \end{gathered}
  \end{equation*}
\end{theo}
\par\bigskip
\noindent\textbf{Anmärkning:}\par
\begin{itemize}
  \item Om $f,g$ är injektiva, så är $g\circ f$ injektiv
  \item Om $f,g$ är surjektiva, så är $g\circ f$ surjektiv 
  \item Om $f,g$ är bijektiv, så är $g\circ f$ bijektiv
\end{itemize}
\newpage
\subsection{Pullback \& Pushforward}\hfill\\\par
\noindent Antag att vi har en avbildning $f: X\to Y$, då gäller följande:
\par\bigskip
\begin{theo}[Pullback]{thm:pullback}
  Inte hela kodomänen träffas av en avbildning/funktion. Det beror på vad domänen är (och hur funktionen ser ut).\par
  \noindent Vi kan däremot korta ner domänen och undersöka hur det påverkar bilden av avbildningen, detta är \textit{pullback}:
  \begin{equation*}
    \begin{gathered}
      f_*(A) = \left\{f(x)\in Y\;|\; x\in A\right\} = f(A)\qquad A\subseteq\mathcal{P}(X)
    \end{gathered}
  \end{equation*}
  \par\bigskip
  \noindent\textbf{Anmärkning:}\par
  $A\subseteq X\Rightarrow X\subseteq\mathcal{P}(X)$, samt att $f_*\subseteq Y\Lrarr f_*\subseteq\mathcal{P}(Y)$\par
  Vi har då $f_*:\mathcal{P}(X)\to\mathcal{P}(Y)$
\end{theo}
\par\bigskip
\begin{theo}[Pushforward]{thm:pushforward}
  Liknande/Motsatsen gäller för \textit{pushforward}. Här vill vi undersöka vad som händer med domänen om vi betraktar en delmängd till kodomänen:
  \begin{equation*}
    \begin{gathered}
      f^*(B) = \left\{x\in X\;|\; f(x)\in B\right\}\qquad B\subseteq\mathcal{P}(Y)
    \end{gathered}
  \end{equation*}
\end{theo}
\par\bigskip
\noindent\textbf{Anmärkning:}\par
\noindent Pushforward är invariant under union \textbf{och} snitt\par
\noindent Pullback är \textbf{enbart} invariant under union (annars, låt $f(x) = x^2$ och visa vad som händer med $f_*(\left\{-1\right\})$ resp. $f_*(\left\{1\right\})$)
\par\bigskip
\noindent Detta följer från definitionen och påminner lite om lagen om total sannolikhet från kursen Sannolikhetsteori 1.
\par\bigskip
\subsection{Abstrakt algebra}\hfill\\\par
\begin{theo}[Ordnad kropp]{thm:orderedfield}
  Låt $\mathbb{F}$ vara en kropp med en strikt ordning $<$ så att:\par
  \begin{itemize}
    \item $x,y,z\in\mathbb{F}$ och $y<z$ så är även $x+y<x+z$
    \item $x,y\in\mathbb{F}$ och $x,y>0$ så är även $xy>0$
  \end{itemize}
\end{theo}
\par\bigskip
\begin{theo}[Vektorrum över kropp]{thm:vectorrmoverfield}
  En mängd $V$ är ett vektorrum över en kropp $\mathbb{F}$ om vi har följande 2 avbildningar:
  \begin{equation*}
    \begin{gathered}
      V\times V\stackrel{+}{\to}V\qquad \mathbb{F}\times V\stackrel{\cdot}{\to} V\\
      (v,w)\mapsto v+w\qquad(\alpha, v)\mapsto\alpha v
    \end{gathered}
  \end{equation*}
\end{theo}
\par\bigskip
\begin{theo}[Stödet till en avbildning]{thm:support}
  Låt $f:M\to\mathbb{F}$ där $\mathbb{F}$ är en kropp. Vi definierar
  \begin{equation*}
    \begin{gathered}
    \text{supp}(f) = \left\{m\in M\;|\; f(m)\neq0\right\}
    \end{gathered}
  \end{equation*}\par
  \noindent Vi kan generalisera detta till avbildningar över vektorrum $V$ och definiera:
  \begin{equation*}
    \begin{gathered}
      V_{\text{fin}} = \left\{f\in V\;|\; \text{supp}(f)\text{ ändlig}\right\}
    \end{gathered}
  \end{equation*}
\end{theo}
\par\bigskip
\noindent Vi har i andra kurser kikat på hur vi kan kvota mängder med ideal. Vi kan även kvota med ekvivalensrelationer enligt följande:
\begin{equation*}
  \begin{gathered}
    M/_\sim\quad = \left\{[x]\in\mathcal{P}(M)\;|\; x\in M\right\}
  \end{gathered}
\end{equation*}
\par\bigskip
\noindent Givet en binär komposition $*$ över $M$ och en ekvivalensrelation $\sim$ säger vi att $\sim$ \textbf{respekterar} $*$ om:
\begin{equation*}
  \begin{gathered}
    x\sim x^{\prime}\quad\& y\sim y^{\prime} \Rightarrow x*y\sim x^{\prime}*y^{\prime}\qquad\forall x,y\in M
  \end{gathered}
\end{equation*}
\par\bigskip
\noindent Vi kan då defniera den \textit{inducerade binära kompositionen} $\overline{*}$ på $M/_\sim$ genom:
\begin{equation*}
  \begin{gathered}
    M/_\sim\times M/_\sim\stackrel{\overline{*}}{\to} M/_\sim\\
    \left([x],[y]\right)\mapsto [x*y]
  \end{gathered}
\end{equation*}
\par\bigskip
\subsection{Kardinalitet}\hfill\\\par
\noindent Två mängder har samma kardinalitet om det finns en bijektion mellan de.
\par\bigskip
\noindent Vi säger att en mängd $A$ är uppräknelig om det finns en bijektion mellan $A$ och $\N$.\par
Om det inte finns en bijektion säger vi att $A$ är överuppräknelig.
\par\bigskip
\noindent\textbf{Anmärkning:}\par
\noindent Om $A_1,A_2,\cdots,$ är uppräkneliga så är deras union uppräkneliga.
\par\bigskip
\begin{theo}[Cantors sats]{thm:cantorsthm}
  Låt $X$ vara en mängd. Det finns \textbf{ingen} surjektion från $X$ till $\mathcal{P}(X)$ 
\end{theo}
\par\bigskip
\begin{theo}[Schröder-Bernsteins sats]{thm:schroderbernthm}
  Betrakta följande:
  \begin{equation*}
    \begin{gathered}
      X\stackrel{f}{\to}Y\quad Y\stackrel{g}{\to}X
    \end{gathered}
  \end{equation*}\par
  \noindent Om $f,g$ är injektiva så finns en bijektion $X\stackrel{h}{\to}Y$
\end{theo}
\par\bigskip
\noindent För att visa detta krävs följande sats:
\par\bigskip
\begin{theo}[Tarskis fixpunktssats]{thm:tarskifixpointthm}
  Antag att $F:\mathcal{P}(X)\to\mathcal{P}(X)$ är monotont växande, dvs
  \begin{equation*}
    \begin{gathered}
      A\subseteq B\Rightarrow F(A)\subseteq F(B)
    \end{gathered}
  \end{equation*}\par
  \noindent Då finns det $M\subseteq X$ så att $F(M) = M$
\end{theo}
