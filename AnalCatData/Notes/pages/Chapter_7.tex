\section{Chapter 7}
\subsection{Slide 5}\hfill\\
\noindent For $F_\varepsilon$, what distribution you might ask? Logistic distribution:
\begin{equation*}
  \begin{gathered}
    \varepsilon\sim F(X) = \dfrac{\text{exp}\left\{x\right\}}{1+\text{exp}\left\{x\right\}}\quad \pi\Rightarrow\dfrac{\text{exp}\left\{x^T\beta\right\}}{1+\text{exp}\left\{x^T\beta\right\}}\Lrarr \ln{\left(\dfrac{\pi}{1-\pi}\right)} = x^T\beta
  \end{gathered}
\end{equation*}\par
\noindent Choice of link-function needs to be motivated for the error.
\par\bigskip
\subsection{Slide 13}\hfill\\
\noindent Here it is assumed $y_i\in\left\{0,1\right\}$
\par\bigskip
\subsection{Slide 14}\hfill\\
\noindent This is for the Frequentist only. We assume we have a distribution with known $\theta$. $Y_i$ is observed from distribution.\par
\noindent A \textit{statistic }$T = T(Y_1,\ldots, Y_n)$, conditional distribution of data given sufficient statistic does not depend on $\theta$, all information about $\theta$ is contained in the sufficient statistic.\par
\noindent \textit{Minimal sufficient statistic}  uses the smallest dimension. \par
\noindent As an example, assume we have two samples $X_i, Y_j$. Take the ratio $\dfrac{f(Y\mid\theta)}{f(X\mid\theta)}$. A statistic $T(Y)$ is minimal sufficient if the ratio does not depend on $\theta\Lrarr T(X) = T(Y)$\par
\noindent In order to not be dependent on $\alpha, $, we need 
\begin{equation*}
  \begin{gathered}
    \sum_i y_i = \sum_i y_i^*
  \end{gathered}
\end{equation*}\par
\noindent so that they cancel. Thus, $\sum_i y_i$  is a minimal sufficient statistic.
\par\bigskip
\subsection{Slide 15}\hfill\\
\begin{equation*}
  \begin{gathered}
    \ln{\left(\dfrac{\pi}{1-\pi}\right)} = \alpha+\beta_1x_1+\beta_2x_2
  \end{gathered}
\end{equation*}\par
\noindent Here, $\alpha$ and $\beta_2$ are nuisance parameters, and the minimal sufficent statistic for \par
\begin{itemize}
  \item $\alpha$: $\sum y$
  \item $\beta_j$: $\sum y_i x_{ij}$
\end{itemize}
