\section{Lokalisering}\par
\noindent Är en process som är en generalisering av hur vi fick bråkkroppen till ett integritetsområde\par
\noindent Den beror lite på hur generell man vill vara
\par\bigskip
\noindent Lite repetition om $R$ integritetsområde:
\begin{equation*}
  \begin{gathered}
  R\text{x}(R\left\{0\right\})/\sim\\
  \dfrac{p}{r}\sim \dfrac{q}{s}\Lrarr ps=qr\Rightarrow Q(R)
  \end{gathered}
\end{equation*}
\par\bigskip
\noindent Detta går att generalisera. Syftet med bråkkroppen är att invertera element, men vi utesluter 0. Däremot, ibland kanske man inte vill invertera \textit{alla} element, men ändå få ett vettigt resultat.\par
\noindent Ibland vill vi kunna invertera bara vissa element i $R$
\par\bigskip
\begin{theo}
  ALåt $R$ vara en ring. $S\subset R$ kallas för en \textit{multiplikativ mängd} om:\par
  \begin{itemize}
    \item $1\in S$
    \item $S$ sluten under multiplikation $\cdot$ med element i $S$
  \item $0\notin S$
  \end{itemize}
\end{theo}
\par\bigskip
\begin{theo}
  OOm $R$ är ett integritetsområde och $S$ en multiplikativ mängd. Lokalisering av $R$ med avseende på $S$ brukar betecknas $S^{-1}R$ är följande:
  \begin{equation*}
    \begin{gathered}
      S^{-1}R=R\text{x}S/\sim\qquad(p,r)\sim(q,s)\Lrarr ps=qr
    \end{gathered}
  \end{equation*}
\end{theo}
\par\bigskip
\noindent\textbf{Exempel:}\par
\noindent Första exemplet är givetvis $R = R$ och $S = R\backslash\left\{0\right\}$
\par\bigskip
\noindent Ett annat exempell är $\Z = R$ och $S = \left\{1,3,3^2,3^3,\cdots\right\}$\par
\noindent Då är $S^{-1}\Z = \left\{\dfrac{p}{3^n}\:|\: n\in\N\right\}$\par
\noindent Notera att detta är även en ring. 
\par\bigskip
\noindent Ibland används följande notation för att visa när vi lokaliserar ett element: $R_i$, i exemplet ovan blir det $\Z_3$ (här menar vi \textit{inte} $\Z/3\Z$, lite förvirrande) 
\par\bigskip
\noindent\textbf{Exempel:}\par
\noindent Om $R=\Z$ och $S = R\backslash\left\{0\right\} = Z\backslash3\Z$ (multiplikativ ty $3\Z$ är primideal), det vill säga invertera allt som inte har en faktor 3:
\begin{equation*}
  \begin{gathered}
    S^{-1}\Z = \left\{\dfrac{p}{q}\:|\: \dfrac{q}{3}\notin\Z\right\}
  \end{gathered}
\end{equation*}
\par\bigskip
\noindent\textbf{Exempel:}\par
\noindent Säg att $R = \C[x]$, låt $m=(x)$ vara ett maximalt ideal\par
\noindent Låt $S = \C[x]\backslash(x) = \left\{p\in\C[x]\:|\: p(0)\neq0\right\}$ (vi inverterar allt som inte har en faktor $x$)\par
\noindent Med andra ord, vi vill invertera polynom som inte är 0 i 0\par
\noindent Då är $S^{-1}R$:
\begin{equation*}
  \begin{gathered}
    = \C[x]_x = \left\{\dfrac{p}{q}\:|\:q(0)\neq0\right\}
  \end{gathered}
\end{equation*}
\par\bigskip
\noindent\textbf{Exempel:}\par
\noindent Nu låter vi $R = \C[x,y]$ och ett maximalt ideal $m = (x-a,y-b)$. Lokalisering i $(a,b)$ ger (lokalisering i en punkt menas med avseende på komplementet till maximala idealet i den punkten):
\begin{equation*}
  \begin{gathered}
    \C[x,y]_m = \left\{\dfrac{p}{q}\:|\: q(a,b)\neq0\right\}
  \end{gathered}
\end{equation*}
\par\bigskip
\begin{theo}
  ADen lokala ringen i punkten $p=(a,b)\in\C^2$ är:
  \begin{equation*}
    \begin{gathered}
      \mathcal{O}_p =\C[x,y]_m\qquad \text{där }m=(x-a,y-b)
    \end{gathered}
  \end{equation*}
\end{theo}
\par\bigskip
\noindent\textbf{Kommentar:}\par
\noindent Mer allmänt kallas en ring lokal om den har exakt ett maximalt ideal. Detta $\mathcal{O}_p$ är ett exempel på en lokal ring, det unika maximala idealet är $m\mathcal{O}_p\subseteq\mathcal{O}_p$
\par\bigskip
\begin{theo}
  ALåt $R$ vara en ring (där det kan finnas nolldelare) och $S$ multiplikativ mängd (kan också innehålla nolldelare)
  \par\bigskip
  \noindent Lokaliseringen ges fortfarande av $S^{-1}R = R\text{x}S/\sim$
  \par\bigskip
  \noindent Om $S$ innehåller nolldelare, kan vi inte visa att $\sim$ är en ekvivalensrelation (transitiviteten funkar inte ty vi behöver kancellationslagar, men vi kan inte kancellera nolldelare)
  \par\bigskip
  \noindent Vi lägger därmed till:
  \begin{equation*}
    \begin{gathered}
      (p,r)\sim(q,s)\Lrarr\exists u\in S:\; u(ps-qr)\neq0
    \end{gathered}
  \end{equation*}
\end{theo}
