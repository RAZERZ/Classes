\section{Vad vi vet nu om Bezouts sats}\par
\noindent Målet med denna kurs är i någon mening att studera denna sats, men för att göra detta behövde vi införa ett par saker såsom $\C$. Sen har vi infört $I_p$ för att räkna antal, det som verkar fattas är den projektiva delen.
\par\bigskip
\noindent Vi ska fortsätta tala lite om $I_p$, där vi tidigare kunde finna lösningar till grafen av en funktion av $x$ eftersom vi kunde dela bort $f = y-p(x)$ från $g$
\par\bigskip
\noindent Vi har en sats som vi skulle vilja ha till följd av våra axiom, om 2 kurvor skär varandra transversellt så borde skärningstalet vara 1. Nu ska vi visa att det är så:
\par\bigskip
\begin{theo}[$I_p=1$ för transversella skärningar]{thm:trjbsg}
  Om $f(p) = g(p)=0$ där $p=(a,b)$ och $\nabla f(p)\neq0\neq0\nabla g(p)$
  \par\bigskip
  \noindent Om $\nabla\neq0$ vet vi från implicita funktionssatsen vet vi att de konvergerar mot glatta funktioner. Vi vet också att vi kan uttrycka $f,g$ som en funktion av en variabel, men det följer.\par
  \noindent Vi antar också att $\nabla f(p)\neq\lambda\nabla g(p)$
  \par\bigskip
  \noindent Då är $I_p(f,g)=1$
\end{theo}
\par\bigskip
\noindent Hur går vi till väga för att bevisa detta? I geometri ska vi inte bara betrakta punkter i ett plan, vi måste tänka att vi inte bara ska betrakta $f,g$ eftersom vi lever nu i den affina världen, så kom alltid ihåg att vi kan alltid använda affina transformationer för att göra det lite lättare att räkna ut.
\par\bigskip
\begin{theo}[Följdsats]{thm:pjgw}
  Om $\nabla f(p)\neq0$ och $L$ är en linje genom $p$ sådan att $L\neq$ tangentlinjen till $f$, vilket betyder att linjen är transversell till $f$.
  \par\bigskip
  \noindent Då är skärningstalet mellan $L$ och $f$ $=1$
  \par\bigskip
  \noindent Om $I_p(f,L)>1$ så kan det vara så att gradienterna är parallella, dvs $L$ är parallel med tangentlinjen enligt satsen.
\end{theo}
\par\bigskip
\noindent\textbf{Standardexempel}
\par
\noindent Skärning mellan $y=x^2$ och $y=c$ var noll i $\R^2$, detta löste vi med att införa $\C$. Här hittar vi då 2 lösningar, $(\pm i\sqrt{\left|c\right|},c)\in\C^2$
\par\bigskip
\noindent\textbf{Övning:} Visa att $I_{(\pm i\sqrt{\left|c\right|},c)}(y-x^2,c)=1$
\newpage
\section{Projektiv Geometri över $\R$}\par
\noindent Vår "vanliga" synsätt på geometri är euklidisk geometri. De ser ut precis som vi mäter de.
\par\bigskip
\noindent En rektangel är bestämd av sina vinklar som är räta och längden av sina motstående sidor som är samma. Har vi 2st rektanglar där längder skiljer sig så är även rektanglarna sklida. Vi kan inte rotera och flytta den för att den ska bli den andra rektangeln, inga euklidiska transformationer åstadkommer detta. Skalärprodukten är inte bevarad, dvs längder och vinklar.
\par\bigskip
\noindent I euklidisk geometri finns det cirklar som inte är samma sak som ellipser exempelvis.
\par\bigskip
\noindent Vi kan säga att euklidisk geometri är "som geometri är i den verkliga världen", men det må vara så det är i den verkliga världen men det är inte så det ser ut med våra ögon. Exmepelvis med tågräls som är paralella som enligt euklidisk geometri inte skär i $\infty$, men våra ögon säger att de gör de.
\par\bigskip
\noindent Vad projektiv geomtri gör är att omvandla euklidisk geometri till vad vi faktiskt ser. Folk som tittar på saker i världen kallar vi konstnärer, särskillt om de försöker avbilda det. Det var så projektiv geometri uppkom (under renässansen i florens)
\par\bigskip
\noindent Vi vill i den här kursen göra plan geometri, men sättet vi kommer göra det på är inte genom ortonormala axlar, utan betrakta planet från från $z$-axeln, vi vill "stå på" $xy$-planet och titta på vad som händer.
\par\bigskip
\begin{theo}[Rella projektiva planet]{thm:wpigh}
  Det reella projektiva planet $\R P^2$ ($= P^2(\R) = P_{\R}^2$) är en följande mängd av ekvivalensklasser:
  \begin{equation*}
    \begin{gathered}
      \R P^2 = \R^3\left\{\bar{0}\right\}/\sim
    \end{gathered}
  \end{equation*}\par
  \noindent där $\sim$ är relationen. $\bar{x}\sim\bar{y}\Lrarr\exists\lambda\in\R \left\{0\right\}: \bar{x}=\lambda\bar{y}$
\end{theo}
\par\bigskip
\noindent Man kan säga att $\R P^2 = \left\{\text{linjer genom orio i $\R^3$}\right\}$
\par\bigskip
\noindent Punkter i $\R P^2$ representeras av en trippel av tal (3-dimensionell vektor) i $\R^3$, exempelvis på följande sätt $(x,y,z)$ och skall alltså betraktas som lika med $(tx,ty,tz)$. Det enda villkoret vi har är att $(x,y,z)\neq(0,0,0)$.\par
\noindent Detta sätt att betrakta $(x,y,z)$ kallas för \textit{homogena koordinater}. Punkter i $\R P^2$ kallas \textit{projektiva punkter}.
\par\bigskip
\noindent\textbf{Anmärkning:}\par
\noindent Varje punkt $(x,y,z)$ i $\R P^2$ där $z\neq0$, har en unik representant där $z=1$ $\left(\dfrac{x}{z},\dfrac{y}{z},1\right)$. Varje sådan punkt ($x,y,z$) motsvarar förstås också en projektiv punkt.\par
\noindent Vi har en injektiv funktion $(x,y)\mapsto(x,y,1)$ vars bild täcker det mesta av $\R P^2$.\par
\noindent För att beskriva dessa punkter räcker det alltså med att ge 2 koordinater $(x,y)$. Precis som vi kan använda latitud och longitud och rita en bit av jordytan i en kartbok.\par
\noindent Denna del av $\R P^2$ kallas för \textit{den affina ($x,y$) kartan}
\par\bigskip
\noindent\textbf{Exempel:}\par
\noindent Punkten $(2,1,3)$  har $z\neq0$ och har $\left(\dfrac{2}{3},\dfrac{1}{3}\right)$ som projektiv punkt i $\R P^2$
\par\bigskip
\noindent\textbf{Exempel:}\par
\noindent Mängden av projektiva punkter som uppfyller $zy = x^2+z^2$ och dessutom har $z\neq0$ utgör kurvan $y=x^2+1$ i $xy$-kartan eftersom vi sätter $z=1$.\par
\noindent Men, om $z=0$ så får vi $x=0$, medan $y$ kan anta vilket värde som helst, det finns alltså en projektiv punkt kvar, nämligen $(0,1,0)$ (egentligen $(0,y,0)$ men de är homogena koordinater för samma punkt). Det motsvarar "horizonten".
\par\bigskip
\noindent De andra affina kartorna $xz$ och $yz$  definieras analogt. Tänk på det som $\R^2\subset \R P^2$
\par\bigskip
\noindent\textbf{Exempel:}\par
\noindent $(2,1,3)$ har koordinater $(2,3)$ i $xz$-kartan (dela med $y$, vilket är 1)
\par\bigskip
\noindent\textbf{Exempel:}\par
\noindent Den återstående punkten $(0,1,0)$ syns bara i en karta, dvs i $xz$-kartan, där den är origo.
\par\bigskip
\begin{theo}
  GGivet säg, $xy$-kartan, så är linjen i oändligheten mängden:
  \begin{equation*}
    \begin{gathered}
      L_{\infty}^{xy} = \left\{(x,y,z)\in\R P^2 : z=0\right\}
    \end{gathered}
  \end{equation*}
\end{theo}
\par\bigskip
\begin{theo}[Homogent polynom]{thm:homo}
  Ett polynom $F(x_1,\cdots,x_n)\in k[x_1,\cdots, x_n]$ kallas \textit{homogent} av grad $d$ om alla termer har total grad $d$.\par
  \noindent Exmepelvis:
  \begin{equation*}
    \begin{gathered}
      t^dF(x,y,z)=F(tx,ty,tz)
    \end{gathered}
  \end{equation*}
\end{theo}
\par\bigskip
\noindent\textbf{Anmärkning:}\par
\noindent Om $F(x,y,z)\in\R[x,y,z]$ är homogent och $F(a,b,c)=0$ $\Rightarrow$ $F(ta,tb,tc)=0$\par
\noindent Detta betyder att $V(F)$ i $\R P^2$ är väldefinierad. Vi kallar $V(F)$ för en \textit{reell plan projektiv kurva} 
\par\bigskip
\noindent\textbf{Exempel:}\par
\noindent $x+y+z=0$ är ett homogent polynom av grad 1. Detta beskriver ett plan, men vi ska tänka på det som en projektiv linje. Mer allmänt kan vi sätta vilka koefficienter som hels, dvs $ax+by+cz=0$.\par
\noindent I kartan $z=1$ har linjen ekvationen $ax+by+c=0$ i $xy$-planet
\par\bigskip
\noindent\textbf{Exempel:}\par
\noindent Om vi bygger vidare på föregående exempel, om $a=b=0$ och $z=0$ r linjen i $\infty$ $xy$-kartan
\par\bigskip
\begin{theo}
  GGivet två olika projektiva punkter, finns en entydig projektiv linje genom den (som i euklidisk geometr).
  \par\bigskip
  \noindent Givet två olika linjer i $\R P^2$ finns en entydig projektiv punkt som ligger på båda
\end{theo}
\par\bigskip
\begin{prf}
  TVå projektiva punkter är två olika linjer. En entydig projektiv linje är då ett plan, och detta plan spänns up av de två olika linjerna
  \par\bigskip
  \noindent Två olika linjer i det projektiva planet är två olika plan, och dessa plan skär varandra i en linje, som projektivt är en punkt
\end{prf}
\par\bigskip
\begin{theo}
  GGivet ett polynom $f(x,y)\in\R[x,y]$ av grad $d$, definierar vi dess \textit{homogenisering} som $F(x,y,z) = z^df\left(\dfrac{x}{z},\dfrac{y}{z}\right)$ 
\end{theo}
\par\bigskip
\noindent\textbf{Exempel:}\par
\noindent $f(x,y) = y-x^2-1\Rightarrow F(x,y,z) = z^2\left(\dfrac{y}{z}-\left(\dfrac{x}{z}\right)^2-1\right) = zy-x^2-z^2$
\par\bigskip
\noindent\textbf{Anmärkning:}\par
\noindent Ger oss en projektiv kurva givet en affin kurva. Det omvända gäller om vi betraktar kartor till det projektiva planet.
\par\bigskip
\noindent\textbf{Exempel:}\par
\noindent $f(y,z)=zy-z^2-1\Rightarrow zy-z^2-x^2=F(x,y,z)$. Vad är $zy-z^2=1$? En hyperbel förstås!\par
\noindent Detta är ju det vi visat med våra klassificeringar som vi gjort, det vill säga en cirkel = parabel = hyperbel.
\par\bigskip
\noindent Med homogenisering kan vi tänka oss att vi kan gå från affina kurvor till projektiva kurvor med hjälp av homogenisering.
\par\bigskip
\begin{theo}[Projektiva tangentlinjen]{thm:ligma}
  Den projektiva tangentlinjen till $F(x,y,z) = 0$ i punkten $P=(a,b,c)$ ges av (vi antar $\nabla F(a,b,c)\neq0$):
  \begin{equation*}
    \begin{gathered}
      \dfrac{\partial F}{\partial x} (a,b,c)x+\dfrac{\partial F}{\partial y}(a,b,c,)y+ \dfrac{\partial F}{\partial z}(a,b,c)z =0
    \end{gathered}
  \end{equation*}
\end{theo}
\par\bigskip
\subsection{Projektiva transformationer}\hfill\\\par
\noindent Kommer ersätta affina transformationer, är en större klass av avbildningar men kan vara lättare att hantera. För att begripa dessa är det bra att ha koll på linjära transformationer.
\par\bigskip
\noindent $\bar{F}:\R^3\to\R^3$ - en linjär avbildning med ker$\bar{F} = \left\{\bar{0}\right\}$. Detta betyder att den är injektiv (men eftersom vi går från $\R^3\to\R^3$ så är den även bijektiv).
\par\bigskip
\noindent Eftersom det är en bijektiv homomorfi, är det även en isomorfi.
\par\bigskip
\noindent\textbf{Anmärkning:}\par
\noindent Om $L$ är en linje genom origo, så är $\bar{F}(L)$ också det. Den här avbildning \textit{inducerar} en avbildning $F_P:\R P^2\to\R P^2$ på det projektiva planet.\par
\noindent Det händer nästan ingenting här, i homogena koordinater ges $L=(x,y,z)$ $F_P(x,y,z)$ av $F(x,y,z)$
\par\bigskip
\noindent\textbf{Exempel:}\par
\noindent Betrakta följande linjär avbildning $\bar{F}(x,y,z) = (2x,2y,2z)$, då är $F_P = id_{\R P^2}$ eftersom de projektivt är samma (kom ihåg ekvivalensrelationen)
\par\bigskip
\noindent\textbf{Exempel:}\par
\noindent En allmän transformation $\bar{F}$ ges av en matris $\begin{pmatrix}x\\y\\z\end{pmatrix}\mapsto\begin{pmatrix}a&b&c\\d&e&f\\g&h&i\end{pmatrix}\begin{pmatrix}x\\y\\z\end{pmatrix}$\par
\noindent Att beräkna $\bar{F}_P$ är bara att betrakta $\begin{pmatrix}x\\y\\z\end{pmatrix}$ som homogena koordinater, och räkna som vanligt.
\par\bigskip
\noindent\textbf{Exempel:}\par
\noindent Om matrisen är exempelvis $\begin{pmatrix}a&b&c\\d&e&f\\0&0&1\end{pmatrix}\begin{pmatrix}x\\y\\z\end{pmatrix} = \begin{pmatrix}ax+by+cz\\dx+ey+fz\\z\end{pmatrix}$
\par\noindent Låt oss titta på den här i $xy$-kartan ($z=1$). Då ser den ut som $(x,y)\leftrightarrow(x,y,1) \Rightarrow \begin{pmatrix}ax+by+c\\dx+ey+f\\1\end{pmatrix}$\par
\noindent I $xy$-kartan ser alltså avbildningen ut på följande vis $\begin{pmatrix}x\\y\end{pmatrix}\mapsto\begin{pmatrix}ax+by+c\\dx+ey+f\end{pmatrix}$
\par\bigskip
\noindent Men notera! Detta är ju bara en affin avbildning! Med andra ord, projektiva avbildningar generaliserar affina avbildningar (en projektiv avbildning med $z=1$ visade vi nu var affin).
\par\bigskip
\noindent\textbf{Exempel:}\par
\noindent Vilka avbildningar bevarar linjen $z=0$? Jo det måste vara en affin avbildning på följande form:
\begin{equation*}
  \begin{gathered}
    \begin{pmatrix}a&0&c\\0&a&f\\0&0&i\end{pmatrix}\Rightarrow (ax,ay,0) = (x,y,0)
  \end{gathered}
\end{equation*}
\par\bigskip\bigskip
\noindent\textbf{Exempel:}\par
\noindent Låt $\bar{F}(x,y,z)=(x,z,y)$. Detta är en spegling i $y=z$. Hur ser den ut i $xy$-kartan?\par
\noindent Vi gör följande: $(x,y)\mapsto(x,y,1)=(x,1,y) = \left(\dfrac{x}{y},\dfrac{1}{y},1\right)\leftrightarrow\left(\dfrac{x}{y},\dfrac{1}{y}\right)$
\par\bigskip
\noindent Säg att vi har 4 punkter i planet (hörn i en rektangel), vad kan hända med dessa under en affin transformation? Under affin transformation får exempelvis vi parallelogram.
\par\bigskip
\begin{theo}
  SLåt $z_1,\cdots, z_4$ och $w_1,\cdots, w_4$ vara punkter i det projektiva planet. De ska ha egenskaperna så att 3 av de ligger aldrig på en rät linje (se det som hörn i en fyrhörning)
  \par\bigskip
  \noindent Då finns en entydig projektiv transformation $F$ så att $F(z_i) = w_i\quad i=1\cdots4$
\end{theo}
\par\bigskip
\begin{prf}
  BRepresentera $z_i\in\R P^2$ med en vektor $\bar{z_i}\in\R$ och samma med $w_i$
  \par\bigskip
  \noindent Villkoret betyder att ingen deltrippel av $\left\{\bar{z_1},\cdots,\bar{z_4}\right\}$ är i samma plan (dvs linjärt beroende). De utgör alltså en bas för $\R^3$. Detsamma för $\bar{w_i}$
  \par\bigskip
  \noindent Eftersom 3 nu utgör en bas så kan vi uttrycka den fjärde som en linjärkombination av de andra:
  \begin{equation*}
    \begin{gathered}
      \bar{z_4} = \sum_{i=1}^{3}a_i\bar{z_i}\\
      \bar{w_4} = \sum_{i=1}^{3}b_i\bar{w_i}
    \end{gathered}
  \end{equation*}
  \par\bigskip
  \noindent Vi definierar nu en linjär avbildning $\bar{F}:\R^3\to\R^3$ genom:
  \begin{equation*}
    \begin{gathered}
      \bar{F}(\bar{z_i}) = \dfrac{b_i}{a_i}\bar{w_i}\qquad i=1,2,3
    \end{gathered}
  \end{equation*}
  \par\bigskip
  \noindent Då får vi:
  \begin{equation*}
    \begin{gathered}
      \bar{F}(\bar{z_4}) = \bar{F}\left(\sum_{i=1}^{3}a_i\bar{z_i}\right) = \sum_{i=1}^{3}a_i\bar{F}(\bar{z_i}) = \sum_{i=1}^{3}a_i\dfrac{b_i}{a_i}\bar{w_i} = \sum_{i=1}^{3}b_i\bar{w_i} = \bar{w_4}
    \end{gathered}
  \end{equation*}
  \par\bigskip
  \noindent Detta visar existens. Om $G:\R^3\to\R^3$ också har samma egenskap, dvs $G(\bar{z_i})=\lambda_i\bar{w_i}$\par
  \begin{equation*}
    \begin{gathered}
      G(\bar{z_4}) = \sum_{i=1}^{3}a_i\bar{z_i} = \sum_{i=1}^{3}a_iG(z_i) = \sum_{i=1}^{3}a_i\lambda_i\bar{w_i}\\
      \text{Å andra sidan } G(\bar{z_4}) = \lambda_4\bar{w_4} = \sum_{i=1}^{3}\lambda_4b_i\bar{w_i}\\
      \Rightarrow \lambda_i = \lambda_4\dfrac{b_i}{a_i}
    \end{gathered}
  \end{equation*}
  \par\bigskip
  \noindent Så $\bar{G}(\bar{z_i}) = \lambda_4\dfrac{b_i}{a_i}\bar{w_i} = \lambda_4\bar{F}(\bar{z_i})$\par
  \noindent Från detta följer det att $\bar{G} = \lambda_4\bar{F}$ så $\bar{G_P} = \bar{F_p}$
\end{prf}
\newpage
\subsection{Ekvivalenta polynom upp till projektiv transformation}\hfill\\\par
\noindent Homogena polynom i 3 variabler betecknas som $\R_h[x,y,z]\subseteq\R[x,y,z]$
\par\bigskip
\noindent Som vanligt, om vi har ett homogent polynom $F(x,y,z)$ och en projektiv transformation $G_p(\bar{G})$ kan vi sammansätta polynomet med transformationen och göra ett "variabelbyte" (detta är inverterbart). Då kan vi betrakta polynom som ekvivalenta om vi kan få vardera polynom med en sådan transformation, dvs $F\circ\bar{G} = \bar{G}^*(F)$
\par\bigskip
\noindent Vår reella affina klassificeringar hade en hel klassificeringar av $\R$-kurvor: 
\begin{itemize}
  \item $x^2+y^2-1$ (cirkel)
  \item $x^2-y^2-1$ (hyperbel)
  \item $x^2-y$ (parabel)
  \item $x^2-y^2$ (linjekon)
  \item $x^2+y^2$ (punkt)
  \item$x(x-1)$ (två parallella linjer)
  \item $x^2$ ("dubbellinje")
  \item $x^2+1$ (tom)
  \item $x^2+y^2+1$ (tom)
\end{itemize}
\par\bigskip
\noindent Vi påminner oss dock om att när vi hade parabeln $y=x^2+1$ kunde vi få en cirkel, och en hyperbel genom en bijektiv affin transformation. Då kommer alltså följande ske:\par
\begin{equation*}
  \begin{gathered}
    \begin{rcases*}
      x^2+y^2-1\\
      x^2-y^2-1\\
      y-x^2
    \end{rcases*}\Rightarrow x^2+y^2-z^2\text{ (kon)}\\
    \begin{rcases*}
      x^2-y^2\\
      x(x-1)
    \end{rcases*}\Rightarrow x^2-y^2\text{ (parallella linjer)}\\
    \begin{rcases*}
      x^2+y^2\\
      x^2+1
    \end{rcases*}\Rightarrow x^2+y^2 \text{ (punkt)}\\
    x^2\Rightarrow\text{ (punkt)}\\
    x^2+y^2+1\Rightarrow\\O
  \end{gathered}
\end{equation*}
\par\bigskip
\noindent Projektivt sätt finns det bara ett icke-singulärt polynom.
\par\bigskip
\subsection{Skärningstalet i $\R P^2$}\hfill\\\par
\noindent Vi har ju bara satt upp axiom för det, men nu kommer vi defniera det genom att välja en affin karta där punkten vi vill räkna skärningstalet för ligger i, och sen använda axiomen som vanligt.\par
\noindent Det finns en risk med detta, och det är att det kan bero på vilken karta vi använder (kurvorna kanske tangerar istället). Vi ändrar detta genom att säga att det inte är så, det vill säga inför:
\par\bigskip
\noindent\textbf{Axiom 0}:\par
\noindent Vi vill kunna titta på skärningen i vilket affint koordinatssystem som helst (bevaras under affin transformation): (Affint invariant)
\begin{equation*}
  \begin{gathered}
    I_p(f,g) = I_{F^{-1}(p)}(F^*f,F^*g)\qquad\forall F\in\text{Aff}_\C(2)
  \end{gathered}
\end{equation*}
\par\bigskip
\noindent\textbf{Axiom 1:}\par
\noindent Säger att $f$ och $g$ skär varandra lika många gånger som $g$ och $f$ skär varandra:
\begin{equation*}
  \begin{gathered}
    I_p(f,g) = I_p(g,f)
  \end{gathered}
\end{equation*}
\par\bigskip
\noindent\textbf{Axiom 2:}\par
\noindent Skärningstalet mellan $f$ och $g$ skall vara nollskillt omm punkten $p$ ligger på båda kurvorna:
\begin{equation*}
  \begin{gathered}
    I_p(f,g)\neq0\Lrarr f(p) = g(p) = 0
  \end{gathered}
\end{equation*}
\par\bigskip
\noindent\textbf{Axiom 3:}\par
\noindent Koordinataxlarna (speciella kurvor) skär varandra i en enda punkt, alltså borde de har skärningstalet 1:
\begin{equation*}
  \begin{gathered}
    I_0(x,y) = 1
  \end{gathered}
\end{equation*}
\par\bigskip
\noindent\textbf{Axiom 4:}\par
\noindent $I_p(f,gh) = I_p(f,g)+I_p(f,h)$ 
\par\bigskip
\noindent\textbf{Axiom 5:}\par
\noindent Skärningstalet i $p$ mellan $f$ och $g$ är detsamma som skärningstalet mellan $f$ och $g+fh$:
\begin{equation*}
  \begin{gathered}
    I_p(f,g) = I_p(f,g+fh)\qquad\forall h\in\C[x,y]
  \end{gathered}
\end{equation*}
\par\bigskip
\noindent\textbf{Axiom 6}: $I_p$ är invariant under projektiv transformation.
\par\bigskip
\noindent\textbf{Exempel:}\par
\noindent Finn de punkter i $\R P^2$ där
\begin{equation*}
  \begin{gathered}
    C=x^3-3x^2y+2xy-4y-10=0
  \end{gathered}
\end{equation*}\par
\noindent Skär linjen $L_\infty (z=0)$\par
\noindent Beräkna också skärningstalet $I_p(C,L_\infty)$ i dessa punkter.
\par\bigskip
\noindent\textit{Lösning:} Börja med att homogenisera:
\begin{equation*}
  \begin{gathered}
    x^3-3x^2y+2xyz-4yz^2-10z^3=0
  \end{gathered}
\end{equation*}\par
\noindent Var skär denna kurva linjen $z=0$? Sätt $z=0$ och vi får då:
\begin{equation*}
  \begin{gathered}
    x^3-3x^2y=0\Lrarr x^2(x-3y)=0
  \end{gathered}
\end{equation*}\par
\noindent Vi har alltså lösningar $x=0, z=0$ och $x=3y, z=0$, dvs $(0,1,0), (3,1,0)$\par
\noindent Nu ska vi hitta skärningstalet i dessa punkter. Då bör vi hitta en lämplig affin karta att räkna i, vi väljer $xz$-kartan eftersom båda punkter ligger där.\par
\noindent Våra punkter blir då $(0,0)$ resp. $(3,0)$, men hur ser ekvationerna ut här? Sätt $y=1$:
\begin{equation*}
  \begin{gathered}
    \begin{cases}
      \overbrace{x^3-3x^2+2xz-4z^2-10z^3=0}^{\text{$g(x,z)$}}\\
      z=0
    \end{cases}
  \end{gathered}
\end{equation*}\par
\noindent Här är $z=0$, vilket är grafet till ett polynom ($p(x)=0$), men det är inte $g(x,z)$ (det är inte en graf till ett envariabelpolynom)\par
\noindent Notera, enligt Theorem 7.1 får vi skärningstalet som multipliciteten till motsvarande nollstället till $g(x,p(x)) = g(x,0) = x^3-3x^2 =x^2(x-3)$. Detta ger att $x=0, x=3$ är en lösning, och multipliciteten av $x=0$ är 2, alltså är skärningstalet i origo $=2$ och i $x=3=1$
\par\bigskip
\subsection{Generalisering av $\R P^2$}\hfill\\\par
\noindent Projektiva rum existerar i varje dimension, låt oss betrakta reella projektiva \textit{linjen} $\R P^1$.\par
\noindent Istället för att börja med $\R^3$ börjar vi med $\R^2$ där alla linjer genom genom origo är ekvivalenta. Homogena koordinater blir bara en dimension mindre $(x,y)=(\lambda x, \lambda y)$
