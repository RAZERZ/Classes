\section{Plana algebraiska kurvor}
\par\bigskip
\noindent Vi inleder med definition:
\par\bigskip
\begin{theo}[Plan affin algebraisk kurva]{thm:planeaffinecurve}
  En \textbf{plan affin algebraisk kurva} är nollställesmängden till ett icke-konstant polynom $f(x,y)\in\R[x,y]$ där $\R[x,y]$ är mängden av alla polynom med 2 variabler med reella koefficienter.
  \par\bigskip
  \noindent \textbf{Nollställesmängden} kan betecknas $V(f) = \{(x,y)\in\R^2:f(x,y)=0\}$ 
\end{theo}
\par\bigskip
\begin{theo}[Affin-avbildning]{thm:affinproj}
  En linjär avbildning är på formen $x\mapsto ax$, medan en affin avbildning är "ungefär linjär", dvs $x\mapsto ax+b$
\end{theo}
\par\bigskip
\noindent Ett sätt att betrakta polynom är att de är ett ändligt antal utförande av operatorer på kropp-element.
\par\bigskip
\noindent\textbf{Exempel:}\par
\noindent Betrakta följande polynom i $\R^2$, $ax+by+c = f(x,y)$. Polynomet är av grad 1, och är därför därmed ett linjärt polynom.
\par\bigskip
\noindent\textbf{Exempel:}\par
\noindent Vi kan även ha nollställesmängden som parabel med följande funktion $f(x,y) = y-x^2$ 
\par\bigskip
\noindent Bygger vi vidare på föregående exempel kommer vi fram till följande mer generella formel för att "omvandla" ett endimensionellt polynom till en flerdimensionell:
\begin{equation*}
  \begin{gathered}
    f(x,y) = y-p(x)
  \end{gathered}
\end{equation*}\par
\noindent Där $p(x)$ är ett godtyckligt polynom.
\par\bigskip
\noindent\textbf{Exempel:}\par
\noindent Om vi betraktar följande funktion $f(x,y) = x^2+y^2$ (enhetscirkeln) så har den en nollställesmängd som är en punkt.
\par\bigskip
\noindent\textbf{Exempel:}\par
\noindent Om vi betraktar tomma-mängden som nollställesmängd (dvs exempelvis $f(x,y) = x^2+y^2+1$) så är det absolut en valid nollställesmängd, men en obehaglig sådan ty det inte finns en intuitiv geometrisk bild, kan vi kalla den för en kurva? $f(x,y) = x^2+1$ har ju samma nollställesmängd!
\par\bigskip
\noindent\textbf{Exempel:}\par
\noindent Betrakta följande funktion $f(x,y)=xy$. Denna har unionen av $x$-axeln och $y$-axeln som lösningsmängd
\par\bigskip
\noindent En affin funktion från flervarren som vi kanske minns är faktiskt linjäriseringen av $f$:
\begin{equation*}
  \begin{gathered}
    f(\bar{r})\approx f(\bar{r}_0))+\nabla f(\bar{r}_0)\cdot (\bar{r}-\bar{r}_0)
  \end{gathered}
\end{equation*}
\par\bigskip
\noindent I den här kursen tillåter vi allmänna linjära basbyten, alltså ej bara isometriska avbildningar utan vi kan skala om ena axeln och krympa den/deformera den!
\par\bigskip
\begin{theo}[Singulära punkter]{thm:singpoint}
  En punkt $\bar{r}_0$ sådant att $f(\bar{r}_0)=0$ sådant att $\nabla f(\bar{r}_0=(0,0))$ kallas \textbf{singulär}. Singulära punkter bevaras under affin transformation. 
end{theo}
\par\bigskip
\begin{theo}[Transversell skärning]{thm:cross}
  Två kurvor $f(\bar{r})= 0$ och $g(\bar{r})=0$ sägs skära varandra transversellt i $\bar[r]_0$ om $f(\bar{r}_0) = 0 = g(\bar{r}_0)$ och $\nabla f(\bar{r}_0) \neq 0 \neq \nabla g(\bar{r}_0)$ och $\nabla f(\bar{r}_0)$ och $\nabla g(\bar{r}_0)$ är \textit{inte} parallella (linjärkombinationer av varandra)
\end{theo}
