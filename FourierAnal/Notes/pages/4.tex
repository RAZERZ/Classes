\section{Tillämpningar på differentialekvationer}\par
\noindent Innan vi börjar, ska vi notera unikheten av Laplacetransformationen:
\par\bigskip
\begin{theo}
  OOm Laplacetransformationen för 2 kontinuerliga funktioner sammanfaller, så måste funktionerna vara samma.
  \noindent Om de enbart är kontinuerliga i en punkt, så måste de sammanfalla i den punkten om Laplacetransformationen är densamma. 
\end{theo}
\par\bigskip
\noindent Då kan vi dra slutsatsen att det finns en invers $\mathcal{L}^{-1}$ (på grund av unikhet)
\par\bigskip
\noindent\textbf{Exempel:}\par
\noindent Lös följande linjära förstaordningens IVP med konstanta koefficienter med Laplace:
\begin{equation*}
  \begin{gathered}
    \begin{cases*}
      y^{\prime}(t) +y(t)=3\\
      y(0) = 2
    \end{cases*}
  \end{gathered}
\end{equation*}
\par\bigskip
\noindent\textbf{Lösning:}\par
\noindent Det första vi gör är att köra Laplacetransformation på allt:
\begin{equation*}
  \begin{gathered}
    \mathcal{L}[y^{\prime}(t)]+\mathcal{L}[y(t)] = \mathcal{L}[3]\\
    s\cdot\mathcal{L}[y(t)]-\underbrace{y(0)}_{\text{2}}+\mathcal{L}[y(t)] = \mathcal{L}[3]\\
    \mathcal{L}[y(t)](s+1) -2 = \mathcal{L}[3] = \dfrac{3}{s}\\
    \mathcal{L}[y(t)] = \dfrac{\left(\dfrac{3}{s}+2\right)}{s+1} = \dfrac{3+2s}{s(s+1)}
  \end{gathered}
\end{equation*}
\par\bigskip
\noindent Nu ska vi finna inversen till denna Laplacetransformation, då får vi $y(t)$. Vi kan kika i formelbladet för att hitta detta:
\begin{equation*}
  \begin{gathered}
    \dfrac{3+2s}{s(s+1)} = \dfrac{A}{s}+\dfrac{B}{s+1} = \dfrac{3}{s}-\dfrac{1}{s+1}\\
    y(t) = 3-e^{-t}\quad t\geq0
  \end{gathered}
\end{equation*}
\par\bigskip
\noindent\textbf{Anmärkning:}\par
\noindent Lösningen till en ODE är kontinuerlig
\par\bigskip
\noindent\textbf{Anmärkning:}\par
\noindent Integralens värde ändras inte om funktionen är diskontinuerlig i ändligt många punkter. 
\par\bigskip
\noindent Laplacetransformationer är hjälpsamma för att inte bara lösa differentialekvationer, men \textit{integralekvationer}
\par\bigskip
\subsection{Tillämpningar till integralekvationer}\hfill\\\par
\noindent\textbf{Exempel:}\par
\noindent Hitta en funktion $f(t)$ sådant att den löser följande ekvation:
\begin{equation*}
  \begin{gathered}
    2\underbrace{\int_{0}^{t}\cos(t-x)f(x)dx}_{\text{Konvolution av $\cos*f$}} = f(t)+3
  \end{gathered}
\end{equation*}
\newpage
\noindent\textbf{Lösning:}\par
\noindent Börja med Laplacetransformation och använd egenskapen att Laplacetransformationen av konvolutio blir produkt:
\begin{equation*}
  \begin{gathered}
    2\dfrac{s}{s^2+1}\cdot\mathcal{L}[f] = \mathcal{L}[f]+\dfrac{3}{s}\\
    \mathcal{L}[f]\left(1-\dfrac{2s}{s^2+1}\right) = -\dfrac{3}{s}\\
    \Lrarr \mathcal{L}[f]\left(\dfrac{s^2-2s+1}{s^2+1}\right) = \mathcal{L}[f]\dfrac{(s-1)^2}{s^2+1} = -\dfrac{3}{s}\\
    \Lrarr\mathcal{L}[f] = -\dfrac{3s^2+3}{s(s-1)^2}
  \end{gathered}
\end{equation*}
\par\bigskip
\noindent Nu kan vi hitta inversen till Laplacetransformationen, vilket vi kan göra med partialbråksuppdelning, vi får då: (\textbf{CHECK})
\begin{equation*}
  \begin{gathered}
    \mathcal{L}^{-1} = f(t) = -3-6te^t
  \end{gathered}
\end{equation*}
