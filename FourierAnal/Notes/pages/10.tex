\section{Mean-Square konvergens av Fourierserier och Semi-inre-produktrum}\par
\noindent Tanken här är att undersöka konvergens av funktioner i vektorrum (med funktionen som bas). Vi börjar med att bevisa följande:
\par\bigskip
\begin{theo}
  OOm $f:\R\to\C$ är $2\pi$-periodisk och integrerbar i intervallet $[0,2\pi]$, så:
  \begin{equation*}
    \begin{gathered}
      \lim_{N\to\infty}\dfrac{1}{2\pi}\int_{0}^{2\pi}\left|S_N(f)(x)-f(x)\right|^2dx \to0
    \end{gathered}
  \end{equation*}\par
  \noindent "Om genomsnittet av skillnaden konvergerar"
\end{theo}
\par\bigskip
\noindent\textbf{Anmärkning:}\par
\noindent Kallas för $L^2$ konvergens.
\par\bigskip
\noindent För att visa detta kommer vi kräva en del linjär algebra.
\par\bigskip
\noindent Låt $V$ vara ett vektorrum över $\C$. Vi påminner oss om vad inre-produkt är:
\par\bigskip
\begin{theo}[Semi Inre-produkt]{thm:innerprod}
  En semi inre-produkt (semi för att vi inte har Lebesgue integraler) på vektorrummet $V$, är en avbildning $<\cdot,\cdot>:V\times V\to\C$ som uppfyller:\par
  \begin{itemize}
    \item $<u,v> = \overline{<v,u>}$ (skew-kommutativ, "taket" över är komplexa konjugatet)
    \item $<a_1u_1+a_2u_2,v> = a_1<u_1,v>+a_2<u_2,v>$ (linjäritet i första input)
    \item $<u,u>\geq0\in\R$
  \end{itemize}
\end{theo}
\par\bigskip
\noindent\textbf{Anmärkning:}\par
\noindent Sista punkten, det följer att det är ett reellt tal efterom reella tal är de enda som uppfyller första punkten.
\par\bigskip
\begin{theo}[Inre-produkt]{thm:innerprodd}
  Om vi lägger till endast ett till krav på definitionen till semi inre-produkt, så får vi inre-produkt, nämligen att\par
  \begin{itemize}
    \item $<u,u> = 0\Rightarrow u = \overline{0}$
  \end{itemize}
\end{theo}
\par\bigskip
\begin{theo}[Semi norm]{thm:seminorm}
  Definieras på följande:
  \begin{equation*}
    \begin{gathered}
      \left|\left|u\right|\right| = \sqrt{<u,u>}
    \end{gathered}
  \end{equation*}
\end{theo}
\par\bigskip
\noindent\textbf{Exempel (*):}\par
\noindent Låt $V = \C^n$ och:
\begin{equation*}
  \begin{gathered}
    <\underbrace{(u_1,\cdots,u_n)}_{\text{$\in\C^n$}},\underbrace{(w_1,\cdots,w_n)}_{\text{$\in \C^n$}}>\\
    u_1\overline{w_1}+u_2\overline{w}+\cdots+u_n\overline{w_n}
  \end{gathered}
\end{equation*}\par
\noindent Detta är en inre-produkt
\par\bigskip
\noindent\textbf{Exempel:}\par
\noindent Låt $V = C(a,b) = \left\{\text{kontinuerliga funktioner } f:[a,b]\to\C\right\}$\par
\noindent Här kan vi även definiera en inre-produkt $<f,g>$ på följande vis:
\begin{equation*}
  \begin{gathered}
    <f,g> = \dfrac{1}{b-a}\int_{a}^{b}f(x)\overline{g(x)}dx
  \end{gathered}
\end{equation*}
\newpage
\noindent\textbf{Anmärkning:}\par
\noindent Detta är en giltig inre-produkt eftersom om:
\begin{equation*}
  \begin{gathered}
    f\in C(a,b)\quad \int_{a}^{b}\left|f(x)\right|^2dx = 0\Rightarrow f(x) = 0\;\forall x
  \end{gathered}
\end{equation*}
\par\bigskip
\noindent\textbf{Anmärkning:}\par
\noindent Det komplexa konjugatet för en funktion tar komplexa konjugatet av evalueringen i en punkt.
\par\bigskip
\noindent\textbf{Anmärkning:}\par
\noindent Generellt sagt, genom att fixera en kontinuerlig funktion $w:[a,b]\to(0,\infty)$, så kan vi definiera inre produkten $<f,g>$ till:
\begin{equation*}
  \begin{gathered}
    <f,g> = \dfrac{1}{b-a}\int_{a}^{b}f(x)\overline{g(x)}w(x)dx
  \end{gathered}
\end{equation*}\par
\noindent Detta uppfyller fortfarande kraven för en inre-produkt på $C(a,b)$\par
\noindent Vi kan se $w$ som en slags "vikt" funktion som är praktisk att ha om vi undersöker funktioner som beter sig intressant i ett intervall. 
\par\bigskip
\noindent\textbf{Exempel:}\par
\noindent Låt $V = I(a,b) = \left\{\text{integrerbara funktioner } f:[a,b]\to\C\right\}$, vi definierar $<f,g>$ på fölajnde:
\begin{equation*}
  \begin{gathered}
    <f,g> = \dfrac{1}{b-a}\int_{a}^{b}f(x)\overline{g(x)}dx
  \end{gathered}
\end{equation*}\par
\noindent Detta är en \textit{semi-inre-produkt}.\par
\noindent Anledningen till varför det är en \textit{semi} och inte en vanligt inre-produkt är följande\par
Ponera att vi har $f\in I(2,3) = \begin{cases}1\quad x=2\\0\quad x\neq2\end{cases}$\par
Denna funktion är inte 0-funktionen, men kör vi $<f,f>$ så får vi 0
\par\bigskip
\noindent\textbf{Exempel:}\par
\noindent Låt $V = l^2(\Z) = \left\{\text{funktioner } f:\Z\to\C\;(k\mapsto a_k)\text{ så att } \sum_{k=-\infty}^{\infty}\left|a_k\right|^2<\infty\right\}$
\par\bigskip
\noindent Detta påminner ju om fourierserier och fourierkoefficienterna! Givet en funktion, så kan vi associera ett element i rummet $V$.\par
\noindent Vår inre-produkt $<\left\{a_k\right\}.\left\{b_k\right\}>$ är:
\begin{equation*}
  \begin{gathered}
  <\left\{a_k\right\}, \left\{b_k\right\}> = \sum_{k=-\infty}^{\infty}a_k\overline{b_k}
  \end{gathered}
\end{equation*}\par
\noindent Detta är en inre-produkt och en generalisering av exempel (*)
\par\bigskip
\noindent\textbf{Anmärkning:}\par
\noindent Från och med nu ska vi låta $(V,<\cdot,\cdot>)$  vara ett semi-inre-produkt-rum ($V$ vektorrum, och $<\cdot,\cdot>$ semi-inre-produkt)
\par\bigskip
\begin{theo}[Egenskaper hos semi-inre-produkt]{thm:correswarre}
  \begin{itemize}
    \item $\left|<v,w>\right|\leq \left|\left|v\right|\right|\cdot\left|\left|w\right|\right|$ (Cauchy-Schwartz)
    \item $\left|\left|v+w\right|\right| \leq \left|\left|v\right|\right|+\left|\left|w\right|\right|$ (Ye' Olde Triangelolikheten)
  \end{itemize}
\end{theo}
\par\bigskip
\begin{theo}[Ortogonala vektorer]{thm:ortoganalvec}
  Vektorerna $v,w\in V$ är \textit{ortogonala} om $<v,w> = 0$\par
  \noindent Betecknas $v_{\perp}w$
\end{theo}
\par\bigskip
\begin{theo}[Ortonormala vektorer]{thm:orthonormal}
  Vektoerna $v,w\in V$ är \textit{ortonormala} om de är ortogonala \textit{och} $\left|\left|v\right|\right| = 1 = \left|\left|w\right|\right|$
\end{theo}
\par\bigskip
\begin{theo}[Pythagoras sats]{thm:pyrresarre}
  Om $v_{\perp}w$, så:
  \begin{equation*}
    \begin{gathered}
      \left|\left|v+w\right|\right|^2 = \left|\left|v\right|\right|^2+\left|\left|w\right|\right|
    \end{gathered}
  \end{equation*}
\end{theo}
\par\bigskip
\begin{prf}[Pythagoras sats]{prf:pyrresarre}
  \begin{equation*}
    \begin{gathered}
      \left|\left|v+w\right|\right|^2 \stackrel{def}{=} <v+w,v+w> = \underbrace{<v,v+w>}_{\text{$\overline{<v+w,v>} = \overline{<v,w>}+\underbrace{\overline{<w,v>}}_{\text{=0}}$}}+\overbrace{<w,v+w>}^{\text{$\overbrace{\overline{<w,v>}}^{\text{=0}}+\overline{<w,w>}$}}\\
      \Rightarrow <v,v>+<w,w> = \left|\left|v\right|\right|^2+\left|\left|w\right|\right|^2
    \end{gathered}
  \end{equation*}
\end{prf}
\par\bigskip
\noindent\textbf{Exempel:}\par
\noindent Funktioner på formen $e^{inx}$ där $n\in\Z$ bildar en \textit{ortonormal mängd} av $V = I(0,2\pi)$\par
\noindent Vad menas en ortonormal mängd? En mängd med parvis ortonormala vektorer. Vi kan visa att detta gäller i vårat fall:
\begin{equation*}
  \begin{gathered}
    <e^{imx},e^{inx}> \stackrel{def}{=} \dfrac{1}{2\pi}\int_{0}^{2\pi}e^{imx}\underbrace{\overline{e^{inx}}}_{\text{$e^{-inx}$}}dx\\
    \Rightarrow = \dfrac{1}{2\pi}\int_{0}^{2\pi}e^{i(m-n)x}dx = \begin{cases}1\quad m=n\\0\quad m\neq n\end{cases}
  \end{gathered}
\end{equation*}\par
\noindent Fortsätter vi, tag vilken funktion som helst i detta semi-inre-produkt $f\in I(0,2\pi)$, vad är då $<f,e^{inx}>$?
\begin{equation*}
  \begin{gathered}
    <f,e^{inx}> = \dfrac{1}{2\pi}\int_{0}^{2\pi}f(x)\underbrace{e^{-inx}}_{\text{$\overline{e^{inx}}$}}dx = c_n
  \end{gathered}
\end{equation*}\par
\noindent Det visar sig vara de $n$:te fourierkoefficienterna! 
\par\bigskip
\noindent Något som vi kanske minns från linjär algebra var att vi kunde ta ett delrum från ett ortonormalt inre-produkt-rum och så kunde vi projicera. Vi undersöker vad som händer om vi gör detta med semi-inre-produkt-rum:
\par\bigskip
\begin{theo}[Ortogonal projection]{thm:orthoproj}
  Låt $v_1,\cdots,v_N\in V$ vara en ortonormal mängd (parvis ortonormal)\par
  \noindent Låt $W = \text{span}(v_1,\cdots,v_n)\subset V$
  \par\bigskip
  \noindent Vi definierar den \textit{ortogonala projectionen} $P:V\to W$ är:
  \begin{equation*}
    \begin{gathered}
      P(v) = \sum_{k=1}^{N}<v,v_k>\cdot v_k\in W
    \end{gathered}
  \end{equation*}
  \par\bigskip
  \noindent Vi kallar $v-P(v)$ \textit{residualen} (\textbf{SVENSKA}). Residulaen är ortogonal till hela $W$
\end{theo}
\newpage
\noindent\textbf{Exempel:}\par
\noindent Vi återgår lite till föregående exempel. Låt $V = I(0,2\pi)$. Fixera $N>0$ och tag \par
$W = \text{span}\underbrace{(e^{-iNx},e^{i(-N+1)x},\cdots,1,e^{ix},\cdots,e^{iNx})}_{\text{$2N+1$}}$
\par\bigskip
\noindent Givet $f\in I(0,2\pi)$:
\begin{equation*}
  \begin{gathered}
    P(f) = \sum_{k=-N}^{N}<f,e^{ikx}>e^{ikx} = \sum_{k=-N}^{N}c_ke^{ikx} = S_N(f)(x)
  \end{gathered}
\end{equation*}
\par\bigskip
\noindent När vi säger att fourierserien konvergerar så konvergerar givetvis även partiella summorna, som är funktionsföljder. Nu har vi översätt funktionsföljderna i termer av projektion vilket kommer låta oss genomföra bevis.
\par\bigskip
\noindent\textbf{Anmärkning:}\par
\noindent Residulaen är ortogonal mot $W$ (delrummet), dvs $(v-P(v))\perp W$
\par\bigskip
\begin{theo}[Minsta-kvadrat approximering]{thm:lsa}
  Givet en vektor $v\in V$ ($V$ är semi-inre-produkt-rum).\par
  \noindent Elementet $w\in W$ (där $W$ är uppspänd av spannet av en ändlig ortonormal mängd) som är närmast $v$, det vill säga $\left|\left|v-w\right|\right|$ är minimal
  \par\bigskip
  \noindent Detta är den ortogonala projektionen $P(v)$
\end{theo}
\par\bigskip
\begin{prf}
  TTag ett $v\in W$, givet något $w\in W$, då är normen:
  \begin{equation*}
    \begin{gathered}
      \left|\left|v-w\right|\right|^2 = \underbrace{\left|\left|\underbrace{v-P(v)}_{\text{residualen}}+\underbrace{P(v)-w}_{\text{$\in W$}}\right|\right|}_{\text{Ortogonala}}^2\\
      (v-P(v))\perp(P(v)-w)\stackrel{Pyth.}{\Rightarrow}\left|\left|v-P(v)\right|\right|^2+\underbrace{\left|\left|P(v)-w\right|\right|^2}_{\text{$\geq0$}}
    \end{gathered}
  \end{equation*}\par
  \noindent Som \textit{minst}, gäller när uttrycket $\left|\left|P(v)-w\right|\right|^2 = 0$
\end{prf}
\par\bigskip
\begin{theo}[Fullständig / Komplett ortonormal mängd]{thm:comportho}
  En ortonormal mängd $\left\{v_k\right\}$ där $k=1,\cdots$ (kan fortsätta åt oändligheten) är \textit{fullständig}/\textit{komplett} om för varje $v\in V$ och för varje $\varepsilon>0$ gäller att:
  \begin{equation*}
    \begin{gathered}
      \left|\left|v-\sum_{k=1}^{N}a_kv_k\right|\right|<\varepsilon\qquad N<\infty
    \end{gathered}
  \end{equation*}
  \par\bigskip
  \noindent Detta gäller för någon \textit{ändlig} mängd av koefficienter $a_1,a_2,\cdots,a_N\in\C$ 
\end{theo}
\newpage
\begin{theo}[Mean-square konvergens av fourierserier]{thm:mskf}
  Mängden vektorer $\left\{e^{ikx}\right\}$ där $k\in\Z$ (oändligt och ortonormal mängd) är fullständig i $I(\mathbb{T})$
  \par\bigskip
  \noindent Mer konkret, för vilken vektor som helst i mitt vektorrum $V$ (eller $f\in I(\mathbb{T})$) gäller:
  \begin{equation*}
    \begin{gathered}
      \lim_{N\to\infty}\left|\left|f-\underbrace{S_N(f)}_{\text{$=P(f)$ om $W=$span$(e^{-iNx},\cdots,1,e^{iNx})$}}\right|\right| = 0
    \end{gathered}
  \end{equation*}
  \par\bigskip
  \noindent Detta ska påminna lite om hur Taylorserier ser ut/fungerar. En funktion $f$ har bättre Taylorapproximation $S_N(f)$ ju fler termer man har med.
\end{theo}
\par\bigskip
\noindent\textbf{Anmärkning:}\par
\noindent Glum inte att:
\begin{equation*}
  \begin{gathered}
    S_N(f) = \sum_{k=-N}^{N}c_ke^{ikx}
  \end{gathered}
\end{equation*}\par
\noindent och:
\begin{equation*}
  \begin{gathered}
    \left|\left|f-S_N(f)\right|\right| = \sqrt{\dfrac{1}{2\pi}\int_{0}^{2\pi}\left|f(x)-S_N(f)(x)\right|^2dx}
  \end{gathered}
\end{equation*}
\par\bigskip
\begin{theo}[Parsevals formel]{thm:parseval}
  Om vi har en funktion $f\in I(\mathbb{T})$, så kan vi skriva en numerisk serie med hjälp av dess fourierkoefficienter:
  \begin{equation*}
    \begin{gathered}
      \sum_{k=-\infty}^{\infty}\left|c(f)_k\right|^2 = \dfrac{1}{2\pi}\int_{0}^{2\pi}\left|f(x)\right|^2dx
    \end{gathered}
  \end{equation*}
\end{theo}
\par\bigskip
\noindent Varför är det hjälpsamt? Jo, Dirichlets konvergens kriterier gav oss verktyg för att lösa konvergenser samt evaluering av serier. Parsevals formel ger (gratis) ett annat sätt att explicit räkna ut summor av oändliga serie
\par\bigskip
\begin{prf}
  A
  \begin{equation*}
    \begin{gathered}
      \dfrac{1}{2\pi}\int_{0}^{2\pi}\left|f(x)\right|^2dx = \left|\left|f(x)\right|\right|^2\\
      = ||\underbrace{\underbrace{f-\underbrace{S_N(f)}_{\text{$P(f)$}}}_{\text{residual}}+S_N(f)}_{\text{Ortogonala}}||^2\\
      (f-S_N(f))\perp S_N(f)\stackrel{Pyth}{\Rightarrow}\underbrace{\left|\left|f-S_N(f)\right|\right|^2}_{\text{$\stackrel{N\to\infty}{\to}0$}}+\left|\left|S_N(f)\right|\right|^2 = \lim_{N\to\infty}\left|\left|S_N(f)\right|\right|^2\\
      =\left|\left|\sum_{k=-N}^{N}c_ke^{ikx}\right|\right|^2\stackrel{Pyth}{=}\sum_{k=-N}^{N}\left|c_k\right|^2\underbrace{\left|\left|e^{ikx}\right|\right|}_{\text{=1}}\\
      \Rightarrow \lim_{N\to\infty}\sum_{k=-N}^{N}\left|c_k\right|^2 = \sum_{k=-\infty}^{\infty}\left|c_k\right|^2
    \end{gathered}
  \end{equation*}
\end{prf}
\par\bigskip
\noindent\textbf{Exempel:}\par
\noindent Om $f:\R\to\C$ är $2\pi$-periodisk och $f(x) = x$ för $-\pi\leq x<\pi$, så gäller:
\begin{equation*}
  \begin{gathered}
    f(x)\sim \sum_{n=1}^{\infty}\dfrac{2(-1)^{n+1}}{n}\sin(nx)
  \end{gathered}
\end{equation*}
\par\bigskip
\noindent Vi ska visa här att Parsevals formel kommer ge oss summan för $\dfrac{1}{n^2}$. För att göra det ska vi omvandla Parsevals formel i trigonometrisk form:
\begin{equation*}
  \begin{gathered}
    f(x)\sim\dfrac{a_0}{2}+\sum_{k=1}^{\infty}a_k\cos(kx)+b_k\sin(kx)\\
    \Rightarrow \dfrac{1}{2\pi}\int_{0}^{2\pi}\left|f(x)\right|^2dx = \dfrac{\left|a_0\right|^2}{4}+\dfrac{1}{2}\sum_{k=1}^{\infty}\left(\left|a_k\right|^2+\left|b_k\right|^2\right)
  \end{gathered}
\end{equation*}
\par\bigskip
\noindent Nu kan vi använda Parsevals trigonometriska formel:
\begin{equation*}
  \begin{gathered}
    \dfrac{1}{2\pi}\int_{-\pi}^{\pi}\left|x^2\right|dx = \dfrac{1}{2}\sum_{k=1}^{\infty}\left|b_k\right|^2
  \end{gathered}
\end{equation*}
\par\bigskip
\noindent\textbf{Anmärkning:}\par
\noindent Eftersom vår funktion är udda så har vi att $a_k = 0$, och eftersom vi inte har konstanter så är $a_0=0$. Därför har vi bara $b_k$ kvar:
\begin{equation*}
  \begin{gathered}
    \dfrac{1}{2\pi}\int_{-\pi}^{\pi}x^2dx = \dfrac{1}{2\pi}\dfrac{x^3}{3}|_{_\pi}^\pi = \dfrac{2\pi^3}{3\cdot2\pi} = \dfrac{\pi^2}{3}\\
    = \dfrac{1}{2}\sum_{k=1}^{\infty}\dfrac{4}{k^2} = 2\sum_{k=1}^{\infty}\dfrac{1}{k^2}\\
    \Rightarrow \sum_{k=1}^{\infty}\dfrac{1}{n^2} = \dfrac{\pi^2}{6}
  \end{gathered}
\end{equation*}
\par\bigskip
\noindent Vi har sett att Parsevals formel var en konsekvens av Mean-square konvergens. Vi ska nu kika på en följd av Parsevals formel
\par\bigskip
\noindent\textbf{Corollarium:} (Unikhet av fourierkoefficienter)\par
\noindent Om $f,g\in C(\mathbb{T})$ (kontinuerliga och $2\pi$-periodiska) och $c(f)_k = c(g)_k$ för alla $k\in\Z$, så måste $f = g$
\par\bigskip
\noindent\textbf{Anmärkning:}\par
\noindent Om $f,g$ är styckvis kontinuerliga på intervallet $[0,2\pi]$ och de har samma fourierkoefficienter, så kan vi fortfarande säga att $f=g$ för alla $x\in[0,2\pi]$ \textit{förutom} i de (ändliga) punkter där $f$ eller $g$ är diskontinuerliga. 
\par\bigskip
\begin{prf}[Skiss]{prf:kiss}
  \begin{equation*}
    \begin{gathered}
      \dfrac{1}{2\pi}\int_{0}^{2\pi}\left|f(x)-g(x)\right|^2dx \stackrel{Parse.}{=}\sum_{k=-\infty}^{\infty}|\underbrace{c(f-g)_k}_{\text{$c(f)_k-c(g)_k = 0$}}|^2 = 0
    \end{gathered}
  \end{equation*}
\end{prf}
\newpage
\begin{prf}[(Skiss) Mean.-square konvergens av fourierserier]{prf:mskf}
  För att satsen ska gälla vill vi alltså säga att för varje $\varepsilon>0$ finns det ett $N_0$ så att om $N>N_0$ så gäller $\left|\left|f-S_N(f)\right|\right|<\varepsilon$
  \par\bigskip
  \noindent Det första vi gör är att fixera ett $f$ och ett $\varepsilon>0$. Vi vill hitta:\par
  \begin{itemize}
    \item$g$ som är styckvis konstant så att $\left|\left|f-g\right|\right|<\dfrac{\varepsilon}{3}$
    \item$h\in C^2(\mathbb{T})$ så att $\left|\left|g-h\right|\right|<\dfrac{\varepsilon}{3}$
    \item $S_N(h)\to h$ konvergerar likformigt, så det finns $N_0$  så att för alla $N>N_0$ gäller $\left|\left|h-S_N(h)\right|\right|<\dfrac{\varepsilon}{3}$
  \end{itemize}
  \par\bigskip
  \noindent Nu vill vi visa att $\left|\left|f-S_N(f)\right|\right|<\varepsilon$, men vad var $S_N(f)$? Jo, den ortogonala projectionen på något ändligt delrum (vi har betecknat det $W$), och från minsta-kvadrat approximationen sade den att fourierpolynomet (ortogonala projektionen) var den bästa approximationen vi kunde hitta i det delrummet, då gäller:
  \begin{equation*}
    \begin{gathered}
      \left|\left|f-S_N(f)\right|\right|\leq \underbrace{\left|\left|f-S_N(h)\right|\right|}_{\text{Min. kvad. approx.}} = \left|\left|f-g+g-h+h-S_N(h)\right|\right|
    \end{gathered}
  \end{equation*}
  \par\bigskip
  \noindent Nu kör vi ye' olde triangelolikheten:
  \begin{equation*}
    \begin{gathered}
      \left|\left|f-g+g-h+h-S_N(f)\right|\right|\leq \left|\left|f-g\right|\right|+\left|\left|g-h\right|\right| + \left|\left|h-S_N(h)\right|\right|
    \end{gathered}
  \end{equation*}
  \par\bigskip
  \noindent Från konstruktionen av $g,h$ och hur vi valde $N$, gäller då:
  \begin{equation*}
    \begin{gathered}
      \dfrac{\varepsilon}{3} +\dfrac{\varepsilon}{3}+\dfrac{\varepsilon}{3} = \varepsilon\\
      \Rightarrow\left|\left|f-g\right|\right|+\left|\left|g-h\right|\right| + \left|\left|h-S_N(h)\right|\right|<\varepsilon
    \end{gathered}
  \end{equation*}
\end{prf}
\par\bigskip
\noindent\textbf{Anmärkning:}\par
\noindent Likformig konvergens är så pass starkt att det implicerar (vanlig) konvergens.
\par\bigskip
\subsection{Gram-Schmidt Processen}\hfill\\
\noindent Säg att vi har ett ändligt antal vektorer $v_1,\cdots,v_N$. Om dessa utgör en bas för ett semi-inre-produkt-vektorrummet $V$, så kan vi hitta en ortonormal bas $w_1,\cdots,w_N$  genom följande:\par
\begin{itemize}
  \item $w_1 = \dfrac{v_1}{\left|\left|v_1\right|\right|}\qquad \left|\left|v_1\right|\right|\neq0$
    \par\bigskip
  \item $w_2 = \dfrac{\overbrace{v_2-<v_2,w_1>w_1}^{\text{Residualen}}}{\underbrace{\left|\left|v_2-<v_2,w_1>w_1\right|\right|}_{\text{orthog. proj. av $v_2$ på span$(w_1)$}}}$
    \par\bigskip
  \item $w_3 = \dfrac{v_3-<v_3,w_1>w_1-<v_3,w_2>w_2}{\left|\left|v_3-<v_3,w_1>w_1-<v_3,w_2>w_2\right|\right|}$
\end{itemize}
\par\bigskip
\noindent\textbf{Anmärkning:}\par
\noindent Eftersom vi är i ett \textit{semi}-produktrum så kan en vektor $v_1$ inte vara nollvektorn trots att den har längd 0, därför måste vi specifiera att $\left|\left|v_1\right|\right|\neq0$ \par
\noindent Därmed antas att nämnarna $\neq0$
\par\bigskip
\noindent\textbf{Exempel:}\par
\noindent Låt vårat vektorrum $V = C(-1,1) = \left\{f:[-1,1]\to\C\text{ $f$ kontinuerlig}\right\}$\par
\noindent Och låt $<f,g> = \int_{-1}^{1}f(x)\overline{g(x)}dx$
\par\bigskip
\noindent\textit{Hitta en ortonormal bas för delrummet $W = \text{span}(1,x,x^2) = \text{span}(v_1,v_2,v_3)$}
\par\bigskip
\noindent Låt oss kalla $w_1,w_2,w_3$ basen för delrummen.\par
\begin{itemize}
  \item$w_1$\par
    \begin{equation*}
      \begin{gathered}
        \left|\left|v_1\right|\right| = \sqrt{\int_{-1}^{1}1dx}\Rightarrow w_1 = \dfrac{v_1}{\left|\left|v_1\right|\right|} = \dfrac{\sqrt{2}}{2} = \dfrac{1}{\sqrt{2}}
      \end{gathered}
    \end{equation*}
    \par\bigskip
  \item$w_2$:\par
    \begin{equation*}
      \begin{gathered}
        <v_2,w_1> = \int_{-1}^{1}x\dfrac{\sqrt{2}}{2}dx = 0\\
        v_2-<v_2,w_1>w_1 = v_2\\
        \left|\left|v_2\right|\right| = \sqrt{\int_{-1}^{1}x^2dx} = \sqrt{\dfrac{2}{3}}\\
        w_2 = \dfrac{v_2}{\left|\left|v_2\right|\right|} = \sqrt{\dfrac{3}{2}}x
      \end{gathered}
    \end{equation*}
\end{itemize}
\par\bigskip
\noindent En övning är att hitta $w_3$
