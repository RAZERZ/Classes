\section{Bakgrund}\par
\noindent Låt oss betrakta $f:[0,\pi]\to\R$ så att $f(0) = f(\pi) = 0$\par
\noindent När kan vi skriva denna funktion $f(x)$ som en analytisk funktion (potensserie), det vill säga:
\begin{equation}
  \begin{gathered}
    f(x) = \sum_{n=1}^{\infty} a_n\cdot\sin(n\cdot x)
  \end{gathered}
\end{equation}\par
\noindent Där $a_n\in\R$ är konstanter.
\par\bigskip
\noindent Inte alla funktioner tillfredställer att intervallet $[0,\pi]$ ger en konvergerande potensserie för $f$, frågan man kan ställa sig är \textit{när kan vi skriva $f$ som en serie av trigonometriska funktioner?} 
\par\bigskip
\noindent Vi kommer inse att \textit{om} $f$ går att skriva som en potensserie av trigonometriska funktioner, så behöver vi hitta våra koefficienter. I fallet med MacLaurin serier så kom de ($a_n$) från derivatan.\par
\noindent I detta fall kommer det från:
\begin{equation*}
  \begin{gathered}
    a_n  = \dfrac{1}{n}\int_{0}^{\pi}f(x)\sin(nx)dx
  \end{gathered}
\end{equation*}
\par\bigskip
\noindent I någon mening kommer analys-delen av denna kurs från att vi studerar funktioner utifrån integraler, såsom den ovan.\par
\noindent Integralen ovan är integral-transform. 
\par\bigskip
\noindent Vi kan även skriva:
\begin{equation*}
  \begin{gathered}
    f(x) = \sum_{n=0}^{\infty}a_n\sin(nx)+b_n\cos(nx)
  \end{gathered}
\end{equation*}
\par\bigskip
\noindent Något mer vi kommer undersöka, är om vår fourierserie konvergerar, och om den konvergerar mot vår funktion (detta är inte alltid uppenbart)
\par\bigskip
\subsection{Komplexa exponentialer}\hfill\\\par
\noindent Det finns en viktig eulerformel. Vi alla känner till $e^x$, men vad händer om $x = a+bi$?
\par\bigskip
\begin{theo}[Eulers formel]{thm:eulersform}
  Vi får då $e^{a+bi} = \underbrace{e^a}_{\text{$\in\R$}}e^{ib} = e^a\left(\cos(b)+i\sin(b)\right)$ för varje $a,b\in\R$\par
\end{theo}
\par\bigskip
\noindent Kom ihåg att vi kan representera komplexa tal med polära koordinater.\par
\noindent Vi har då att varje komplext tal $a+bi$ kan representeras som $r = \sqrt{a^2+b^2}$. Vi får då $a+bi = re^{i\theta} = e^{\log(r)+i\theta}$
\par\bigskip
\noindent\textbf{Övning:}\par
\noindent Använd Eulers formel för att visa att $\cos(2x) = \left(\cos^2(x)-\sin^2(x)\right)$ och $\sin(2x) = 2\sin(x)\cos(x)$
\par\bigskip
\noindent\textbf{Anmärkning:}\par
\noindent Komplexa exponentialer är \textit{inte} injektiva, alltså fungerar inte logaritmen.\par
\noindent Exempelvis kan vi betrakta $e^{i2\pi} = e^{i0} = 1$
\par\bigskip
\begin{theo}[Fourierpolynomial]{thm:fourierpoly}
  Är på formen:
  \begin{equation*}
    \begin{gathered}
      \sum_{k=-N}^{N}c_k\cdot e^{ikx}
    \end{gathered}
  \end{equation*}
  \par\bigskip
  \noindent Kallas för polynom för att vi har $e^{ikx} = (e^{ix})^k$ som är ett monom i $e^{ix}$
\end{theo}
\par\bigskip
\noindent Vi kan uttrycka Fourierpolynom m.h.a sinus och cosinus enligt följande:
\begin{equation*}
  \begin{gathered}
      \sum_{k=-N}^{N}c_k\cdot e^{ikx} = \sum_{k=-N}^{N}c_k\left(\cos(kx)+i\sin(kx)\right)\\
      = c_0 +\sum_{k=1}^{N}(c_k+c_{-k})\cos(kx)+i(c_k-c_{-k})\sin(kx)
  \end{gathered}
\end{equation*}
\par\bigskip
\noindent\textbf{Anmärkning:}\par
\noindent Detta är samma fourierpolynom som var på \textit{exponential form} i \textit{trigonometrisk form}
\par\bigskip
\subsection{Lebesgue integralen}\hfill\\\par
\noindent Den vanliga definitionen av integralen som vi alla är vana vid är Riemann-integralen.
