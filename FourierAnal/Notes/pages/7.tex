\section{Analys av Fourierserier}\par
\noindent Vi vill undersöka\par
\begin{itemize}
  \item Konvergerar serien:\par
    \begin{itemize}
      \item Punktvis
      \item Likformigt
      \item Absolut
      \item På något sätt?
    \end{itemize}\par
  \item Om serien konvergerar, konvergerar den till $f(x)$? 
\end{itemize}
\par\bigskip
\noindent Om vi analyserar följande:
\begin{equation*}
  \begin{gathered}
    f(x)\sim \dfrac{\pi^2}{3}+\sum_{n=1}^{\infty}\dfrac{4}{n^2}(-1)^n\cos(nx)
  \end{gathered}
\end{equation*}
\par\bigskip
\noindent\textbf{Konvergerar serien?}\par
\noindent Vi kan använda Weirstrass $M$-test eftersom:
\begin{equation*}
  \begin{gathered}
    \left|\dfrac{4}{n^2}(-1)^n\cos(nx)\right| = M_n\\
    \sum_{n=1}^{\infty}M_n = \sum_{n=1}^{4}\dfrac{4}{n^2}\Leftarrow\text{ konvergerar}
  \end{gathered}
\end{equation*}
\par\bigskip
\noindent Av Weirstrass $M$ test konvergerar serien absolut, likformigt, och punktvis
\par\bigskip
\noindent\textbf{Anmärkning:}\par
\noindent Om vi sätter absolutbelopp på vår fourierserie så försvinner de trigonometriska formerna och vår kandidat för $M_n$ blir då våra koefficienter
\par\bigskip
\noindent\textbf{Konvergerar serien till $f(x)$?}\par
\noindent NÄSTA FÖRELÄSNING SER VI ATT SVARET ÄR JA
\par\bigskip
\noindent\textbf{Evaluering av Fourierserien}\par
\noindent Låt oss anta för tillfället att serien konvergerar till funktionen\par
\noindent Vi vet att $f(\pi) = \pi^2$\par
\noindent Fourierserien i $\pi$:
\begin{equation*}
  \begin{gathered}
    \dfrac{\pi^2}{3}+\sum_{n=1}^{\infty}\dfrac{4}{n^2}(-1)^n\underbrace{\cos(n\pi)}_{\text{$(-1)^n$}}\\
    = \dfrac{\pi^2}{3}+\sum_{n=1}^{\infty}\dfrac{4}{n^2}\\
    \Rightarrow \pi^2 = \dfrac{\pi^2}{3}+4\sum_{1}^{n^2}\Rightarrow \sum_{n=1}^{\infty}\dfrac{1}{n^2} = \dfrac{2\pi^2}{3\cdot4} = \dfrac{\pi^2}{6}
  \end{gathered}
\end{equation*}
\par\bigskip
\subsection{Konvergerande Fourierserier}\hfill\\\par
\noindent I en talföljd vet vi att serien konvergerar om exempelvis $\lim_{n\to\infty}a_n = 0$
\par\bigskip
\noindent Vi kan översätta detta till fourierserier genom att undersöka om koefficienterna går till 0 
\par\bigskip
\noindent För att göra detta behöver vi införa lite nya verktyg:
\par\bigskip
\begin{theo}[Absolutintegrerbar]{thm:absoluteintegrable}
  Om vi tar ett intervall $I\subset \R$ av de reella talen och vi har en $f:I\to\C$, säger vi att $f$ är absolutintegrerbar om:\par
  $\int_I |f(x)|dx$ konvergerar (integrerbar)\par
\end{theo}
\par\bigskip
\noindent\textbf{Anmärkning:}\par
\begin{itemize}
  \item Om $I = [a,b]$ är ett ändligt intervall och $f:I\to\C$ integrerbar, så är $f$ absolutintegrerbar\par
  \item Detta gäller \textit{inte} om intervallet $I$ är oändligt:
    \begin{equation*}
      \begin{gathered}
        \int_{0}^{\infty}f(x)dx = \sum_{n=1}^{\infty}\dfrac{(-1)^{n+1}}{n}<\infty\\
        \int_{0}^{\infty}\left|f(x)\right|dx = \sum_{n=1}^{\infty}\dfrac{1}{n} = \infty
      \end{gathered}
    \end{equation*}
\end{itemize}
\par\bigskip
\begin{theo}[Riemann-Lebesgue sats]{thm:riemannlebbsats}
  Om vi har ett intervall $I\subset \R$ och en funktion $f:I\to\C$ som är integrerbar \textit{och} absolutintegrerbar, då gäller:
  \begin{equation*}
    \begin{gathered}
      \lim_{\lambda\to\infty}\int_{I}f(x)\cdot\cos(\lambda x)dx = 0
    \end{gathered}
  \end{equation*}
  \par\bigskip
  \noindent Samma gäller för följande (och för $-\infty$):
  \begin{equation*}
    \begin{gathered}
      \lim_{\lambda\to\infty}\int_{I}f(x)\sin(\lambda x)dx = 0\\
      \lim_{\lambda\to\infty}\int_{I}f(x)e^{i\lambda x}dx = 0
    \end{gathered}
  \end{equation*}
\end{theo}
\par\bigskip
\noindent\textbf{Anmärkning:}\par
\noindent En funktion kan vara absolutintegrerbar men \textit{inte} integrerbar, exempelvis:
\begin{equation*}
  \begin{gathered}
    f:[0,1]\to\R\\
    f(x) = \begin{cases}1\quad x\in\Q\\-1\quad x\in\R\backslash\Q\end{cases}
  \end{gathered}
\end{equation*}
\par\bigskip
\noindent Här gäller att $\left|f(x)\right| = 1$ som är integrerbar, men funktionen går knappt att rita (ej Riemann integrerbar men Lebesgue integrerbar)
\par\bigskip
\noindent Varför gillar vi Riemann-Lebesgues sats? Tänk på koefficienterna! De är integrerbara på den formen (ett intervall $[0,2\pi]\subset \R$, integrerbar, absolutintegrerbar?)
\newpage
\noindent\textbf{Corollarium}:\par
\noindent Om vi har en funktion $f:\R\to\C$ som är $2\pi$ periodisk och integrerbar på intervallet $[0,2\pi]$ så:
\begin{equation*}
  \begin{gathered}
    \lim_{n\to\infty}c_n = 0 = \lim_{n\to-\infty}c_n
  \end{gathered}
\end{equation*}
\par\bigskip
\begin{prf}
  A
  \begin{equation*}
    \begin{gathered}
      c_n = \dfrac{1}{2\pi}\int_{0}^{2\pi}f(x)e^{-inx}dx
    \end{gathered}
  \end{equation*}
  \par\bigskip
  \noindent Då följer det från Riemann-Lebesgues sats att:
  \begin{equation*}
    \begin{gathered}
      \lim_{n\to\infty}c_n = 0\\
      I = [0,2\pi]\qquad\lambda = -n
    \end{gathered}
  \end{equation*}
  \par\bigskip
  \noindent Vi kan göra detta för $-\infty$
\end{prf}
\par\bigskip
\noindent\textbf{Anmärkning:}\par
\noindent Notera att kraven vi ställde för $c_n$ är samma för de för Fourierserier. 
\par\bigskip
\noindent För vanliga gränsvärden kan vi säga något om deras hastigheter. $\dfrac{1}{n}$ går mot 0 segare än vad $\dfrac{1}{n^2}$ gör.
\par\bigskip
\noindent Vi kan göra något liknande med funktioner. Det finns en "slogan" som säger följande:\par
Ju mer integrerbar en funktion är, desto snabbare går deras fourierkoefficienter mot 0\par
\noindent Vi skriver detta som en sats:
\par\bigskip
\begin{theo}
  AOm $f\in C^k(\mathbb{T})$, så är $f^{(k)}$ kontinuerlig och därmed integrerbar med fourierkoefficienter, med följande:
  \begin{equation*}
    \begin{gathered}
      c_n\text{ av } f^{(k)}(n) = (in)^kc_n
    \end{gathered}
  \end{equation*}
\end{theo}
\par\bigskip
\noindent Detta påminner om Laplacetransformationens egenskap med avseende på derivatan. Om vi struntar i IVP så har vi ju något liknande\par
\noindent Detta följer från att Laplacetransformationen och Fourier är \textit{integral transformationer} 
\par\bigskip
\begin{theo}
  OOm $f\in C^k(\mathbb{T})$, så finns det en konstant $c>0$ så att:
  \begin{equation*}
    \begin{gathered}
      \forall n\in\Z\quad \left|c_n\right|\leq\dfrac{c}{\left|n\right|^k}
    \end{gathered}
  \end{equation*}
\end{theo}
\par\bigskip
\noindent\textbf{Anmärkning:}\par
\noindent Det är nästan som att koefficienterna vet något om funktionen :o
\newpage
\noindent\textbf{Corollarium}\par
\noindent Om $f\in C^2(\mathbb{T})$, så är funktionens fourierserie likformigt, absolut, och punktvis konvergent 
\par\bigskip
\begin{prf}
  DDet låter väldigt lämpligt att använda Weirstrass $M$-test:
  \begin{equation*}
    \begin{gathered}
      f\sim \sum_{n=-\infty}^{\infty}c_ne^{inx}\\
      \left|c_ne^{inx}\right| = \left|c_n\right|\leq \dfrac{c}{\left|n\right|^2} = M_n\\
      \sum_{n=-\infty}^{\infty}M_n\text{ konvergerar}\Rightarrow \sum_{n=-\infty}^{\infty}c_ne^{inx}\text{ konvergerar}
    \end{gathered}
  \end{equation*}
\end{prf}
