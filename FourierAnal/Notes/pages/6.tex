\section{Fourierserier}\par
\subsection{Notation \& Terminologi}\hfill\\\par
\noindent Skillnaden mellan Fourierserier och Laplacetransformationen har mest att göra med funktionens domän. I Laplacetransformationen så begränar vi oss till $\R_+$, men med Fourierserier vill vi kunna nyttja hela $\R$
\par\bigskip
\begin{theo}[Periodisk funktion]{thm:periodic}
  Vi säger att en funktion $f:\R\to\C$ är $2\pi$-periodisk om:
  \begin{equation*}
    \begin{gathered}
      f(x+2\pi) = f(x)\qquad\forall x\in\R
    \end{gathered}
  \end{equation*}
\end{theo}
\par\bigskip
\noindent\textbf{Exempel:}\par
\noindent $\sin(kx)$ och $\cos(kx)$ där $k\in\Z$ är $2\pi$ periodiska.
\par\bigskip
\noindent Iden är att försöka uttrycka en godtycklig $2\pi$ periodisk funktion som en oändlig linjärkombination av $\sin(x)$ och $\cos(x)$
\par\bigskip
\noindent\textbf{Anmärkning:}\par
\noindent En bra fördel med $2\pi$ periodiska funktioner är att trots att funktionen är definierad på hela $\R$ så räcker det med att studera $[0,2\pi]$
\par\bigskip
\noindent\textbf{Anmärkning:}\par
\noindent Om man försöker omvandla en funktion till $2\pi$ periodisk, så kan det hända att vi introducerar diskontinuitet vid ändpunkten $k\cdot2\pi$
\par\bigskip
\begin{theo}
  VVi skriver följande:\par
  \begin{itemize}
  \item $C(\mathbb{T}) = \left\{\text{kontinuerliga funkioner} f:\R\to\C\;|\; 2\pi\text{ periodiska}\right\}$\par
  \item $C^k(\mathbb{T}) = \left\{f:\R\to\C\;|\;2\pi\text{ periodiska med } f^{(k)}(x)\text{ kontinuerlig}\right\}$
  \end{itemize}
\end{theo}
\par\bigskip
\noindent\textbf{Anmärkning:}\par
\noindent Bara för en funktion ligger i $C(\mathbb{T})$ betyder det \textbf{inte} att den ligger i $C^1(\mathbb{T})$
\par\bigskip
\noindent\textbf{Anmärkning:}\par
\noindent Anledningen till varför vi studerar just $2\pi$ periodiska funktioner är för att om vi har en annan funktion med period $p\neq 2\pi$, så kan vi med variabelbyte få den funktionen att bli $2\pi$ periodisk 
\par\bigskip
\begin{equation*}
  \begin{gathered}
    f:\R\to\C\qquad p>0\text{ - periodisk}\\
    g:\R\to\C\quad g(x) = f\left(\dfrac{p}{2\pi}\cdot x\right)\qquad2\pi\text{ - periodisk}
  \end{gathered}
\end{equation*}
\par\bigskip
\noindent\textbf{Fråga:}\par
\noindent Låt $f:\R\to\C$ vara en $2\pi$ periodisk funktion. Kan vi skriva $f(x)$ i termer av en fourierserie?
\begin{equation*}
  \begin{gathered}
    f(x) = \sum_{k=-\infty}^{\infty}c_k\underbrace{e^{ikx}}_{\text{$\cos(kx)+i\sin(kx)$}}\qquad c_k\in\C
  \end{gathered}
\end{equation*}
\newpage
\noindent\textbf{Anmärkning:}\par
\noindent Om jag har en serie $\sum_{k=-\infty}^{\infty}c_ke^{ikx}$ så kan jag skriva den som:
\begin{equation*}
  \begin{gathered}
    \underbrace{\sum_{k=\infty}^{\infty}c_ke^{ikx}}_{\text{Exponentialform}}= \underbrace{\dfrac{a_0}{2} + \sum_{k=1}^{\infty}a_k\cos(kx)+b_k\sin(kx)}_{\text{Trigon. form}}
  \end{gathered}
\end{equation*}\par
\noindent där:
\begin{equation*}
  \begin{gathered}
    a_0 = 2c_0\\
    a_k = c_k+c_{-k}\\
    b_k = i(c_k-c_{-k})
  \end{gathered}
\end{equation*}\par
\noindent Anledningen till varför vi delar på 2 är för att det blir en finare formel senare när man väl ska använda den
\par\bigskip
\noindent\textbf{Anmärkning:}\par
\noindent Konstanterna $c_k$ kommer från $f(x)$, vi kan därmed uttrycka $c_k$ som en funktion av $f$
\par\bigskip
\begin{lem}
  GGivet $n\in\Z$
  \begin{equation*}
    \begin{gathered}
      \int_{0}^{2\pi}e^{inx}dx = \begin{cases}\dfrac{e^{inx}}{in}|_{_0}^{^{2\pi}} \stackrel{2\pi-\text{per.}}{=} 0\quad n\neq0\\\\2\pi\quad n=0\end{cases}
    \end{gathered}
  \end{equation*}
  \par\bigskip
  \noindent Givet $n,m\in\Z$
  \begin{equation*}
    \begin{gathered}
      \int_{0}^{2\pi}\cos(nx)\cos(mx)dx = \int_{0}^{2\pi}\sin(nx)\sin(mx)dx = \begin{cases}0\quad n\neq m\\\pi\quad n=m\neq0\end{cases}
    \end{gathered}
  \end{equation*}
\end{lem}
\par\bigskip
\noindent Antag att
\begin{equation*}
  \begin{gathered}
    f(x) = \sum_{-\infty}^{\infty}c_ke^{ikx}
  \end{gathered}
\end{equation*}\par
\noindent där serien är likformigt konvergent. Då måste $f$ vara kontinuerlig.\par
\noindent Hur kan vi få ut $c_k$ ur funktionen $f(x)$? Låt oss testa lite grejs:
\begin{equation*}
  \begin{gathered}
    \int_{0}^{2\pi}f(x)e^{ikx} dx = \int_{0}^{2\pi}\sum_{m=-\infty}^{\infty}c_me^{imx}e^{-ikx}dx\\
    = \int_{0}^{2\pi}\sum_{m=-\infty}^{\infty}c_ne^{i(m-k)x}dx\stackrel{\text{likf. kont.}}{=} \sum_{m=-\infty}^{\infty}\int_{0}^{2\pi}c_me^{i(m-k)x}dx\\
    \stackrel{\text{Lem}6.1}{=} 2\pi\cdot c_k\qquad\forall k\in\Z\\
    \Rightarrow c_k = \dfrac{1}{2\pi}\int_{0}^{2\pi} f(x)e^{-ikx}dx
  \end{gathered}
\end{equation*}
\par\bigskip
\noindent Detta gäller för nice funktioner, funktioner som kommer från likformigt konvergenta fourierserier.
\par\bigskip
\noindent Då kan man få för sig att bara studera dessa typer av funktioner, men det finns en till tolkning.
\par\bigskip
\noindent\textbf{Formeln för $c_k$ funkar för alla funktioner som är integrerbara.} Vi kan konstruera en fourierserie av en given funktion, men kommer serien konvergera? Kommer serien konvergera till funktionen vi började med?\par
\noindent Det är vad som studien av fourierserier försöker besvara. 
\newpage
\begin{theo}[Fourierserien av en funktion]{thm:fouriersrofcun}
  Låt $f:\R\to\C$ vara $2\pi$ periodisk och integrerbar på intervallet $[0,2\pi]$
  \par\bigskip
  \noindent Funktionens fourierserie anges av:
  \begin{equation*}
    \begin{gathered}
      f(x)\sim\sum_{k=-\infty}^{\infty}c_ke^{ikx}
    \end{gathered}
  \end{equation*}\par
  \noindent där:
  \begin{equation*}
    \begin{gathered}
      c_k = \dfrac{1}{2\pi}\int_{0}^{2\pi}f(x)e^{-ikx}dx
    \end{gathered}
  \end{equation*}
\end{theo}
\par\bigskip
\noindent Vi skall nu försöka besvara föregående frågor:\par
\begin{itemize}
  \item Konvergerar serien:\par
    \begin{itemize}
      \item Punktvis
      \item Likformigt
      \item Absolut
      \item På något sätt?
    \end{itemize}\par
  \item Om serien konvergerar, konvergerar den till $f(x)$? 
\end{itemize}
\par\bigskip
\noindent\textbf{Anmärkning:}\par
\noindent Vi antar här att vår funktion är $2\pi$ periodisk (men det spelar egentligen ingen roll vilken period eftersom vi kan med variabelbyte få den att bli $2\pi$, viktigt att den är periodisk dock!)
\par\bigskip
\noindent\textbf{Anmärkning:}\par
\noindent Eftersom vi kan gå från exponentialform till trigonometrisk form, så gäller följande för fourierserier:
\begin{equation*}
  \begin{gathered}
    f(x)\sim \dfrac{a_0}{2}+\sum_{k=1}^{\infty}a_k\cos(kx)+b_k\sin(kx)\\
    a_k = \dfrac{1}{\pi}\int_{0}^{2\pi}f(x)\cos(kx)dx\\
    b_k = \dfrac{1}{\pi}\int_{0}^{2\pi}f(x)\sin(kx)dx
  \end{gathered}
\end{equation*}
\par\bigskip
\noindent\textbf{Anmärkning:}\par
\noindent Tidigare sa vi att $b_k$ var en multipel av $i$, men detta gällde för att $c_k$ var komplex, så $b_k = ic_k\in\R$. Det är därför vi inte behöver bry oss så mycket om $i$ när vi definierar den som vi gjorde här.
\par\bigskip
\noindent\textbf{Exempel:}\par
\noindent Vi ska köra ett exempel och speciellt se om vi kan svara på frågorna ovan.
\par\bigskip
\noindent Låt $f:\R\to\C$ så att $f(x) = x$ i intervallet $[-\pi,\pi)$ ($f$ är $2\pi$ periodisk)
\par\bigskip
\noindent Lägg märke till att $f(x)$ är en udda funktion. En linjärkombination av jämna funktioner kommer vara jämn, alltså kommer vi ha en linjärkombination av udda funktioner.\par
\noindent Eftersom $\cos(x)$ är jämn, så kommer alla $a_k$ koefficienter vara 0 (inga jämna funktioner). Vi behöver då bara räkna fram $b_k$:\par
$k\geq1$
\begin{equation*}
  \begin{gathered}
    b_k = \dfrac{1}{\pi}\int_{-\pi}^{\pi}x\sin(kx)dx \stackrel{\text{parts}}{=} \dfrac{2(-1)^{k+1}}{k}\\
    \Rightarrow f(x)\sim\sum_{k=1}^{\infty}\dfrac{2(-1)^{k+1}}{k}\sin(kx)
  \end{gathered}
\end{equation*}
\par\bigskip
\noindent Nu kan vi besvara frågorna.
\newpage
\noindent\textbf{Exempel:}\par
\noindent Låt $f:\R\to\C$ vara en $2\pi$ periodisk funktion där $f(x) = x^2$ då $x\in[-\pi,\pi]$
\par\bigskip
\noindent Denna funktion har nollställen i jämna multiplar av $\pi$ (inkl. 0)
\par\bigskip
\noindent Bestäm funktionens fourierserie. Notera att vår funktion är jämn, så vi kommer \textit{inte} ha några $\sin$ eftersom sinus är udda. Detta gör det lättare att representera funktionen i trigonometrisk form:
\begin{equation*}
  \begin{gathered}
    f(x)\sim \dfrac{a_0}{2}+\sum_{n=1}^{\infty}a_n\cdot\cos(nx)
  \end{gathered}
\end{equation*}\par
\noindent Vi kan använda att funktionen är symmetrisk och därmed räkna halva koefficienterna\par 
\noindent Där $a_n$:
\begin{equation*}
  \begin{gathered}
    a_n= \dfrac{1}{\pi}\int_{-\pi}^{\pi}f(x)\cos(nx)dx = \dfrac{1}{\pi}\int_{-\pi}^{\pi}x^2\cos(nx)dx\\
    \stackrel{\text{parts}}{=} \dfrac{4}{n^2}(-1)^n\qquad n\neq0\\
    a_0 = \dfrac{1}{\pi}\int_{-\pi}^{\pi}x^2dx = \dfrac{1}{\pi}\dfrac{x^3}{3}\Rightarrow \dfrac{\pi^2}{3}+\dfrac{\pi^2}{3} = \dfrac{2}{3}\pi^2\\
    f(x)\sim \dfrac{\pi^2}{3}+\sum_{n=1}^{\infty}\dfrac{4}{n^2}(-1)^n\cos(nx)
  \end{gathered}
\end{equation*}
