\section{Fouriertransformation}\par
\noindent Iden bakom detta är att försöka anpassa (läs: finna en version) fourierserier till funktioner som inte enbart är $2\pi$ periodiska, utan till \textit{alla} funktioner med inte bara integraler från $[0,\infty)$ utan \textit{överallt} :o
\par\bigskip
\noindent Vi påminner oss om $f\in I(\mathbb{T})$, så gällde:
\begin{equation*}
  \begin{gathered}
    \begin{cases}
      c_k = \dfrac{1}{2\pi}\int_{0}^{2\pi}f(x)e^{-ikx} dx\qquad k\in\Z\quad(1)\\
      f(x)\sim \sum_{k=-\infty}^{\infty}c_ke^{ikx}\quad(2)
    \end{cases}
  \end{gathered}
\end{equation*}
\par\bigskip
\noindent Vi söker alltså dessa saker för funktioner $f:\R\to\C$ som \textit{inte} är periodisk.
\par\bigskip
\begin{theo}[Fouriertransformation]{thm:fouriertransf}
  \noindent Låt:
  \begin{equation*}
    \begin{gathered}
      \begin{cases}
        c_\omega = \int_{-\infty}^{\infty}f(x)e^{-i\omega x}dx\qquad\omega\in\R\quad(1^{\prime})\\
        f(x)\sim \dfrac{1}{2\pi}\sum_{-\infty}^{\infty}c_\omega e^{i\omega x}d\omega\qquad(2^{\prime})
      \end{cases}
    \end{gathered}
  \end{equation*}
\end{theo}
\par\bigskip
\noindent\textbf{Anmärkning:}\par
\noindent Vi flyttar $\dfrac{1}{2\pi}$ från koefficienterna till framför serien 
\par\bigskip
\begin{theo}
  OOm $f:\R\to\C$ är \textit{integrerbar} \textbf{och} \textit{absolutintegrerbar}, så får vi en ny funktion som vi kallar för \textit{Fouriertransformationen}:
  \begin{equation*}
    \begin{gathered}
      c_\omega:\R\to\C\Leftarrow c_\omega = \int_{-\infty}^{\infty}f(x)e^{-i\omega x}dx
    \end{gathered}
  \end{equation*}
\end{theo}
\par\bigskip
\noindent\textbf{Anmärkning:}\par
\noindent Se till att ha koll på Lemma 7.1: Riemann-Lebesgue framöver!
\par\bigskip
\noindent\textbf{Anmärkning:}\par
\begin{equation*}
  \begin{gathered}
    c_0 = \int_{-\infty}^{\infty}f(x)dx
  \end{gathered}
\end{equation*}
\par\bigskip
\noindent Eftersom den är integrerbar (och absolutintegrerbar), så kan vi få information om funktionen vid "tid" = 0 
\par\bigskip
\noindent\textbf{Exempel:}\par
\noindent Låt $f(x) = e^{-\left|x\right|}$, hitta Fouriertransformationen.
\begin{equation*}
  \begin{gathered}
    c_\omega = \int_{-\infty}^{\infty}e^{-\left|x\right|}e^{-\omega x}dx = \int_{-\infty}^{\infty}e^{-(\left|x\right|+i\omega x)} dx\\
    = \int_{-\infty}^{0}e^{x}e^{-\omega x}dx + \int_{0}^{\infty}e^{-x}e^{-i\omega x}\\
    \lim_{A\to-\infty}\int_{A}^{0}\underbrace{e^{x}e^{-\omega x}}_{\text{$e^{(1-i\omega)x}$}}dx = \lim_{A\to-\infty}\left[\dfrac{e^{(1-i\omega)x}}{(1-i\omega)}\right]_{A}^0 = \lim_{A\to-\infty}\dfrac{1}{1-i\omega}-\underbrace{\dfrac{e^{(1-i\omega)A}}{1-i\omega}}_{\text{$\to0$}} = \dfrac{1}{1-i\omega}\\
    \int_{0}^{\infty}e^{-x}e^{-i\omega x}dx = \cdots = \dfrac{1}{1+i\omega}\\
    c_\omega = \dfrac{1}{1-i\omega} + \dfrac{1}{1+i\omega} = \dfrac{2}{1+\omega^2}
  \end{gathered}
\end{equation*}
\newpage
\noindent\textbf{Anmärkning:}\par
\noindent Får funktion var en reell-värd funktion, och vi gör grejs med komplexa tal (för att räkna ut Fouriertransformationen) men får ändå ut en reell-värd funktion
\par\bigskip
\noindent\textbf{Anmärkning:}\par
\noindent Det finns mycket bus som kan hända när vi tar bort $2\pi$-periodiciteten, och när vi kräver integrerbarhet av funktionen så tappar vi väldigt många funktioner (exempelvis har inte polynom fouriertransformationer)
