\section{Fouriertransformation}\par
\noindent Iden bakom detta är att försöka anpassa (läs: finna en version) fourierserier till funktioner som inte enbart är $2\pi$ periodiska, utan till \textit{alla} funktioner med inte bara integraler från $[0,\infty)$ utan \textit{överallt} :o
\par\bigskip
\noindent Vi påminner oss om $f\in I(\mathbb{T})$, så gällde:
\begin{equation*}
  \begin{gathered}
    \begin{cases}
      c_k = \dfrac{1}{2\pi}\int_{0}^{2\pi}f(x)e^{-ikx} dx\qquad k\in\Z\quad(1)\\
      f(x)\sim \sum_{k=-\infty}^{\infty}c_ke^{ikx}\quad(2)
    \end{cases}
  \end{gathered}
\end{equation*}
\par\bigskip
\noindent Vi söker alltså dessa saker för funktioner $f:\R\to\C$ som \textit{inte} är periodisk.
\par\bigskip
\begin{theo}[Fouriertransformation]{thm:fouriertransf}
  \noindent Låt:
  \begin{equation*}
    \begin{gathered}
      \begin{cases}
        c_\omega = \int_{-\infty}^{\infty}f(x)e^{-i\omega x}dx\qquad\omega\in\R\quad(1^{\prime})\\
        f(x)\sim \dfrac{1}{2\pi}\sum_{-\infty}^{\infty}c_\omega e^{i\omega x}d\omega\qquad(2^{\prime})
      \end{cases}
    \end{gathered}
  \end{equation*}
\end{theo}
\par\bigskip
\noindent\textbf{Anmärkning:}\par
\noindent Vi flyttar $\dfrac{1}{2\pi}$ från koefficienterna till framför serien 
\par\bigskip
\begin{theo}
  OOm $f:\R\to\C$ är \textit{integrerbar} \textbf{och} \textit{absolutintegrerbar}, så får vi en ny funktion som vi kallar för \textit{Fouriertransformationen}:
  \begin{equation*}
    \begin{gathered}
      c_\omega:\R\to\C\Leftarrow c_\omega = \int_{-\infty}^{\infty}f(x)e^{-i\omega x}dx
    \end{gathered}
  \end{equation*}
\end{theo}
\par\bigskip
\noindent\textbf{Anmärkning:}\par
\noindent Likt notationen för Laplace så är notationen för en fouriertransformerad funktion $\mathcal{F}[f](\omega)$ 
\par\bigskip
\noindent\textbf{Anmärkning:}\par
\noindent Se till att ha koll på Lemma 7.1: Riemann-Lebesgue framöver!
\par\bigskip
\noindent\textbf{Anmärkning:}\par
\begin{equation*}
  \begin{gathered}
    c_0 = \int_{-\infty}^{\infty}f(x)dx
  \end{gathered}
\end{equation*}
\par\bigskip
\noindent Eftersom den är integrerbar (och absolutintegrerbar), så kan vi få information om funktionen vid "tid" = 0 
\par\bigskip
\noindent\textbf{Exempel:}\par
\noindent Låt $f(x) = e^{-\left|x\right|}$, hitta Fouriertransformationen.
\begin{equation*}
  \begin{gathered}
    c_\omega = \int_{-\infty}^{\infty}e^{-\left|x\right|}e^{-\omega x}dx = \int_{-\infty}^{\infty}e^{-(\left|x\right|+i\omega x)} dx\\
    = \int_{-\infty}^{0}e^{x}e^{-\omega x}dx + \int_{0}^{\infty}e^{-x}e^{-i\omega x}\\
    \lim_{A\to-\infty}\int_{A}^{0}\underbrace{e^{x}e^{-\omega x}}_{\text{$e^{(1-i\omega)x}$}}dx = \lim_{A\to-\infty}\left[\dfrac{e^{(1-i\omega)x}}{(1-i\omega)}\right]_{A}^0 = \lim_{A\to-\infty}\dfrac{1}{1-i\omega}-\overbrace{\dfrac{e^{(1-i\omega)A}}{1-i\omega}}^{\text{$\to0$}} = \dfrac{1}{1-i\omega}\\
    \int_{0}^{\infty}e^{-x}e^{-i\omega x}dx = \cdots = \dfrac{1}{1+i\omega}\\
    c_\omega = \dfrac{1}{1-i\omega} + \dfrac{1}{1+i\omega} = \dfrac{2}{1+\omega^2}
  \end{gathered}
\end{equation*}
\newpage
\noindent\textbf{Anmärkning:}\par
\noindent Får funktion var en reell-värd funktion, och vi gör grejs med komplexa tal (för att räkna ut Fouriertransformationen) men får ändå ut en reell-värd funktion
\par\bigskip
\noindent\textbf{Anmärkning:}\par
\noindent Det finns mycket bus som kan hända när vi tar bort $2\pi$-periodiciteten, och när vi kräver integrerbarhet av funktionen så tappar vi väldigt många funktioner (exempelvis har inte polynom fouriertransformationer)
\par\bigskip
\subsection{Egenskaper hos fouriertransformationen}\hfill\\\par
\noindent Låt $f:\R\to\C$ vara en integrerbar och absolutintegrerbar funktion. Då gäller följande för fouriertransformationen av $f$:\par
\begin{itemize}
  \item $\left|\mathcal{F}[f](\omega)\right|\leq \underbrace{\dfrac{1}{2\pi}}_{\text{behövs ej}}\int_{-\infty}^{\infty}\left|f(x)\right|dx<\infty$ (begränsad, oavsett val på $\omega$)
    \par\bigskip
  \item $\lim_{\left|w\right|\to\infty}\mathcal{F}[f](\omega) = 0$
    \par\bigskip
  \item $\mathcal{F}[f](\omega)$ är likformigt kontinuerlig (och därmed kontinuerlig)
    \par\bigskip
  \item $\mathcal{F}[(f*g)](\omega) = \mathcal{F}[f](\omega)\cdot\mathcal{F}[g](\omega)$ där faltningen definieras enligt Definition 11.3
    \par\bigskip
  \item $\mathcal{F}[\alpha f+\beta g](\omega) = \alpha\mathcal{F}[f](\omega)+\beta\mathcal{F}[g](\omega)$ där $\alpha, \beta\in\C$
    \par\bigskip
  \item $\mathcal{F}[f^{\prime}](\omega) = (i\omega)\mathcal{F}[f](\omega)$ om $f^{\prime}$ är integrerbar och absolutintegrerbar
    \par\bigskip
  \item $\mathcal{F}[x\cdot f(x)](\omega) = i\dfrac{d}{d\omega}\mathcal{F}[f](\omega)$ om $x\cdot f(x)$ är integrerbar och absolutintegrerbar
    \par\bigskip
  \item $\mathcal{F}[f(x)\cdot e^{iax}](\omega) = \mathcal{F}[f](\omega-a)$ där $a\in\R$
    \par\bigskip
  \item $\mathcal{F}[f(x-a)](\omega) = e^{-ia\omega}\mathcal{F}[f](\omega)$ där $a\in\R$
    \par\bigskip
  \item $\mathcal{F}[ax](\omega) = \dfrac{1}{\left|a\right|}\mathcal{F}[f]\left(\dfrac{\omega}{a}\right)$ där $a\neq0\in\R$
\end{itemize}
\par\bigskip
\par\bigskip

\begin{theo}[Konvolution (Faltning)]{thm:convolutionv3}
  Om både $f,g$ är funktioner från $\R\to\C$ (och integrerbara och absolutintegrerbara), så definieras deras \textit{faltning} som:
  \begin{equation*}
    \begin{gathered}
      (f*g)(x) = \int_{-\infty}^{\infty}f(x-t)\cdot g(t)dt
    \end{gathered}
  \end{equation*}
\end{theo}
\par\bigskip
\noindent\textbf{Anmärkning:}\par
\noindent $(f*g)(x) = (g*f)(x)$
\par\bigskip
\noindent\textbf{Exempel:}\par
\noindent Låt $g(x) = \chi_{[a,b]}$ (där $\chi_{[a,b]} =$ karaktäristisk funktion på intervallet $[a,b]$), dvs:
\begin{equation*}
  \begin{gathered}
    \chi_{[a,b]} = \begin{cases}1\quad x\in[a,b]\\0\quad x\not\in[a,b]\end{cases}
  \end{gathered}
\end{equation*}
\par\bigskip
\noindent Vi har:
\begin{equation*}
  \begin{gathered}
    (f*g)(x) = \int_{-\infty}^{\infty}f(x-t)g(t)dt = \int_{a}^{b}f(x-t)dt
  \end{gathered}
\end{equation*}
\par\bigskip
\noindent Vi ser vad som händer om $u = x-t\Rightarrow du = -dt$:
\begin{equation*}
  \begin{gathered}
    -\int_{x-a}^{x-b}f(u)du = --\int_{x-b}^{x-a}f(x)dx = \int_{x-b}^{x-a}f(x)dx\\
    \Rightarrow (f*(\dfrac{1}{2b}\chi_{[-b,b]}))(x) = \dfrac{1}{2b}\int_{x-b}^{x+b}f(u)du
  \end{gathered}
\end{equation*}
\par\bigskip
\noindent Detta är "genomsnittet" av funktionen i intervallet $[-b,b]$ centrerad kring $x$
\par\bigskip
\begin{prf}[Punkt 1,2]{prf:urin}
  \begin{itemize}
    \item\textbf{Punkt 1}:\par
      \begin{equation*}
        \begin{gathered}
          \left|\mathcal{F}[f](\omega)\right| = \left|\dfrac{1}{2\pi}\int_{-\infty}^{\infty}f(x)e^{-i\omega x}dx\right|\stackrel{\text{triang.}}{\leq}\dfrac{1}{2\pi}\int_{-\infty}^{\infty}\underbrace{\left|f(x)e^{-i\omega x}\right|}_{\text{$\left|f(x)\right|\cdot\underbrace{\left|e^{-i\omega x}\right|}_{\text{=1}}$}}dx\\
          = \dfrac{1}{2\pi}\int_{-\infty}^{\infty}\left|f(x)\right|dx<\infty\text{ eftersom $f$ per antagande är absolutintegrerbar}
        \end{gathered}
      \end{equation*}
      \par\bigskip
    \item\textbf{Punkt 2}:
      \begin{equation*}
        \begin{gathered}
          \lim_{\omega\to\infty}\mathcal{F}[f](\omega) = \lim_{\omega\to\infty}\dfrac{1}{2\pi}\int_{-\infty}^{\infty}f(x)e^{-i\omega x} = 0\text{ p.g.a Riemann-Lebesgue}
        \end{gathered}
      \end{equation*}
  \end{itemize}
\end{prf}
\par\bigskip
\noindent\textbf{Anmärkning:}\par
\noindent Notera att för fouriertransformationer så bryr vi oss inte om exponentiell ordning, detta eftersom vi kräver att funktionen ska vara integrerbar \textit{och} absolutintegrerbar, så kraven att $f(x)$ inte ska "växa för mycket" uppfylls därmed
\par\bigskip
\noindent\textbf{Exempel:}\par
\noindent Vi ska hitta fouriertransformationen till den så kallade \textit{Gausiska} funktionen:
\begin{equation*}
  \begin{gathered}
    f(x) = e^{-x^2/2}\qquad f^{\prime}(x) = -xe^{-x^2/2} = -x\cdot f(x)
  \end{gathered}
\end{equation*}
\par\bigskip
\noindent Detta kan vi betrakta som en ODE ($f^{\prime}(x) = -x\cdot f(x)$) och använda några av egenskaperna ovan: 
\begin{equation*}
  \begin{gathered}
    \Rightarrow (i\omega)\mathcal{F}[f](\omega) = -i\dfrac{d}{d\omega}\mathcal{F}[f](\omega)\\
    \stackrel{\times i}{\Rightarrow}\dfrac{d}{d\omega}\left(\mathcal{F}[f](\omega)\right) = -\omega\mathcal{F}[f](\omega)\Rightarrow \mathcal{F}[f](\omega) = Ce^{-\omega^2/2}
  \end{gathered}
\end{equation*}
\par\bigskip
\noindent Det är ju samma ODE som vi löste för! Eftersom lösningen till en ODE är unik, så får vi att de skiljer sig med en konstant $C$. Vi behöver något initalvärde (som vi inte har) men som vi kan hitta genom att stoppa in några värden!
\begin{equation*}
  \begin{gathered}
    \mathcal{F}[f](0) = \int_{-\infty}^{\infty}f(x)dx = \int_{-\infty}^{\infty}e^{-x^2/2} = \sqrt{2\pi}\\
    \Rightarrow C = \sqrt{2\pi}\\
    \Rightarrow \mathcal{F}[f](\omega) = \sqrt{2\pi}e^{-\omega^2/2}
  \end{gathered}
\end{equation*}
\par\bigskip
\noindent\textbf{Anmärkning:}\par
\noindent $\int_{-\infty}^{\infty}e^{-x^2/2} = \sqrt{2\pi}$ är en standardintegral
\par\bigskip
\noindent Vad händer om vi betraktar \textit{alla} Gausiska funktioner på formen $e^{-ax^2}$ där $a>0$?
\begin{equation*}
  \begin{gathered}
    \underbrace{\mathcal{F}[e^{-ax^2}](\omega)}_{\text{$e^{(-(\sqrt{2a}x)^2/2)}$}} = \dfrac{1}{\sqrt{2a}}\mathcal{F}[e^{-x^2/2}]\left(\dfrac{\omega}{\sqrt{2a}}\right) = \dfrac{1}{\sqrt{2a}}\sqrt{2\pi}e^{-\left(\dfrac{\omega}{\sqrt{2a}}\right)^2/2}\\
    = \dfrac{\sqrt{\pi}}{\sqrt{a}}e^{-\dfrac{\omega^2}{4a}}
  \end{gathered}
\end{equation*}
\par\bigskip
\subsection{Inversen till fouriertransformationen}\hfill\\\par
\noindent Det kanske inte är superuppenbart först, men förhoppningsvis blir det tydligare ju mer vi går igenom det. Sättet vi ska se det på är via Dirichlets konvergenser till fourierserier
\par\bigskip
\noindent Här rekommenderas det att påminna sig om Horizontella/laterala gränsvärden samt derivator och var fourierserien konvergerar (\textit{hint:} genomsnittet)
\par\bigskip
\begin{theo}[Inverssatsen till fouriertransformationen]{thm:inversefourier}
  Låt $f:\R\to\C$ vara en integrerbar och absolutintegrerbar funktion (så vi kan ta fouriertransformationen) 
  \par\bigskip
  \noindent Antag att $f$ har \textbf{både} Horizontella/laterala gränsvärden \textit{och} \textbf{både} Horizontella/lateral derivator i någon punkt $x_0\in\R$. Då gäller:
  \begin{equation*}
    \begin{gathered}
      \lim_{A\to\infty}\dfrac{1}{2\pi}\int_{-A}^{A}\mathcal{F}[f](\omega)e^{i\omega x_0}d\omega = \dfrac{f(x_0^-)+f(x_0^+)}{2}
    \end{gathered}
  \end{equation*}
  \par\bigskip
  \noindent\textbf{Notera }att $f$ må vara integrerbar osv, men vi vet inte om det som står innanför integraltecknet är integrerbar och absolutintegrerbar. Varpå vi har gränsvärdet 
  \par\bigskip
  \noindent\textbf{Notera} att om $f$ är integrerbar och absolutintegrerbar \& differentierbar, så gäller:
  \begin{equation*}
    \begin{gathered}
      \lim_{A\to\infty}\dfrac{1}{2\pi}\int_{-A}^{A}\mathcal{F}[f](\omega)e^{i\omega x_0}d\omega = f(x_0)\qquad\forall x_0\in\R
    \end{gathered}
  \end{equation*}
\end{theo}
\par\bigskip
\noindent\textbf{Anmärkning:}\par
\noindent Förutsatt att det som står i satsen ovan finns, så har vi alltså en ekvation för inversa fouriertransformationen.
\par\bigskip
\noindent Om vi vet att $\mathcal{F}[f](\omega)$ är absolutintegrerbar, så kan vi dra slutsatsen att:
\begin{equation*}
  \begin{gathered}
    \lim_{A\to\infty}\int_{-A}^{A}\mathcal{F}[f](\omega)e^{i\omega x_0} d\omega = \int_{-\infty}^{\infty}\mathcal{F}[f](\omega)e^{i\omega x_0}d\omega
  \end{gathered}
\end{equation*}
\par\bigskip
\noindent\textbf{Corollarium:}\par
\noindent Om $f$ är integrerbar och absolutintegrerbar \& differentierbar och $\mathcal{F}[f](\omega)$ är absolutintegrerbar, så gäller:
\begin{equation*}
  \begin{gathered}
    \dfrac{1}{2\pi}\int_{-\infty}^{\infty}\mathcal{F}[f](\omega)e^{i\omega x_0}d\omega = f(x_0)\qquad\forall x_0\in\R
  \end{gathered}
\end{equation*}
\par\bigskip
\noindent\textbf{Anmärkning:}\par
\noindent Om dessa krav (integrerbar, absolutintegrerbar, differentierbar, fouriertransformationen är absolutintegrerbar) hålls, så är fouriertransformationen unik. Det går även att visa unikhet med mindre krav (Horizontella/lateral differentierbar istället för differentierbar), detta görs längre ner.
\newpage
\noindent\textbf{Exempel:}\par
\noindent Vi har sett att $\mathcal{F}[e^{-\left|x\right|}](\omega) = \dfrac{2}{1+\omega^2}$\par
\noindent Här är $e^{-\left|x\right|}$ integrerbar och absolutintegrerbar, och samtidigt är dess fouriertransformation absolutintegrerbar. Då har vi alltså följande:
\begin{equation*}
  \begin{gathered}
    e^{-\left|x\right|} = \dfrac{1}{2\pi}\int_{-\infty}^{\infty}\dfrac{2}{1+\omega^2}e^{i\omega x}d\omega
  \end{gathered}
\end{equation*}
\par\bigskip
\noindent\textbf{Corollarium:}\par
\noindent Om $f,g$ är integrerbara, absolutintegrerbara, kontinuerliga, Horizontella/laterala differentierbara.\par
\noindent Om $\mathcal{F}[f](\omega) = \mathcal{F}[g](\omega)$ för alla $\omega\in\R$ så kan vi dra slutsatsen att $f(x) = g(x)$ för alla $x\in\R$
\par\bigskip
\begin{theo}[Inversa fouriertransformationen]{thm:inversefouriertransf}
  Om vi har en funktion $f:\R\to\C$ som är integrerbar och absolutintegrerbar, så gäller:
  \begin{equation*}
    \begin{gathered}
      \mathcal{F}^{-1}[f](\omega) = \dfrac{1}{2\pi}\int_{-\infty}^{\infty}f(x)e^{i\omega x}dx
    \end{gathered}
  \end{equation*}
  \par\bigskip
  \noindent Denna kallar vi för \textit{inversa fouriertransformationen}.
\end{theo}
\par\bigskip
\begin{prf}[Unikhet av fouriertransformationen]{prf:unique}
  Låt $f,g$ vara kontinuerliga, integrerbara, och absolutintegrerbara. Tag, $f-g$ (som också uppfyller kraven). Då uppfylls inversa fouriertransformationen för varje $x\in\R$, alltså:
  \begin{equation*}
    \begin{gathered}
      \lim_{A\to\infty}\underbrace{\int_{-A}^{A}\mathcal{F}[f-g](\omega)}_{\text{$\mathcal{F}[f](\omega)-\mathcal{F}[g](\omega) = 0\quad\forall \omega\in\R$}}e^{i\omega x}d\omega = f(x)-g(x)\qquad\forall x\in\R\\
      \Rightarrow 0 = f(x)-g(x)\Lrarr f(x) = g(x)
    \end{gathered}
  \end{equation*}
  \par\bigskip
  \noindent Detta eftersom per antagande så är $\mathcal{F}[f](\omega) = \mathcal{F}[g](\omega)$
\end{prf}
\par\bigskip
\noindent\textbf{Anmärkning:}\par
\noindent Inversa funktionen borde ta in en funktion av $\omega$ och spotta ut en funktion av $x$, men den verkar göra motsatsen. Det ska vi kika mer på nästa gång!
\par\bigskip
\noindent\textbf{Anmärkning:}\par
\noindent Om vi byter ut $\omega$ mot $-\omega$ i inversa fouriertransformationen, så får vi ju fouriertransformationen!
\begin{equation*}
  \begin{gathered}
    \dfrac{1}{2\pi}\mathcal{F}[f](\omega) = \mathcal{F}^{-1}[f](-\omega)
  \end{gathered}
\end{equation*}
\par\bigskip
\noindent Vi kan göra följande omskrivning:
\begin{equation*}
  \begin{gathered}
    \mathcal{F}^{-1}[f](x) = \dfrac{1}{2\pi}\mathcal{F}[f](-x)\stackrel{\mathcal{F}}{\Rightarrow}f(x) = \dfrac{1}{2\pi}\mathcal{F}[\mathcal{F}[f]](-x)\\
    \Rightarrow \mathcal{F}[\mathcal{F}[f]](x) = 2\pi f(-x)\\
    \Rightarrow \mathcal{F}[\mathcal{F}[\mathcal{F}[\mathcal{F}[f]]]](x) = (2\pi)^2f(x)=\hat{\hat{\hat{\hat{f}}}}(x)
  \end{gathered}
\end{equation*}
\par\bigskip
\noindent Här ser vi att inversen ser \textit{nästan} ut som fouriertransformationen.
\newpage
\noindent\textbf{Exempel:}\par
\noindent Vi ska kika på ett konkret exempel på dessa busiga formler. Vi har sett att
\begin{equation*}
  \begin{gathered}
    \mathcal{F}[e^{-\left|x\right|}](\omega) = \dfrac{2}{1+\omega^2}\\
    \Rightarrow \mathcal{F}^{-1}\left[\dfrac{2}{1+x^2}\right](\omega) = e^{-\left|\omega\right|}
  \end{gathered}
\end{equation*}
\par\bigskip
\noindent Om vi använder omskrivningen över:
\begin{equation*}
  \begin{gathered}
    \mathcal{F}\left[\dfrac{2}{1+x^2}\right](\omega) =2\pi\mathcal{F}^{-1}\left[\dfrac{2}{1+(-x)^2}\right](\omega) = 2\pi e^{-\left|\omega\right|}
  \end{gathered}
\end{equation*}
\par\bigskip
\noindent\textbf{Anmärkning:}\par
\noindent Nu kan vara en bra ide att påminna sig själv om Parsevals formel
\par\bigskip
\begin{theo}[Plancherels formel]{thm:plancherel}
  Detta är en slags DIY Parsevals formel för fouriertransformationer genom att modifiera sista kravet för fouriertransformationbarhet\par
  \noindent Vi kräver alltså:\par
  \begin{itemize}
    \item $f$ är integrerbar
    \item $\int_{-\infty}^{\infty}\left|f(x)\right|^2$ konvergerar
  \end{itemize}
  \par\bigskip
  \noindent Då gäller följande: 
  \begin{equation*}
    \begin{gathered}
      \int_{-\infty}^{\infty}\left|f(x)\right|^2dx = \dfrac{1}{2\pi}\int_{-\infty}^{\infty}\left|\mathcal{F}\left[f\right](\omega)\right|^2d\omega
    \end{gathered}
  \end{equation*}\par
\end{theo}
\par\bigskip
\noindent\textbf{Exempel:}\par
\noindent Vi tittar på:
\begin{equation*}
  \begin{gathered}
    \mathcal{F}\left[e^{-\left|x\right|}\right](\omega) = \dfrac{2}{1+\omega^2}
  \end{gathered}
\end{equation*}
\par\bigskip
\noindent Använder vi Plancherels formel får vi:
\begin{equation*}
  \begin{gathered}
    \int_{-\infty}^{\infty}\left(e^{-\left|x\right|}\right)^2dx = \dfrac{1}{2\pi}\int_{-\infty}^{\infty}\left(\dfrac{2}{1+\omega^2}\right)^2d\omega\\
    \Rightarrow\int_{-\infty}^{\infty}e^{-2\left|x\right|}dx = \dfrac{1}{2\pi}\int_{-\infty}^{\infty}\dfrac{4}{(1+\omega^2)^2}d\omega
  \end{gathered}
\end{equation*}
\par\bigskip
\noindent Notera att integralen i VL är betydligt lättare att räkna än HL. Nice gratis verktyg från fouriertransformationer! 
