\section{Laplace Transform}\par
\noindent Vi kommer inte se många komplexa exponentialer, detta stöter vi på senare när vi ska kika på fourierserier och fouriertransformationer.
\par\bigskip
\noindent Precis som alla integraltranformationer, så är tanken bakom att de ska hjälpa att lösa en ODE/PDE. Vi kan tänka på den som en maskin som man stoppar in en funktion, och så spottar den ut en annan funktion:
\begin{equation*}
  \begin{gathered}
    f(t)\sim \mathcal{L}[f](s) = (\stackrel{\sim}{f}(s))
  \end{gathered}
\end{equation*}
\par\bigskip
\noindent Den nya funktionen har lite andra egenskaper. Notera att vi byter variabler från $t$ till $s$.  
\par\bigskip
\noindent Laplace transformationen har lite nice egenskaper:\par
\begin{itemize}
  \item $f^{\prime}(t)\sim \mathcal{L}[f^{\prime}](s) = s\cdot\mathcal{L}[f](s) - f(0)$
\end{itemize}
\par\bigskip
\noindent Det finns såklart en "inverse-maskin", som tar en laplace transformation som input och ger ursprungliga funktionen:
\begin{equation*}
  \begin{gathered}
    \mathcal{L}[f](s)\stackrel{\mathcal{L}^-1}{\sim}f(t)
  \end{gathered}
\end{equation*}
\par\bigskip
\noindent Genom att använda den nice egenskapen på en hel differentialekvation får vi istället en ekvation som består av $s\cdot$ gånger någon funktion, vilket är betydligt lättare att lösa. 
\par\bigskip
\begin{theo}[Laplace Transformationen]{thm:laplacetransform}
  \textit{Laplace Transformationen} av en funktion $f:[0,\infty)\to\R$ är
  \begin{equation*}
    \begin{gathered}
      \mathcal{L}[f](s) = \int_{0}^{\infty}f(t)\cdot e^{-st}dt
    \end{gathered}
  \end{equation*}
  \par\bigskip
  \noindent\textbf{Anmärkning:}\par
  \noindent Laplace transformationen bryr sig inte om vad som händer på den negativa delen av domänen, den vill bara att den är definierad över $\R_+$, alltså:
  \begin{equation*}
    \begin{gathered}
      \mathcal{L}[f(t)](s) = \mathcal{L}[H(t)\cdot f(t)](s)\\
      H(t) = 
      \begin{cases*}
        1,\quad t\geq0\\
        0,\quad t<0
      \end{cases*}
    \end{gathered}
  \end{equation*}\par
  \noindent Vi tappar alltså vad som händer med funktionen i dens negativa domän.
  \par\bigskip
  \noindent\textbf{Anmärkning:}\par
  \noindent Enbart om integralen är definierad. Inte alla funktioner har Laplacetransformationer
  \par\bigskip
  \noindent Vi kan tänka oss att vi "integrerar" bort $t$, kvar får vi en funktion som beror på $s$.
\end{theo}
\par\bigskip
\noindent Ibland får man ett svar som är definierad på negativa värden av $t$ och \textit{ibland} löser den DE:n, men absolut inte alltid och detta måste verifieras för hand och kan inte hänvisas till teori.
\par\bigskip
\noindent\textbf{Exempel:}\par
\noindent Beräkna $\mathcal{L}[t^n](s)$:
\begin{equation*}
  \begin{gathered}
    \int_{0}^{\infty}t^ne^{-st}dt\stackrel{\text{parts}}{=} \left[t^n\cdot(-\dfrac{e^{-st}}{s})\right]_0^\infty+\int_{0}^{\infty}nt^{n-1}\cdot\dfrac{e^{-st}}{s}dt\\
    \text{Om } s>0\Rightarrow 0+\dfrac{n}{s}\int_{0}^{\infty}t^{n-1}\cdot e^{-st}dt\stackrel{\text{parts}}{=}\dfrac{n(n-1)}{s^2}\int_{0}^{\infty}t^{n-2}e^{-st}dt\\
    =\cdots=\dfrac{n!}{s^n}\int_{0}^{\infty}e^{-st}dt = \dfrac{n!}{s^n}\left[\dfrac{e^{-st}}{-s}\right]_0^\infty = \dfrac{n!}{s^{n+1}}
  \end{gathered}
\end{equation*}
\par\bigskip
\noindent\textbf{Anmärkning:}\par
\noindent Eftersom integralen för $s<0$ divergerar, så säger vi att Laplacetransformationen inte är definierad för $s<0$
\par\bigskip
\noindent\textbf{Exempel:}\par
\noindent Beräkna $\mathcal{L}[e^{at}](s)$:
\begin{equation*}
  \begin{gathered}
    \int_{0}^{\infty}e^{at}e^{-st}dt = \int_{0}^{\infty}e^{t(a-s)}dt = \left[\dfrac{e^{t(a-s)}}{a-s}\right]_0^\infty\\
    \text{Om } s>a\quad\Rightarrow\dfrac{1}{s-a}
  \end{gathered}
\end{equation*}
\par\bigskip
\begin{theo}
  OOm $f:[0,\infty)\to\R$ är:\par
  \begin{itemize}
    \item Styckvis kontinuerlig (om i varje begränsat intervall $[0,b]$, så är $f$ begränsad och har ändligt många punkter av diskontinuitet)
      \par\bigskip
    \item Av exponential ordning, dvs det finns konstanter $M>0$ och $\alpha\in\R$ så att $\left|f(t)\right|\leq Me^{\alpha t}$ för alla $t\geq0$
  \end{itemize}
  \par\bigskip
  \noindent Då kommer integralen $\mathcal{L}[f](s)$ konvergera $\forall\;s>\alpha$
\end{theo}
\par\bigskip
\noindent\textbf{Anmärkning:}\par
\noindent Med exponential ordning menas att:
\begin{equation*}
  \begin{gathered}
    \lim_{t\to\infty}\dfrac{f(t)}{Me^{\alpha t}} = 0
  \end{gathered}
\end{equation*}
\par\bigskip
\noindent\textbf{Anmärkning:}\par
\noindent Om Laplacetransformationen konvergerar för $s_0$, så kommer den konvergera $\forall\;s>s_0$
\par\bigskip
\noindent\textbf{Exempel:}\par
\noindent Varje polynom $p(t)$ är av exponentiell ordning med $\alpha=\varepsilon$. Detta för att oavsett hur litet $\alpha$ är, så kommer en exponentialfunktion växa mycket snabbare än ett polynom tillslut. 
\par\bigskip
\noindent\textbf{Anmärkning:}\par
\noindent Laplacetransformationen är linjär (följer från att integralen är linjär)
\par\bigskip
\noindent\textbf{Exempel:}\par
\noindent Funktionen $f(t) = e^{t^2}$ är \textit{inte} av exponentiell ordning. $t^2$ växer snabbare än vilket $\alpha$ som helst. Denna funktion har inga Laplacetransformationer oavsett värde på $s$ 
\par\bigskip
\noindent\textbf{Anmärkning:}\par
\noindent Det finns funktioner som inte är styckvis kontinuerliga men som har konvergerande Laplacetransformation. Vi kommer i denna kurs bara bry oss om de som är styckvis kontinuerliga.
\par\bigskip
\noindent Även om man inte vet att lösningen till en DE uppfyller kriterierna kan man alltid testa! 
\par\bigskip
\begin{theo}[Convolution]{thm:convolution}
  \textit{Convolution} * of $f,g:[0,\infty)\to\R$ is:
  \begin{equation*}
    \begin{gathered}
      (f*g)(t)\int_{0}^{t}f(t-u)\cdot g(u)du
    \end{gathered}
  \end{equation*}
\end{theo}
\par\bigskip
\noindent\textbf{Anmärkning:}\par
\noindent $f*g = g*f$
\newpage
\subsection{Egenskaper hos Laplacetransformation}\hfill\\\par
\begin{itemize}
  \item Om Laplacetransformationen existerar vid $s_0$, då existerar Laplacetransformationen för samma funktion i $s$ för alla $s>s_0$
    \par\bigskip
  \item Laplacetransformationen är linjär
    \par\bigskip
  \item $\mathcal{L}[e^{at}f(t)](s) = \mathcal{L}[f(t)](s-a)$
    \par\bigskip
  \item $\mathcal{L}[H(t-a)f(t-a)](s) = e^{-as}\cdot\mathcal{L}[f](s)\quad a>0$\par
    \noindent Vi vill döda vad som händer på negativa sidan när vi skiftar med $a$, för vi vet inte vad som händer där, varpå "heavyside" funktionen $H(t-a)$ kommer in
    \par\bigskip
  \item $\mathcal{L}[f(at)](s) = \dfrac{1}{a}\mathcal{L}[f(t)]\left(\dfrac{s}{a}\right)\quad a>0$
    \par\bigskip
  \item $\mathcal{L}[f^{\prime}(t)](s) = s\mathcal{L}[f(t)](s)-f(0)$ (måste hålla koll på initialvärdena på DE:n)
    \par\bigskip
  \item $\mathcal{L}[t^nf(t)](s) = (-1)^n\dfrac{d^n}{ds^n}\left(\mathcal{L}[f](s)\right)$ (missbildad spegling av föregående punkt)
    \par\bigskip
  \item $\mathcal{L}[f*g](s) = \mathcal{L}[f](s)\cdot\mathcal{L}[g](s)$
\end{itemize}
\par\bigskip
\noindent\textbf{Exempel:}\par
\noindent Vad är $\mathcal{L}[\sin(at)]$ och $\mathcal{L}[\cos(at)]$?
\begin{equation*}
  \begin{gathered}
    \mathcal{L}[\sin(t)] = \mathcal{L}[-\cos^{\prime}(t)] = -\mathcal{L}[\cos^{\prime}(t)] = -s\cdot\mathcal{L}[\cos(t)]+\cos(0)\\
    = -s\cdot\mathcal{L}[\cos(t)]+1\\
    \mathcal{L}[\cos(t)] = \mathcal{L}[\sin^{\prime}(t)]=s\mathcal{L}[\sin(t)] - \sin(0)\\
    \Rightarrow \mathcal{L}[\sin(t)] = -s^2\mathcal{L}[\sin(t)]+1\\
    \mathcal{L}[\sin(t)](1+s^2) = 1\Rightarrow \mathcal{L}[\sin(t)](s) = \dfrac{1}{s^2+1}\\
    \Rightarrow\mathcal{L}[\cos(t)] = \dfrac{s}{s^2+1}
  \end{gathered}
\end{equation*}
\par\bigskip
\noindent Vi påminner om att Laplarretransformationen är bara definierad över de positiva reella talen, så vi delar upp i fall då $a<0$ och $a>0$:
\par\bigskip
\textbf{Fall $a>0$}:
\begin{equation*}
  \begin{gathered}
    \mathcal{L}[\sin(at)](s) = \dfrac{1}{a}\cdot\mathcal{L}[\sin(t)]\left(\dfrac{s}{a}\right) = \dfrac{1}{a}\cdot\dfrac{1}{\left(\dfrac{s}{a}\right)^2+1} = \dfrac{a}{s^2+a^2}\\
    \mathcal{L}[\cos(at)](s) = \dfrac{1}{a}\cdot\mathcal{L}[\cos(t)]\left(\dfrac{s}{a}\right) = \dfrac{1}{a}\cdot\dfrac{s/a}{(s/a)^2+1} = \dfrac{s}{s^2+a^2}
  \end{gathered}
\end{equation*}
\par\bigskip
\textbf{Fall $a<0$}:
\begin{equation*}
  \begin{gathered}
    \mathcal{L}[\sin(at)](s) = \mathcal{L}[\sin(-(-a)t)](s) = \mathcal{L}[-\sin(-at)](s) = -\mathcal{L}[\sin(-at)](s)\\
    \Rightarrow -\dfrac{-a}{s^2+a^2} = \dfrac{a}{s^2+a^2}\\
    \mathcal{L}[\cos(at)](s) = \mathcal{L}[\cos(-(-a)t)] = \mathcal{L}[\cos(-at)](s) = \dfrac{s}{s^2+a^2}
  \end{gathered}
\end{equation*}
\par\bigskip
\textbf{Fall a=0}:
\begin{equation*}
  \begin{gathered}
    \mathcal{L}[\sin(at)](s) = \mathcal{L}[0](s) = 0\\
    \mathcal{L}[\cos(at)](s) = \mathcal{L}[1](s) = \dfrac{1}{s}
  \end{gathered}
\end{equation*}
\newpage
\noindent Vi visar att Laplarre av en deriverad funktion (en gång!) är $s$ gånger Laplarren:
\par\bigskip
\begin{prf}
  VVi skriver vad vi vet:
  \begin{equation*}
    \begin{gathered}
      \mathcal{L}[f^{\prime}(t)](s) = \int_{0}^{\infty}f^{\prime}(t)e^{-st}dt = \lim_{A\to\infty}\int_{0}^{A}f^{\prime}(t)e^{-st}dt\\
      \stackrel{\text{parts}}{=} \lim_{A\to\infty}[f(t)e^{-st}]_0^A+s\int_{0}^{A}f(t)e^{-st}dt\\
      =\lim_{A\to\infty} f(A)e^{-sA}-f(0)\cdot1+\int_{0}^{A}f(t)e^{-st}dt\\
      = 0-f(0)+s\mathcal{L}[f(t)](s)\Lrarr s\mathcal{L}[f(t)](s)-f(0)
    \end{gathered}
  \end{equation*}
\end{prf}
