\section{Lösa PDE:er med Fourierserier}\par
\subsection{Viktiga PDE:er som löses med separation of variabler}\hfill\\\par
\subsubsection{Vågekvationben}\hfill\\
\begin{equation*}
  \begin{gathered}
    u(x,t)\\
    \Rightarrow\dfrac{\partial^2u}{\partial t^2} = \dfrac{\partial^2u}{\partial x^2}
  \end{gathered}
\end{equation*}
\par\bigskip
\subsubsection{Värmeekvationen}\hfill\\\par
\begin{equation*}
  \begin{gathered}
    \dfrac{\partial u}{\partial t} = \dfrac{\partial u^2}{\partial x^2}
  \end{gathered}
\end{equation*}
\par\bigskip
\noindent Det var värmeekvationen som Fourier började med
\par\bigskip
\subsubsection{Schrödingers ekvation}\hfill\\\par
\begin{equation*}
  \begin{gathered}
    i\dfrac{\partial u}{\partial t}=-\dfrac{\partial^2u}{\partial x^2}
  \end{gathered}
\end{equation*}
\par\bigskip
\subsection{Separation av variabler}\hfill\\\par
\noindent Vi inleder med ett exempel:
\par\bigskip
\noindent\textbf{Exempel:}\par
\noindent Lös följande IVP (initalvärdesproblem) för $u(x,t)$:
\begin{equation*}
  \begin{gathered}
    (*)\begin{cases}
      u_t = u_{xx}+\cos(x)\quad t>0,\; 0<x<\pi\\
      x=0\Rightarrow u(0,t) = 0\\
      x=\pi\Rightarrow u(\pi,t) = 1\quad\forall t>0\\
      t=0\Rightarrow u(x,0)=\dfrac{3}{\pi}x+\cos(x)\quad\forall 0< x< \pi
    \end{cases}
  \end{gathered}
\end{equation*}
\par\bigskip
\noindent Problemet är inte så lätt som den kan vara. Iden är som en vanlig DE, lös det generella problemet (som ger parametrar), stoppa in randvärderna och få lösningen. 
\par\bigskip
\noindent Vanligtvis kan vi ta linjärkombinationer av lösningarn för att få fler lösningar.
\par\bigskip
\noindent Eftersom vi har $\cos(x)$ i PDE:n och $=1$ i randvärderna, så har vi ett \textit{icke-homogent} problem. Vi måste förenkla ner problemet till ett homogent problem:
\begin{equation*}
  \begin{gathered}
    u(x,t) = v(x,t) +\varphi(x)
  \end{gathered}
\end{equation*}\par
\noindent Här slänger vi all icke-homogenitet till $\varphi(x)$ som är en funktion av \textit{en} variabel (och ger oss en ODE), och $v(x,t)$ är då en homogen PDE som blir lättare att lösa
\par\bigskip
\noindent Nu ger vi lite info om $v$ och $\varphi$:
\begin{equation*}
  \begin{gathered}(**)
    \begin{cases*}
      v_t = v_{xx}\quad t>0,\; 0<x<\pi\\
      v(0,t) = 0\;\&\;\underbrace{v(\pi,t)=0}_{\text{$\Rightarrow$ homogen}}\quad\forall t>0\\
      v(x,0)=?\;=1
    \end{cases*}\qquad(***)
    \begin{cases*}
      \varphi^{\prime\prime} = -\cos(x)\\
      \varphi(0) = 0\\
      \varphi(\pi) = 1
    \end{cases*}\\
    \begin{cases*}
      u_t=v_t\\
      u_{xx} = v_{xx}+\varphi^{\prime\prime}\\
      \Rightarrow v_t = v_{xx} +\varphi^{\prime\prime}+\cos(x)
    \end{cases*}
  \end{gathered}
\end{equation*}
\par\bigskip
\noindent Nu behöver vi lösa ODE:n $(***)$ för att bestämma initialvärdena till $v(x,0)$ i $(***)$ 
\begin{equation*}
  \begin{gathered}
    \varphi^{\prime\prime} = -\cos(x)\\
    \varphi^{\prime} = \int-\cos(x)dx = -\sin(x)+C\\
    \varphi = \int\int-\cos(x)dx = \int-\sin(x)+C = \cos(x)+Ax+B\\
    \Rightarrow \varphi(0) = 1+B = 0\Lrarr B = -1\\
    \Rightarrow \varphi(\pi) = -1+A\pi-1 = 1\Lrarr A = \dfrac{3}{\pi}\\
    \Rightarrow \varphi(x) = \cos(x)+\dfrac{3}{\pi}x-1
  \end{gathered}
\end{equation*}
\par\bigskip
\noindent Vi vet att $u(x,0) = \dfrac{3}{\pi}x+\cos(x)$\par
\noindent Däremot, per konstruktion har vi att $ u(x,0) = v(x,0)+\varphi(x)$:
\begin{equation*}
  \begin{gathered}
    v(x,0)+\varphi(x) = v(x,0)+\cos(x)+\dfrac{3}{\pi}x-1\\
    u(x,0) = v(x,t)+\varphi(x) = v(x,t)+\cos(x)+\dfrac{3}{\pi}x-1\\
    u(x,0) = \dfrac{3}{\pi}x+\cos(x) = v(x,0)+\cos(x)+\dfrac{3}{\pi}x-1\\
    \Rightarrow v(x,0) = 1
  \end{gathered}
\end{equation*}
\par\bigskip
\noindent Nu löser vi $(**)$ genom separation av variabler metoden
\par\bigskip
\noindent Iden med separation av variabler metoden är att försöka hitta speciella lösningar på formen\par\noindent $v(x,t) = X(x)\cdot T(t)$ där $X$ är en funktion som bara beror på $x$ och $T$ är en funktion som bara beror på $t$
\par\bigskip
\noindent Vi kikar närmare på $(**)$:
\begin{equation*}
  \begin{gathered}
    v_t = v_{xx}\Lrarr X(x)\cdot T^{\prime}(t) = X^{\prime\prime}(x)\cdot T(x)\\
    \Rightarrow \dfrac{T^{\prime}(t)}{T(t)} = \dfrac{X^{\prime\prime}(x)}{X(x)} = -\lambda
  \end{gathered}
\end{equation*}
\par\bigskip
\noindent Nu har vi fått ner problemet till en ODE:
\begin{equation*}
  \begin{gathered}
    \begin{cases*}
      X^{\prime\prime}(x) = -\lambda X(x)\\
      T^{\prime}(t) = -\lambda T(t)
    \end{cases*}
  \end{gathered}
\end{equation*}\par
\noindent Detta är ett system av ODE:er som beror på en parameter $\lambda$
