\section{Lösa PDE:er med Fourierserier}\par
\subsection{Viktiga PDE:er som löses med separation of variabler}\hfill\\\par
\subsubsection{Vågekvationben}\hfill\\
\begin{equation*}
  \begin{gathered}
    u(x,t)\\
    \Rightarrow\dfrac{\partial^2u}{\partial t^2} = \dfrac{\partial^2u}{\partial x^2}
  \end{gathered}
\end{equation*}
\par\bigskip
\subsubsection{Värmeekvationen}\hfill\\\par
\begin{equation*}
  \begin{gathered}
    \dfrac{\partial u}{\partial t} = \dfrac{\partial u^2}{\partial x^2}
  \end{gathered}
\end{equation*}
\par\bigskip
\noindent Det var värmeekvationen som Fourier började med
\par\bigskip
\subsubsection{Schrödingers ekvation}\hfill\\\par
\begin{equation*}
  \begin{gathered}
    i\dfrac{\partial u}{\partial t}=-\dfrac{\partial^2u}{\partial x^2}
  \end{gathered}
\end{equation*}
\par\bigskip
\subsection{Separation av variabler}\hfill\\\par
\noindent Vi inleder med ett exempel:
\par\bigskip
\noindent\textbf{Exempel:}\par
\noindent Lös följande IVP (initalvärdesproblem) för $u(x,t)$:
\begin{equation*}
  \begin{gathered}
    (*)\begin{cases}
      u_t = u_{xx}+\cos(x)\quad t>0,\; 0<x<\pi\\
      x=0\Rightarrow u(0,t) = 0\\
      x=\pi\Rightarrow u(\pi,t) = 1\quad\forall t>0\\
      t=0\Rightarrow u(x,0)=\dfrac{3}{\pi}x+\cos(x)\quad\forall 0< x< \pi
    \end{cases}
  \end{gathered}
\end{equation*}
\par\bigskip
\noindent Problemet är inte så lätt som den kan vara. Iden är som en vanlig DE, lös det generella problemet (som ger parametrar), stoppa in randvärderna och få lösningen. 
\par\bigskip
\noindent Vanligtvis kan vi ta linjärkombinationer av lösningarn för att få fler lösningar.
\par\bigskip
\noindent Eftersom vi har $\cos(x)$ i PDE:n och $=1$ i randvärderna, så har vi ett \textit{icke-homogent} problem. Vi måste förenkla ner problemet till ett homogent problem:
\begin{equation*}
  \begin{gathered}
    u(x,t) = v(x,t) +\varphi(x)
  \end{gathered}
\end{equation*}\par
\noindent Här slänger vi all icke-homogenitet till $\varphi(x)$ som är en funktion av \textit{en} variabel (och ger oss en ODE), och $v(x,t)$ är då en homogen PDE som blir lättare att lösa
\par\bigskip
\noindent Nu ger vi lite info om $v$ och $\varphi$:
\begin{equation*}
  \begin{gathered}(**)
    \begin{cases*}
      v_t = v_{xx}\quad t>0,\; 0<x<\pi\\
      v(0,t) = 0\;\&\;\underbrace{v(\pi,t)=0}_{\text{$\Rightarrow$ homogen}}\quad\forall t>0\\
      v(x,0)=?\;=1
    \end{cases*}\qquad(***)
    \begin{cases*}
      \varphi^{\prime\prime} = -\cos(x)\\
      \varphi(0) = 0\\
      \varphi(\pi) = 1
    \end{cases*}\\
    \begin{cases*}
      u_t=v_t\\
      u_{xx} = v_{xx}+\varphi^{\prime\prime}\\
      \Rightarrow v_t = v_{xx} +\varphi^{\prime\prime}+\cos(x)
    \end{cases*}
  \end{gathered}
\end{equation*}
\par\bigskip
\noindent Nu behöver vi lösa ODE:n $(***)$ för att bestämma initialvärdena till $v(x,0)$ i $(***)$ 
\begin{equation*}
  \begin{gathered}
    \varphi^{\prime\prime} = -\cos(x)\\
    \varphi^{\prime} = \int-\cos(x)dx = -\sin(x)+C\\
    \varphi = \int\int-\cos(x)dx = \int-\sin(x)+C = \cos(x)+Ax+B\\
    \Rightarrow \varphi(0) = 1+B = 0\Lrarr B = -1\\
    \Rightarrow \varphi(\pi) = -1+A\pi-1 = 1\Lrarr A = \dfrac{3}{\pi}\\
    \Rightarrow \varphi(x) = \cos(x)+\dfrac{3}{\pi}x-1
  \end{gathered}
\end{equation*}
\par\bigskip
\noindent Vi vet att $u(x,0) = \dfrac{3}{\pi}x+\cos(x)$\par
\noindent Däremot, per konstruktion har vi att $ u(x,0) = v(x,0)+\varphi(x)$:
\begin{equation*}
  \begin{gathered}
    v(x,0)+\varphi(x) = v(x,0)+\cos(x)+\dfrac{3}{\pi}x-1\\
    u(x,0) = v(x,t)+\varphi(x) = v(x,t)+\cos(x)+\dfrac{3}{\pi}x-1\\
    u(x,0) = \dfrac{3}{\pi}x+\cos(x) = v(x,0)+\cos(x)+\dfrac{3}{\pi}x-1\\
    \Rightarrow v(x,0) = 1
  \end{gathered}
\end{equation*}
\par\bigskip
\noindent Nu löser vi $(**)$ genom separation av variabler metoden
\par\bigskip
\noindent Iden med separation av variabler metoden är att försöka hitta speciella lösningar på formen\par\noindent $v(x,t) = X(x)\cdot T(t)$ där $X$ är en funktion som bara beror på $x$ och $T$ är en funktion som bara beror på $t$
\par\bigskip
\noindent Vi kikar närmare på $(**)$:
\begin{equation*}
  \begin{gathered}
    v_t = v_{xx}\Lrarr X(x)\cdot T^{\prime}(t) = X^{\prime\prime}(x)\cdot T(x)\\
    \Rightarrow \dfrac{T^{\prime}(t)}{T(t)} = \dfrac{X^{\prime\prime}(x)}{X(x)} = -\lambda
  \end{gathered}
\end{equation*}
\par\bigskip
\noindent Nu har vi fått ner problemet till en ODE:
\begin{equation*}
  \begin{gathered}
    \begin{cases*}
      X^{\prime\prime}(x) = -\lambda X(x)\\
      T^{\prime}(t) = -\lambda T(t)
    \end{cases*}
  \end{gathered}
\end{equation*}\par
\noindent Detta är ett system ODE:er som beror på en parameter $\lambda\in\R$. Det är inte tydligt att denna lösning existerar eller är på denna form, men vi ska visa att det faktiskt finns. \par
\par\bigskip
\noindent $\lambda$ är en konstant, occh det finns olika val av $\lambda$, men de som faktiskt löser kommer vara de negativa $\lambda$, så vi kollar vad som händer om:
\par\bigskip
\noindent\textbf{Fall $\lambda<0$}.
\begin{equation*}
  \begin{gathered}
    X^{\prime\prime}(x) = -\lambda X(x)
  \end{gathered}
\end{equation*}\par
\noindent Här får vi en karaktäristisk ekvation: $\alpha^2=-\lambda\Rightarrow\alpha = \pm\sqrt{-\lambda}$\par
\noindent Generella lösningen till ODE:n med $X$ blir då:
\begin{equation*}
  \begin{gathered}
    X(x) = c_1e^{-\sqrt{-\lambda}x}+c_2e^{\sqrt{-\lambda}x}
  \end{gathered}
\end{equation*}
\par\bigskip
\noindent Vi kan använda initialvärdena för att lösa ut $c_1,c_2$:
\begin{equation*}
  \begin{gathered}
    v(0,t) = 0\Lrarr X(0)\cdot T(t) = 0\\
    v(\pi,t) = 0\Lrarr X(\pi)\cdot T(t) = 0\\
    \Rightarrow\begin{cases}X(0) = 0\\X(\pi) = 0\end{cases}
  \end{gathered}
\end{equation*}\par
\noindent Hur vet vi att inte $T(t) = 0\quad\forall t$? Ett av initialvärdena är $v(x,0) = 1$.
\begin{equation*}
  \begin{gathered}
    \Rightarrow c_1e^{0}+c_2e^{0} \Rightarrow c_2 = -c_1\\
    c_1\left(e^{-\sqrt{-\lambda}\pi}-e^{\sqrt{-\lambda}\pi}\right)\Rightarrow c_1 = 0\\
    \Rightarrow X(x) = 0
  \end{gathered}
\end{equation*}
\par\bigskip
\noindent Med $\lambda<0$ får vi alltså en ganska tråkig lösning, vi undersöker vad som händer om vi har andra värden
\par\bigskip
\noindent\textbf{Fall $\lambda = 0$}
\begin{equation*}
  \begin{gathered}
    X^{\prime\prime}(x) = 0\Rightarrow \int\int X^{\prime\prime}(x) dx = Ax+B\\
    X(0) = B = 0\\
    X(\pi) = A\pi =0\Rightarrow A = 0\\
    \Rightarrow A,B = 0
  \end{gathered}
\end{equation*}
\par\bigskip
\noindent Tråkigt igen, vi kikar på positiva lambdor
\par\bigskip
\noindent\textbf{Fall $\lambda>0$}:
\par\bigskip
\noindent När vi stoppar in i vårat karaktäristiska polynom så kommer vi få icke-reella lösningar. Vi kan då skriva lösningen i termer av $e$, eller så använder vi trigonometrisk form: 
\begin{equation*}
  \begin{gathered}
    X^{\prime\prime}(x) = -\lambda X(x)\\
    \Rightarrow X(x) = c_1\cos((\sqrt{\lambda}x))+c_2\sin(\sqrt{\lambda}x)\\
    X(0) = c_1 = 0\\
    X(\pi) = c_2\sin(\sqrt{\lambda}\pi) = 0\Rightarrow\sqrt{\lambda}\in\Z\Rightarrow \lambda = k^2\;k\in\Z\\
    \Rightarrow X(x) = c_2\sin(kx)\qquad k\in\N^+, c_2\in\R
  \end{gathered}
\end{equation*}
\par\bigskip
\noindent Vi påminner oss om att vi hittar lösningar på formen:
\begin{equation*}
  \begin{gathered}
    v(x,t) = X(x)T(t)\\
    T^{\prime}(t) = -\lambda T(t)\Rightarrow T^{\prime}(t) = -k^2 T(t)\\
    T(t) = c_3e^{-k^2t}
  \end{gathered}
\end{equation*}\par
\noindent Syftet med att kolla på $T(t)$ är för att kunna fixera $k$:et i $X(x)$, det gör det lättare att lösa
\begin{equation*}
  \begin{gathered}
    \Rightarrow v(x,t) = c_2c_3\sin(kx)e^{-k^2t}
  \end{gathered}
\end{equation*}
\par\bigskip
\noindent\textbf{Superposition principle}\par
\noindent Vi får nya lösningar till vår homogena system\par
\noindent VI försöker hitta alla ekvationer innan vi stoppar in initalvärderna. Vi tar alla linjärkombinationer:
\begin{equation*}
  \begin{gathered}
    v(x,t) = \sum_{k=1}^{\infty} b_k\sin(kx)e^{-k^2t}\qquad b_k\in\R
  \end{gathered}
\end{equation*}
\par\bigskip
\noindent Frågan är nu vad $b_k$ är, och för att hitta det så kan vi använda oss av vårat initalvärde $v(x,0) = 1$:
\begin{equation*}
  \begin{gathered}
    v(x,0) = \sum_{k=1}^{\infty}b_k\sin(kx) = 1
  \end{gathered}
\end{equation*}\par
\noindent Detta är en fourierserie! $b_k$ är koefficienter till funktionen $1$, men notera att vi har en fourierserie med bara sinus (som är en udda funktion), medan funktionen $=1$ är en jämn funktion.\par
\noindent Men! Vi behöver bara ha $v = 1$ i intervallet $0<x<\pi$. Fourierserien är $2\pi$ periodisk och udda, men det är inte högerledet 1, så vi gör om högerledet så att det blir det.
\par\bigskip
\noindent Låt $f:\R\to\C$ vara\par
\begin{itemize}
  \item $2\pi$-periodisk
  \item $f(x) = 1$ om $0<x<\pi$
  \item udda
\end{itemize}\par
\noindent Detta är funktionen som har värdet 1 mellan 0 till $\pi$ och -1 mellan $-\pi$ till 0, och inversen överallt. Denna är udda och uppfyller resten av kraven
\par\bigskip
\noindent Nu vill vi hitta en fourierserie för den funktionen:
\begin{equation*}
  \begin{gathered}
    b_k = \dfrac{1}{\pi}\int_{-\pi}^{\pi}\underbrace{f(x)\sin(kx)}_{\text{udda$\cdot$udda = jämn}}dx\\
    \Rightarrow \dfrac{2}{\pi}\int_{0}^{\pi}\sin(kx)dx = \dfrac{2}{\pi}\dfrac{-\cos(x)}{k} = \dfrac{2}{k\pi}\underbrace{\left((-1)^{k+1}+1\right)}_{\text{$k = 2m+1\Rightarrow 2$}}\\
    \Rightarrow v(x,t) = \sum_{m=0}^{\infty}\dfrac{4}{(2m+1)\pi}\sin\left((2m+1)x\right)e^{-(2m+1)^2t}
  \end{gathered}
\end{equation*}
\par\bigskip
\noindent Vi har hittat allt vi behöver nu! Vi påminner oss om att $u(x,t) = v(x,t)+\varphi(x)$:
\begin{equation*}
  \begin{gathered}
    u(x,t) = \left(\sum_{m=0}^{\infty}\dfrac{4}{(2m+1)\pi}\sin((2m+1)x)e^{-(2m+1)^2t}\right) + \cos(x)+\dfrac{3}{\pi}x-1
  \end{gathered}
\end{equation*}
\par\bigskip
\noindent\textbf{Anmärkning}\par
\noindent Funktionen $f(x)$ som vi definierade uppfyller Dirichlets konvergens kriterium. Vi "iggar" konvergens för positiva $t$
\par\bigskip
\noindent\textbf{Anmärkning}\par
\noindent Föreläsaren uppmuntrar att igga konvergens
