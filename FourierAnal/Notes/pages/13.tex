\section{(Tempered) Distributioner}\par
\noindent Än så länge har vi enbart kikat på Fouriertransformationer som har ganska nice egenskaper, bland annat att funktionen över hela $\R$ är integrerbar och detsamma för absolutintegralen.\par
\noindent Men detta gäller generellt sett inte för polynom, eller andra funktioner. Det är en ganska "liten" del av funktionerna som uppfyller kraven.
\par\bigskip
\noindent Det är detta som distributioner ska försöka åtgärda.\par
\noindent Detta gör vi genom att hitta en väldigt liten mängd funktioner (mindre än integrerbar och absolutintegrerbar) så att fouriertransformationen är en bijektiv automorfi med mängden.
\par\bigskip
\noindent Vi påminner oss om följande egenskap hos fouriertransformationen:
\begin{equation*}
  \begin{gathered}
    \mathcal{F}\left[f^{\prime}\right](\omega) = (i\omega)\mathcal{F}\left[f\right](\omega)\\
    \Rightarrow \left|\mathcal{F}\left[f^{\prime}\right](\omega)\right| = \left|\omega\right|\mathcal{F}\left[f\right](\omega)\\
    \Rightarrow\left|\mathcal{F}\left[f\right](\omega)\right| \leq \dfrac{\left|\mathcal{F}\left[f^{\prime}\right](\omega)\right|}{\omega}\leq\dfrac{C}{\left|\omega\right|}
  \end{gathered}
\end{equation*}\par
\noindent Där $C$ är någon konstant (eftersom fouriertransformationen är begränsad).
\par\bigskip
\noindent Generellt säger vi följande:\par
\textit{Om $f^{(k)}(x)$ är integrerbar och absolutintegrerbar gäller:}
\begin{equation*}
  \begin{gathered}
    \left|\mathcal{F}\left[f\right](\omega)\right|\leq\dfrac{C}{\left|\omega\right|^k}
  \end{gathered}
\end{equation*}
\par\bigskip
\noindent Detta påminner lite om fourierkoefficienter, där ju mer differentierbar vi hade en funktion desto snabbare gick funktionens fourierkoefficienter till 0\par
\noindent Samma här, ju mer differentierbar $f$ är, desto snabbare går $\left|\mathcal{F}\left[f\right](\omega)\right|$ mot 0 då $\omega\to\pm\infty$
\par\bigskip
\noindent Detta är i någon mening något vi vill att vår mängd av funktioner ska uppfylla:
\par\bigskip
\begin{theo}[Funktioner som dör snabbt]{thm:fsds}
  Vi säger att $f:\R\to\C$ är en \textit{snabbt döende funktion} om följande gäller:\par
  \begin{itemize}
    \item $f$ är glatt (oändligt gånger differentierbar)
      \par\bigskip
    \item För alla $n$ och varje $k$ $\in\N$ vill vi att $x^nf^{(k)}(x)$ är begränsad, alltså finns det någon konstant $C_{n,k}\geq0$ så att $\left|x^nf^{(k)}(x)\right|\leq C_{n,k}$
  \end{itemize}
\end{theo}
\par\bigskip
\noindent Denna mängd av funktioner som dör snabbt kallas för Schwartzrummet och betecknas $\mathscr{S}(\R)$. Det är ett vektorrum.
\par\bigskip
\noindent\textbf{Exempel}:\par
\noindent Gaussianen $f(x) = e^{-ax^2}\in\mathscr{S}(\R)\quad\forall a>0$ finns i Schwartzrummet  
\par\bigskip
\noindent\textbf{Anmärkning}\par
\noindent Det är oftare lättare att \textit{inte} räkna ut vad denna konstant $C$ är, det funkar lika bra att veta \textit{att} vi har en konstant.
\par\bigskip
\noindent\textbf{Övning:}\par
\noindent Varför är inte $e^{-x}\in\mathscr{S}(\R)$ 
\par\bigskip
\noindent\textbf{Exempel:}\par
\noindent $g(x) = e^{-\sqrt{1+x^2}}\in\mathscr{S}(\R)$
\newpage
\begin{theo}
  OOm $f\in\mathscr{S}(\R)$, så är dess fouriertransformation $\mathcal{F}\left[f\right](\omega)\in\mathscr{S}(\R)$
  \par\bigskip
  \noindent Om vi ser på fouriertransformationen som $\mathcal{F}:\mathscr{S}(\R)\to\mathscr{S}(\R)$, så är den en linjär bijektion till vektorrummet $\mathscr{S}(\R)$. Den skickar även inre-produkten till inre-produkten. \par
\end{theo}
\par\bigskip
\noindent Vi ska försöka göra lite analys i Schwartzrummet!
\par\bigskip
\begin{theo}
  EEn följd $\varphi_j\in\mathscr{S}(\R)$ konvergerar till något $\varphi\in\mathscr{S}(\R)$ om:
  \begin{equation*}
    \begin{gathered}
      \forall n\in\N\quad\forall k\in\N\quad\lim_{j\to\infty}\left|x^n\right|\left|\varphi_j^{(k)}(x)-\varphi^{(k)}(x)\right|
    \end{gathered}
  \end{equation*}\par
  \noindent Detta är samma som att säga:
  \begin{equation*}
    \begin{gathered}
      \forall n\in\N\quad\forall k\in\N\quad\lim_{j\to\infty}\sup{\left|x^n\right|\left|\varphi_j^{(k)}(x)-\varphi^{(k)}(x)\right|}
    \end{gathered}
  \end{equation*}
\end{theo}
\par\bigskip
\noindent Nu kommer det roliga!
\par\bigskip
\begin{theo}[Distribution]{thm:distribution}
  En (tempered) \textit{distribution }är en avbildning $f:\mathscr{S}(\R)\to\C$ sådant att:\par
  \begin{itemize}
    \item $f$ är linjär, det vill säga $f(a\varphi_1+b\varphi_2) = af(\varphi_1)+bf(\varphi_2)$ där $a,b\in\C$ och $\varphi_1,\varphi_2\in\mathscr{S}(\R)$
      \par\bigskip
    \item $f$ är kontinuerlig, det vill säga $f$ bevarar konvergens för följder, det vill säga om\par
      \noindent$\varphi_j\stackrel{\mathscr{S}(\R)}{\to}\varphi$ så $f(\varphi_j)\stackrel{\C}{\to}f(\varphi)$
  \end{itemize}
  \par\bigskip
  \noindent Rummet av dessa distributioner betecknas $\mathscr{S}^{\prime}(\R)$
\end{theo}
\par\bigskip
\noindent\textbf{Exempel:}\par
\noindent Låt $f:\R\to\C$ vara:\par
\begin{itemize}
  \item lokalt integrerbar (det vill säga, $f$ är integrerbar på varje slutet intervall $[a,b]\subset\R$)
    \par\bigskip
  \item Moderat (\textbf{?}) tillväxt, det vill säga den växer lite segare än ett polynom, men inte som en exponential\par
    \noindent Det finns $M>0$ och $n\geq0$ så att för alla $x\in\R$ har vi $\left|f(x)\right|\leq M(\left|x\right|^n+1)$ (polynomiell ordning)
\end{itemize}
\par\bigskip
\noindent För sådana $f$ kan vi definiera en distribution $f\in\mathscr{S}^{\prime}(\R)$ genom:\par
För varje $\varphi\in\mathscr{S}(\R)$ så är:
\begin{equation*}
  \begin{gathered}
    f(\varphi) = \int_{-\infty}^{\infty}f(x)\varphi(x)dx\in\C
  \end{gathered}
\end{equation*}
\par\bigskip
\noindent\textbf{Anmärkning}\par
\noindent Om $f:\R\to\C$ är kontinuerlig, så bestäms den helt av distributionen $f\in\mathscr{S}^{\prime}(\R)$ 
\par\bigskip
\noindent\textbf{Exempel:}\par
\noindent Polynom och de trigonometriska funktionera definierar distributioner.
\par\bigskip
\noindent\textbf{Icke-Exempel:}\par
\noindent $f(x) = e^{x^2}$ ger \textit{inte} en distribution. Detta eftersom om $\varphi(x) = e^{-x^2}\in\mathscr{S}(\R)$ får vi:
\begin{equation*}
  \begin{gathered}
    \int_{-\infty}^{\infty}f(x)\varphi(x)dx = \int_{\infty}^{\infty}1dx = \infty+\infty \not<\infty
  \end{gathered}
\end{equation*}
\par\bigskip
\noindent Den växer för snabbt helt enkelt och är inte begränsad av polynom.
\par\bigskip
\noindent\textbf{Exempel:}\par
\noindent Detta är ett exempel på en distribution som \textit{inte} kommer från en funktion.
\par\bigskip
\begin{theo}[Dirac-delta]{thm:dirarre}
  \textit{Dirac delta} $\delta\in\mathscr{S}^{\prime}(\R)$ definieras enligt:\par
  \begin{itemize}
    \item För varje $\varphi\in\mathscr{S}(\R)$ så är $\delta(\varphi) = \varphi(0)$
  \end{itemize}
  \par\bigskip
  \noindent Notera här att $f(\varphi)\neq\int_{-\infty}^{\infty}f(x)\varphi(x)dx$ eftersom vi då behöver en funktion som är noll överallt men inte i origo, men detta är inte en funktion.
\end{theo}
