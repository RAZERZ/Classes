\section{Bakgrund}\par
\noindent Låt oss betrakta $f:[0,\pi]\to\R$ så att $f(0) = f(\pi) = 0$\par
\noindent När kan vi skriva denna funktion $f(x)$ som en analytisk funktion (potensserie), det vill säga:
\begin{equation}
  \begin{gathered}
    f(x) = \sum_{n=1}^{\infty} a_n\cdot\sin(n\cdot x)
  \end{gathered}
\end{equation}\par
\noindent Där $a_n\in\R$ är konstanter.
\par\bigskip
\noindent Inte alla funktioner tillfredställer att intervallet $[0,\pi]$ ger en konvergerande potensserie för $f$, frågan man kan ställa sig är \textit{när kan vi skriva $f$ som en serie av trigonometriska funktioner?} 
\par\bigskip
\noindent Vi kommer inse att \textit{om} $f$ går att skriva som en potensserie av trigonometriska funktioner, så behöver vi hitta våra koefficienter. I fallet med MacLaurin serier så kom de ($a_n$) från derivatan.\par
\noindent I detta fall kommer det från:
\begin{equation*}
  \begin{gathered}
    a_n  = \dfrac{1}{n}\int_{0}^{\pi}f(x)\sin(nx)dx
  \end{gathered}
\end{equation*}
\par\bigskip
\noindent I någon mening kommer analys-delen av denna kurs från att vi studerar funktioner utifrån integraler, såsom den ovan.\par
\noindent Integralen ovan är integral-transform. 
\par\bigskip
\noindent Vi kan även skriva:
\begin{equation*}
  \begin{gathered}
    f(x) = \sum_{n=0}^{\infty}a_n\sin(nx)+b_n\cos(nx)
  \end{gathered}
\end{equation*}
\par\bigskip
\noindent Något mer vi kommer undersöka, är om vår fourierserie konvergerar, och om den konvergerar mot vår funktion (detta är inte alltid uppenbart) 
