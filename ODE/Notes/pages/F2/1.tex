\section{Föreläsning - 1:a ordningens ODE och 3 typer}

\noindent Vi kommer kolla på 3 olika typer av 1:a ordningens ODE:
\par\bigskip
\begin{itemize}
  \item Linjära
  \item Separabla
  \item Exakta
\end{itemize}
\par\bigskip
\noindent I alla dessa fall kommer vi kunna lösa dessa och få fram explicita lösningar - i allmänhet inte möjligt. Däremot är det viktigt att notera att det finns 1:a ordningens ODE som inte täcks av dessa fallen!
\par\bigskip
\noindent Vi kommer primärt undersöka ODE:er på formen $y^{\prime}=f(t,y)$
\par\bigskip
\subsection{Linjära 1:a ordningens ODE}\hfill\\

\noindent Kan skrivas på formen:
\par\bigskip

\begin{equation*}
  \begin{gathered}
    a(x)\dfrac{dy}{dx}+b(x)y=c(x) \Lrarr \dfrac{dy}{dx}=\dfrac{c(x)-b(x)y}{a(x)}
  \end{gathered}
\end{equation*}
\par\bigskip
\noindent Här antas $a(x)\neq 0$

\subsection{Motiverande exmpel}\hfill\\

\noindent Antag att vi har en ODE på formen:


\begin{equation*}
  \begin{gathered}
    xy^{\prime}+y=e^x\\
    x>0
  \end{gathered}
\end{equation*}
\par\bigskip
\noindent Vi har $xy^{\prime}+y=(xy)^{\prime} = e^x$ där vi nu kan integrera båda sidorna:
\par\bigskip


\begin{equation*}
  \begin{gathered}
    xy = \int e^x dx +C \Lrarr xy=e^x+C \Lrarr y = \dfrac{e^x+C}{x}
  \end{gathered}
\end{equation*}
\par\bigskip

\noindent Här hade vi riktigt tur att vi kunde inse att derivatan av produkten var lika med ODE:n vi ville lösa. Givetvis går det inte alltid att göra så. Men vad vi kan göra är att vi kan multiplicera ekvationen med en \textit{faktor} för att få ODE:n på den formen.
\par\bigskip
\noindent Denna faktor brukar betecknas $\mu(x)$ och kallas för den \textit{integrerande faktorn}. Låt oss kolla på den allmänna lösningsmetoden:
\par\bigskip

\begin{itemize}
  \item Skriv på formen $y\prime+p(x)y=f(x)$
  \item Beräkna integrerande faktorn $\mu(x) = e^{\int p(x)dx}$
  \item Tag integrerande faktor och multiplicera ekvationen med den: $\mu(x)y^{\prime}+\mu(x)p(x)y=\mu(x)f(x)$
  \item Nu har vi $(\mu(x)y)^{\prime}= \mu(x)y^{\prime}+\mu(x)p(x)y$
  \item Vi kan skriva om ekvationen som $(\mu(x)y)^{\prime}=\mu(x)f(x)\Lrarr \mu(x)y = \int\mu(x)f(x)dx$
\end{itemize}
\par\bigskip
\noindent Detta ger oss slutgiltigen lösningen $y=\dfrac{1}{\mu(x)}\int\mu(x)f(x)dx$
\par\bigskip
\noindent Notera att $\mu(x)\neq0$ ty $e^x\neq0$
\par\bigskip
\noindent Detta funkar "alltid", så länge vi kan integrera.

\subsection{Exempel}\hfill\\


\begin{equation*}
  \begin{gathered}
    y^{\prime}+3x^2y=x^2
  \end{gathered}
\end{equation*}
\par\bigskip
\noindent Vi noterar att vi är på rätt form, dvs $p(x)=3x^2, f(x)=x^2$. Då kan vi räkna den integrerande faktorn:

\begin{equation*}
  \begin{gathered}
    \mu(x)=e^{\int p(x)dx}=e^{\int 3x^2dx}=e^{x^3}\\
    \Lrarr e^{x^3}y^{\prime}+e^{x^3} 3x^2y=e^{x^3}\cdot x^2 \Lrarr (e^{x^3}\cdot y)^{\prime}=e^{x^3}x^2\\
    e^{x^3}y=\int x^2\cdot e^{x^3}dx \Lrarr e^{x^3}y=\dfrac{e^{x^3}}{3}+C\\
    y=\dfrac{1}{3}+C\cdot e^{-x^3}
  \end{gathered}
\end{equation*}
\par\bigskip

\subsection{Separabla ekvationer}\hfill\\

\noindent Namnet är ganska beskrivande i det här fallet, där är ekvationer där vi kan separera variablerna. Formellt menas det att ekvationer på denna form är separabla:
\par\bigskip

\begin{equation*}
  \begin{gathered}
    \dfrac{dy}{dx}=g(x)h(y)
  \end{gathered}
\end{equation*}
\par\bigskip

\noindent En lösningsmetod ser ut på följande:
\par\bigskip
\begin{itemize}
  \item Skriv som $\dfrac{dy}{h(y)}=g(x)dx$ (flyttat över allt med $y$ på ena sidan och allt med $x$ på andra)
  \item Integrera båda sidorna: $\int\dfrac{dy}{h(y)}=\int g(x)dx$
\end{itemize}
\par\bigskip
\noindent En rimlig fråga man kan ställa sig är "varför funkar det att betrakta $\dfrac{dy}{dx}$ som ett bråk?":
\par\bigskip

\begin{equation*}
  \begin{gathered}
    y^{\prime}(x)=g(x)h(y(x)) \Lrarr \dfrac{y^{\prime}(x)}{h(y(x))}=g(x)\\
    \int\dfrac{y^{\prime}(x)}{h(y(x))}dx = \int g(x)dx \text{HL är ok, men VL, är den verkligen samma som vi kom fram till? Vi skriver om den}\\
    \int\dfrac{y^{\prime}(x)}{h(y(x))}dx = \left[u(x)=y(x), \dfrac{du}{dx}=y^{\prime}(x)\right] = \int\dfrac{1}{h(u)}du \text{Men $u$ kan lika gärna vara $y$}
  \end{gathered}
\end{equation*}
\par\bigskip

\subsection{Exempel}\hfill\\


\begin{equation*}
  \begin{gathered}
    k=1, n=1000, x(0)=1\\
    \dfrac{dx}{dt}=kx(n+1-x), 0<x<n+1\\
    \dfrac{dx}{dt}=x(1001-x)\\\dfrac{dx}{x(1001-x)}=dt\\
    \int\dfrac{dx}{x(1001-x)}dx = \int1dt\\
    \int\dfrac{dx}{x(1001-x)}dx = \dfrac{1}{1001}\int\dfrac{1}{x}+\dfrac{1}{1001-x}dx \text{vi använde PBU}\\
    \Lrarr \dfrac{1}{1001}\cdot(\log(x)-\log(1001-x))=t+C\\
    \log\left(\dfrac{x}{1001-x}\right) = 1001\cdot t+C\\
    \dfrac{x}{1001-x}=e^{1001t+C}\\
    x(0)=1 \Rightarrow \dfrac{1}{1001-1}=e^{0+C}\Lrarr e^C=\dfrac{1}{1000}\\
    x=\dfrac{1001e^{1001t}}{1000+e^{1001t}}
  \end{gathered}
\end{equation*}
\par\bigskip

\subsection{Exempel 2}\hfill\\


\begin{equation*}
  \begin{gathered}
    (e^{2y}+y)\cdot\cos(x)\dfrac{dy}{dx}=e^{y}\sin(2x)
  \end{gathered}
\end{equation*}
\par\bigskip
\noindent Här är det inte helt uppenbart att den är separabel, vi kan testa att flytta runt saker och se vad vi får:
\par\bigskip

\begin{equation*}
  \begin{gathered}
    \dfrac{e^{2y}+y}{e^y} dy=\dfrac{\sin(2x)}{\cos(x)}dx, (\cos(x)\neq0)
  \end{gathered}
\end{equation*}
\par\bigskip
\noindent Nu ser vi att den är separabel och vi kan köra på!


\begin{equation*}
  \begin{gathered}
    \int\dfrac{e^{2y}+y}{e^y}dy=\int e^y+ye^{-y}dy = e^y -ye^{-y}-e^{-y}+C = \text{VL}\\
    \int\dfrac{\sin(2x)}{\cos(x)}dx = \int\dfrac{2\sin(x)\cos(x)}{\cos(x)}dx = 2\int\sin(x)dx = -2\cos(x)+D = \text{HL}\\
    \text{VL = HL} \Lrarr e^y-ye^-y-e^{-y}=-2\cos(x)+C
  \end{gathered}
\end{equation*}
\par\bigskip
\noindent Detta är en lösning på implicit form, och det går inte att göra så mycket bättre än så ty inga startvärden. Detta är vanligt för separabla ekvationer.
\par\bigskip
\noindent Värt att notera, när vi delar på $\cos(x)$ antar vi att den inte antar värdet 0, men sen i slutet spelar det ingen roll om vi har $\cos(x)=0$. Detta gäller för att vi har kontinuitet och är okej och giltigt.
\par\bigskip
\noindent När $\sin(x)=0$ är $\cos(x)=0$ samtidigt (i bråket), vi får kolla på gränsvärdet då och vi ser att det finns ett G.V. utan problem. Det blir så kallat härbar singularitet.
\par\bigskip

\subsection{Exakta ekvationer}\hfill\\

\noindent Diff. ekvationer på formen:

\begin{equation*}
  \begin{gathered}
    M(x,y)+N(x,y)\cdot\dfrac{dy}{dx}=0
  \end{gathered}
\end{equation*}
\par\bigskip

\noindent Detta betyder nödvändigtvis inte att den är exakt, så vi måste ställa krav på $M, N$. Därför ställer vi lite krav som de bör uppfylla.
\par\bigskip

\begin{theo}[Exakt ekvation]{thm:exakt}
  Ekvationen är \textit{exkakt} om det finns en funktion $F(x,y)$ så att $\dfrac{\partial F(x,y)}{\partial x}=M(x,y)$ och $\dfrac{\partial F(x,y)}{\partial y} = N(x,y)$

\end{theo}
\par\bigskip

\begin{prf}[Bevisskiss: exakt ekvation]{prf:exakt}
  \begin{equation*}
    \begin{gathered}
      M(x,y)dx+N(x,y)dy=0\\
      dF(x,y)=\dfrac{\partial F}{\partial x}\cdot dx+\dfrac{\partial F}{\partial y}\cdot dy = M(x,y)dx+N(x,y)dy = dF(x,y)=0\\
      \Lrarr F(x,y)=C
    \end{gathered}
\end{equation*}

\end{prf}


\subsection{Enkelt exempel}\hfill\\


\begin{equation*}
  \begin{gathered}
    ydx+xdy=0 \Lrarr F(x,y)=xy \Rightarrow \dfrac{\partial F}{\partial x}=y, \dfrac{\partial F}{\partial y}=x\\
    \Lrarr F(x,y)=C \Rightarrow xy = C \Rightarrow y=\dfrac{C}{x}
  \end{gathered}
\end{equation*}
\par\bigskip

\noindent Kuriosa: Varför kallas dessa för \textit{exakta}? Inom differentialgeometri så kallas en differential på denna form $ydx+xdy=0$ för exakt. En viktigare fråga man törs fråga sig är kanske \textit{när är en differentialekvation exakt?}.
\par\bigskip
\noindent Vi har sagt att den är det om det finns ett $F$, men hur kan vi hitta det?
\par\bigskip

\begin{theo}[När kan vi hitta $F$]
  Låt $M(x,y)$ och $N(x,y)$ vara två kontinuerliga funktioner med kontinuerliga första ordningens partiella derivator (vi antar att det här gäller i någon rektangel $a<x<b, c<y<d$). Då är $M(x,y)dx+N(x,y)dy=0$ exakt omm:


  \begin{equation*}
    \begin{gathered}
      \dfrac{\partial M}{\partial y}=\dfrac{\partial N}{\partial x}
    \end{gathered}
  \end{equation*}

\end{theo}
\par\bigskip
\pagebreak

\begin{prf}[När kan vi hitta $F$]
   VVi börjar med ena hållet, om ekvationen är exakt vill vi visa att det här gäller.
  \par\bigskip
  \noindent Att den är exakt implicerar att $\dfrac{\partial F}{\partial x}=M$, $\dfrac{\partial F}{\partial y}=N$.
  \par\bigskip
  \noindent  Vi får:

  \begin{equation*}
    \begin{gathered}
      \dfrac{\partial M}{\partial y}= \dfrac{\partial}{\partial y}\dfrac{\partial F}{\partial x}\\
      \dfrac{\partial N}{\partial x}=\dfrac{\partial}{\partial x}\dfrac{\partial F}{\partial y}\\
      \Lrarr \text{De är lika}
    \end{gathered}
  \end{equation*}
  \par\bigskip
  Eftersom $M,N$ har kontinuerliga derivator så kommuterar dem (enligt flervariabelanalys)
  \par\bigskip

  Vissa ekvationer kan göras exakta genom att multiplicera med en integrerande faktor. Inte så allmänt
  \par\bigskip
\end{prf}
\par\bigskip

\subsection{Exempel}\hfill\\
\noindent Betrakta följande differentialekvation:


\begin{equation*}
  \begin{gathered}
    2xydx+(x^2-1)dy
  \end{gathered}
\end{equation*}
\par\bigskip

\noindent Här är $2xy=M$ och $(x^2-1)=N$. Nu vill vi kolla om den här differentialekvation är exakt:


\begin{equation*}
  \begin{gathered}
    \dfrac{\partial M}{\partial y}= 2x = \dfrac{\partial N}{\partial x}
  \end{gathered}
\end{equation*}
\par\bigskip
\noindent Nu vill vi hitta ett $F$ så att $\dfrac{\partial F}{\partial x} = M = 2x$ och att $\dfrac{\partial F}{\partial y}=N=x^2-1$.
\par\bigskip
\noindent Låt oss ta integralen på $N$:

\begin{equation*}
  \begin{gathered}
    F(x,y)=\int(x^2-1)dy=(x^2-1)\cdot y + h(x)
  \end{gathered}
\end{equation*}
\par\bigskip
\noindent Viktigt att notera att vi får $h(x)$ som konstant, ty vi vet inte om  konstanten beror på $x$ när vi integrerar med avseende på $y$.
\par\bigskip
\noindent Vi vill hitta $h(x)$ så att den första partialen ($M$) uppfylls:
\par\bigskip


\begin{equation*}
  \begin{gathered}
    \dfrac{\partial}{\partial x}\left((x^2-1)y+h(x)\right)=2xy\\
    2xy+h^{\prime}(x)=2xy \Lrarr h^{\prime}(x)=0 \Lrarr h(x)=C
  \end{gathered}
\end{equation*}
\par\bigskip
\noindent Vi behöver \textit{ett} $F$, så vi kan ta $C=0$:

\begin{equation*}
  \begin{gathered}
    F(x,y)=(x^2-1)y\\
    dF(x,y)=0 \Lrarr F(x,y)=C\\
    (x^2-1)y=C \Lrarr y=\dfrac{C}{x^2-1}
  \end{gathered}
\end{equation*}

\subsection{Exempel på differentialekvationer som ej linjär, exakt, eller separabel}\hfill\\

\begin{itemize}
  \item $y^{\prime}=\sin(xy)$
  \item $e^{y^{\prime}}=x$
\end{itemize}





