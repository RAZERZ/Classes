\section{Uppgifter}

\subsection{Potensserielösningar}\hfill\\
\par

\begin{enumerate}

  \item De singulära punkterna är nollställen till $A(x)$
  \item De singulära punkterna är punkterna där antingen $p(x)$ eller $q(x)$ inte är analytisk
  \item Lösningen till ekvationen är analytisk i närheten av en ordinär punkt
  \item Lösningen till ekvationen är inte analytisk i närheten av en singulär punkt
  \item En singulär punkt är reguljär om $x\cdot p(x)$ och $x^2\cdot q(x)$ är analytisk
  \item En singulär punkt är reguljär om $x\cdot p(x)$ och $x\cdot q(x)$ är analytis
  \item För att kontrollera om $x=a$ är en ordinär punkt räcker det att kontrollera om $p(x)$ och $q(x)$ båda är ändliga när $x\to a$

\end{enumerate}
\par\bigskip

\subsection{Eulerekvationen}\hfill\\
\par

\begin{enumerate}

  \item Indikalekvationen för Eulerekvationen ges av $r^2+(a-1)r+b=0$
  \item Eulerekvationen har alltid 2 linjärt oberoende lösningar
  \item Lösningarna till Eulerekvationen är analytiska vid $x=0$
  \item Eulerekvationen har en irreguljär singulär punkt vid $x=0$
  \item Om $r$ är en rot till indikalekvationen är en lösning $x^r$
  \item Om indikalekvationen har en dubbelrot är en lösning $x^r\cdot\log(x)$

\end{enumerate}
\par\bigskip

\subsection{Frobeinus metod}\hfill\\
\par

\begin{enumerate}

  \item Metoden ska användas för reguljära singulära punkter
  \item Indikalekvationen ges av $r^2+p_0\cdot r+q_0=0$
  \item Om $r$ är en rot till indikalekvationen är en lösning på formen $x^{r\cdot f(x)}$ där $f$ är analytisk och $f(0)\neq0$
  \item Det är 2 olika fall att hantera beroende på rötterna till indikalekvationen
  \item Det finns alltid 2 linjärt oberoende lösningar
  \item Metoden ger i allmänhet en differensekvation för koefficienterna i serierna för de 2 lösningarna 

\end{enumerate}
\par\bigskip

\subsection{Gammal inlupp uppgift}\hfill\\
\par

\begin{enumerate}

  \item Betrakta $(1+x^2)\sin(x)y^{\prime\prime}+xy^{\prime}+(1+x^2)xy=0$

    \begin{enumerate}

      \item Bestäm alla reella singulära punkter
      \item Klassifiera singulära punkterna som antingen reguljära eller irreguljära
      \item Om vi söker en potensserielösning i punkten $x_0=1$, det vill säga en serie på formen
        \begin{equation*}
          \begin{gathered}
            \sum_{n=0}^{\infty}c_n(x-1)^n
          \end{gathered}
        \end{equation*}
        \noindent vad är det största intervallet $I$ där vi kan förvänta oss att serien konvergerar?

    \end{enumerate}
    \par\bigskip

  \item Betrakta $x^2y^{\prime\prime}-x(x+3)y^{\prime}+(x+3)y=0$

    \begin{enumerate}

      \item Visa att $x=0$ är en reguljär singulär punkt
      \item Finn indikalekvationen och bestäm dess rötter
      \item Finn en potensserielösning för $x>0$ (en lösning räcker). Det räcker att ge differensekvationen för koefficienterna såväl som de tre första nollskillda termer. 

    \end{enumerate}

\end{enumerate}
