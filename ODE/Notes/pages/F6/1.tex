\section{Föreläsning}
\noindent Idag kommer vi lösa ekvationer på formen $y^{\prime\prime}+p(x)y^{\prime}+q(x)y=f(x)$. Tidigare har vi kollat på det homogena fallet, det vill säga när $f(x) = 0$, nu ska vi hantera det inhomogena fallet.
\par\bigskip

\begin{theo}[Generell lösning]{thm:part}
  Den generella lösningen till $y^{\prime\prime}+p(x)y^{\prime}+q(x)y=f(x)$ kan skrivas på formen:

  \begin{equation*}
    \begin{gathered}
      y(x) = y_h+y_p(x)
    \end{gathered}
  \end{equation*}
  \par\bigskip
  \noindent Där $y_h = C_1y_1(x)+C_2y_2(x)$ är den allmänna lösnigen till den associerade homogena ekvationen, och $y_p$ är \textit{någon} lösning till $y^{\prime\prime}+p(x)y^{\prime}+q(x)y=f(x)$ (kallas för \textit{partikulär} lösning). Partikulärlösningen är alltid oberoende.
\end{theo}
\par\bigskip
\noindent Vi ser då att lösnigen är summan av lösnigen till den associerade homogena ekvationen och \textit{en} lösning till ekvationen.
\par\bigskip

\begin{prf}[Generell lösning]{prf:gensol}
  Låt $y$ vara en lösning till $y^{\prime\prime}+p(x)y^{\prime}+q(x)y=f(x)$. Vi vill visa att $y$ går att skriva på formen:

  \begin{equation*}
    \begin{gathered}
      y = C_1y_1+C_2y_2 +y_p
    \end{gathered}
  \end{equation*}
  \par\bigskip
  \noindent Vi har $y^{\prime\prime}+p(x)y^{\prime}+q(x)y=f(x)$, men vi har också antagit att $y_p^{\prime\prime}+p(x)y_p^{\prime}+q(x)y_p = f(xx)$. Om vi subtraherar båda från varandra får vi:


  \begin{equation*}
    \begin{gathered}
      (y-y_p)^{\prime\prime}+p(x)(y-y_p)^{\prime}+q(x)(y-y_p)=0
    \end{gathered}
  \end{equation*}
  \par\bigskip
  \noindent Denna ekvation är homogen! Det ger oss:


  \begin{equation*}
    \begin{gathered}
      y-y_p = C_1y_1+C_2y_2\\
      \Lrarr y=C_1y_1+C_2y_2+y_p
    \end{gathered}
  \end{equation*}
\end{prf}
\par\bigskip
\noindent Hur ska vi hitta $y_p$? Detta är föreläsningens premiss. Partikulärlösningen är \textit{inte} unik, även om vi hänvisar till den som "partikulärlösning\textit{en}". Oftast får man den på en specifik form, men den är givetvis inte entydig.
\par\bigskip
\noindent Kommentar: Om vi har 2 olika partikulärlösningar $y_{p_1}, y_{p_2}$ så förhåller de sig till varandra:\par\noindent $y_{p_2} = C_1y_1+C_2y_2+y_{p_1}$, men vi skulle kunna vända på det och skriva $y_{p_1}=D_1y_1+D_2y_2+y_{p_2}$
\par\bigskip
\noindent Vi kommer gå igenom 2 olika metoder för att hita partikulärlösningen:

\begin{itemize}
  \item Metoden med obestämda koefficienter (ansatsmetoden) (mer gissning)
  \item Variation av parameter-metoden (mer allmän, mer räkna)
\end{itemize}
\par\bigskip

\subsection{Metoden med obestämda koefficienter}\hfill\\

\noindent Idén går ut på att gissa en lösning och verifiera den.
\par\bigskip
\noindent Exempel: Hitta en partikulärlösning till:
\par\bigskip

\begin{itemize}
  \item $y^{\prime\prime}+3y^{\prime}+4y=3x+2$\\
  \item $y^{\prime\prime}-y = 2e^{3x}$\\
  \item $3y^{\prime\prime}+y^{\prime}-2y=2\cos(x)$\\
\end{itemize}
\par\bigskip
\noindent Anmärkning: De har alla konstanta koefficienter.
\par\bigskip
\noindent Vi börjar med att kolla på första. Vi söker ett $y$ så att vi får $3x+2$ på HL. Det låter rimligt att det vi söker är ett polynom, så vi tar ett allmänt polynom och stoppar in och ser vad som händer.
\par\bigskip
\noindent Graden av polynomet på HL är av grad 1, så rimligtvis borde $y$ vara ett förstagradspolynom. Det generella förstagradspolynomet är $y = Ax+B$. Vi stoppar in och ser vad som händer:


\begin{equation*}
  \begin{gathered}
    \begin{rcases*}
      y_p^{\prime}=A\\
      y_p^{\prime\prime}=0
    \end{rcases*}\\
    \Lrarr 0+3A+4(Ax+B)=3x+2\\
    4Ax+(3A+4B) = 3x+2\\
    \text{Här kan man jämföra vad som står framför $x$ och vilka som är konstanter. Vi får:}\\
    \begin{rcases*}
      4A = 3\\
      3A+4B = 2
    \end{rcases*}= A=\dfrac{3}{4}\text{ , } B=\dfrac{-1}{16}
  \end{gathered}
\end{equation*}
\par\bigskip
\noindent En partikulärlösning är $y_p = \dfrac{3}{4}x-\dfrac{1}{16}$
\par\bigskip
\noindent Nästa kommer vara att inse att vi har en exponentialfunktion. Vi gissar:


\begin{equation*}
  \begin{gathered}
    y_P = Ae^{3x}\\
    \begin{rcases*}
      y_p^{\prime} = 3Ae^{ex}\\
      y_p^{\prime\prime}=9Ae^{3x}
    \end{rcases*}\\
    \Lrarr 9Ae^{ex}-Ae^{3x}=2e^{3x} = 8Ae^{3x}=2e^{3x}\\
    \Lrarr A = \dfrac{1}{4} \Rightarrow y_p = \dfrac{1}{4}e^{3x}
  \end{gathered}
\end{equation*}
\par\bigskip
\noindent Ibland får man en gissning som ger 0 på VL, då har man gissat fel. Vissa folk på tentor har gissat fel och börjar lösa för $x$ men det är inte riktigt det vi vill, vi vill hitta för alla $x$.
\par\bigskip
\noindent Vi kör sista (men inte minsta!). Vi ska testa vad som händer om man gissar fel. Vi gissar $y_p = A\cos(x)$:


\begin{equation*}
  \begin{gathered}
    \begin{rcases*}
      y_p^{\prime} = -A\sin(x)\\
      y_p^{\prime\prime}=-A\cos(x)
    \end{rcases*}\\
    \text{Första anmärkningen på att något har gått snett är att vi har fått $\sin(x)$, vi forts.}\\
    -3A\cos(x)-A\sin(x)-2A\cos(x)=2\cos(x)\\
    -5A\cos(x)-A\sin(x)=2\cos(x)\\
    \text{För att detta skall gälla bör vi ha:}\\
    -5A = 2 \Rightarrow A=\dfrac{-2}{5}\text{ men vi har ingen sinus term i HL!}
  \end{gathered}
\end{equation*}
\par\bigskip
\noindent Vad vi skulle kunna göra är att vi gör en ny gissning, eftersom vi vet att vid derivering får vi motsatt funktion kan vi ansätta $y_p = A\cos(x)+B\sin(x)$. Vi får:
\par\bigskip


\begin{equation*}
  \begin{gathered}
    \begin{rcases*}
      y_p^{\prime} = -A\sin(x)+B\cos(x)\\
      y_p^{\prime\prime}=-A\cos(x)-B\sin(x)
    \end{rcases*}\\
    3(-A\cos(x)-B\sin(x))+(-A\sin(x)+B\cos(x))-2(A\cos(x)+B\sin(x))=2\cos(x)\\
    (-5A+B)\cos(x)+(-A-5B)\sin(x) = 2\cos(x)\\
    \begin{rcases*}
      -5A+B=2\\
      -A-5B=0
    \end{rcases*}=
    \begin{rcases*}
      A=\dfrac{-5}{13}\\\\
      B=\dfrac{1}{13}
    \end{rcases*}\\\\
    y_p = \dfrac{-5}{13}\cos(x)+\dfrac{1}{13}\sin(x)
  \end{gathered}
\end{equation*}
\par\bigskip
\noindent Denna metod fungerade bra i dessa exempel. Vi ska kika på ett exempel där den första gissningen inte riktigt skulle funka och tänka på "vad kan vi göra då?".
\par\bigskip
\noindent Exempel: Hitta en partikulärlösning till $y^{\prime\prime}-5y^{\prime}+4y=8e^x$ (ganska lik ett tidigare exempel). Rimlig gissning är $y_p = Ae^x$:


\begin{equation*}
  \begin{gathered}
    Ae^x-5Ae^x+4Ae^x=8e^x\\
    0Ae^x=8e^x \Rightarrow\text{Bus! Kan ej välja $A$ så att vi får en lösning}.
  \end{gathered}
\end{equation*}
\par\bigskip
\noindent Varför hände detta och varför funkar det inte? Jo, för att om vi kollar på den associerade homogena ekvationen $y^{\prime\prime}-5y^{\prime}+4y=0$ så har den den allmänna lösningen $y_h=C_1e^x+C_2e^{4x}$. Så $y_p =Ae^x$ löser den homogena ekvationen, löser den den homogena kan den inten rimligtvis lösa den inhomogena.
\par\bigskip
\noindent En smidig lösning på detta är att man tar det man gissade på, och multiplicerar med $x$. Detta funkar i allmänhet, får man noll så multiplicera med $x$.$y_p = Axe^x$. Vi kollar:


\begin{equation*}
  \begin{gathered}
    \begin{rcases*}
      y_p^{\prime} = Ae^x+Axe^x\\
      y_p^{\prime\prime}=Ae^x+Ae^x+Axe^x = 2Ae^x+Axe^x
    \end{rcases*}\\
    \Lrarr 2Ae^x+Axe^x-5(Ae^x+Axe^x)+4Axe^x=8e^x\\
    -3Ae^x+0Axe^x=8e^x\\
    A=\dfrac{-8}{3}\Rightarrow y_p=\dfrac{-8}{3}xe^x
  \end{gathered}
\end{equation*}
\par\bigskip

\begin{center}
  \begin{tabular}{|c|c|}
    \hline
    Om $f(x)$ är&Gissa $y_p$\\
    \hline
    Polynom&$x^s$Polynom (där $s$ oftast är noll men om vi får 0 öka)\\
    \hline
    $a\cos(kx)+b\sin(kx)$&$x^s(A\cos(kx)+B\sin(kx))$\\
    \hline
    $e^{rx}(a\cos(kx)+b\sin(kx))$&$x^se^{rx}(A\cos(kx)+B\sin(kx))$\\
    \hline
    $P_ne^{rx}$&$x^s(P_n)e^{rx}$\\
    \hline
    $P_n(a\cos(kx)+b\sin(kx))$&$x^s(P_n)(A\cos(kx)+B\sin(kx))$
  \end{tabular}
\end{center}
\par\bigskip
\noindent Vad händer om $f(x)$ inte är på en form liknande ovan? Då kan vi troligen inte gissa lösningen. Men, om den är en summa av saker i tabellen, då kan vi använda att den är linjär och därmed dela upp i fall.
\par\bigskip

\subsection{Variation av parameter-metoden}\hfill\\
\par\bigskip

\noindent Mer allmän, men kräver mer beräkning. Risken är därmed större att man räknar fel.
\par\bigskip
\noindent Den ger oss en lösning till $y^{\prime\prime}+p(x)y^{\prime}+q(x)y=f(x)$ om vi har en fundamental lösningsmängd $\{y_1, y_2\}$ till den associerade homogena ekvationen.
\par\bigskip
\noindent En partikulärlösning ges då på formen $y_p = u_1(x)y_1(x)+u_2(x)y_2(x)$. Hade $u_1, u_2$ varit konstanter hade vi fått en homogen ekvation.
\par\bigskip
\noindent Här är $u_1^{\prime} = \dfrac{W_1}{W(y_1, y_2)}$ och $u_2^{\prime} = \dfrac{W_2}{W(y_1, y_2)}$ där $W$ är Wronskianen som inte är något annat än en determinant av en 2x2 matris:


\begin{equation*}
  \begin{gathered}
    W = det\left(\begin{pmatrix}y_1&y_2\\y_1^{\prime}&y_2^{\prime}\end{pmatrix}\right) = y_1y_2^{\prime}-y_1^{\prime}y_2\\
    W_1 = det\left(\begin{pmatrix}0&y_2\\f(x)&y_2^{\prime}
    \end{pmatrix}\right) = -f(x)y_2\\
        W_2 = det\left(\begin{pmatrix}y_1&0\\y_1^{\prime}&f(x)\end{pmatrix}\right) = f(x)y_1
  \end{gathered}
\end{equation*}
\par\bigskip
\noindent Exempel: Lös $y^{\prime\prime}+3y^{\prime}+2y=\dfrac{1}{1+e^x}$
\par\bigskip
\noindent Om vi vill använda denna metod, så behöver vi den homogena ekvationen, så steg 1, lös homogen:


\begin{equation*}
  \begin{gathered}
    y^{\prime\prime}+3y^{\prime}+2y=0\\
    \text{Karaktäristiska polynomet } r^2+3r+2=0\\
    \begin{rcases*}
      r_1 = -1\\
      r_2 = -2
    \end{rcases*}\\
    \text{Lösningen är } y_h = C_1e^{-x}+C_2e^{-2x}\\
    \text{Där } y_1 = e^{-x}, y_2 = e^{-2x}\\
    \text{Vi söker } y_p = u_1(x)y_1(x)+u_2(x)y_2(x)\\
    W(y_1,y_2) = det\left(\begin{pmatrix}e^{-x}&e^{-2x}\\-e^{-x}&-2e^{-2x}\end{pmatrix}\right) = -2e^{-3x}+e^{-3x} = -e^{-3x}\\
    W_1 = \dfrac{-1}{1+e^x}e^{-2x}\\
    W_2 = \dfrac{1}{1+e^x}e^{-x}\\\\
    u_1^{\prime} = \dfrac{W_1}{W} = \dfrac{-e^{-2x}}{1+e^x}\cdot\dfrac{1}{-e^{-3x}} = \dfrac{e^x}{1+e^x}\\
    u_2^{\prime} = \dfrac{e^{-x}}{1+e^x}\cdot\dfrac{1}{-e^{-3x}}=\dfrac{-e^{2x}}{1+e^x}\\\\
    u_1 = \int\dfrac{e^x}{1+e^x}dx = \left[u = e^x, du = e^xdx\right] = \int\dfrac{1}{1+u}du\\
    = \ln|1+u| \text{ (vi struntar i +$C$ ty partikulärlösning är \textit{en})} = \ln\left|1+e^x\right|\\\\
    u_2 = \int \dfrac{e^2x}{1+e^x} = -\int\dfrac{e^x(1+e^x)-e^x}{1+e^x}dx = -\int e^xdx+\int\dfrac{e^x}{1+e^x}dx = -e^x+\ln\left|1+e^x\right|
  \end{gathered}
\end{equation*}
\par\bigskip
\noindent Vi får $y_p = \ln\left|1+e^x\right|\cdot e^{-x} + (-e^x+\ln\left|1+e^x\right|\cdot e^{-2x}) = \ln\left|1+e^x\right|(e^{-x}+e^{-2x})-e^{-x}$
\par\bigskip
\noindent Allmänna lösningen ges av:

\begin{equation*}
  \begin{gathered}
    y = y_p+y_h = C_1e^{-x}+C_2e^{-2x}+\ln\left|1+e^x\right|(e^{-x}+e^{-2x})-e^{-x}
  \end{gathered}
\end{equation*}
\par\bigskip
\noindent Varför funkar det här? Om vi stoppar in så löser det, men \textit{varför?} Sätter vi in $y_p(x)=u_1(x)y_1(x)+u_2(x)y_2(x)$ i ekvationen och förenklar, får vi:

\begin{equation*}
  \begin{gathered}
    (y_1u_1^{\prime}+y_2u_2^{\prime})^{\prime} + p(x)(y_1u_1^{\prime}+y_2u_2^{\prime})+y_1^{\prime}u_1^{\prime}+y_2^{\prime}u_2^{\prime} = f(x)
  \end{gathered}
\end{equation*}
\par\bigskip
\noindent Notera att vi har $(y_1u_1^{\prime}+y_2u_2^{\prime})^{\prime}$ 2 gånger. Vi kan då kräva
\par\bigskip
$
\begin{rcases*}
  (y_1u_1^{\prime}+y_2u_2^{\prime})^{\prime} = 0\\
  y_1^{\prime}u_1^{\prime}+y_2^{\prime}u_2^{\prime} = t(x)
\end{rcases*}$
\par\bigskip
\noindent Detta är en linjärt system! Vi har slutna formler. Vi kan alltså kolla om systemet är lösbart genom Cramers regel. 






