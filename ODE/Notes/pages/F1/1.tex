\section{Föreläsning - Intro}
\subsection{Vad är en differentialekvation?}
\par\bigskip


\noindent En ekvation som innehåller derivator med avseende på en eller flera oberoende variabler.\par
Om det enbart är en variabel som är oberoendekallas det för en \textit{ordinär} differentialekvation.

Fråga: Vad är poängen med att variablerna är oberoende?
Svar: Man måste ha oberoende variabler som man löser för

\subsection{Exemepel på ODE}
\begin{itemize}
  \item $\dfrac{dy}{dx} + 5y = e^x$
  \item $y^{\prime\prime}-y^{\prime}+6y=0$
  \item $\dfrac{dx}{dt}+\dfrac{dy}{dt}=2x+y$
\end{itemize}
\par\bigskip

\noindent I alla dessa fall hade vi \textit{en} oberoende variabel, har vi flera så är det en \textit{partiell differentialekvation} (PDE).

\subsection{Exemepel på PDE}

\begin{itemize}
  \item $\dfrac{\partial^2u}{\partial x^2}+\dfrac{\partial^2u}{\partial y^2}=0$
\end{itemize}

\subsection{Newtons andra lag}
\par\bigskip

$F=m\cdot a$ där $F$ är kraften, $m$ massan, $a$ accelerationen.
Säg att vi har en sten:

\begin{figure}[ht]
    \centering
    \incfig{figur}
    \caption{Figur}
    \label{fig:figur}
\end{figure}
\par\bigskip

\noindent Då får vi följande samband

\begin{equation*}
  a = \dfrac{d^2s}{dt^2}\\
  F = -mg\\
  -mg=m\cdot\dfrac{d^2s}{dt^2}\\
  \dfrac{d^2s}{dt^2}=-g
\end{equation*}

\subsection{Spridning av en sjukdom}

\noindent Ett virus sprids genom en population. Spridning sker när folk som är infekterade kommer i kontakt med folk som inte är smittade.
\par\bigskip
 Vad säger detta om spridningen? Antag att vid tid $t$ så är $x(t)$ personer infekterade och $y(t)$ personer är inte infekterade. Spridning bör vara proportionell mot hur många från $x, y$ som mötes, det vill säga ett rimligt antagande är:
 \par\bigskip


 \begin{equation*}
   \dfrac{dx}{dt} = kxy
 \end{equation*}
 \par\bigskip

 \noindent Något man kan studera är vad som händer när en infekterad person introduceras till en grupp icke-infekterade. Detta betyder att $x(0)=1$. Antal personer är $x+y=n+1$. Stoppar vi in detta i ekvationen får vi:


 \begin{equation*}
   \dfrac{dx}{dt}= k\cdot x\cdot(n+1-x)
 \end{equation*}
 \par\bigskip

\noindent Ofta kan man inte lösa ODE:er med hand, då kan man lösa det grafiskt/m.h.a dator. Det finns flera verktyg som kan åstadkomma detta men ett av dessa verktyg är ett vektorfält/riktningsfält.

\subsection{Riktningsfält}
\par\bigskip

\noindent Grafiska verktyg är ofta väldigt kraftfulla för att förstå hur lösningen till en ODE ser ut.\par
Tänk oss att vi har följande ODE:


\begin{equation*}
  y^{\prime}= \dfrac{y}{x^2}
\end{equation*}
\par\bigskip

\noindent De flesta ODE:er går inte att lösa exakt. Därför är grafiska verktyg/numreriska metoder väldigt användbara när man vil lösa ODE:er.
\par\bigskip

\subsection{Lösning av en ODE}
\par\bigskip


\begin{equation*}
  \begin{gathered}
    \dfrac{dp}{dt}=\dfrac{p}{2}-450\\
    \dfrac{dp}{dt}=\dfrac{1}{2}(p-450)\\
    p=900 \Rightarrow \dfrac{dp}{dt}=0 \Rightarrow p(t)=900 \text{är en lösning}\\
    p\neq 900 \Rightarrow \dfrac{\dfrac{dp}{dt}}{p-900}=\dfrac{1}{2}\\
    \dfrac{d}{dt}(\log(|p-900|))=\dfrac{1}{2} \text{vi kan nu integrera m.a.p t}\\
    \log(|p-900|)=\int\dfrac{1}{2}dt=\dfrac{t}{2}+C\\
    |p-900|=e^{\dfrac{t}{2}+C}=e^Ce^{\dfrac{t}{2}}\\
    p=900 \pm e^Ce^{\dfrac{t}{2}} \text{där $e^C$ är en positiv konstant}
  \end{gathered}
\end{equation*}
\par\bigskip

Kom ihåg att $p=900$ är en lösning, alltså:
\par\bigskip

\begin{equation*}
  p=900+Ce^{t/2}
\end{equation*}
\par\bigskip

ger alla lösningar. Vi kan få fram C genom att kolla på startpopulation:
\par\bigskip


\begin{equation*}
  \begin{split}
    p(0)=1000 \Rightarrow C=100
    p(t)=900+100e^{t/2}
  \end{split}
\end{equation*}
\par\bigskip

Liknande beteende $\forall p(0)>900$
\par\bigskip

\subsection{Klassificering av ODE:er}
\par\bigskip

\noindent I funktionen $y(t)$ är $y$ en beroende variabel (beror på $t$) och $t$ en oberoende variabel. Detta är en reellvärd funktion på intervallet $(a,b)$:

\begin{equation*}
  y:(a-b)\to\R
\end{equation*}

\begin{equation*}
  y^{\prime}=\dfrac{dy}{dt}, y^{\prime\prime}=\dfrac{d^2y}{dt}, y^{\prime\prime\prime}=\dfrac{d^3y}{dt}
\end{equation*}
\par\bigskip
\noindent Alternativt:

\begin{equation*}
  y^{(0)}=y, y^{(1)}=\dfrac{dy}{dt}\cdots
\end{equation*}

\begin{theo}[ODE]{thm:ODE}
  En ordinär differentialekvation för funktionen $y=y(t)$ är en ekvation på formen:

  \begin{equation*}
    F(t,y,y^{(1)}, \cdots y^{(n)}) = 0
  \end{equation*}
  \par\bigskip

\end{theo}
\par\bigskip

\begin{theo}[Graden av en ODE]{thm:deg}
  \textit{Graden} av en ODE är ordningen på den högsta derivatan av $y$ som förekommer. I sats 1.1 hade det då varit $n$.
\end{theo}
\par\bigskip

\begin{theo}[Linjär ODE]{thm:linjODE}
  En ODE kallas för \textit{linjär} om den är på formen:

  \begin{equation*}
    \sum_{i=0}^{n}a_i(t)y^{(i)}(t) = g(t)
  \end{equation*}
\end{theo}
\par\bigskip

\begin{theo}[Icke-linjär ODE]{thm:nonLin}
  Om en ODE inte är linjär kallas den \textit{icke-linjär}
\end{theo}
\par\bigskip

\noindent Hur definierar vi en lösning till en ODE?
\par\bigskip
En lösning till en ODE på ett intervall $(a,b)$ är en funktion $y(t)$ så att:

\begin{itemize}
  \item $y$ och alla dess derivator är kontinuerliga $\forall t\in(a,b)$
  \item $y$ löser ekvationen $\forall t$
\end{itemize}
\par\bigskip
Om detta uppfylls kallas $(a,b)$ \textit{lösningsintervallet}.

\subsection{Exemepel}
\par\bigskip

\begin{equation*}
  t^5y^{(4)}-t^3-y^{(2)}+6y=0
\end{equation*}
\par\bigskip

Denna ODE är linjär och av grad 4.
\par\bigskip

\subsection{Exemepel}
\par\bigskip


\begin{equation*}
  u^{\prime\prime}=\sqrt{1+(u^{\prime})^2}
\end{equation*}
\par\bigskip

\begin{theo}[Initialvärdesproblem]{thm:init}
  Ett \textit{initialvärdesproblem} (IVP) är en ODE tillsammans med ett startvärde för den oberoende variabeln. För en ODE av grad 1:
  \par\bigskip

  \begin{equation*}
    F(t,y,y^{\prime})=0 \text{och} y(x_0)=y_0
  \end{equation*}
  \par\bigskip

  Där $y(x_0)=y_0$ kallas \textit{initialvillkoret}
\end{theo}
\par\bigskip

\noindent Ofta kommer vi ha:
\begin{equation*}
  y^{(n)}(t)= F(t,y^{(1)}\cdots y^{(n-1)}(t))
\end{equation*}
\par\bigskip

\subsection{Idén i den här kursen}
\par\bigskip

\begin{itemize}
  \item Givet en ODE, finns det lösning?
  \item Om ja, hur många lösningar finns det? (Inget initialvärde kommer vi ha oändligt)
  \item Hitta explicita lösningar till enkla ODE:er
  \item Analysera och approximera lösningar till komplicerade ODE:er med serier (typ som Taylor-serier men med mer krut)
  \item Kvalitativa egenskaper (hur påverkar initialvillkoret lösningen?)
  \item Numreriska metoder
\end{itemize}
