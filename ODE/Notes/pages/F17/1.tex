\section{Periodiska lösningar och gränscykler}
\par\bigskip
\noindent Många fenomen i naturen/verkligheten beter sig periodiskt, eller i alla fall approximativt periodiska. I många fall beskriver periodiska lösningar någon slags gräns/slutfas hos ett system. Den kanske inte är periodisk från början, men när tiden går så blir den periodisk. Detta är premissen till dagens föreläsning.
\par\bigskip

\begin{theo}[Periodisk lösning]{thm:periodicsol}
  En lösning $X=X(t)$ av det autonoma systemet $X^{\prime} = F(x)$.\par
  \noindent En lösning till detta system kallas \textit{periodisk} om det finns ett $t\in\R$ (perioden) så att:
  \begin{equation*}
    \begin{gathered}
      X(t+T)=X(t) \quad \forall t\in\R
    \end{gathered}
  \end{equation*}\par
  \noindent Konstanta lösningar är periodiska.
\end{theo}
\par\bigskip
\noindent Vi bryr oss främst om icke-konstanta periodiska lösningar.
\par\bigskip
\noindent Vi har sätt exempel på periodiska lösningar när vi betraktade system av ODE:er med rent imaginära egenvärden och lösningen berodde på $\sin(x)$ och $\cos(x)$
\par\bigskip
\noindent Vi betraktar de fallen då lösningen inte är explicit periodisk utan är slutfasen.\par
\noindent Detta betyder att det kommer finnas en periodisk lösning (lokalt) och andra lösningar närmar sig denna lösning.
\par\bigskip
\noindent En sådan periodisk lösning kallas för en \textit{gränscykel}. De flesta system kommer vi inte epxlicit kunna hitta gränscykeln, det kan gå numeriskt men oftast inte explicit.
\par\bigskip
\noindent\textbf{Exempel:} Betrakta följande system:
\begin{equation*}
  \begin{gathered}
    \begin{rcases*}
      x^{\prime} = x+y-x(x^2+y^2) = P(x,y)\\
      y^{\prime} = -x+y-y(x^2+y^2) = Q(x,y)
    \end{rcases*}
  \end{gathered}
\end{equation*}
\par\bigskip
\noindent Det som gör detta system till ett snällt system är att vi har $x^2+y^2$ som gör att vi kan skriva om till polära koordinater.\par
\noindent Till att börja med så kan vi notera att $(0,0)$ är en kritisk punkt. Betraktar vi Jacobianen i punkten $(0,0)$ har egenvärdena:
\begin{equation*}
  \begin{gathered}
    1\pm i
  \end{gathered}
\end{equation*}
\par\bigskip
\noindent Origo är en instabil punkt och det kommer vara någon slags spiral. Detta är inte av super intresse, utan bara en notering.\par
\noindent Hade vi haft ett linjärt system hade vi haft att spiralen går mot oändligheten. Eftersom det är icke-linjärt kan vi inte säga samma sak.
\par\bigskip
\noindent Vi kommer se att lösningar som börjar nära $(0,0)$ närmar sig en gränscykel.
\par\bigskip
\noindent Byte till polära koordinater ger:
\begin{equation*}
  \begin{gathered}
    x = r\cos(\theta),\qquad y=r\sin(\theta)\\
  \end{gathered}
\end{equation*}\par
\noindent Vi vet att följande gäller:
\begin{equation*}
  \begin{gathered}
    x^2+y^2=r^2\text{, derivering med avseende på $t$ ger:}\\
  2r\dfrac{dr}{dt} = 2x\dfrac{dx}{dt}+2y\dfrac{dy}{dt}\\
  = r\dfrac{dr}{dt}=x(x+y-x(x(x^2+y^2)))+(-x+y-y(x^2+y^2))\\
  = x^2+y^2-(x^2+y^2)^2\\
  \Lrarr r^2-r^4\qquad \dfrac{dr}{dt} = r(1-r^2)\Leftarrow\text{Autonom ekvation}
  \end{gathered}
\end{equation*}
\par\bigskip
\noindent Tecknet på derivatna beror på $r(1-r^2)$.\par
\noindent Vi får om $r\in(0,1)\Rightarrow\dfrac{dr}{dt}>0\Rightarrow r$ ökar.
\noindent Om $r\in(1,\infty)\Rightarrow\dfrac{dr}{dt}<0\Rightarrow r$ minskar\par
\noindent Om $r=1\Rightarrow\dfrac{dr}{dt}=0\Rightarrow r$ konstant (motsvarar enhetscirkeln)
\par\bigskip
\noindent (Denna ODE är separabel och kan lösas explicit).
\par\bigskip
\noindent För $\theta$ får vi om vi deriverar $x=r\cos(\theta)$, $y=r\sin(\theta)$ med avseende på $t$:
\begin{equation*}
  \begin{gathered}
    (1) \quad\dfrac{dx}{dt} = \dfrac{dr}{dt}\cos(\theta)-r\sin(\theta)\dfrac{d\theta}{dt}\\\\
    (2) \quad\dfrac{dy}{dt} = \dfrac{dr}{dt}\sin(\theta)+r\cos(\theta)\dfrac{d\theta}{dt}
  \end{gathered}
\end{equation*}
\par\bigskip
\noindent Vi vill ha $\dfrac{d\theta}{dt}$.\par
\noindent Vi kan få bort $\dfrac{dr}{dt}$ om vi tar $(1)\cdot\sin(\theta)-(2)\cdot\cos(\theta)$.\par
\noindent Om vi samtidigt multiplicerar med $r$ får vi:
\begin{equation*}
  \begin{gathered}
    \underbrace{r\sin(\theta)}_{\text{$y$}}\dfrac{dx}{dt}-\underbrace{r\cos(\theta)}_{\text{$y$}}\dfrac{dy}{dt}\\
    = r\sin(\theta)\left(\dfrac{dr}{dt}\cos(\theta)-r\sin(\theta)\dfrac{d\theta}{dt}\right)-r\cos(\theta)\left(\dfrac{dr}{dt}\sin(\theta)+r\cos(\theta)\dfrac{d\theta}{dt}\right)\\
    \Lrarr -r^2\dfrac{d\theta}{dt} = \text{HL}
  \end{gathered}
\end{equation*}
\par\bigskip
\noindent För VL har vi:
\begin{equation*}
  \begin{gathered}
    y\dfrac{dx}{dt}-x\dfrac{dy}{dt} = x^2+y^2 = r^2\\
    \Lrarr r^2 = -r^2\dfrac{d\theta}{dt}\Rightarrow\dfrac{d\theta}{dt}=-1
  \end{gathered}
\end{equation*}
\par\bigskip
\noindent Vi får systemet:
\begin{equation*}
  \begin{gathered}
    \begin{rcases*}
      \dfrac{dr}{dt} = r(1-r^2)\\\\
      \dfrac{d\theta}{dt}=-1
    \end{rcases*}\\
    \Rightarrow
    \begin{cases*}
      r = \dfrac{1}{\sqrt{1+(r_0^{-2}-1)e^{-2t}}}\\
      \theta = -(t-\theta_0)
    \end{cases*}
  \end{gathered}
\end{equation*}
\par\bigskip
\noindent Om $r_0=1$ får vi $r(t) = 1$
\par\bigskip
\begin{theo}[Gränscykel]{thm:limitcykel}
  En sluten bana (ej har början eller slut) i fasplanet (vi tar endast upp 2-dim ty i 3-dim kan vi snurra runt) som uppfyller att andra banor närmar sig gränscykeln (antingen utifrån eller innifrån) när $t\to\infty$ kallas för en \textit{gränscykel}. \par
  \begin{itemize}
    \item Om alla banor nära gränscykeln närmar sig, kallas gränscykeln för \textit{stabil}
    \item Om enbart en sida närmar sig kallas den \textit{semi-stabil}
    \item Om ingen sida närmar sig kallas den \textit{instabil}, men är ej gränscykel per definition
  \end{itemize}\par\bigskip
  \noindent\textbf{OBS!} En gränscykel måste inte vara en cirkel. Utseendet på gränscykeln är kopplad till $t$. 
\end{theo}
\par\bigskip
\noindent I allmänhet är det svårt att visa att att vi ens har gränscykler. Det är i allmänhet väldigt svårt att studera (även för polynom). Vi kommer visa fall då de \textit{inte} finns gränscykler.
\par\bigskip
\noindent Betrakta följande system:
\begin{equation*}
  \begin{gathered}
    \begin{rcases*}
      x^{\prime} = P(x,y)\\
      y^{\prime} = Q(x,y)
    \end{rcases*}
  \end{gathered}
\end{equation*}
\par\bigskip

\begin{theo}
  OOm $P,Q$ har kontinuerliga deriverar i en enkelt sammanhängande område $D\subseteq\R^2$.\par
  \noindent En sluten bana \textit{måste} omsluta/omringa minst en kritisk punkt.\par
  \noindent Om den endast omsluter/omringar \textit{en} kritisk punkt, kan den punkten \textit{inte} vara en sadelpunkt. 
\end{theo}
\par\bigskip
\noindent Från Sats 24.3 följer följande:
\begin{itemize}
  \item Om ett område \textit{inte} har några kritiska punkter, så existerar det inte slutna banor i det området
  \item Om ett område endast har en kritisk punkt som är en sadelpunkt, då finns inga slutna banor i det området.
\end{itemize}
\par\bigskip
\begin{theo}
  AAntag att $P,Q$ har kontinuerliga derivator i ett enkelt sammanhängande område $D\subseteq\R^2$\par
  \noindent Om $P_x+Q_y$ har samma tecken i hela $D$ så finns inga slutna banor. 
\end{theo}
\par\bigskip
\textbf{Anmärkning:} Enkelt sammanhängande = inga hål
\par\bigskip

\noindent\textbf{Exempel:} Betrakta följande system:
\begin{equation*}
  \begin{gathered}
    \begin{rcases*}
      x^{\prime} = -2x-3y-xy^2 = P(x,y)\\
      y^{\prime} = y+x^3-x^2y = Q(x,y)
    \end{rcases*}
  \end{gathered}
\end{equation*}
\par\bigskip
\noindent Vi undrar om den har några slutna banor eller gränscykler.\par
\noindent Vi deriverar:
\begin{equation*}
  \begin{gathered}
    P_x+Q_y = (-2-y^2)+(1-x^2)=-1-(x^2+y^2)
  \end{gathered}
\end{equation*}
\par\bigskip

\noindent Vi noterar att den är negativ i hela $\R^2$\par
\noindent Det finns då inga slutna banor enligt Sats 24.4 
