\section{Homogena Linjära system med konstanta koefficienter - Forts}
\par\bigskip
\noindent Förra gången tittade vi på $X^{\prime} = AX$ där $A$ är en $n$x$n$-matris med konstanta koefficienter. Lösningen är en kolonnvektor $X = \begin{pmatrix}x_1(t)\\\vdots\\x_n(t)\end{pmatrix}$. Om $A$ har $n$ distinkta egenvärden är lösningen given på formen $X = C_1e^{\lambda_1t}K_1 + \cdots + C_ne^{\lambda_nt}K_n$
\par\bigskip
\noindent I dagens föreläsning ska vi kolla på om vi har komplexa egenvärden och eller dubbla.
\par\bigskip
\subsection{Komplexa egenvärden}\hfill\\
\par\bigskip
\noindent Vi kollar på fallet $n=2$ (dvs $2$x$2$-matriser), men det är relativt tydligt hur man generaliserar till högre dimensioner.\par
\noindent Vi har då $\lambda = \alpha+i\beta$ är ett egenvärde, då är motsvarande egenvektor komplex:
\begin{equation*}
  \begin{gathered}
    K = K_1+iK_2
  \end{gathered}
\end{equation*}\par
\noindent En lösning till $X^{\prime} = AX$ ges på samma sätt som vi gjorde förra gången:
\begin{equation*}
  \begin{gathered}
    X = e^{\lambda t}\cdot K = e^{(\alpha+i\beta)t}\left(K_1+K_2\right)
  \end{gathered}
\end{equation*}\par
\noindent Däremot vill vi oftast skriva detta på en mer reell form. Vi har gjort liknande när vi gjorde detta för 2:a ordningens ekvationer.
\par\bigskip
\noindent Exempel:\par
\noindent Lös $X^{\prime} = \underbrace{\begin{pmatrix}4&-5\\5&-4\end{pmatrix}}_{\text{$A$}}X$\par\bigskip
\noindent \textbf{Steg 1}, bestäm egenvärden:
\begin{equation*}
  \begin{gathered}
    det(A-\lambda I) = \lambda^2+9 = 0 \Lrarr \lambda=\pm3i
  \end{gathered}
\end{equation*}
\par\bigskip
\noindent\textbf{Steg 2}, hitta egenvektor (vi behöver bara räkan ut 1, ty även deras egenvektorer är konjugat).\par
\noindent För $\lambda_1=3i$ får vi:
\begin{equation*}
  \begin{gathered}
    \begin{pmatrix}4-3i&-5\\5&-4-3i\end{pmatrix}\begin{pmatrix}k_1\\k_2\end{pmatrix}=0
  \end{gathered}
\end{equation*}\par
\noindent Notera att raderna är linjärt beroende, så det räcker med att kolla på en rad. Vi kikar på den övre:
\begin{equation*}
  \begin{gathered}
    (4-3i)k_1-5k_2 = 0
  \end{gathered}
\end{equation*}\par
\noindent Denna löses genom $k_1 = 5$ och $k_2 = 4-3i$.\par
\noindent Vår första egenvektor är då:
\begin{equation*}
  \begin{gathered}
    K_1 = \begin{pmatrix}5\\4-3i\end{pmatrix} = \begin{pmatrix}5\\4\end{pmatrix}+\begin{pmatrix}0\\-3\end{pmatrix}i
  \end{gathered}\\
  K_2 = \begin{pmatrix}5\\4\end{pmatrix}+\begin{pmatrix}0\\3\end{pmatrix}i
\end{equation*}\par
\noindent Vi har då:
\begin{equation*}
  \begin{gathered}
    X_1 = e^{\lambda_1t}K_1 = e^{3it}\left(\begin{pmatrix}5\\4\end{pmatrix}+\begin{pmatrix}0\\-3\end{pmatrix}i\right) = (\cos(3t)+i\sin(3t))\left(\begin{pmatrix}5\\4\end{pmatrix}+\begin{pmatrix}0\\-3\end{pmatrix}i\right)\\
    \Lrarr\begin{pmatrix}5\cos(3t)\\4\cos(3t)+3\sin(3t)\end{pmatrix}+\begin{pmatrix}5\sin(3t)\\-3\cos(3t)+4\sin(3t)\end{pmatrix}i\\\\
    X_2 = \cdots = \begin{pmatrix}5\cos(3t)\\4\cos(3t)+3\sin(3t)\end{pmatrix}-\begin{pmatrix}5\sin(3t)\\-3\cos(3t)+4\sin(3t)\end{pmatrix}i
  \end{gathered}
\end{equation*}\par
\noindent Om vi tittar på de reella lösningarna (vilket vi kan få från linjärkombinationer):
\begin{equation*}
  \begin{gathered}
    \dfrac{1}{2}X_1+\dfrac{1}{2}X_2 = \begin{pmatrix}5\cos(3t)\\4\cos(3t)+3\sin(3t)\end{pmatrix}\\
    \dfrac{1}{2i}X_1-\dfrac{1}{2i}X_2 = \begin{pmatrix}5\sin(3t)\\-3\cos(t)+4\sin(3t)\end{pmatrix}
  \end{gathered}
\end{equation*}\par
\noindent Men skalärer är $\in\R$? De flesta saker fungerar även i $\C$, detta även här. Inga konstigheter alltså!
\par\bigskip
\noindent Nu har vi löst ekvationen, men m.h.a fasporträtt får vi en "känsla" av hur lösningen är och beter sig. Detta kan vi göra genom att sätta $C_1 = 0$ och $C_2=0$ där $C_i$ är konstanten i linjärkombinationen. Vi har då i första fallet:
\begin{equation*}
  \begin{gathered}
    X = \begin{pmatrix}5\cos(3t)\\4\cos(3t)-3\sin(3t)\end{pmatrix}
  \end{gathered}
\end{equation*}\par
\noindent Notera här att lösningen blir periodiskt med period $\dfrac{2p}{3}$ (detta kommer från att vi har argumentet $3t$). Vi kan testa stoppa in några värden på $t$ och plotta:
\par\bigskip
\begin{center}
  \begin{tabular}{c|c}
    $t$&$=$\\
    \hline
    0&$\begin{pmatrix}5\\4\end{pmatrix}$\\
    \hline
    $\dfrac{\pi}{6}$&$\begin{pmatrix}0\\3\end{pmatrix}$\\
    \hline
    $\dfrac{\pi}{3}$&$\begin{pmatrix}-5\\-4\end{pmatrix}$\\
    \hline
    $\dfrac{\pi}{2}$&$\begin{pmatrix}0\\-3\end{pmatrix}$
  \end{tabular}
\end{center}\par
\noindent Om läsaren testar att plotta detta så kommer en ellips uppenbara sig. Generellt, om egenvärden är rent imaginära får vi slutna lösningar (periodisk) som roterar runt origo.
\par
\noindent Fasporträttet kallas för \textit{stabilt}
\par\bigskip
\noindent Om $\lambda = \alpha+i\beta$ med $\alpha<0$ får vi spiraler som rör sig mot origo. Kallas för stabil spiral men även för \textit{asymptotiskt stabil}\par
\noindent Däremot, om $\alpha>0$ rör sig spiralerna utåt. Denna är en instabil spiral ty vi rör oss bort från origo.
\par\bigskip
\subsection{Dubbla egenvärden}\hfill\\
\par\bigskip
\noindent Det som skiljer sig här från tidigare är att vi har algebraiskmultiplicitet, men även geometrisk multiplicitet. Återigen kommer vi kolla på fallet $n=2$. Det som händer i högre dimensioner är liknande men vi kan få lite problem.
\par\bigskip
\noindent Antag $X^{\prime} = AX$ där $A$ har bara ett egenvärde $\lambda$. Problemet här är att vi behöver 2 lösningar men har barar 1 egenvärde. Men! Det visar sig att det går att lösa om vi kikar på den geometriska multipliciteten. Vi delar in i fall:
\par\bigskip
\noindent\textbf{Fall 1}: Geometrisk multiplicitet 2 (hur många tillhörande egenvektorer som finns)\par
\noindent $(A-\lambda I)K=0$ har 2 linjärt oberoende lösningar. Då är det inga problemis, ty vi får vårt $K_1$ och $K_2$. Vi får:
\begin{equation*}
  \begin{gathered}
    X_1 = e^{\lambda t}K_1,\qquad X_2 = e^{\lambda t}K_2
  \end{gathered}
\end{equation*}
\par\bigskip
\noindent Här händer inget särskillt. Det enda som skiljer sig lite är att vanligtvis rör sig fasporträtt till den största egenvektorn, men här är de lika stora.
\par\bigskip
\noindent\textbf{Fall 2}: Geometrisk multiplicitet 1:\par
\noindent $(A-\lambda I)K=0$ har bara 1 lösning $K_1$. Vi får $X_1 = e^{\lambda t}K_1$, men vi måste hitta $K_2$. Idén är lik den med 2:a ordningens, vi testar:
\begin{equation*}
  \begin{gathered}
    X_2 = e^{\lambda t}tK_1
  \end{gathered}
\end{equation*}\par
\noindent Detta kommer inte funka, men det vi kommer göra är:
\begin{equation*}
  \begin{gathered}
    X_2 = e^{\lambda t}tK_1+e^{\lambda t}p
  \end{gathered}
\end{equation*}\par
\noindent Detta kommer däremot inte funka i det endimensionella fallet. Vi tar det som ansats och stoppar in i $X^{\prime} = AX$. Vi får:
\begin{equation*}
  \begin{gathered}
    K_1\lambda te^{\lambda t}+K_1e^{\lambda t}+p\lambda e^{\lambda t}+Ape^{\lambda t}\\
    \Lrarr te^{\lambda t}\underbrace{\left(\lambda K_1-AK_1\right)}_{\text{=0}}+e^{\lambda t}\underbrace{\left(K_1+\lambda p-AP\right)}_{\text{=0}}=0
  \end{gathered}
\end{equation*}\par
\noindent Det enda sättet den är lika med noll är om det som står i måsvingarna är noll, annars har vi linjärt beroende.
\par\bigskip

\begin{equation*}
  \begin{gathered}
    \left(\lambda K_1-AK_1\right) = (A-\lambda I)K_1 = 0\\
    \left(K_1+\lambda p-AP\right) = (A-\lambda I)p=K_1
  \end{gathered}
\end{equation*}\par
\noindent Notera att den första ekvationen är uppfyllt eftersom $K_1$ är en egenvektor med egenvärde $\lambda$. Det enda vi behöver göra är alltså att hitta $p$ som löser den andra ekvationen. Man kommer alltid kunna hitta det här $p$:et givet denna situation.\par
\noindent Den generella lösningen är:
\begin{equation*}
  \begin{gathered}
    X = C_1e^{\lambda t}K_1 + C_2e^{\lambda t}(tK_1+p)
  \end{gathered}
\end{equation*}
\par\bigskip
\noindent Exempel:\par
\noindent Lös $X^{\prime} = \begin{pmatrix}-1&3\\-3&5\end{pmatrix}X$
\par\bigskip
\noindent\textbf{Steg 1}, egenvärden!:

\begin{equation*}
  \begin{gathered}
    det(A-\lambda I) = (\lambda-2)^2 \Rightarrow\lambda_1=\lambda_2=2
  \end{gathered}
\end{equation*}\par
\noindent Egenvektorer är:
\begin{equation*}
  \begin{gathered}
    \begin{pmatrix}-3&3\\-3&3\end{pmatrix}\begin{pmatrix}k_1\\k_2\end{pmatrix}=\begin{pmatrix}0\\0\end{pmatrix}
  \end{gathered}
\end{equation*}\par
\noindent Notera att den övre och undre är indentisk, vi får $k_1 = k_2$. Vi får 1 egenvektor $K=\begin{pmatrix}1\\1\end{pmatrix}$ :(. Det som kan hända att man får 2 linjärt oberoende lösningar (om man har geometrisk multiplicitet > 1) men det fick vi inte i detta fall.\par
\noindent Vi har en lösning $X_1 = e^{2t}\begin{pmatrix}1\\1\end{pmatrix}$
\par\bigskip
\noindent För $X_2$ söker vi $p$ så att $(A-\lambda I)p=K$:
\begin{equation*}
  \begin{gathered}
    \begin{pmatrix}-3&3\\-3&3\end{pmatrix}\begin{pmatrix}p_1\\p_2\end{pmatrix} = \begin{pmatrix}1\\1\end{pmatrix}
  \end{gathered}
\end{equation*}\par
\noindent Vi kan ta $p_1 = 0$ och $p_2 = \dfrac{1}{3}\Rightarrow p=\begin{pmatrix}0\\\dfrac{1}{3}\end{pmatrix}$. Även här finns det många lösningar till denna ekvation och det spelar ingen roll vilken man tar. Det vi \textit{inte} kan göra är att multiplicera med en konstant. Detta ger oss:
\begin{equation*}
  \begin{gathered}
    X_2 = e^{2t}\left(t\begin{pmatrix}1\\1\end{pmatrix}+\begin{pmatrix}0\\\dfrac{1}{3}\end{pmatrix}\right) = e^{2t}\begin{pmatrix}t\\t+\dfrac{1}{3}\end{pmatrix}
  \end{gathered}
\end{equation*}
\par\bigskip
\noindent Den generella lösningen ges av:
\begin{equation*}
  \begin{gathered}
    X = C_1X_1+C_2X_2 = C_1e^{2t}\begin{pmatrix}1\\1\end{pmatrix}+C_2e^{2t}\begin{pmatrix}t\\t+\dfrac{1}{3}\end{pmatrix}
  \end{gathered}
\end{equation*}
\par\bigskip
\noindent Kuriosa, om spiralen motsvarar gyllene snittet/avståndet mellan spiralen är gyllene snittet, vad säger det om ODE:en?
