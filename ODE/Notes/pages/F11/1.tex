\section{Homogena linjära system med konstanta koefficienter}
\par\bigskip
\noindent Från tidigare hade vi:
\begin{equation*}
  \begin{gathered}
    X^{\prime} = \begin{pmatrix}1&3\\5&3\end{pmatrix}X \Rightarrow X = C_1X_1+C_2X_2\\
    X_1 = \begin{pmatrix}1\\-1\end{pmatrix}e^{-2t}, \qquad X_2 = \begin{pmatrix}3\\5\end{pmatrix}e^{6t}
  \end{gathered}
\end{equation*}
\par\bigskip
\noindent Notera här att både $X_1$ och $X_2$ är på formen:
\begin{equation*}
  \begin{gathered}
    \begin{pmatrix}k_1\\k_2\end{pmatrix}e^{\lambda t}
  \end{gathered}
\end{equation*}
\par\bigskip
\noindent Då kan vi ställa oss frågan, gäller detta generellt? JA!!
\par\bigskip
\noindent Låt $A$ vara en $n$x$n$ matris med konstanta koefficienter. Vi studerar $X^{\prime} = AX$.\par
\noindent Ansätt $X = Ke^{\lambda t}$ (där $K$ kolonnvektor). Vi har då:

\begin{equation*}
  \begin{gathered}
    X^{\prime} = \lambda\cdot K e^{\lambda t}\\
    \text{Insättning ger: } \lambda K e^{\lambda t} = AKe^{\lambda t}\\
    \Lrarr\lambda K = AK
  \end{gathered}
\end{equation*}
\par\bigskip
\noindent Det vill säga, $K$ är en egenvektor till $A$ med egenvärde $\lambda$.
\par\bigskip
\subsection{Repetition - Hitta egenvärden och egenvektorer}\hfill\\
\par\bigskip
\begin{itemize}
  \item Skriv $AK=\lambda K$ som ($A-\lambda I$)$K=0$. Vi vill att determinanten till $(A-\lambda I)$ ska vara 0, ty då vet vi att $(A-\lambda I)$ inte är 0 och vi kan då hitta $K$
  \item Lös polynomekvationen $det(A-\lambda I)=0$. Ekvation av grad $n$, $n$ rötter ($\lambda_1,\cdots, \lambda_n$) om vi räknar multipliciteten och tar med komplexa rötter
  \item Sätt in $\lambda_i$ i $(A-\lambda I)K=0$. En nollskilld lösning till detta system ger en egenvektor.
\end{itemize}
\par\bigskip
\noindent I denna (och nästa) föreläsning kommer vi diskutera vad som händer i tre fall:
\begin{itemize}
  \item rella distinkta rötter
  \item Rötter med multiplicitet > 1 (repeterande rötter)
  \item Komplexa rötter
\end{itemize}
\par\bigskip
\noindent Vi kommer även studera beteendet hos lösningen genom att betrakta dess fasporträtt.
\par\bigskip
\subsection{Fall 1 - Reella distinkta rötter}\hfill\\
\par\bigskip
\noindent Det här är, som tidigare, det enkla fallet. Vi behöver inte göra så många busigheter.
\par\bigskip
\begin{theo}
  BBetrakta ett system $X^{\prime} = AX$ där $A$ är en $n$x$n$ matris med konstanta koefficienter. Om $A$ har $n$ distinkta reella egenvärden $\lambda_1,\cdots\lambda_n$ med egenvektor $K_1,\cdots, K_n$, då är den generella lösningen till ekvationen:
  \begin{equation*}
    \begin{gathered}
      X = C_1K_2e^{\lambda_1 t}+\cdots + C_nK_ne^{\lambda_n t}
    \end{gathered}
  \end{equation*}
\end{theo}
\par\bigskip
\noindent Exempel:\par
\begin{equation*}
  \begin{gathered}
    \begin{cases*}
      \dfrac{dx}{dt} = -4x+y+z\\
      \dfrac{dy}{dt}=x+5y-z\\
      \dfrac{dz}{dt}=y-3x
    \end{cases*}
  \end{gathered}
\end{equation*}
\par\bigskip
\noindent Första steget kommer vara att skriva om detta på matrisform för att sedan kunna skriva det på formen $X^{\prime} = AX$ har vi:

\begin{equation*}
  \begin{gathered}
    A = \begin{pmatrix}-4&1&1\\1&5&-1\\0&1&-3\end{pmatrix},\qquad X = \begin{pmatrix}x\\y\\z\end{pmatrix}
  \end{gathered}
\end{equation*}
\par\bigskip
\noindent Vi hittar egenvärdena genom att kolla på determinanten av $(A-\lambda I)=0$:
\begin{equation*}
  \begin{gathered}
    \begin{vmatrix}-4-\lambda&1&1\\1&5-\lambda&-1\\0&1&-3-\lambda\end{vmatrix} = 0\\
    \text{Kofaktorutveckla längst sista raden eller Sarrus ger: }\\
    -(4+\lambda-1)-(3+\lambda)(-(4+\lambda)(5-\lambda)-1)= -(\lambda+3)+(\lambda+3)((4+\lambda)(5-\lambda)+1)\\
    (\lambda+3)(4+\lambda)(5-\lambda)\\
    \Lrarr \lambda_1 = -3,\qquad\lambda_2 = -4,\qquad\lambda_3 = 5
  \end{gathered}
\end{equation*}
\par\bigskip
\noindent Nu gäller det att bestämma egenvektorerna:
\begin{equation*}
  \begin{gathered}
    \lambda_1 = -3 \Rightarrow (A-\lambda_1 I) = \begin{pmatrix}-1&1&1\\1&8&-1\\0&1&0\end{pmatrix}
  \end{gathered}
\end{equation*}
\par\bigskip
\noindent Vi vill lösa:
\begin{equation*}
  \begin{gathered}
    \begin{pmatrix}-1&1&1\\1&8&-1\\0&1&0\end{pmatrix}\begin{pmatrix}k_1\\k_2\\k_3\end{pmatrix}
  \end{gathered}
\end{equation*}
\par\bigskip
\noindent Vi gör Gauzz med den:
\begin{equation*}
  \begin{gathered}
    \begin{pmatrix}-1&1&1\\1&8&-1\\0&1&0\end{pmatrix} = 0\\
    \begin{rcases*}
      k_1 = k_3\\
      k_2 = 0
    \end{rcases*}\Rightarrow\text{Vi kan ta $k_1 = 1$: }\\
    K_1 = \begin{pmatrix}1\\0\\1\end{pmatrix}\Rightarrow K_1e^{-3t}
  \end{gathered}
\end{equation*}
\par\bigskip
\noindent $\lambda=-4$:
\begin{equation*}
  \begin{gathered}
    A-\lambda_2I = \begin{pmatrix}0&1&1\\1&9&-1\\0&1&1\end{pmatrix}\\
    \begin{rcases*}
      k_2+k_3 = 0\\
      k_1-10k_3 = 0
    \end{rcases*}\text{Väljer vi $k_3=1$ får vi heltalsgrejer, najs}:\\
    k_3 = 1,\qquad k_1=10,\qquad k_2=-1\\
    K_2 = \begin{pmatrix}10\\-1\\1\end{pmatrix}\Rightarrow X_2 = K_2e^{-4t}
  \end{gathered}
\end{equation*}
\par\bigskip
\noindent För $\lambda = 5$ får vi $K_3=\begin{pmatrix}1\\8\\1\end{pmatrix}$
\par\bigskip
\noindent Generella lösningen ges av linjärkombination av alla:
\begin{equation*}
  \begin{gathered}
    X = C_1\begin{pmatrix}1\\0\\1\end{pmatrix}e^{-3t}+C_2\begin{pmatrix}10\\-1\\1\end{pmatrix}e^{-4t}+C_3\begin{pmatrix}1\\8\\1\end{pmatrix}e^{5t}
  \end{gathered}
\end{equation*}
\par\bigskip
\noindent Vi vill nu lägga grunden för fasporträtt. Vi gör detta genom ett exempel:
\par\bigskip
\noindent Rörelsen för en partikel beskrivs av ODE:en
\begin{equation*}
  \begin{gathered}
    X^{\prime} = AX \text{ där } X= \begin{pmatrix}x\\y\end{pmatrix}\text{ och $A$ uppfyller}:\\
    A\begin{pmatrix}1\\2\end{pmatrix} = \begin{pmatrix}3\\6\end{pmatrix} \text{ och } A\begin{pmatrix}1\\-2\end{pmatrix} = \begin{pmatrix}-1\\2\end{pmatrix}
  \end{gathered}
\end{equation*}
\par\bigskip
\noindent Hitta den generella lösningen och beskriv vad som händer med en partikel som vid $t=2$ är $\begin{pmatrix}2\\2\end{pmatrix}$ när $t\to\infty$.
\par\bigskip
\noindent Vad vi noterar är att det enda som händer vid matrismultiplikationen är att i första fallet så får vi $3\cdot\begin{pmatrix}1\\2\end{pmatrix}$ och andra fallet $-1\cdot\begin{pmatrix}1\\-2\end{pmatrix}$ så vi kan enkelt läsa ut vilka våra egenvärden och egenvektorer är:
\begin{equation*}
  \begin{gathered}
    \lambda_1 = 3\text{ dess egenvektor är } \begin{pmatrix}1\\2\end{pmatrix}\\
    \lambda_2 = -1\text{ dess egenvektor är } \begin{pmatrix}1\\-2\end{pmatrix}
  \end{gathered}
\end{equation*}
\par\bigskip
\noindent Då har vi allt för att hitta lösningen:
\begin{equation*}
  \begin{gathered}
    X = C_!\begin{pmatrix}1\\2\end{pmatrix}e^{3t}+C_2\begin{pmatrix}1\\-2\end{pmatrix}e^{-t}
  \end{gathered}
\end{equation*}
\par\bigskip
\noindent Nu skall vi svara på frågan. Vi söker $C_1, C_2$ så att $X(2)=\begin{pmatrix}2\\2\end{pmatrix}$:
\begin{equation*}
  \begin{gathered}
    C_1\begin{pmatrix}1\\2\end{pmatrix}e^6 + C_2\begin{pmatrix}1\\-2\end{pmatrix}e^{-2} = \begin{pmatrix}2\\2\end{pmatrix}\Lrarr\begin{pmatrix}e^6&e^{-2}\\2e^{6}&-2e^{-25}\end{pmatrix}\begin{pmatrix}C_1\\C_2\end{pmatrix} = \begin{pmatrix}2\\2\end{pmatrix}\\
    \Lrarr C_1 = 
  \end{gathered}
\end{equation*}
\par\bigskip
\noindent Vi får då:
\begin{equation*}
  \begin{gathered}
    \begin{rcases*}
      C_1 = \dfrac{3}{2}e^{-6}\\
      C_2 = \dfrac{1}{2}e^2
    \end{rcases*}
  \end{gathered}
\end{equation*}
\par\bigskip
\noindent Lösningen är då:
\begin{equation*}
  \begin{gathered}
    X(t)=e^{3t}\begin{pmatrix}\dfrac{3}{2}e^{-6}\\3e^{-6}\end{pmatrix}+\dfrac{1}{2}e^{-t}\begin{pmatrix}\dfrac{1}{2}e^{2}\\-e^2\end{pmatrix}
  \end{gathered}
\end{equation*}
\par\bigskip
\noindent Vad händer när $t\to\infty$?:
\begin{equation*}
  \begin{gathered}
    e^{3t}\to\infty\text{ när }t\to\infty\\
    e^{-t}\to0\text{ när }t\to\infty
  \end{gathered}
\end{equation*}
\par\bigskip
\noindent Det vi kan se är att partikeln går mot $\infty$ i riktningen som ges av $\begin{pmatrix}\dfrac{3}{2}e^{-6}\\3e^{-6}\end{pmatrix}$, eftersom vi bara bryr oss om riktningen kan vi ta bort $e^{-6}$, dvs $\begin{pmatrix}\dfrac{3}{2}\\3\end{pmatrix}$
\par\bigskip
\subsection{Fasporträtt}\hfill\\
\par\bigskip
\noindent Den generella lösningen till $X^{\prime} = AX$ går att beskriva grafiskt. Exempelvis för lösningen från tidigare exemplet har vi:
\begin{equation*}
  \begin{gathered}
    \begin{pmatrix}x(t)\\y(t)\end{pmatrix} = \begin{pmatrix}C_1e^{3t}+C_2e^{-t}\\2C_1e^{3t}-2C_2e^{-t}\end{pmatrix}
  \end{gathered}
\end{equation*}
\par\bigskip
\noindent Denna ritar upp en kurva i planet. För varje val av $C_1$ och $C_2$ får vi en kurva. Dessa kurvor har det formella namnet \textit{banor}. I detta fall kommer banorna befinna sig i $\R^2$, vi kallar $\R^2$ för \textit{fasplanet}. Anledningen till varför det kallas fasplan är att vi kommer se att beroende på var partiekln börjar kommer vi få olika faser.
\par\bigskip
\noindent Fasporträttet ges av att rita ut ett par banor i fasplanet. Man kan inte välja dessa banor hur som helst utan måste göra det representativt.\par
\noindent Hur kan vi systematiskt rita ut fasporträtt? Om vi börjar att kolla på lättare fall, säg $C_2=0$, får vi $X = C_1\begin{pmatrix}1\\2\end{pmatrix}e^{3t}$. Hur beter sig denna när den går mot $\pm\infty$?\par
\noindent När $t\to-\infty$ går denna mot origo. När $t\to\infty$ får vi olika fall beroende på om $C_1$ är positivt eller negativt. Om $C_1$ är positivt går denna i riktningen $\begin{pmatrix}1\\2\end{pmatrix}$, om $C_1$ är negativ går den i riktningen $-\begin{pmatrix}1\\2\end{pmatrix}$:

\begin{figure}[ht]
    \centering
    \incfig{fasporträtt}
    \caption{Fasporträtt}
    \label{fig:fasporträtt}
\end{figure}
\par\bigskip
\noindent Om $C_1 = 0$ får vi $X = C_2\begin{pmatrix}1\\-2\end{pmatrix}e^{-t}$. När $t\to+\infty$ går denna mot origo, när $t\to-\infty$ får viockså 2 olika fall beroende på om $C_2$ är positiv eller negativ:

\begin{figure}[ht]
    \centering
    \incfig{fasporträtt-2}
    \caption{Fasporträtt 2}
    \label{fig:fasporträtt-2}
\end{figure}
\par\bigskip
\noindent Om $C_1\neq0\neq C_2$ får vi:
\begin{equation*}
  \begin{gathered}
    X = C_1\begin{pmatrix}1\\2\end{pmatrix}e^{3t}+C_2\begin{pmatrix}1\\-2\end{pmatrix}e^{-t}
  \end{gathered}
\end{equation*}
\par\bigskip
\noindent När $t\to+\infty$ närmar sig $X$ $C_1\begin{pmatrix}1\\2\end{pmatrix}e^{3t}$ och när $t\to-\infty$ $C_2\begin{pmatrix}1\\-2\end{pmatrix}$. "Kombinerar" vi graferna får vi:

\begin{figure}[ht]
    \centering
    \incfig{fasporträtt3}
    \caption{Fasporträtt3}
    \label{fig:fasporträtt3}
\end{figure}
