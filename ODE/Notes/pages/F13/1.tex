\section{Icke-homogena linjära system med konstanta koefficienter}
\par\bigskip
\noindent Idag kommer vi prata om system av första ordningens ODE:er, likt det vi har gjort de tidigare föreläsningar, men vi kommer betrakta det inhomogena fallet, dvs $X^{\prime} = AX+F$.\par
\noindent Lösningen är väldigt likt det vi gjorde för andra ordningens ODE:er. Den generella lösningen ges på formen:
\begin{equation*}
  \begin{gathered}
    X = X_h+X_p
  \end{gathered}
\end{equation*}\par
\noindent Där $X_h$ är lösningen till den associerade homogena ekvationen, och $X_p$ är en partikulärlösning. Man kan säga att förra föreläsningarna har gått igenom hur vi hittar $X_h$ om $A$ endast har konstanta koefficienter.
\par\bigskip
\noindent\textbf{Hur hittar vi $X_p$?}
\par\bigskip
\noindent Metoden är i princip samma som den med andra ordningens ODE. Det finns ett par metoder att betrakta:
\begin{itemize}
  \item Obestämda koefficienter, "gissa en lösning och sätt in"
  \item Variation-av-parameter metoden
    \begin{itemize}
      \item Här kommer vi få en "fundamentalmatris"
    \end{itemize}
\end{itemize}
\par\bigskip
\subsection{Metoden med obestämda koefficienter}\hfill\\
\par\bigskip
Vi kör ett introducerande exempel, som vanligt. Idén går ut på att man gissar en lösning (systemet måste vara enkelt, annars kan vi oftast inte göra en bra gissning).\par
\noindent Det krävs 2 saker för att göra en gissning:
\begin{itemize}
  \item $A$ har konstanta koefficienter
  \item $F$ är "lätt":
    \begin{itemize}
      \item Polynom
      \item Exponentialfunktioner
      \item Trigonometriska funktioner
      \item Produkter av ovanstående
    \end{itemize}
\end{itemize}
\par\bigskip
\noindent Exempel:\par
\noindent Betrakta:
\begin{equation*}
  \begin{gathered}
    X^{\prime} = \begin{pmatrix}1&1\\4&1\end{pmatrix}X+\begin{pmatrix}2\\-1\end{pmatrix}e^t
  \end{gathered}
\end{equation*}\par
\noindent Här är $F$ "lätt". Det associerade homogena systemet ges av:
\begin{equation*}
  \begin{gathered}
    X^{\prime} = \begin{pmatrix}1&1\\4&1\end{pmatrix}\\
    X_h = C_1e^{3t}\begin{pmatrix}1\\2\end{pmatrix}+C_2e^{-t}\begin{pmatrix}1\\2\end{pmatrix}
  \end{gathered}
\end{equation*}
\par\bigskip
\noindent Nästa steg är att hitta en partikulärlösning.\par
\noindent Vi har $F(t) = \begin{pmatrix}2\\-1\end{pmatrix}e^{t}$, en rimlig gissning är en gissning på samma form, dvs $X_p = \begin{pmatrix}a\\b\end{pmatrix}e^t$\par
\noindent Vi stoppar in och räknar:
\begin{equation*}
  \begin{gathered}
    \begin{pmatrix}a\\b\end{pmatrix}e^t=\begin{pmatrix}1&1\\4&1\end{pmatrix}\begin{pmatrix}a\\b\end{pmatrix}e^t+\begin{pmatrix}2\\-1\end{pmatrix}e^t\\
    \Lrarr \begin{pmatrix}a\\b\end{pmatrix}=\begin{pmatrix}1&1\\4&1\end{pmatrix}\begin{pmatrix}a\\b\end{pmatrix}+\begin{pmatrix}2\\-1\end{pmatrix}\\
    \Lrarr \begin{pmatrix}a\\b\end{pmatrix} = \begin{pmatrix}a+b+2\\4a+b-1\end{pmatrix}\\
    \Lrarr \begin{pmatrix}0\\0\end{pmatrix} = \begin{pmatrix}b+2\\4a-1\end{pmatrix}\\
    a = \dfrac{1}{4}\qquad b=-2\\
    \Lrarr X_p = \begin{pmatrix}1/4\\-2\end{pmatrix}e^t
  \end{gathered}
\end{equation*}\par
\noindent Generella lösningen ges då av:
\begin{equation*}
  \begin{gathered}
    X = X_h+X_p = C_1e^{3t}\begin{pmatrix}1\\2\end{pmatrix}+C_2e^{-t}\begin{pmatrix}1\\-2\end{pmatrix}+e^t\begin{pmatrix}1/4\\-2\end{pmatrix}
  \end{gathered}
\end{equation*}
\par\bigskip
\noindent Detta var ett väldigt snällt exempel, men idén är smma sak i komplicerade fall men man får räkna lite mer.
\par\bigskip
\noindent Precis som tidigare så finns det en slags "mall" över lämpliga gissningar:
\par\bigskip
\noindent Om vi har $F(t) = \begin{pmatrix}6t\\-10t+4\end{pmatrix}$ ser vi att vi kan skriva om det som $\begin{pmatrix}6\\-10\end{pmatrix}t+\begin{pmatrix}0\\4\end{pmatrix}$\par
\noindent Då blir en rimlig gissning $\begin{pmatrix}a_1\\b_2\end{pmatrix}t+\begin{pmatrix}a_2\\b_2\end{pmatrix}$
\par\bigskip
\noindent Om vi har $F(t) = \begin{pmatrix}-1e^{-t}+1\\e^{-t}-5t+7\end{pmatrix}$ kan vi, som tidigare, dela upp det i $\begin{pmatrix}-2\\1\end{pmatrix}e^{-t}+\begin{pmatrix}0\\-5\end{pmatrix}t+\begin{pmatrix}1\\7\end{pmatrix}$\par
\noindent Då blir en rimlig gissning $\begin{pmatrix}a_1\\b_1\end{pmatrix}e^{-t}+\begin{pmatrix}a_2\\b_2\end{pmatrix}t+\begin{pmatrix}a_3\\b_3\end{pmatrix}$
\par\bigskip
\noindent Notera, man kan även fördela räkningen av partikulärlösning genom att först räkna partikulärlösningen för första termen ($\begin{pmatrix}a_1\\b_1\end{pmatrix}e^{-t}$) och sedan andra och sedan summera dem.\par
\noindent Det blir snabbt mycket räkningar!
\par\bigskip
\noindent Vi får snabbt problem om gissningen löser det homogena systemet. I exmeplet, om det hade stått $F = \begin{pmatrix}2\\-1\end{pmatrix}e^{3t}$ så hade det inte funkat. Tidigare har vi kunnat multiplicera $x$ eller logaritmer men det funkar inte riktigt här utan vi får göra lite mer komplicerade saker som vi inte kommer gå igenom. Det kan funka i vissa fall, men inte i alllmänhet.
\par\bigskip
\subsection{Variation av parameter metoden}\hfill\\
\par\bigskip
\noindent För att kunna beskriva denna metod behöver vi ett nytt koncept, \textit{fundamentalmatris}:
\par\bigskip
\begin{theo}[Fundamentalmatrisen]{thm:fundmat}
  Låt $X_1,\cdots,X_n$ vra en fundamental lösningsmängd till $X^{\prime} = AX$\par på intervallet $I$
  \noindent Matrisen:
  \begin{equation*}
    \begin{gathered}
      \Phi = \begin{pmatrix}\vdots&\vdots&\vdots\\X_1&\hdots&X_n\\\vdots&\vdots&\vdots\end{pmatrix}
    \end{gathered}
  \end{equation*}\par
  \noindent kallas för \textit{fundamentalmatrisen} till systemet. Notera att detta är en $n$x$n$-matris
\end{theo}
\par\bigskip
\noindent Denna matris har behändiga egenskaper, såsom:
\begin{itemize}
  \item Eftersom $X_i$ löser $X_i^{\prime} = AX_i\qquad\forall i$ har vi även $\Phi^{\prime} = A\Phi$
  \item Eftersom $X_1,\cdots,X_n$ är linjärt oberoende följer det att $det(\Phi)\neq0$ gäller för alla $t\in I$, matrisen är alltså inverterbar
    \begin{itemize}
      \item Notera: determinanten är även känd som Wronskianen
    \end{itemize}
\end{itemize}
\par\bigskip
\noindent Alla lösningar till $X^{\prime} = AX$ är på formen:
\begin{equation*}
  \begin{gathered}
    X = C_1X_1+\cdots+C_nX_n \text{ (linjärkombination av lösnignar)}
  \end{gathered}
\end{equation*}\par
\noindent Denna linjärkombination går att skriva på ett smidigt sätt med hjälp av denna fundamentalmatris:
\begin{equation*}
  \begin{gathered}
    X = \Phi\cdot C\qquad C= \begin{pmatrix}C_1\\\vdots\\C_n\end{pmatrix}
  \end{gathered}
\end{equation*}
\par\bigskip
\noindent Vi provar $X_p = \Phi(t)\cdot U(t)$ där $U=\begin{pmatrix}u_1(t)\\\vdots\\u_n(t)\end{pmatrix}$ istället för $y_p = u_1y_1+u_2y_2$. Nu är frågan om vi kan stoppa in det i systemet och finna $U$ så att vi kan finna en explicit lösning.\par
\noindent Vi stoppar in i $X^{\prime} = AX$:
\begin{equation*}
  \begin{gathered}
    \Phi^{\prime}U+\Phi U^{\prime} = A\Phi U + F
  \end{gathered}
\end{equation*}
\par\bigskip
\noindent Vi kan använda från tidigare att $\Phi$ är en lösning till det homogena systemet:
\begin{equation*}
  \begin{gathered}
    A\Phi U +\Phi U^{\prime} = A\Phi U+F
  \end{gathered}
\end{equation*}\par
\noindent Vi vet att $\Phi^{-1}$ finns eftersom kolonnerna är linjärt oberoende, då kan vi göra:
\begin{equation*}
  \begin{gathered}
    \Phi U^{\prime} = F \Rightarrow \Phi^{-1}\Phi U^{\prime} = \Phi^{-1}F\\
    U^{\prime} = \Phi^{-1}F\Rightarrow U = \int\Phi^{-1}F dt \text{ (integralen ges av att elementvis integrera)}
  \end{gathered}
\end{equation*}
\par\bigskip
\noindent Om vi väljer $U$ till detta, dvs:
\begin{equation*}
  \begin{gathered}
    X_p = \Phi\int\Phi^{-1}F dt
  \end{gathered}
\end{equation*}\par
\noindent Så löser $X_p$ det inhomogena systemet $X^{\prime} = AX+F$
\par\bigskip
\noindent Exempel:\par
\noindent Betrakta: $X^{\prime} = \begin{pmatrix}0&-1\\1&0\end{pmatrix}X+\begin{pmatrix}1/\cos(t)\\0\end{pmatrix}$
\par\bigskip
\noindent $A$ har konstanta koefficienter men $F$ är inte på "enkel" form, vi kan inte gissa oss fram så det är ett ypperligt tillfälle att använda det vi precis lärt oss!
\par\bigskip
\noindent\textbf{Steg 1:} \textit{Lös det associerade homogena systemet}:
\par\bigskip

\begin{equation*}
  \begin{gathered}
    A = \begin{pmatrix}0&-1\\1&0\end{pmatrix} \Rightarrow \lambda = \pm i \text{ (egenvärden)}\\
  \end{gathered}
\end{equation*}\par
\noindent Vi får att en egenvektor till $\lambda_1 = i$ är $\begin{pmatrix}i\\1\end{pmatrix}$, för $\lambda_2 = -i$ har vi då $\begin{pmatrix}-i\\1\end{pmatrix}$ (ges av konjugatet).\par
\noindent Den allmänna/generella lösningen (om vi skriver upp det i termer av dessa 2) blir:
\begin{equation*}
  \begin{gathered}
    C_1e^{it}\begin{pmatrix}i\\1\end{pmatrix}+C_2e^{-it}\begin{pmatrix}-i\\1\end{pmatrix}
  \end{gathered}
\end{equation*}\par
\noindent Men detta vill vi skriva på reell form (i termer av sinus och cosinus):
\begin{equation*}
  \begin{gathered}
    \Lrarr (C_1+C_2)\begin{pmatrix}-\sin(t)\\\cos(t)\end{pmatrix}+i(C_1-C_2)\begin{pmatrix}\cos(t)\\\sin(t)\end{pmatrix}
  \end{gathered}
\end{equation*}\par
\noindent Om vi väljer $C_1$ och $C_2$ på rätt sätt kan vi få lösningen uttryckt i bara reella termer.\par
\noindent Vi kan ta $X_1 = \begin{pmatrix}-\sin(t)\\\cos(t)\end{pmatrix}$ och $X_2 = \begin{pmatrix}\cos(t)\\\sin(t)\end{pmatrix}$\par
\noindent Vi får:

\begin{equation*}
  \begin{gathered}
    X_h = C_1X_1+C_2X_2
  \end{gathered}
\end{equation*}\par
\noindent Men det är inte $X_h$ som vi bryr oss om här, det vi vill komma åt är $\Phi$:
\begin{equation*}
  \begin{gathered}
    \Phi = \begin{pmatrix}X_1&X_2\end{pmatrix} = \begin{pmatrix}-\sin(t)&\cos(t)\\\cos(t)&\sin(t)\end{pmatrix}
  \end{gathered}
\end{equation*}
\par\bigskip
\noindent\textbf{Steg 2:}\textit{Hitta en partikulärlösning}:
\par\bigskip
\noindent Då tar vi får $\Phi$na formel som vi har hittat och stoppar in:
\begin{equation*}
  \begin{gathered}
    X_p = \Phi\int\Phi^{-1}F dt
  \end{gathered}
\end{equation*}\par
\noindent Vi har $\Phi$, vi har $F$, vi har inte $\Phi^{-1}$. Vi räknar fram det. Tacksamt är det en 2x2 matris, och inversen ges av en sluten formel:
\begin{equation*}
  \begin{gathered}
    \dfrac{1}{ad-bc}\begin{pmatrix}d&-b\\-c&a\end{pmatrix}\\
    \Lrarr \Phi = \dfrac{1}{-\sin^2(t)-\cos^2(t)}\begin{pmatrix}\sin(t)&-\cos(t)\\-\cos(t)&-\sin(t)\end{pmatrix} = \begin{pmatrix}-\sin(t)&\cos(t)\\\cos(t)&\sin(t)\end{pmatrix}
  \end{gathered}
\end{equation*}\par
\noindent Notera, inversen är sig sjäv, coolt va! (Detta är absolut inte det allmänna fallet).\par
\noindent Nu har vi $\Phi^{-1}$ så vi räknar:
\par\bigskip

\begin{equation*}
  \begin{gathered}
    \int\Phi^{-1}(t)F(t)dt = \int\begin{pmatrix}-\sin(t)&\cos(t)\\\cos(t)&\sin(t)\end{pmatrix}\begin{pmatrix}1/\cos(t)\\0\end{pmatrix}dt\\
    \Lrarr \int\begin{pmatrix}-\dfrac{\sin(t)}{\cos(t)}\\1\end{pmatrix}dt = \int\begin{pmatrix}-\tan(t)\\1\end{pmatrix}dt\\
    \Lrarr \begin{pmatrix}\log(\left|\cos(t)\right|)\\t\end{pmatrix}
  \end{gathered}
\end{equation*}\par
\noindent Vi får inte glömma bort multiplikationen med $\Phi$:
\par\bigskip

\begin{equation*}
  \begin{gathered}
    \Phi\int\Phi^{-1}Fdt = \begin{pmatrix}-\sin(t)&\cos(t)\\\cos(t)&\sin(t)\end{pmatrix}\begin{pmatrix}\log(\left|\cos(t)\right|)\\t\end{pmatrix}\\\\
    \Lrarr \begin{pmatrix}-\sin(t)\log(\left|\cos(t)\right|)+t\cos(t)\\\cos(t)\log(\left|\cos(t)\right|)+t\sin(t)\end{pmatrix}
  \end{gathered}
\end{equation*}
\par\bigskip

\noindent Detta ger en partikulärlösning! Vi har vår homogena, vi har vår partikulärlösning, vi får alltså:
\begin{equation*}
  \begin{gathered}
    X = X_h+X_p \Rightarrow C_1X_1+C_2X_+X_p
  \end{gathered}
\end{equation*}
\par\bigskip
\subsection{Flera egenskaper hos fundamentalmatrisen}\hfill\\
\par\bigskip
\noindent Fundamentalmatrisen är som linjär algebra själv, det finns otroligt många tillämpningsområden och coola egenskaper man kan studera och som kan hjälpa en med ODE:er. Vi ska kika på ett par:
\par\bigskip

\noindent Om man skall hitta en specifik lösning till ett IVP behöver vi bestämma $C_1, C_2,\cdots, C_n$. Man kan stoppa in och lösa, men det visar sig att det går att smidigt använda sig av fundamentalmatrisen för att göra saker lite enklare.\par
\noindent Vi har sett att lösningen till $X^{\prime} = AX$ kan skrivas som $X = \Phi C$ där $C = \begin{pmatrix}C_1\\\vdots\\C_n\end{pmatrix}$. Om vi nu vill hitta en lösning så att vi har ett specifikt initialvillkor, exempelvis $X(0)=X_0$, så får vi ju en matrisekvation att lösa ty vi har $X_0 = \Phi(0)C$. Vi vet att $\Phi$ är inverterbar, så vi kan helt enkelt skriva det som $C = \Phi^{-1}(0)X_0$\par
\noindent Vi kan skriva detta på en sats:
\par\bigskip
\begin{theo}
  LLåt $\Phi(t)$ vara fundamentalmatrisen för $X^{\prime} = AX$. Den (unika) lösningen till systemet med $X(0)=X_0$ ges av:
  \begin{equation*}
    \begin{gathered}
      X(t)=\Phi(t)\cdot C = \Phi(t)\Phi^{-1}(0)X_0
    \end{gathered}
  \end{equation*}
\end{theo}
\par\bigskip
\noindent Exempel:\par
\noindent Betrakta
\begin{equation*}
  \begin{gathered}
    \begin{rcases*}
      x^{\prime} = 4x+2y\qquad x(0)=1\\
      y^{\prime} = 3x-y\qquad y(0)=-1
    \end{rcases*}\\
    X^{\prime} = \underbrace{\begin{pmatrix}4&2\\3&-1\end{pmatrix}}_{\text{$A$}}X
  \end{gathered}
\end{equation*}\par
\noindent Hitta $\Phi$, vi måste räkna fram egenvärden vilket blir $\lambda_1=-2$ och $\lambda_2 = 5$. Tillhörande egenvektorer är $K_1 = \begin{pmatrix}1\\-3\end{pmatrix}$ och $K_2 = \begin{pmatrix}2\\1\end{pmatrix}$\par
\noindent Lösningar ges av $X_1 = e^{-2t}\begin{pmatrix}1\\-3\end{pmatrix},\qquad X_2 = e^{5t}\begin{pmatrix}2\\¡\end{pmatrix}$
\par\bigskip

\begin{equation*}
  \begin{gathered}
    \Phi = \begin{pmatrix}e^{-2t}&2e^{5t}\\-3e^{-2t}&e^{5t}\end{pmatrix}
  \end{gathered}
\end{equation*}\par
\noindent Vi vill hitta $X$ så att $X(0) = \begin{pmatrix}1\\-1\end{pmatrix}$. Vi använder formen direkt:
\begin{equation*}
  \begin{gathered}
    \Phi(0) = \begin{pmatrix}1&2\\-3&1\end{pmatrix}\Rightarrow\Phi^{-1}(0) = \dfrac{1}{1+6}\begin{pmatrix}1&-2\\3&1\end{pmatrix}\\
    \Lrarr \dfrac{1}{7}\begin{pmatrix}1&-2\\3&1\end{pmatrix}
  \end{gathered}
\end{equation*}\par
\noindent Vi stoppar in i formeln vi hade för att hitta $C$:
\begin{equation*}
  \begin{gathered}
    X(t) = \Phi\Phi^{-1}(0)X_0\\
    = \begin{pmatrix}e^{-2t}&2e^{5t}\\-3e^{-2t}&e^{5t}\end{pmatrix}\dfrac{1}{7}\begin{pmatrix}1&-2\\3&1\end{pmatrix}\begin{pmatrix}1\\-1\end{pmatrix}\\\\
    \Lrarr\dfrac{1}{7}\begin{pmatrix}3e^{-2t}+4e^{5t}\\-9e^{-2t}+2e^{5t}\end{pmatrix}
  \end{gathered}
\end{equation*}
