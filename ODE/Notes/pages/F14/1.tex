\section{Numeriska metoder - Datorprojekt}
\par\bigskip
\noindent Oftast kan vi inte hitta en explicit lösning, men en approximation är allt som behövs.\par
\noindent Vi kommer titta på:
\begin{itemize}
  \item Eulers metod
  \item Förbättrad Eulers metod
  \item Runge-Kutta metoden
\end{itemize}
\par\bigskip
\begin{theo}
  Antag $y^{\prime} = f(x,y)$ med $y(x_0) = y_0$ har en unik lösning på intervallet $[x_0, a]$. Då finns en konstant $c$, så att:
  \par\bigskip
  \noindent Om $y_1, \cdots, y_n$ är approximationerna med Eulers metod beräknad med steglängd $h$, då har vi:
  \begin{equation*}
    \begin{gathered}
      \left|y_i-y(x_i)\right|\leq C\cdot h
    \end{gathered}
  \end{equation*}
\end{theo}
