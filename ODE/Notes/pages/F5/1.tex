\section{Föreläsning - Homogena ekvationer}

\noindent Vi vill hitta dessa linjärt oberoende funktionerna, det är syftet med dagens föreläsning. Vi kommer börja med att undersöka följande:


\begin{equation*}
  \begin{gathered}
    y^{\prime\prime}+p(x)y^{\prime}+q(x)y=0
  \end{gathered}
\end{equation*}
\par\bigskip
\noindent Alla lösningar är på formen $y=C_1y_1+C_2y_2$ där $\{y_1,y_2\}$ är en \textit{fundamental lösningsmängd}, dvs linjärt oberoende.
\par\bigskip
\noindent Hur ska vi hitta $y_1, y_2$? Vi kommer använda 2 olika metoder:

\begin{itemize}
  \item Reduction of order (reduction av ordning, om vi har $y_1$ så kan vi hitta $y_2$)
  \item $p, q$ är konstanter så finns det en annan metod
\end{itemize}
\par\bigskip

\subsection{Reduktion av ordning}\hfill\\

\noindent Antag att vi har en lösning $y_1$ till $y^{\prime\prime}+p(x)y^{\prime}+q(x)y=0$.
\par\bigskip
\noindent Denna metod kan användas för att hitta en till (linjärt oberoende) lösning $y_2$. Idén:

\begin{itemize}
  \item Hitta $y_2$ som ej är beroende av $y_1$
  \item Vi sätter $y_2(x)=u(x)\cdot y_1(x)$, om linj. ber. så är $u(x)$ en konstant så vi vill hitta $u(x)$
\end{itemize}
\par\bigskip
\noindent Vi kan börja med att se att:


\begin{equation*}
  \begin{gathered}
    y_{2}^{\prime}=u^{\prime}\cdot y_1+uy_{1}^{\prime}\\
    y_{2}^{\prime\prime}= u^{\prime\prime}\cdot y_1+2u^{\prime}y_{1}^{\prime}+uy_{1}^{\prime\prime}\\
    \text{Insättning av $y_2, y_2^{\prime}, y_2^{\prime\prime}$ i ekvationen ger: }\\
    (u^{\prime\prime}\cdot y_1+2u^{\prime}y_{1}^{\prime}+uy_{1}^{\prime\prime})+p(u^{\prime}\cdot y_1+uy_{1}^{\prime}) +quy_1\\
    u\underbrace{(y_{1}^{\prime\prime}+py_{1}^{\prime}+qy_1)}_{\text{Ekvationen! = 0}}+u^{\prime\prime}y_1+u^{\prime}(2y_{1}^{\prime}+py_1)\\
    \text{Det som står framför $u$ är lika med noll ty det är precis det vi börjar med:}\\
    u^{\prime\prime}y_1+u^{\prime}(2y_{1}^{\prime}+py_1)=0\\
    \text{Låt } w=u^{\prime} \text{ :}\\
    y_1w^{\prime}+w(2y_{1}^{\prime}+py_1)=0 \text{ Detta är en linjär första ordningens ODE}
  \end{gathered}
\end{equation*}
\par\bigskip

\noindent Exempel: Givet att $x^4$ är en lösning till $x^2y^{\prime\prime}-7xy^{\prime}+16y=0$, hitta en annan lösning. Det vi kommer se är att metoden vi precis gick igenom funkar ävenom vi har något framför $y^{\prime\prime}$
\par\bigskip
\noindent Ansätt $y_2=uy_1$ där $y_1=x^4$:

\begin{equation*}
  \begin{gathered}
    x^2(u^{\prime\prime}y_1+2u^{\prime}y_{1}^{\prime}+uy_{1}^{\prime\prime})+p(u^{\prime}y_1+uy_{1}^{\prime})+quy_1=0\\
    u(x^2y_{1}^{\prime\prime}+py_{1}^{\prime}+qy_{1})+x^2u^{\prime\prime}y_1+u^{\prime}(2x^2y_{1}^{\prime}+py_1)=0\\
    0+x^2u^{\prime\prime}y_1+u^{\prime}(2x^2y_{1}^{\prime}+py_1)=0\\
    \text{Låt } w = u^{\prime} \text{ och sätt in } y_1=x^4:\\
    x^2\cdot x^4\cdot w^{\prime}+w(2x^2\cdot4x^3-7x\cdot x^4)=0 \text{ OBS: $q$ finns inte med}\\
    x^6w^{\prime}+w(8x^5-7x^5)=0\\
    x^6w^{\prime}+x^5w=0\\
    xw^{\prime}+w=0\\
    (xw)^{\prime}=0\\
    xw=C\\
    w=\dfrac{C}{x}\\
    u^{\prime}=\dfrac{C}{x}\Lrarr u=C\log(|x|)+D\\
    y_2 = u\cdot y_1 = (C\log(|x|)+D)x^4 \text{ men vi vill få \textit{en} lösning, detta är en familj av lösning}\\
    \text{Oftast väljer man $D=0$. $C$ får helst inte vara noll, annars får vi att $y_2=Cy_1$, alltså linjärt beroende}\\
     \text{Väljer vi $C=1$, $D=0$ får vi:}\\
     y_2=x^4\log(|x|)
  \end{gathered}
\end{equation*}
\par\bigskip
\noindent Alla lösningar är på formen:


\begin{equation*}
  \begin{gathered}
    C_1y_1+C_2y_2=C_1x^4+C_2x^4\log(|x|)=x^4(C_2\log(|x|)+C_1)
  \end{gathered}
\end{equation*}
\par\bigskip
\noindent Kommentar: Detta fungerar bara om vi har en lösning redan. Påminner lite om polynom, där om man redan har en lösning så kan vi dela bort den och få ett polynom med lägre grad.

\subsection{Homogena 2:a orndingens linj. ODE med konstanta koefficienter $p,q$}\hfill\\

\noindent Vi kommer betrakta $y^{\prime\prime}+py^{\prime}+qy=0$ för $p, q \in\R$. Det visar sig att det alltid finns en lösning på formen $y(x)=e^{r\cdot x}$ där $r$ beror på $p, q$. Insättning ger:


\begin{equation*}
  \begin{gathered}
    r^2e^{rx}+pre^{rx}+qe^{rx}=0 \Lrarr e^{rx}(r^2+pr+q)=0
    \text{$e^{rx}$ är aldrig noll, alltså} r^2+pr+q=0
  \end{gathered}
\end{equation*}
\par\bigskip
\noindent För att det skall vara lika med noll, så måste $r$ vara en rot, alltså har vi en lösning enbart då.
\par\bigskip

\begin{theo}[Karaktäristiska polynom]{thm:char}
  Polynomet $r^2+pr+q$ kallas för det \textit{karaktäristiska} polynomet till $y^{\prime\prime}+py^{\prime}+qy=0$
\end{theo}
\par\bigskip

\noindent Vi får 3 oilka fall när vi hanterar rötter:

\begin{itemize}
  \item Fall 1: 2 distinkta reela rötter
  \item Fall 2: Dubbelrot
  \item Fall 3: Icke-reela rötter
\end{itemize}
\par\bigskip

\noindent Fall 1: När $r_1\neq r_2$. Vi får 2 lösningar, en för $r_1$ och en för $r_2$: $y_1=e^{r_1x}$ och $y_2=e^{r_2x}$. Dessa är linjärt oberoende eftersom $\dfrac{e^{r_1x}}{e^{r_2x}}=e^{(r_1-r_2)x}$, om de är beroende hade vi fått en konstant det vill säga $e^0$, men eftersom $r_1\neq r_2$ får vi inte det. Alltså är $\{y_1, y_2\}$ en fundamental lösningsmängd. Vi vet att alla lösningar ges på foren $y=C_1e^{r_1x}+C_2e^{r_2x}$
\par\bigskip

\noindent Fall 2: När $r_1=r_2$. Vi får \textit{en} lösning $y_1=e^{rx}$ men vi behöver 2 lösningar. Använd reduktion av ordning för att hitta $y_2$, $y_2 = u(x)y_1 = u(x)(e^{rx})$, derivera 2 gånger och stoppa in så får vi $u^{\prime\prime}y_1+2u^{\prime}y_{1}^{\prime}+u_{1}^{\prime\prime}+p(u^{\prime}y_1+uy_{1}^{\prime})+quy_1=0$. Då försvinner vissa termer och vi får kvar $y_1u^{\prime\prime}+(2y_{1}^{\prime}+py_{1})u^{\prime}=0$. Vi har $y_1=e^{r_1x}$ med $r_1=-\dfrac{p}{2}$. Insättning ger $e^{-\dfrac{p}{2}}u^{\prime\prime}+(2(-\dfrac{p}{2})e^{-\dfrac{p}{2}x}+pe^{-\dfrac{p}{2}x})u^{\prime}=0$:


\begin{equation*}
  \begin{gathered}
    e^{-\dfrac{p}{2}}u^{\prime\prime}=0\\
    u^{\prime\prime}=0\\
    u=Cx+D
  \end{gathered}
\end{equation*}

\noindent Vi har alltså $y_2=(Cx+D)e^{-\dfrac{p}{2}}x$. Sätt $D=0$ och $C=1\rightarrow y_2=xe^{-\dfrac{p}{2}}x=xe^{rx}=xy_1$
\par\bigskip


\noindent Alltså $y=C_1e^{rx}+C_2xe^{rx} = (C_1+C_2x)e^{rx}$ där $C_1\neq C_2$
\par\bigskip
\noindent Fall 3: Komplexa rötter:


\begin{equation*}
  \begin{gathered}
    y_1=e^{(a+bi)x}\text{ och } y_2=e^{(a-bi)x}
  \end{gathered}
\end{equation*}
\par\bigskip
\noindent Dessa är linjärt oberoende. I denna kurs håller vi enbart på med reela ting, så det är fördelaktigt att skriva om så att det blir tydligt att vi arbetar med reela former. Vi kan använda oss av eulers formel $e^{i\theta}=\cos(\theta)+i\sin(\theta)$
\par\bigskip
\noindent Vi får:


\begin{equation*}
  \begin{gathered}
    e^{(a+bi)x}=e^{ax}e^{bix}=e^{ax}(\cos(bx)+i\sin(bx))\\
    e^{(a-bi)x}=e^{ax}e^{-bix}=e^{ax}(\cos(-bx)+i\sin(-bx))=e^{ax}(\cos(bx)-i\sin(bx))
  \end{gathered}
\end{equation*}
\par\bigskip
\noindent Alla lösningar är på formen


\begin{equation*}
  \begin{gathered}
    y=C_1e^{ax}(\cos(bx)+i\sin(bx))+C_2e^{ax}(\cos(bx)-i\sin(bx))=\\
    (C_1+C_2)e^{ax}\cos(bx)+i(C_1-C_2)e^{ax}\sin(bx) \text{ vi har real del och imaginär del}\\
    \text{Tag $C_1=C_2=\dfrac{1}{2}$ } y= e^{ax}\cos(bx)\\
    \text{Tag $C_1=-C_2 = -\dfrac{i}{2}$ } = i(-\dfrac{i}{2}-\dfrac{i}{2})e^{ax}\sin(bx)=e^{ax}\sin(bx)
  \end{gathered}
\end{equation*}
\par\bigskip
\noindent Dessa är linjärt oberoende, om vi delar dem på varandra får vi $\dfrac{\cos(bx)}{\sin(bx)}$ vilket inte är en konstant.
\par\bigskip
\noindent Alltså är $\{e^{ax}\cos(bx), e^{ax}\sin(bx)\}$ en annan fundamental lösningsmängd. Vi använder oftast denna.
\par\bigskip
\noindent Exempel: Hitta den allmänna lösningen till:

\begin{itemize}
  \item $2y^{\prime\prime}-5y^{\prime}-3y=0$
  \item $y^{\prime\prime}-10y^{\prime}+25y=0$
  \item $y^{\prime\prime}+4y^{\prime}+7y=0$
\end{itemize}
\par\bigskip
\noindent Vi kan konstatera att alla dessa är linjära, andra ordningen, konstanta koefficienter. Vi börjar att betrakta första punkten:


\begin{equation*}
  \begin{gathered}
    y^{\prime\prime}-\dfrac{5}{2}y^{\prime}-\dfrac{3}{2}y=0\\
    \text{Karaktäristiska polynomet: } r^2-\dfrac{5}{2}r-\dfrac{3}{2}=0\\
    \text{Rötterna blir: } r_1=3, r_2=-\dfrac{1}{2}\\
    y=C_1e^{-\dfrac{1}{2}x}+C_2e^{3x} \text{ (notera, inget $x$ framför $C_2$ ty ingen dubbelrot)}
  \end{gathered}
\end{equation*}
\par\bigskip
\noindent Nästa:


\begin{equation*}
  \begin{gathered}
    \text{Karaktäristiska polynomet: } r^2-10r+25, \text{ rötterna är } r_1 = 5, r_2=5\\
    \text{Dubbelrotination, allmänna lösningen ges av: }\\
    y = C_1e^{5x}+C_2x^{5x}
  \end{gathered}
\end{equation*}
\par\bigskip
\noindent Sista:


\begin{equation*}
  \begin{gathered}
    \text{Karaktäristiska polynomet: } r^2+4r+7, \text{ rötterna ges av } r_1=-2+i\sqrt(3), r_2=-2-i\sqrt(3)\\
    \text{Två komplexa rötter, vi får: }\\
    y=C_1e^{-2x}\cos(\sqrt(3x))+C_2e^{-2x}\sin(\sqrt(3x))
  \end{gathered}
\end{equation*}










