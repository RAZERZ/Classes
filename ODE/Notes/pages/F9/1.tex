\section{Frobenius metod}
\par\bigskip
\noindent Vi utgår från samma ekvation som alltid
\begin{equation*}
  \begin{gathered}
    A(x)y^{\prime\prime}+B(x)y^{\prime}+C(x)y=0
  \end{gathered}
\end{equation*}\par
\noindent Dagens fokus handlar om reguljära singulära punkter, dvs när vi får problem med att dela bort $A(x)$. Idag kommer vi inte hantera irreguljära singulära punkter. Man kan ha vilka reguljära singulära punkter som helst, men idag kommer vi anta att de ligger i origo (man kan alternativt bara substituera/flytta origo i andra fall).
\par\bigskip
\noindent \textbf{Påminnelse:} En \textit{reguljära} singulär punkt kring $x_0 = 0$ ger:
\begin{itemize}
  \item $x\cdot p(x) = x\dfrac{B(x)}{A(x)}$
  \item $x^2q(x)=x^2\dfrac{C(x)}{A(x)}$
\end{itemize}\par\bigskip
\noindent där båda dessa är \textit{analytiska} (dvs deras potensserie konvergerar till funktionen). Vi kan därför skriva om dessa på deras respektive potensserier:


\begin{equation*}
  \begin{gathered}
    xp(x) = \sum_{n=0}^{\infty}p_nx^n\\
    x^2q(x)=\sum_{n=0}^{\infty}q_nx^n
  \end{gathered}
\end{equation*}
\par\bigskip
\noindent Vi kan nu skriva om  $A(x)y^{\prime\prime}+B(x)y^{\prime}+C(x)y=0$ till:


\begin{equation*}
  \begin{gathered}
    y^{\prime\prime}+p(x)y^{\prime}+q(x)y=0\\
    \Lrarr x^2y^{\prime\prime}+x^2p(x)y^{\prime}+x^2q(x)y=0\\
    \Lrarr x^2y^{\prime\prime}+x(xp(x))y^{\prime}+(x^2q(x))y=0\\
    \Lrarr x^2y^{\prime\prime}+x\left(\sum_{n=0}^{\infty}p_nx^n\right)y^{\prime}+\left(\sum_{n=0}^{\infty}\right)y=0
  \end{gathered}
\end{equation*}
\par\bigskip
\noindent Om $p_n=q_n=0$ för $n\geq1$ får vi
\begin{equation*}
  \begin{gathered}
    x^2y^{\prime\prime}+p_0xy^{\prime}+q_0y=0
  \end{gathered}
\end{equation*}\par
\noindent Detta är en Eulerekvation!
\par\bigskip
\noindent Förra föreläsningen såg vi att Eulerekvationen har lösningar på formen $y=x^r$. Vi provar nu att hitta lösningar på formen
\begin{equation*}
  \begin{gathered}
    y=x^r\cdot f(x)=x^r\sum_{n=0}^{\infty}c_nx^n
  \end{gathered}
\end{equation*}\par
\noindent Om $c_0=0$ blir den första termen som dyker upp $x$ vilket vi kan bryta ut så att vi får $x^{r+1}$, vi vill inte ha saker som vi kan bryta ut. Eftersom $c_1\cdot y$ också är en lösning kan vi ta $c_0=1$.\par
\noindent Att hitta lösningar på den formen kallas för \textit{Frobenius metod}. Den kräver ganska mycket räkning, så vi kommer gå igenom ett långt exempel och senare gå igenom satser och definitioner.
\par\bigskip
\subsection{Exempel}\hfill\\
\par\bigskip
\noindent Betrakta 
\begin{equation*}
  \begin{gathered}
    2x^2y^{\prime\prime}+3xy^{\prime}-(x^2+1)y=0
  \end{gathered}
\end{equation*}\par
\noindent Notera här att om ekvationen hade sett ut på följande $2x^2y^{\prime\prime}+3xy^{\prime}-(1)y=0$ så hade det varit en Eulerekvation.\par
\noindent Vi börjar med att kontrollera att origo är en reguljär singulär punkt. Vi har:
\begin{equation*}
  \begin{gathered}
    p(x)=\dfrac{3x}{2x^2}=\dfrac{3}{2x}\qquad q(x)=-\dfrac{1}{2}-\dfrac{1}{2x^2}
  \end{gathered}
\end{equation*}\par
\noindent Vi ser att $x=0$ är en singulär punkt ty $p$ och $q$ ej är analytiska. vidare skall vi betrakta $x\cdot p(x)$:
\begin{equation*}
  \begin{gathered}
    xp(x)=\dfrac{3}{2}\qquad x^2q(x)=-\dfrac{x^2}{2}-\dfrac{1}{2}
  \end{gathered}
\end{equation*}\par
\noindent Båda dessa är analytiska, vi har alltså en reguljär singulär punkt, då kan vi använda Frobenius metod och se vad som händer.
\par\bigskip
\noindent Det vi vill göra är att ta en ansats som är på formen $y=x^r\sum_{n=0}^{\infty}c_nx^n$. Vi deriverar $y$ och gör det lite enklare genom att multiplicera in $x^r$. Då blir det inte en potensserie ty vi kräver att $n\in\N$ och med $r$ som spökar kan vi inte garantera det, men det spelar ingen roll. Vi får:

\begin{equation*}
  \begin{gathered}
    y^{\prime} = \sum_{n=0}^{\infty}(n+r)c_nx^{n+r-1}\\
    y^{\prime\prime}=\sum_{n=0}^{\infty}(n+r)(n+r-1)c_nx^{n+r-2}
  \end{gathered}
\end{equation*}\par
\noindent Insättning ger:
\begin{equation*}
  \begin{gathered}
    2x^2\cdot\sum_{n=0}^{\infty}(n+r)(n+r-1)c_nx^{n+r-2}+3x\sum_{n=0}^{\infty}(n+r)c_nx^{n+r-1}-(x^2+1)\sum_{n=0}^{\infty}c_nx^{n+r}=0\\
    \sum_{n=0}^{\infty}2(n+r)(n+r-1)c_nx^{n+r}+\sum_{n=0}^{\infty}3(n+r)c_nx^{n+r}-\sum_{n=0}^{\infty}c_nx^{n+r+2}-\sum_{n=0}^{\infty}c_nx^{n+r}=0
  \end{gathered}
\end{equation*}\par
\noindent Vi vill skriva om summorna så att exponenten blir samma för alla. Skriv som den lägsta! (Ej strikt nödvändigt, men gör räkning lättare). I detta fall är det $n+r$. I detta exempel betraktar vi även $x>0$. Den enda sumnan vi behöver skriva om i det här fallet är den med $x^{n+r+2}$, vi vill skriva om den så att den får exponenten $x^{n+r}$:
\begin{equation*}
  \begin{gathered}
    \sum_{n=0}^{\infty}c_nx^{n+r+2}=c_0x^{r+2}+c_1x^{r+3}+c_2x^{r+4}\cdots
  \end{gathered}
\end{equation*}\par
\noindent Vi noterar att om vi istället börjar på $n=2$ så får vi:
\begin{equation*}
  \begin{gathered}
    \sum_{n=2}^{\infty}c_{n-2}x^{n+r}
  \end{gathered}
\end{equation*}\par
\noindent Insättning ger:

\begin{equation*}
  \begin{gathered}
    \sum_{n=0}^{\infty}2(n+r)(n+r-1)c_nx^{n+r}+\sum_{n=0}^{\infty}3(n+r)c_nx^{n+r}-\sum_{n=2}^{\infty}c_{n-2}x^{n+r}-\sum_{n=0}^{\infty}c_nx^{n+r}=0
  \end{gathered}
\end{equation*}\par
\noindent Allt är nästan nice här förutom att vi har en summa som börjar på $n=2$. Notera att den gemensamma summationen börjar vid $n=2$ isåfall. Vi kan "bryta ut" de första 2 termer från de övriga summorna:
\par\bigskip
\begin{equation*}
  \begin{gathered}
    (2r(r-1))c_0+3(rc_0-c_0)x^r+(2(1+r)(1+r-1))c_1+3(1+r)c_1-c_1x^{r+1}\\
    +\sum_{n=2}^{\infty}\left(2(n+r)(n+r-1)c_n+3(n+r)c_n-c_{n-2}-c_n\right)x^{n+r}
  \end{gathered}
\end{equation*}
\par\bigskip
\noindent Dela med $x^r$ och förenkla:

\begin{equation*}
  \begin{gathered}
    (2r(r-1)+3r-1)c_0+(2(1+r)r+3(1+r)-1)c_1x\\
    +\sum_{n=2}^{\infty}\left((2(n+r)(n+r-1)+3(n+r)-1)c_n-c_{n-2}\right)x^n=0\\
    \Lrarr (2r(r-1)+3r-1)c_0 = 0 \Rightarrow 2r(r-1)+3r-1 = 0 \text{\qquad Karaktäristiska ekvation, lös för $r$:}\\
    r_1 = \dfrac{1}{2}\qquad r_2=-1
  \end{gathered}
\end{equation*}
\par\bigskip
\noindent Nu har vi kommit fram till vad $r$ måste vara, nu behöver vi hitta $c_n$. Betraktar vi andra termen har vi:

\begin{equation*}
  \begin{gathered}
    (2(1+r)r+3(1+r)-1)c_1\neq0 \text{ för $r_1$ eller $r_2$, alltså $c_1=0$}
  \end{gathered}
\end{equation*}
\par\bigskip
\noindent Nu vill vi kolla för summan, vi får:
\begin{equation*}
  \begin{gathered}
    (2(n+r)(n+r-1)+3(n+r)-1)c_n-c_{n-2}=0
  \end{gathered}
\end{equation*}
\par\bigskip
\noindent Det vi vill göra är att få fram en differensekvation för $c_n$. Om $(2(n+r)(n+r-1)+3(n+r)-1)\neq=0$. Då kan vi dela bort den så att vi får:

\begin{equation*}
  \begin{gathered}
    c_n = \dfrac{c_{n-1}}{(2(n+r)(n+r-1)+3(n+r)-1)}
  \end{gathered}
\end{equation*}\par
\noindent Hur vet vi att vi kan dela bort den? Jo för att $\underbrace{(2(n+r)(n+r-1)+3(n+r)-1)}_{\text{karaktäristiska ekvationen med $n+r$ istället för $r$}}$ och vi vet att $n\geq2$ (se täljare). Det vi vill göra nu är att hitta 2 linjärt oberoende lösningar. Vi har 2 rötter $r_1$ och $r_2$, så vi delar upp det i 2 fall:
\par\bigskip
\noindent \textbf{Fall 1:} $r=r_1=\dfrac{1}{2}$
\par\bigskip
\noindent Sätt in $r_1$ i differensekvationen för $c_n$. För att inte förvirra oss så kommer vi kalla koefficienterna för $a_n$. Vi får:
\begin{equation*}
  \begin{gathered}
    a_n=\dfrac{a_{n-2}}{2n^2+3n}\qquad n\geq2
  \end{gathered}
\end{equation*}\par
\noindent Vi har då:
\begin{itemize}
  \item $a_0$
  \item $a_1=0$
  \item $a_n = \dfrac{a_{n-2}}{2n^2+3n}$
\end{itemize}\par
\noindent Vi kan räkna ut några $a_n$:

\begin{equation*}
  \begin{gathered}
    a_n = 0 \text{ för alla udda $n$}\\
    a_2 = \dfrac{a_0}{14},\qquad a_4 = \dfrac{a_2}{44}=\dfrac{a_0}{616}
  \end{gathered}
\end{equation*}
\par\bigskip
\noindent Vi har alltså att serien 
\begin{equation*}
  \begin{gathered}
    y_1 = x^{1/2}\sum_{n=0}^{\infty}a_nx^n
  \end{gathered}
\end{equation*}\par
\noindent är en lösning!
\par\bigskip
\noindent \textbf{Fall 2:} $r=r_2=-1$
\par\bigskip
\noindent Vi kallar koefficienterna här för $b_n$. Vi får
\begin{equation*}
  \begin{gathered}
    b_n =\dfrac{b_{n-2}}{2n^2-3n}\qquad n\geq2
  \end{gathered}
\end{equation*}\par
\noindent Återigen har vi $b_1 = 0$ och $b_0=$ fri konsntant. Vi får:
\begin{equation*}
  \begin{gathered}
    y_2 = x^{-1}\sum_{n=0}^{\infty}b_nx^n
  \end{gathered}
\end{equation*}
\newpage
\noindent Sammanfogar vi fallen får vi att den generella lösningen är:

\begin{equation*}
  \begin{gathered}
    y = C_1y_1+C_2y_2
  \end{gathered}
\end{equation*}
\par\bigskip
\noindent Här funkade allt väldigt bra och smidigt, men det kan uppstå problem. Vi kan få 2 problem:
\begin{itemize}
  \item Dubbelrot - löses likt Eulerekvation
  \item Vi kan få division med noll i differensekvationen. Händer om $r_1 = r_2+N\qquad N\in\Z$
\end{itemize}
\par\bigskip
\begin{theo}
  BBetrakta ODE:n

  \begin{equation*}
    \begin{gathered}
      x^2y^{\prime\prime}+x(xp(x))y^{\prime}+x^2q(x)y=0
    \end{gathered}
  \end{equation*}\par
  \noindent Antag att $x=0$ är en reguljär singulär punkt. Då ges $xp(x)$ och $x^2q(x)$ av potensserier:

  \begin{equation*}
    \begin{gathered}
      xp(x) = \sum_{n=0}^{\infty}p_nx^n,\qquad x^2q(x)=\sum_{n=0}^{\infty}q_nx^n
    \end{gathered}
  \end{equation*}\par
  \noindent för $\left|x\right|<\rho$ där $\rho$ är $\leq$ konvergensradien för de två serierna.\par\bigskip
  \noindent Låt $r_1, r_2$ vara rötter till indikalekvationen $r^2+(p_0-1)r+q_0=0$ (notera, samma som för Eulerekvation).\par
  \noindent Antag att $r_1, r_2\in\R$ och $r_1\leq r_2$.\par
  \noindent Då har vi i intervallen $(-\rho,0)$ och $(0,\rho)$ en lösning på formen
  \begin{equation*}
    \begin{gathered}
      y_1=\left|x\right|^{r_1}\sum_{n=0}^{\infty}a_nx^n \qquad a_0\neq0
    \end{gathered}
  \end{equation*}\par\bigskip
  \noindent För den andra (linjärt oberoende lösning) får vi 3 fall:
  \begin{itemize}
    \item $r_1\neq r_2\qquad r_1\neq r_2+N$ för $N\in\N$\\
      Då är $y_2$:
      \begin{equation*}
        \begin{gathered}
          \left|x\right|^{r_2}\sum_{n=0}^{\infty}b_nx^n\qquad b_0\neq0
        \end{gathered}
      \end{equation*}
    \item $r_1 = r_2$ ger:
      \begin{equation*}
        \begin{gathered}
          y_2 = y_1\log(\left|x\right|)+\left|x\right|^{r_1}\sum_{n=1}^{\infty}b_nx^n
        \end{gathered}
      \end{equation*}
    \item $r_1 = r_2 +N\qquad N\in\N$ ger:
      \begin{equation*}
        \begin{gathered}
          y_2 = ay_1\log(\left|x\right|)+\left|x\right|^{r_2}\sum_{n=0}^{\infty}b_nx^n
        \end{gathered}
      \end{equation*} där $a$ är konstant som kan vara 0.
  \end{itemize}
  \noindent Kommentar: alla serier konvergerar för $\left|x\right|<\rho$ och mängden $\{y_1, y_2\}$ ger en fundamental lösnignsmängd
\end{theo}
\par\bigskip
\noindent Satsen säger inte mycet om hur man hittar $a_n$ eller $b_n$, men de går att bestämma genom insättning vilket kan vara mer eller mindre krångligt. 
\par\bigskip
\noindent Beviset liknar metoden för det introducerande exemplet för metoden men mer generellt.
\par\bigskip
\begin{prf}[Bevisskiss - Frobenius metod]{prf:skfrobmeth}
  För att visa att $y_1$ är en lösning gör man som i exemplet.
  \par
  \noindent För $y_2$ kan mann göra samma ansats, men man stöter på probem om $r_1 = r_2$ eller $r_1=r_2+N$. Hur gör man i de fallen? Man kan använad reduktion av ordning.
  \par\bigskip
  \noindent Kommentar: Om $r_1, r_2$ är komplexa så fungerar samma ansats, men du får  komplexa koefficienter. Man kan få fram reella lösningar på samma sätt som vi gjorde för Eulerekvationerna, men detta kommer inte diskuteras vidare 
\end{prf}
\par\bigskip
\noindent Ett problem är att kontrollera sin lösning eftersom man får en potensserielösning. Detta gör att Dahne inte är lika \textbf{hård} om man missar tecken. Denna metod är även väldigt mekanisk så en dator kan göra det. Det finns några program/paket för att göra det vilket ger en taylorlösning.
