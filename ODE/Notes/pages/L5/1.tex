\section{Lektion}
\par\bigskip
\subsection{Kritiska punkter och lokalt linjära system}\hfill\\

\begin{itemize}
  \item En kritisk punkt är antingen stabil eller instabil
  \item Det räcker oftast att studera den linjära delen av ett system för att förstå stabilitet
  \item Ett system kan alltid skrivas som ett lokalt linjärt system
  \item Om Jakobianen vid en krisik punkt har ett positivt egenvärde är punkten instabil
  \item För asymptotiskt stabila kritiska punkter konvergerar närliggande lösningar mot punkten
  \item Om realdelarna för Jakobianens egenvärden alla är noll vid en kritisk punkt får vi ingen information om stabiliteten

  \item Nära en kritsk punkt dominerar den linjära delen av ett lokalt linjärt system
\end{itemize}
\par\bigskip
\subsection{Liapunovs metod}\hfill\\
\begin{itemize}
  \item Att hitta en fungerande Liapunovfunktion är enkelt
  \item En Liapunovfunktion behöver vara negativt definit
  \item Liapunovs satser är användbara när alla egenvärden till Jakobianen är rent imaginära
  \item Det finns bara en Liapunovfunktion som fungerar
  \item En Liapunovfunktion motsvarar ungefär energin hos ett fysikaliskt system
  \item Om det finns en Liapunovfunktion så att derivatan med avseende på systemet är engativt definit då är den kritiska punkten stabil
  \item Om systemet modellerar ett fysikaliskt system funerar energin hos systemet ofta som en Liapunovfunktion
\end{itemize}
\par\bigskip
\subsection{Gränscykler}\hfill\\

\begin{itemize}
  \item En gränscykel är en periodisk lösning
  \item Det finns alltid lösningar som konvergerar mot en gränscykel
  \item Alla lösningar konvergerar mot en gränscykel
  \item En gränscykel går alltid runt origo
  \item Man kan oftast beräkna gränscyklar explicit
  \item För att en gränscykel ska kunna finnas måste systemet ha en kritisk punkt
  \item En gränscykel omsluter exakt en kritisk punkt 
\end{itemize}
