\section{Potensserielösningar}
\noindent På samama sätt som föregående kapitel kommer även detta gälla för linjära ODE:er. Denna metod fungerar för n:te ordningens linjära ODE:er men det blir snabbt mycket räkning så vi kommer begränsa oss till 2:a ordningen. 
\par\bigskip
\noindent Kommer i allmänhet inte ge oss lönsingar på sluten form utan vi får lösningar i termer av en potensserie, därför är det bra att bra grund i potensserier så vi kommer ägna lite tid åt det. 
\par\bigskip
\noindent De flesta ODE:er har inte lösning på sluten form utan måste beskrivas mha potensserier. Exempel på ekvationer:
\par\bigskip
\begin{itemize}
  \item Bessel-ekvationerna $x^2y^{\prime\prime}+xy^{\prime}+(x^2-\alpha^2)y=0$
  \item Legendreekvationen $(1-x^2)y^{\prime\prime}-2xy^{\prime}+\lambda(1+\lambda)y=0$
\end{itemize}
\par\bigskip
\subsection{Potensserier - Repetition}\hfill\\

\noindent För att göra allt på ett snyggt sätt kommer vi kräva lite fler verktyg än vi har tillgång till men det betyder inte att vi inte kan göra saker (- Albert Einsten eller nåt). Nog med banter!
\par\bigskip
\noindent En potensserie är en serie på följande form:

\begin{equation*}
  \begin{gathered}
    \sum_{n=0}^{\infty}c_n(x-a)^n = c_0+c_1(x-a)+c_2(x-a)^2+\cdots
  \end{gathered}
\end{equation*}
\par\bigskip
\noindent Detta är en potensserie med \textit{center i $a$}. Om center är i $a=0$ får vi:


\begin{equation*}
  \begin{gathered}
    \sum_{n=0}^{\infty}c_nx^n = c_0 + c_1x+c_2x^2\cdots
  \end{gathered}
\end{equation*}
\par\bigskip
\noindent Oftast kommer vi arbeta med $a = 0$ (det blir lättast). Men! Med variabelsubstitution är det lätt att byta mellan fallen.
\par\bigskip
\noindent Den viktigaste egenskapen en potensserie har är om den konvergerar eller ej inom ett intervall $I$.
\par\bigskip

\begin{theo}[Potensserie]{thm:expser}
\noindent Potensserien konvergerar om:

\begin{equation*}
  \begin{gathered}
    \sum_{n=0}^{\infty}c_n(x-a)^n = \lim_{N\to\infty}\sum_{n=0}^{N}c_n(x-a)^n
  \end{gathered}
\end{equation*}
\par\bigskip
\noindent existerar för alla $x\in I$. Då är summan:

\begin{equation*}
  \begin{gathered}
    f(x) = \sum_{n=0}^{\infty}c_n(x-a)^n
  \end{gathered}
\end{equation*}
\par\bigskip
\noindent definierad på $I$ och serien kallas för en \textit{potensserierepresentation} av $f(x)$
\end{theo}
\par\bigskip
\noindent Exempel: Vi har:


\begin{equation*}
  \begin{gathered}
    e^x = \sum_{n=0}^{\infty}\dfrac{x^n}{n!}\\
    \cos(x) = \sum_{n=0}^{\infty}\dfrac{(-1)^nx^{2n}}{(2n)!}\\
    \sin(x) = \sum_{n=0}^{\infty}\dfrac{(-1)^nx^{2n+1}}{(2n+1)!}
  \end{gathered}
\end{equation*}
\par\bigskip
\noindent Dessa konvergerar $\forall x$. En som är värd att nämna:


\begin{equation*}
  \begin{gathered}
    \dfrac{1}{1-x} = \sum_{n=0}^{\infty}x^n\\
    \log(1+x)=\sum_{n=0}^{\infty}\dfrac{(-1)^{n+1}x^n}{n}
  \end{gathered}
\end{equation*}
\par\bigskip
\noindent Dessa konvergerar $\left|x\right|<1$ och divergerar då $\left|x\right|>1$.
\par\bigskip
\noindent Alla dessa går att beskriva i termer av taylorserier:
\par\bigskip

\begin{theo}[Taylorserie för funktion]{thm:taylor}
  \noindent Taylorserien för en funktion $f$ centrerad i $x=a$ är:


  \begin{equation*}
    \begin{gathered}
      \sum_{n=0}^{\infty}\dfrac{f^{(n)}a}{n!}\cdot(x-a)^n = f(a)+f^{\prime}(a)(x-a)+\dfrac{f^{\prime\prime}(a)}{2}(x-a)^2\cdots
    \end{gathered}
  \end{equation*}
  \par\bigskip
  \noindent Om $a = 0$ är det MacLaurin serier
  \par\bigskip
  \noindent Om $x = a$ konvergerar serien alltid till $f(a)$.
\end{theo}
\par\bigskip
\noindent Kommer serien konvergera för $x$ utanför de punkterna?
\par\bigskip

\begin{theo}[Konvergensradie]{thm:convrad}
  \textit{Konvergensradien} $r$ av en potensserie är det värde $r\geq0$ så att:
  \begin{itemize}
    \item Serien absolutkonvergerar för $\left|x-a\right|< r$ (notera, strikt mindre)
    \item Serien divergerar för $\left|x-a\right|>r$
  \end{itemize}
  \par\bigskip
  \noindent Serien konvergerar i ett intervall med radie $r$ centrerat kring $a$. Utanför intervallet divergerar den. Notera att vi inte betraktar ändpunkterna i konvergensintervallet; ibland konvergerar den där och ibland inte. 
\end{theo}
\par\bigskip

\begin{theo}[Konvergensradiens existens]{thm:konvradex}
  \noindent Betrakta gränsvärdet:

  \begin{equation*}
    \begin{gathered}
      r = \lim_{n\to\infty}\left|\dfrac{c_n}{c_{n+1}}\right|
    \end{gathered}
  \end{equation*}
  \par\bigskip
  \noindent Om gränsvärdet existerar (och är ändligt) så är $r$ konvergensradien för serien.\par
  \noindent Om gränsvärdet $\infty$ så är konvergensradien $\infty$
\end{theo}
\par\bigskip
\begin{prf}[Konvergensradie]{prf:convrad}
  \noindent Använd kvottest på $\sum_{n=0}^{\infty}a_n$ där $a_n = c_n(x-a)^n$ där kvottestet för en serie $\sum a_n$ görs genom att beräkna:

  \begin{equation*}
    \begin{gathered}
      L = \lim_{n\to\infty}\left|\dfrac{a_{n+1}}{a_n}\right|
    \end{gathered}
  \end{equation*}
  \par\bigskip
\noindent Om $L< 1$ har vi absolutkonvergens, om $L>1$ divergerar serien, om $L=1$ eller om gränsvärdet inte existerar så får vi inte heller någon info.

\end{prf}
\par\bigskip
\noindent Om Taylorserien för $f(x)$ konvergerar till $f(x)$ för alla $x\in I$ där $a\in I$. Då säger vi att $f$ är \textit{analytisk} på $I$ och $f$ är analytisk i $a$.
\par\bigskip
\noindent Kommentar: Begreppet analytisk (även kallat holomorf) kommer från komplex analys. Det är ett \textbf{väldigt} starkt begrepp. Vi kommer använda lite verktyg från komplex analys utan att bevisa dem. De flesta funktioner vi har stött på är analytiska.
\par\bigskip
\noindent Exempel på analytiska funktioner som är analytiska överallt: $\sin(x), \cos(x), e^x$. Exempel på analytiska funktioner som är analytisk nästan överallt är $\dfrac{1}{x}$ för alla $x\neq0$. Exempel på en icke-analytisk funktion är $\left|x\right|$ i $\C$
\par\bigskip
\noindent Om $f, g$ är analytiska så är $f+g$ och $f\cdot g$ analytisk och $\dfrac{f}{g}$ om $g\neq0$. Den kan dock vara analytisk även om $g=0$:
\par\bigskip

\begin{lem}[När $g=0$ ger analytisk funktion]{lem:go}
  \noindent Om $f, g$ är analytiska funktioner i $x = a$ då är $\dfrac{f(x)}{g(x)}$ analytisk i $x=a$ omm $\lim_{x\to a}\dfrac{f(x)}{g(x)}$ existerar (är ändligt).
\end{lem}
\par\bigskip
\noindent Exempelvis $\dfrac{\sin(x)}{x}$ är analytsik för alla $x$. Om $x\neq  0$ är det enkelt att se ty det är ett standardgränsvärde som är lika med 1.
\par\bigskip
\noindent Hur kan vi använda dessa potensserier för ODE:er? Vi måste derivera givetvis! Vi kan begga följande sats:
\par\bigskip

\begin{theo}
  OOm serierepresentationen $f(x) = \sum_{n=0}^{\infty}c_n(x-a)^n$ konvergerar på ett öppet intervall $I$ (som innehåller $a$), då är $f$ glatt (oändligt differentierbar) och derivatan ges av att differentiera termvis, dvs:


  \begin{equation*}
    \begin{gathered}
      f^{\prime}(x) = \sum_{n=0}^{\infty}nc_n(x-a)^{n-1}\\
      f^{\prime\prime} = \sum_{n=0}^{\infty}n(n-1)(x-a)^{n-2}
    \end{gathered}
  \end{equation*}
  \par\bigskip
  \noindent Derivatornas potensserie konvergerar på \textit{samma} intervall $I$. 
\end{theo}
\par\bigskip
\noindent En annan sats som kanske känns lite löjlig men som behövs:
\par\bigskip

\begin{theo}[Identitetsprincipen]{thm:idprin}
  \noindent Om:

  \begin{equation*}
    \begin{gathered}
      \sum_{n=0}^{\infty}a_nx^n = \sum_{n=0}^{\infty}b_nx^n \text{ för } x\in I
    \end{gathered}
  \end{equation*}
  \noindent Då är $a_n = b_n$.\par
  \noindent Om:

  \begin{equation*}
    \begin{gathered}
      \sum_{n=0}^{\infty}a_nx^n = 0
    \end{gathered}
  \end{equation*}
  \noindent Så är $a_n = 0$ för alla $n$

\end{theo}

\subsection{Potensserie-metoden}\hfill\\
\noindent Allt tidigare var bara setup, so let's kick it!
\par\bigskip
\noindent Idé: Ansätt $y(x) = \sum_{n=0}^{\infty}c_nx^n$ i ekvationen och försök bestämma $c_n$
\par\bigskip
\noindent Exempel:

\begin{equation*}
  \begin{gathered}
    y^{\prime} +2y = 0
  \end{gathered}
\end{equation*}
\noindent Denna går att lösa på andra sätt men för sakens skull, låt oss visa hur vi gör med potensserier:


\begin{equation*}
  \begin{gathered}
    \text{Ansätt} y(x) = \sum_{n=0}^{\infty}c_nx^n \rightarrow y^{\prime}(x)=\sum_{n=0}^{\infty}nc_nx^{n-1}\\
    \rightarrow \sum_{n=1}^{\infty}nc_nx^{n-1} + 2\sum_{n=0}^{\infty}c_nx^n = 0
  \end{gathered}
\end{equation*}
\par\bigskip
\noindent Nu vill vi skriva om för att få båda serier med $x^n$:


\begin{equation*}
  \begin{gathered}
    \sum_{n=1}^{\infty}nc_nx^{n-1}=1\cdot c_1\cdot x^0 + 2c_2x^1 + \cdots = \sum_{n=0}^{\infty}(n+1)c_{n+1}x^n\\
    \rightarrow \sum_{n=0}^{\infty}(n+1)c_{n+1}x^n + 2\sum_{n=0}^{\infty}c_nx^n =0\\
    \Lrarr\sum_{n=0}^{\infty}\left((n+1)c_{n+1}+2c_n\right)x^n=0
  \end{gathered}
\end{equation*}
\par\bigskip
\noindent Nu kan vi använda identitetsprincipen som säger att alla termer måste var lika med noll, altså:


\begin{equation*}
  \begin{gathered}
    (n+1)c_{n+1}+2c_n=0
  \end{gathered}
\end{equation*}
\par\bigskip
\noindent Från det här får vi en \textit{rekursiv relation}:


\begin{equation*}
  \begin{gathered}
    c_{n+1}=\dfrac{-2c_n}{n+1}
  \end{gathered}
\end{equation*}
\par\bigskip
\noindent Om vi har något $c_0$ så kan vi få $c_n$ för alla $n\geq1$. Notera att $c_0$ är vår parameter, den kan exempelvis bestämmas av initialvillkoret. I vårat enkla exempel så kan vi få en explicit formel för att räkna $c_n$:


\begin{equation*}
  \begin{gathered}
    c_1 = \dfrac{-2c_0}{1}\\
    c_2 = \dfrac{2^2c_0}{2!}\\
    c_3 = \dfrac{-2^3c_0}{3!}\\
    \rightarrow c_n=\dfrac{(-1)^n2^nc_0}{n!}
  \end{gathered}
\end{equation*}
\par\bigskip
\noindent Så:


\begin{equation*}
  \begin{gathered}
    y(x) = \sum_{n=0}^{\infty}\dfrac{(-1)^n2^nc_0}{n!} = c_0\sum_{n=0}^{\infty}\dfrac{(-2x)^n}{n!} = c_0e^{-2x}
  \end{gathered}
\end{equation*}
\par\bigskip
\noindent Då är frågan var denna serie konvergerar. I detta fall konvergerar den överallt ty vi hittade en taylorekvivalens men i andra fall får man kika närmare på konvergensradien. Alternativt skulle vi kunna räkna ut konvergensradien direkt från $c_{n+1} = \dfrac{-2c_n}{n+1}$:


\begin{equation*}
  \begin{gathered}
    \lim_{n\to\infty}\left|\dfrac{c_n}{c_{n+1}}\right| = \lim_{n\to\infty}\left|\dfrac{c_n}{\dfrac{-2c_n}{n+1}}\right| = \lim_{n\to\infty}\dfrac{n+1}{2}\rightarrow\infty
  \end{gathered}
\end{equation*}
\par\bigskip
\noindent Denna metod fungerar för linjära ekvationer men är relevant för grad $\geq2$
\par\bigskip
\noindent Vi fokuserar på grad 2 och homogena ekvationer (den funkar bra för inhomogena, men det blir lite mer räkningar), alltså:


\begin{equation*}
  \begin{gathered}
    A(x)y^{\prime\prime}+B(x)y^{\prime}+C(x)y=0
  \end{gathered}
\end{equation*}
\par\bigskip
\noindent För att detta skall funka kommer vi behöva anta att $A, B, C$ är analytiska runt en punkt vi vill lösa ekvationen. I många fall kommer de vara polynom vilket är analytiska funktioner. I många fall kommer vi dela bort $A$, så vi får:


\begin{equation*}
  \begin{gathered}
    y^{\prime\prime}+p(x)y^{\prime}+q(x)y =0\\
    p(x) = \dfrac{A(x)}{B(x)}\\
    q(x) = \dfrac{C(x)}{A(x)}
  \end{gathered}
\end{equation*}
\par\bigskip
\noindent Detta funkar när $A\neq0$ men det kan fungera ändå, så länge $p, q$ är analytiska.
\par\bigskip

\begin{theo}
  Om $p(x)=\dfrac{B(x)}{A(x)}$ och $q(x)=\dfrac{C(x)}{A(x)}$ är analytiska i punkten $x=a$, då kallas punkten för en ordinär punkt. Annars kallas punkten för en \textit{singulär punkt} (eller buspunkt som kidsen säger).
\end{theo}
\par\bigskip
\noindent De flesta fall vi kommer jobba med kommer handla om singulära punkter. 
\par\bigskip
\noindent Exempel: $xy^{\prime\prime}+y^{\prime}+xy = 0$, vi vill kolla var den är singulär resp. ordinär:


\begin{equation*}
  \begin{gathered}
    \rightarrow y^{\prime\prime}+\dfrac{1}{x}y^{\prime}+y=0\\
    p(x) = \dfrac{1}{x}\text{, } q(x)= 1
  \end{gathered}
\end{equation*}
\par\bigskip
\noindent Här är $x=0$ en singulär punkt ty $p(x)=\dfrac{1}{x}$ inte är analytisk. Alla andra punkter däremot är ordinära.
\par\bigskip
\noindent Exempel: $xy^{\prime\prime}+\sin(x)y^{\prime}+x^2y=0$:


\begin{equation*}
  \begin{gathered}
    \rightarrow y^{\prime\prime}+\dfrac{\sin(x)}{x}y^{\prime}+xy=0\\
    p(x) = \dfrac{\sin(x)}{x}\text{, } q(x) = x
  \end{gathered}
\end{equation*}
\par\bigskip
\noindent Här är det uppenbart att $q(x)$ är analytisk, men $p(x)$ är inte lika uppenbar. Kikar vi närmare inser vi att man skulle kunna tro att $x=0$ är en singularitet men det är bara ett standardgränsvärde, alltså inga singularitetspunkter, alltså alla punkter är ordinära. 
\par\bigskip
\noindent Notera: Även om am vi har singulära punkter så finns det lösningar, lösningarna kommer i allmänhet inte vara analytiska.
\pagebreak
\begin{theo}
  Om $a$ är en ordinär punkt till $A(x)y^{\prime\prime}+B(x)y^{\prime}+C(x)y=0$, då har ODE:n 2 linjärt oberoende lösningar på formen:

  \begin{equation*}
    \begin{gathered}
      y(x) = \sum_{n=0}^{\infty}c_n(x-a)^n
    \end{gathered}
  \end{equation*}
  \par\bigskip
  \noindent Konvergensradien är åtminstonde lika stor som avståndet till närmsta singulära punkt. Man måste dock även ta hänsyn till komplexa singulära punkter.
\end{theo}
\par\bigskip
\noindent Exempel: Betrakta ekvationen $(x^2+9)y^{\prime\prime}+xy^{\prime}+x^2y=0$:


\begin{equation*}
  \begin{gathered}
    \rightarrow y^{\prime\prime}+\dfrac{x}{x^2+9}+\dfrac{x^2}{x^2+9}y=0
  \end{gathered}
\end{equation*}
\par\bigskip
\noindent Denna har inga reella singularitetspunkter, men den har 2 komplexa, närmare bestämt $x = \pm3i$. En serie lösning på formen:

\begin{equation*}
  \begin{gathered}
    \sum_{n=0}^{\infty}c_nx^n
  \end{gathered}
\end{equation*}
\par\bigskip
\noindent Kommer den ha konvergensradie $\geq$ 3. I en annan punkt exempelvis $\sum c_n(x-4)^n$ får vi $r\geq d(4,\pm3i)= \sqrt{3^2+4^2}=5$




