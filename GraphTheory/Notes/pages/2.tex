\section{Simple graphs}\par
\noindent The idea of simple graphs is to forbid parallell edges and loops. Here, we dont care how things are connected, but which things that \textit{are}.
\par\bigskip
\noindent For example, we can encode the game \textit{Towers of Hanoi} as a simple graph by letting $n$ disks be stacked on $3 pegs$ \par
\noindent We can therefore encode a game state by a string of length $n$
\par\bigskip
\begin{theo}[Simple graph]{thm:simplegraph}
  A \textit{simple graph} is a multigraph  without parallell edges or loops\par
  \noindent An equivalent definition, it is a pair $G = (V,E)$ where $E\subseteq\mathcal{P}_2(V)$
  \par\bigskip
  \noindent By $\mathcal{P}_2$ we mean the powersets of size 2:
  \begin{equation*}
    \begin{gathered}
    \mathcal{P}_2(V) = \left\{A\subseteq V\;|\; \left|A\right|=2\right\}
    \end{gathered}
  \end{equation*}
  \par\bigskip
  \noindent\textbf{Anmärkning:}\par
  \noindent Our $\iota$ is gone! This is because by not having parallell edges and loops, then $\iota:E\to\mathcal{P}_2(V)$ is injective and we can identify the output of $\iota$ with its input, and that is what the definition of a simple graph is 
\end{theo}
\par\bigskip
\noindent\textbf{Anmärkning:}\par
\noindent Every graph is a multigraph
\par\bigskip
\begin{lem}
  AAny simple graph on $n$ vertices has at most $\begin{pmatrix}n\\2\end{pmatrix}$ edges
\end{lem}
\par\bigskip
\begin{prf}
  TThe edge set $E\subseteq\mathcal{P}_2(V)$ and $\left|\mathcal{P}_2(V)\right|=\begin{pmatrix}n\\2\end{pmatrix}$
\end{prf}
\par\bigskip
\noindent\textbf{Anmärkning:}\par
\noindent This implies that simple graphs are finite. In multigraphs, we could put arbitrary edges between vertices, but here it is not accepted. 
\par\bigskip
\noindent Our vertex set is arbitrary, it doesnt matter if $V=\left\{1,2,3,4\right\}$ or $V = \left\{a,b,c,d\right\}$, we need to set up a notion of "sameness" in graphs taking into account that the vertex set is arbitrary.
\par\bigskip
\begin{theo}[Labelled graph]{thm:labelledgraph}
A \textit{labelled graph} is a simple graph with a fixed vertex set, commonly $V = \left\{1,2,\cdots,n\right\}$ if $V$ is finite.
\end{theo}
\newpage
\begin{lem}
  TThere are $2^{\begin{pmatrix}n\\2\end{pmatrix}}$ labelled graphs on $n$ vertices.
\end{lem}
\par\bigskip
\begin{prf}
  SSince $V = \left\{1,2,\cdots,n\right\}$ is fixed, two graphs $(V,E)$ and $(V,E^{\prime})$ coincide iff $E=E^{\prime}$
  \par\bigskip
  \noindent Conversely, any subset of $\mathcal{P}_2(V)$ defines an edge set. We are essentially looking for $\mathcal{P}(\mathcal{P}_2(V))$, and the cardinality of this is $2^{\left|\mathcal{P}_2(V)\right|} = 2^{\begin{pmatrix}n\\2\end{pmatrix}}$
\end{prf}
\par\bigskip
\begin{theo}[Morphism]{thm:morphism}
  Let $G = (V,E)$ and $G^{\prime} = (V^{\prime}, E^{\prime})$ be simple graphs.
  \par\bigskip
  \noindent A \textit{morphism}
  \begin{equation*}
    \begin{gathered}
      \varphi:G\to G^{\prime}
    \end{gathered}
  \end{equation*}\par
  \noindent is a map
  \begin{equation*}
    \begin{gathered}
      \varphi:V\to V^{\prime}
    \end{gathered}
  \end{equation*}\par
\noindent such that $\left\{v,w\right\}\in E\Rightarrow \left\{\varphi(v),\varphi(w)\right\}\in E^{\prime}$
\end{theo}
\par\bigskip
\noindent\textbf{Example:}\par
\noindent See example 15
\par\bigskip
\noindent\textbf{Anmärkning:}\par
\noindent Graph-morphisms do not need to be injective/surjective, nor do they need to exist
\par\bigskip
\noindent Graph-morphisms preserve edges between graphs, thats their whole point
\par\bigskip
\begin{theo}[Identity morphism]{thm:idgmorph}
  For every simple graph $G$, theere is an identity morphism $id_G:G\to G$ where $id_G:V\to V$ is the identity map
  \par\bigskip
  \noindent For simple graphs $G,G^{\prime}, G^{\prime\prime}$ and morphisms
  \begin{equation*}
    \begin{gathered}
      \varphi:G\to G^{\prime}\\
      \varphi^{\prime}:G^{\prime}\to G^{\prime\prime}
    \end{gathered}
  \end{equation*}\par
  \noindent There is a morphisms $\varphi^{\prime}\circ\varphi:G\to G^{\prime\prime}$, given by the map $\varphi^{\prime}\circ\varphi:V\to V^{\prime\prime}$
\end{theo}
\par\bigskip
\begin{theo}[Isomorphism]{thm:isomorphism}
Two graphs $G,G^{\prime}$ are \textit{isomorphic} if there is a bijective morphism $\varphi:V\to V^{\prime}$ and $\left\{v,w\right\}\in E\Lrarr \left\{\varphi(v),\varphi(w)\right\}\in E^{\prime}$
  \par\bigskip
  \noindent Another way of saying this there is $\varphi:G\to G^{\prime}$ and a $\psi:G^{\prime}\to G$ such that $\varphi\circ\psi = id_{G^{\prime}}$ and $\psi\circ\varphi = id_G$
\end{theo}
\par\bigskip
\noindent\textbf{Anmärkning:}\par
\noindent Isomorphic graphs are not necessarily the same if they are labelled. We need to make sure the degree of each vertice coincide, and that we dont lose any edges.
\par\bigskip
\begin{theo}
  TThe number $g_n$ of non-isomorphic simple graphs on $n$ vertices satisfies the following:\par
  \begin{itemize}
    \item $g_n = \dfrac{2^{\begin{pmatrix}n\\2\end{pmatrix}}}{n!}\left(1+\dfrac{n^2-n}{2^{n-1}}+\dfrac{8n!}{(n-4)!}\cdot\dfrac{(3n-7)(3n-9)}{2^{2n}}+\mathcal{O}\left(\dfrac{n^5}{2^{5n/2}}\right)\right)$
  \end{itemize}
  \par\bigskip
  \noindent In particular, $g_n$ behaves asymptotically as $2^{\begin{pmatrix}n\\2\end{pmatrix}}/n!$ in the same way that the probability distribution $Hyp$ becomes $Bin$ for large populations. When we make lots of graphs, eventually, the number of graphs that are isomorphic are so small they dont matter in the grand scheme.
\end{theo}
\par\bigskip
\subsection{Special graphs}\hfill\\\par
\noindent Some graphs are so special that they are given special names:\par
\begin{itemize}
  \item The complete graphs on $n$ vertices, denoted by $K_n$. All $\begin{pmatrix}n\\2\end{pmatrix}$ edges are present (every vertex is a neighbour of everything else)
  \item The path graph of length $l$, denoted by $P_l$ is just a regular path as a graph
  \item The cycle graph on $n$ vertices, denoted by $C_n$ ($n\geq3$)
  \item The complete bipartite graphs, denoted $K_{a,b}$. Here, $V$ is partitioned as the disjoint union $V = V_a\cup V_b$. This means $\left|V\right| = a+b$. There are no edges between two vertices in the same set, but all possible edges are between the two sets.\par
    Notice that $K_{a,b}\cong K_{b,a}$
  \item The complete $r$-partite graphs $K_{a_1,\cdots,a_r}$ has a vertex set $V = \bigcup_{i=1}^{r}V_{a_i}$ such that $\left|V_{a_i}\right| = a_i$\par
    We say that two vertices are neighbours iff they are in different sets.
\end{itemize}
\par\bigskip
\begin{lem}
  TThe complete $r$-partite graph $K_{a_1,\cdots,a_r}$ on $n$ vertices (sum of all $a_i = n$) has
  \begin{equation*}
    \begin{gathered}
      \left|E\right| = \dfrac{1}{2}(n^2-a_1^2-\cdots-a_r^2)
    \end{gathered}
  \end{equation*}
\end{lem}
\par\bigskip
\begin{prf}
  AA vertex in set $V_{a_i}$ has $n-a_i$ neighbours\par
  \noindent By the Handshake lemma, $2\left|E\right|= \sum_{v\in V}\text{deg}(v) = \sum_{i=1}^{r}a_i(n-a_i) = n\sum_{}^{}$
  \begin{equation*}
    \begin{gathered}
      2\left|E\right|= \sum_{v\in V}\text{deg}(v) = \sum_{i=1}^{r}a_i(n-a_i) = n\underbrace{\sum_{i=1}^{r}a_i}_{\text{$=n$}}-\sum_{i=1}^{r}a_i^2\\
      n^2-a_1^2-\cdots-a_r^2
    \end{gathered}
  \end{equation*}
\end{prf}
\par\bigskip
\noindent\textbf{Anmärkning:}\par
\noindent $K_{a,b}$ has $\dfrac{1}{2}(n^2-a^2-b^2) = ab$ edges and $K_n = K_{1,\cdots,1}$ has $\dfrac{1}{2}(n^2-n) = \begin{pmatrix}n\\2\end{pmatrix}$ edges
\par\bigskip
\begin{theo}[Subgraph]{thm:subgraph}
  Let $G = (V,E)$ be a simple graph.\par
  \noindent A simple graph $H = (V^{\prime}, E^{\prime})$ is a \textit{subgraph} of $G$ if $V^{\prime}\subseteq V$ and $E^{\prime}\subseteq E$
\end{theo}
\par\bigskip
\begin{theo}[Induced subgraph]{thm:inducedsubgraph}
  An \textit{induced} subgraph, is a subgraph $H = (V^{\prime}, E^{\prime})$ of $G$, such that $E^{\prime} = \left\{\left\{x,y\right\}\in E\;|\; x,y,\in V^{\prime}\right\}$
  \par\bigskip
  \noindent Denoted by $H = G[V^{\prime}]$
\end{theo}
\par\bigskip
\begin{theo}[Edge-induced subgraph]{thm:edgeinduced}
  An \textit{edge-induced} subgraph is a subgraph $H = (V^{\prime}, E^{\prime})$ such that $V^{\prime} = \left\{v \in V\;|\; v\text{ is incident to some } e\in E^{\prime}\right\}$
  \par\bigskip
  \noindent Denoted by $H = G<E^{\prime}>$
\end{theo}
\par\bigskip
\begin{theo}[Spanning subgraph]{thm:spangraph}
  A subgraph $H = (V^{\prime}, E^{\prime})$ of $G$ is a \textit{spanning} subgraph if $V^{\prime} = V$
\end{theo}
\par\bigskip
\noindent\textbf{Anmärkning:}\par
\noindent There is a way to extend this into multigraphs, but you need to find a way to take care of $\iota$ 
