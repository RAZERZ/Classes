\section{Max-flow-min-cut theorem}
\par\bigskip
\begin{theo}[Directed graph]{thm:directed}
  A \textit{directed simple graph} is a pair $G = (V,E)$ where $E = \subseteq V\times V\backslash\left\{v,v\;|\; v\in V\right\}$ 
  \par\bigskip
  \noindent We interpret an edge $v,w$ as an arrow pointing of $v$ to $w$ 
\end{theo}
\par\bigskip
\noindent\textbf{Anmärkning:}\par
\noindent In a directed graph $(v,w)\neq(w,v)$
\par\bigskip
\begin{theo}[Flow network]{thm:flownetwork}
  A \textit{flow network} consists of a directed simple graph $G = (V,E)$ together with a weight function $w:E\to(0,\infty)$ (to every edge we assign a weight) and two distringuished vertices, a \textit{source} $s$ and a \textit{sink}  $t$
\end{theo}
\par\bigskip
\begin{theo}[Capacity function]{thm:capfunc}
  The function $c:V\times V\to[0,\infty)$ defined by:
  \begin{equation*}
    \begin{gathered}
      c(v,v^{\prime}) = \begin{cases}w(v,v^{\prime})\quad\text{if } (v,v^{\prime})\in E\\0\quad\text{otherwise}\end{cases}
    \end{gathered}
  \end{equation*}
  \par\bigskip
  \noindent is called the \textit{capacity function} of the flow network
\end{theo}
\par\bigskip
\noindent\textbf{Anmärkning:}\par
\noindent We do not want an edge from $v$ to $w$ and an edge from $w$ to $v$, we can do a simplification:
\begin{figure}[ht!]
    \centering
    \begin{tikzpicture}
      \node[state](p0){$v$};
      \node[state, right of=p0, xshift=1cm](p1){$v^{\prime}$};
      \path[-stealth] (p0) edge[above, bend left] node{$w$} (p1);
      \path[-stealth] (p1) edge[below, bend left] node{$w^{\prime}$} (p0);
    \end{tikzpicture}
    \caption{}
\end{figure}
\par\bigskip
\noindent Is replaced by:
\begin{figure}[ht!]
    \centering
    \begin{tikzpicture}
      \node[state](p0){$v$};
      \node[state, right of=p0, xshift=1cm, yshift=.5cm](p1){$v^{\prime}$};
      \node[state, right of=p1, xshift=1cm, yshift=-.5cm](p2){$v^{\prime\prime}$};
      \path[-stealth] (p0) edge[above, bend left] node{$w$} (p1);
      \path[-stealth] (p1) edge[above, bend left] node{$w$} (p2);
      \path[-stealth] (p2) edge[below, bend left] node{$w^{\prime}$} (p0);
    \end{tikzpicture}
    \caption{}
\end{figure}
\newpage
\begin{theo}[Flow function]{thm:flow}
  A flow $f$ on the flow network $G = (V,E)$ with capacity function $c$ is a function $f:V\times V\to [0,\infty)$ such that:\par
  \begin{itemize}
    \item\textbf{Capacity constraint} I cannot pump more water through the pipe than I have capacity
      \begin{equation*}
        \begin{gathered}
          f(v,v^{\prime})\leq c(v,v^{\prime})\quad\forall v,v^{\prime}\in V
        \end{gathered}
      \end{equation*}
      \par\bigskip
    \item\textbf{Conservation constraint} However much water I pump in, I should get out (no loss and no creation out of nothing).
      \par\bigskip
    \noindent For $v\in V\backslash\left\{s,t\right\}$ we have:
    \begin{equation*}
      \begin{gathered}
        \sum_{x\in V}f(x,v) = \sum_{x\in V}f(v,x)
      \end{gathered}
    \end{equation*}
  \end{itemize}
  \par\bigskip
\end{theo}
\par\bigskip
\begin{theo}[Value of flow]{thm:valoflfow}
  The \textit{value} $\left|f\right|$ of the flow is the total out flow at the source:
  \begin{equation*}
    \begin{gathered}
      \left|f\right| = \sum_{x\in V}f(s,x)-\sum_{x\in V}f(x,s)
    \end{gathered}
  \end{equation*}
\end{theo}
\par\bigskip
\noindent\textbf{Anmärkning:}\par
\noindent From the conservation consraint, this difference (value) must go away somewhere and the only place it can go is in the sink.
\par\bigskip
\noindent This means that the value is also equal to the total inflow at the sink, and therefore equal to:
\begin{equation*}
  \begin{gathered}
    \sum_{x\in V}f(x,t)-\sum_{x\in V}f(t,x)
  \end{gathered}
\end{equation*}
\par\bigskip
\noindent How do we know this value is positive? Well, if it is negative, then by the identity above we can swap $s,t$ to ensure we have a positive value.
\par\bigskip
\begin{theo}[$s-t$-cut]{thm:stcut}
  Let $G = (V,E)$ be a flow network with source $s$ and sink $t$ and capacity $c$.
  \noindent An $s-t$-\textit{cut} is a partition of $V$ into sets $S,T$ such that $s\in S$ and $t\in T$ 
\end{theo}
\par\bigskip
\begin{theo}[Capacity of cut]{thm:capcut}
  The \textit{capacity of the cut}, denoted by $c(S,T)$ and is:
  \begin{equation*}
    \begin{gathered}
      c(S,T) = \sum_{(v,v^{\prime})\in S\times T}c(v,v^{\prime})
    \end{gathered}
  \end{equation*}
\end{theo}
\par\bigskip
\noindent\textbf{Anmärkning:}\par
\noindent FOr any flow $f$ and any $s-t$-cut $S,T$, we have that the capacity of the cut is an upperbound to the value of the flow:
\begin{equation*}
  \begin{gathered}
    c(S,T)\geq \left|f\right|
  \end{gathered}
\end{equation*}
\par\bigskip
\noindent We will show that we only have equality in the most extreme case (cut capacity is as high/low as possible)
\newpage
\begin{lem}
  LLet $f$ be a flow on the flow network $G = (V,E)$ and let $S,T$ be an $s-t$ cut of $G$
  \par\bigskip
  \noindent Assume that the value of the flow is equal to the capacity of the cut:
  \begin{equation*}
    \begin{gathered}
      \left|f\right| = c(S,T)
    \end{gathered}
  \end{equation*}
  \par\bigskip
  \noindent Then $\left|f\right|$ is maximal among all flows on $G$, and the capacity of the cut is minimal among all $S,T$ cuts 
\end{lem}
\par\bigskip
\begin{prf}
  AWe need to show there is no flow with strictly larger value and no cut with strictly smaller capacity
  \par\bigskip
  \noindent If $f^{\prime}$ is a flow with $\left|f\right|\geq\left|f\right|$, then it needs to pass through the $S,T$ cut as well.
  \par\bigskip
  \noindent Hence, $\left|f^{\prime}\right|\leq c(S,T) = \left|f\right|\Rightarrow \left|f^{\prime}\right| = \left|f\right|$ and $\left|f\right|$ is maximal
  \par\bigskip
  \noindent Similarly, any $s-t$-cut $S^{\prime},T^{\prime}$ needs to satisfy $c(S^{\prime},T^{\prime})\geq \left|f\right| = c(S,T)$, hence if $c(S^{\prime},T^{\prime})\leq c(S,T)$ then $c(S,T) = c(S^{\prime},T^{\prime})$ and $c(S,T)$ is minimal
\end{prf}
\par\bigskip
\begin{theo}[Residual network]{thm:resnet}
  Let $G = (V,E)$ be a flow network with source $s$ and sink $t$
  \par\bigskip
  \noindent Let $f$ be a flow on $G$.
  \par\bigskip
  \noindent The \textit{residual network} $G_f$ is the flow network with residual capacity $c_f$ set as follows:\par
  \begin{itemize}
    \item For $u,v\in V$, set:\par
      \begin{equation*}
        \begin{gathered}
          c_f(u,v) = \begin{cases}c(u,v)-f(u,v)\quad\text{if } (u,v)\in E\\f(v,u)\quad\text{if } (v,u)\in E\\0\quad\text{otherwise}\end{cases}
        \end{gathered}
      \end{equation*}
  \end{itemize}
\end{theo}
\par\bigskip
\noindent This defines a capacity function. Let $E_f$ be the set of pairs $(u,v)$ for which we have some positive capacity ($c_f(u,v)>0$) 
\par\bigskip
\noindent\textbf{Example:}\par
\noindent Lets look at a flow network $G = (V,E)$ which looks as follows:
\begin{figure}[ht!]
    \centering
    \begin{tikzpicture}
      \node[state](p0){$s$};
      \node[state, right of=p0, xshift=1cm](p1){};
      \node[state, below of=p0, yshift=-1cm](p2){};
      \node[state, below of=p1, yshift=-1cm](p3){$t$};
      \path[-stealth] (p0) edge[above] node{$2/3$} (p1);
      \path[-stealth] (p0) edge[left] node{$0/3$} (p2);
      \path[-stealth] (p1) edge[below] node{$1/1$} (p2);
      \path[-stealth] (p1) edge[right] node{$1/2$} (p3);
      \path[-stealth] (p2) edge[below] node{$1/3$} (p3);
    \end{tikzpicture}
    \caption{}
\end{figure}\par
\par\bigskip
\begin{theo}[Path]{thm:pathh}
  Let $G_f$ be the residual network of a flow network $G$ with respect to a flow $f$
  \par\bigskip
  \noindent A $v_0-v_k$ is a sequence $v_0e_1\cdots e_kv_k$ where $v_0,v_1,\cdots,v_k$ are distinct vertices and all edges point in the same direction, namely $e_i = (v_{i-1},v_i)$  for $i = 1,\cdots,k$
  \par\bigskip
\end{theo}
\par\bigskip
\begin{theo}[Augmenting path]{thm:augpath}
  An $s-t$-path is called an \textit{augmenting path}
\end{theo}
\par\bigskip
\begin{theo}[Capacity of path]{thm:cappath}
  We can define the capacity of a path $P$ by $c(P) = \min_{e\in E(P)}c(e)$
\end{theo}
\par\bigskip
\noindent\textbf{Anmärkning:}\par
\noindent We are working in the residual network, which means by definition that all our capacities are positive, and therefore $c(P)>0$  for any augmenting path $P$
\par\bigskip
\begin{lem}
  ALet $f$ be a flow in $G$  sich that $G_f$ admits an augmenting path $P$
  \par\bigskip
  \noindent Then $f^{\prime}$, defined by:
  \begin{equation*}
    \begin{gathered}
      f^{\prime}(u,v) = \begin{cases}f(u,v)+c_f(P)\quad\text{if }(u,v)\text{ is an edge in }P\\f(u,v)-c_f(P)\quad\text{if }(v,u)\text{ is an edge in }P\\f(u,v)\quad\text{otherwise}\end{cases}
    \end{gathered}
  \end{equation*}
  \par\bigskip
  \noindent for all edges $(u,v)\in E$ (in the original network) is a flow on $G$ with $\left|f^{\prime}\right| = \left|f\right|+c_f(P)>\left|f\right|$ 
\end{lem}
\newpage
\begin{theo}[Ford-Fulkerson]{thm:fordfolk}
  Let $f$ be a flow on the network $G$.\par
  \noindent The following are equivalent:\par
  \begin{itemize}
    \item $f$ is a maximum flow (the value of the flow is maximum)\par
    \item The residual network $G_f$ contains no augmenting path\par
    \item There is an $s-t$-cut $S,T$ with capacity $c(S,T) = \left|f\right|$\par
      \noindent In particular, $\max_{f \text{ flow}}\left|f\right| = \min_{S,T\; s-t-cut}c(S,T)$ 
  \end{itemize}
\end{theo}
\par\bigskip
\begin{prf}
  A
  \begin{itemize}
    \item\textbf{Third implies first}\par
      Already shown above
      \par\bigskip
    \item\textbf{First implies second}\par
      Shown in lemma
      \par\bigskip
    \item\textbf{Second implies third}\par
      Assume $G_f$ does \textit{not} contain an augmenting path $P$. This means I cant walk from $s$ to $t$ in the residual network while only going along the direction of the edges I have.
      \par\bigskip
      Define $S = \left\{v\in V\;|\; v\text{ can be reached by a } s-v-\text{path in } G_f\right\}$\par
      Define $T = V\backslash S$
      \par\bigskip
      Then $s\in S$ and $t\in T$ (otherwise there is an augmenting path in $G_f$)\par
      In particular, these sets define an $s-t$-cut
      \par\bigskip
      By construction, every arrow $(u,v)$ in $G$ where $u\in S$ and $v\in T$, must have its full capacity used by $f$.\par
      Otherwise, in the construction of the residual network, there would still be an arrow $\cdots$ $(u,v)\in G_f$ and $v\in S$
      \par\bigskip
      Moreover, every arrow in the direction $(v,u)$  must have flow 0 (reason is the same)
      \par\bigskip
      Hence, we have that $\left|f\right|\geq c(S,T)$, but we have seen earlier that $c(S,T)\leq \left|f\right|$, which means we have equality
      \par\bigskip
  \end{itemize}
\end{prf}
\par\bigskip
\noindent\textbf{Anmärkning:}\par
\noindent This theorem makes no existence of a maximal flow, it only talks about what happens when we \textit{do} have a maximal flow which is a bit unsatisfying.
\par\bigskip
\noindent A maxflow does however exist! Regard a flow $f$ simply as a vector in $\R^{\left|E\right|}$\par
\noindent Then the $e$-th coordinate of the vector is from $[0,c(e)]$\par
\noindent This means that the set of all flows is closed and bounded (therefore compact in $\R^{\left|E\right|}$) (and therefore has a supremum)\par
\noindent The map $f\mapsto\left|f\right|$ is continous.
\par\bigskip
\noindent We now have a continous map on a compact set. We now know from calcul\textit{us}, there is a $f$ with maximum value $\left|f\right|$ 
\par\bigskip
\noindent What happens if all my capacities are integers?
\par\bigskip
\begin{theo}[Integer flow theorem]{thm:intft}
  If $G$ is a network with an integer-valued capacity function $c:E\to\Z_{>0}$, then there is an integer valued maximum flow. (It only assigns integers flow along every edge)
\end{theo}
\par\bigskip
\noindent\textbf{Algorithm [Ford-Fulkeson]}\par
\noindent Given a flow network $G$, start with the flow that is 0 everywhere $f\equiv 0$
\par\bigskip
\noindent Repeat the following steps:\par
\begin{itemize}
  \item Construct $G_f$\par
  \item Find na augmenting path $P$ in $G_f$\par
    Use it to construct a new flow with larger value\par
    If no augmenting path exists, stop.
\end{itemize}
\par\bigskip
\noindent This is not really an algorithm, since it is not guaranteed to stop.\par
\noindent It will however terminate if all of our capacities are rational numbers. 
