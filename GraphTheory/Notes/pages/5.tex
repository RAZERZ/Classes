\section{Weights and Distances}\par
\begin{theo}[Weighted graph]{thm:weightgraph}
  A \textit{weighted graph} is a finite simple graph $G = (V,E)$ together with a weight function $w:E\to(0,\infty)$
  \par\bigskip
  \noindent If $H = (V^{\prime}, E^{\prime})$ is a subgraph of $G$, then its weight is defined to be $w(H) = \sum_{e\in E^{\prime}}w(e)$ 
\end{theo}
\par\bigskip
\begin{theo}[Minimum spanning tree]{thm:minspantree}
  A \textit{minimum spanning tree} (MST) is a spanning tree $T$ of $G$ such that $w(T)$ is minimal among all spanning trees in $G$ 
\end{theo}
\par\bigskip
\noindent It is not clear that an MST exists. We started with a simple graph which has finite edges and therefore we have finite subgraphs. We are therefore looking for the minimum of a set, which certainly exists, but it may not be unique.
\par\bigskip
\begin{lem}
  LLet $G$ be a connected weighted graph.
  \par\bigskip
  \noindent If $w:E\to(0,\infty)$ is injective, then the MST is unique.
\end{lem}
\par\bigskip
\begin{prf}
  AAssume $w$ is injective and suppose there are MST:s $T_1 = (V,E_1)$ and $T_2 = (V,E_2)$ with $E_1\neq E_2$
  \par\bigskip
  \noindent Then a set $D$ (not incidence matrix) $D = \left\{e\in E\;|\; e\in E_1\vee e\in E_2\text{ but not both}\right\}$
  \par\bigskip
  \noindent Pick $e\in D$ such that $w(e)$ is minimal (we can do this because all of our set is finite, so there must be a minimal weighted edge).\par
  \noindent By definition, this edge lives in one of our 2 trees.
  \par\bigskip
  \noindent WLOG, $e\in E_1$ and therefore $e\not\in E_2$
  \par\bigskip
  \noindent Add $e$ to $T_2$ creates a cycle (a tree is edge-maximal among cycle-free graphs), an don this cycle there is an edge $e^{\prime}\not\in E_1\Rightarrow e^{\prime}\in D$
  \par\bigskip
  \noindent We now have:\par
  \begin{itemize}
    \item $w(e)<w(e^{\prime})$ (by injectivity, they cannot have the same weight)\par
    \item $w(e^{\prime})\leq w(e)$ (because otherwis $T_2$ would not be a MST)
  \end{itemize}
  \par\bigskip
  \noindent Contradiction, and end of proof
\end{prf}
\par\bigskip
\noindent\textbf{Anmärkning:}\par
\noindent Lemma 6.1 is not an equivelance. If there are edges with equal weight, it does not mean our MST is unique. The best example is if we start with a tree.
\newpage
\subsection{Prims Algorithm}\hfill\\\par
\noindent Of course, let $G = (V,E)$ be a connected, simple, weighted graph.
\par\bigskip
\noindent Set $T = (\left\{v\right\}, \O)$ for any $v\in V$ (contains one arbitrary vertex from the graph and no edges)
\par\bigskip
\noindent While $T$ is not spanning, find an edge $e\in E$ between $V(T)$ (all of the vertices that are already in the tree) and $V\backslash V(T)$ (all of the vertices are not already in the tree) such that $w(e)$ is minimal.
\par\bigskip
\noindent Add this edge $e$ to $T$ together with its endpoint from $V\backslash V(T)$
\par\bigskip
\noindent The moment $T$ is spanning, we stop the algorithm.
\par\bigskip
\noindent\textbf{Example:}\par
Example 53 in lecture notes
\par\bigskip
When faced with 2 choices, pick as you want
\par\bigskip
\noindent\textbf{Anmärkning:}\par
\noindent You can get different spanning trees depending on which starting vertex you choose. Therefore, how can we know it is the MST? Theorem time! 
\par\bigskip
\begin{theo}[Prims algorithm produces an MST]{thm:pralgprod}
  \textit{Prims algorithm produces an MST}
\end{theo}
\par\bigskip
\begin{prf}
  LLet PA = Prims algorithm.\par
  \noindent PA produces a spanning subgraph, and in fact it produces a connected spanning subgraph with $n$ vertices and $n-1$ edges (by Lemma 4.2 we have a spanning tree tree $T$)
  \par\bigskip
  \noindent Let $T^{\prime}$ be a MST. We show that the weight $w(T)$ is at most ($\leq$) $w(T^{\prime})$
  \par\bigskip
  \noindent Suppose $T\neq T^{\prime}$. Consider the earliest edge $e$ that was included in $T$ but is not in $T^{\prime}$
  \par\bigskip
  \noindent Partition $V$ in 2 disjoint sets $V_1$ and $V_2$ such that $V_1$ contains all vertices that PA added to $T$ before including $e$.
  \par\bigskip
  \noindent Before including $e$, all of the edges in $T$ were in $T^{\prime}$, so if we look at $T[V_1]$ is a subgraph of $T^{\prime}$
  \par\bigskip
  \noindent $T^{\prime}$ is a tree, so there must be one edge $f\neq e$ (edge $e$ does not occur in $T^{\prime}$) in $T^{\prime}$ that connectes $V_1$ to $V_2$ \textit{in} $T^{\prime}$
  \par\bigskip
  \noindent Transform $T^{\prime}$ by adding $e$ (creates cycle) and removing $f$ (on this cycle is $f$, remove $f$ destroys cycle)
  \par\bigskip
  \noindent PA chose $e$ over $f$, this means that $w(e)\leq w(f)$
  \par\bigskip
  \noindent What happens now when $f\sim T^{\prime\prime}$ (when we rmeove $f$)? Well, the total weight should go down, so $w(T^{\prime\prime})\leq w(T^{\prime})$.
  \par\bigskip
  \noindent Since $T^{\prime}$ is an MST, this implies $w(T^{\prime\prime}) = w(T^{\prime})$
  \par\bigskip
  \noindent We can repeat this until $T^{\prime\prime} = T$. Since we have moved the latest point in which they differ (by repeating this, we essentially move this point as far back as possible)
  \par\bigskip
  \noindent Then it follows that $w(T)\leq w(T^{\prime})$ so $T$ is MST
\end{prf}
\par\bigskip
\noindent\textbf{Anmärkning:}\par
\noindent Prims algorithm is an example of a \textit{greedy algorithm}. This is because we do a locally optimal choice.
\par\bigskip
\noindent\textbf{Anmärkning:}\par
\noindent Regarding runtime, depends on the implimantation and how we choose the minimal edge. A good implimantation has a runtime of $\mathcal{O}(\left|E\right|+\left|V\right|\log(\left|V\right|))$
\par\bigskip
\subsection{Kruskals Algorithm}\hfill\\\par
\noindent We start yet again with a connected weighted graph $G = (V,E)$.
\par\bigskip
\noindent Let $S$ be a list of the edges in $E$, sorted by incfreasing order of weight (lightest edge is first)
\par\bigskip
\noindent Set $T = (V,\O)$ and do the following:
\par\bigskip
\noindent While $T$ is disconnected, delete the first entry in $S$ and att it to $T$ unless it creates a cycle (then just delete it)
\par\bigskip
\noindent\textbf{Example:}\par
Example 54 in lecture notes 
\par\bigskip
\begin{theo}[Kruskals algorithm produces a MST]{thm:kruskal}
  \textit{Kruskals algorithm produces a MST}
\end{theo}
\par\bigskip
\begin{prf}
  ADenote Kruskals algorithm with KA.
  \par\bigskip
  \noindent Notice that KA by design produces a spanning cycle free connected graph (spanning tree). Denote this by $T$
  \par\bigskip
  \noindent If $T$ is \textit{not} MST, denote by $T^{\prime}$ MST that has the maximum number of edges in common $T$ (among all MST)
  \par\bigskip
  \noindent Let $e$ be the earliest edge in $T$ that is not in  $T^{\prime}$ (\textbf{is e the first one to be deleted but not added?}) (no, first one to be included in $T$ but not in $T^{\prime}$)
  \par\bigskip
  \noindent Adding $e$ to $T^{\prime}$ creates a cycle, and this cycle contains an edge $f\neq e$ that is not in $T$
  \par\bigskip
  \noindent Modify $T^{\prime}$ by adding $e$ and removing $f$, this creates a new tree $f\sim T^{\prime\prime}$
  \par\bigskip
  \noindent Since $T^{\prime}$ is an MST, we have that $w(T^{\prime})\leq w(T^{\prime\prime})$, but at the same time, KA chose $e$ over $f$ so by removing $f$ and adding $e$, therefore $w(T^{\prime\prime})\leq w(T^{\prime})\Rightarrow w(T^{\prime}) = w(T^{\prime\prime})$ and $T^{\prime\prime}$ is MST
  \par\bigskip
  \noindent However, $T^{\prime\prime}$ has one more edge in common with $T$ than $T^{\prime}$ had. This is a contradiction since $T^{\prime}$ had the most edges in common. 
\end{prf}
\par\bigskip
\noindent\textbf{Anmärkning:}\par
\noindent A good implimantation has a runtime of $\mathcal{O}(\left|E\right|\log(\left|V\right|))$
\newpage
\noindent You can interpret edge weights, as distances.
\par\bigskip
\begin{theo}[Graph distance]{thm:graphdist}
  Let $G = (V,E)$ be a simple weighted graph with weight function $w:E\to(0,\infty)$ 
  \par\bigskip
  \noindent For vertices $v,v^{\prime}\in V$, we define the \textit{graph distance} between $v$ and $v^{\prime}$ by
  \begin{equation*}
    \begin{gathered}
      d_G(v,v^{\prime}) = \text{min}\left\{\sum_{e\in E(P)}w(e)\;|\;\text{ $P$ is a path from $v$ to $v^{\prime}$}\right\}
    \end{gathered}
  \end{equation*}
  \par\bigskip
  \noindent We set $d_G(v,v^{\prime}) = \infty$ if no path from $v$ to $v^{\prime}$ exists.
\end{theo}
\par\bigskip
\noindent\textbf{Anmärkning:}\par
\noindent For non-weighted graphs, we choose $w(e) = 1$ for all $e\in E$
\par\bigskip
\begin{lem}
  ALet $G = (V,E)$ be a connected weighted graph. Then the following statements are true:\par
  \begin{itemize}
    \item $\forall v,v^{\prime}\in V$, we have $d_G(v,v^{\prime})\geq0$ and $d_G(v,v^{\prime}) =0 \Lrarr v = v^{\prime}$
    \item $d_G(v,v^{\prime}) = d_G(v^{\prime},v)$
    \item Triangular inequality holds: $d_G(v,v^{\prime})+d_G(v^{\prime},v^{\prime\prime})\geq d_G(v,v^{\prime\prime})$
  \end{itemize}
\end{lem}
\par\bigskip
\noindent\textbf{Anmärkning:}\par
\noindent This is a set with a metric.
\par\bigskip
\begin{prf}
  First one is obvious, and so it the second
  \par\bigskip
  \noindent For the Triangular inequality, let $P$ be a path of $d_G(v,v^{\prime})$ from $v$ to $v^{\prime}$  and $Q$ be a path of length $d_G(V^{\prime},v^{\prime\prime})$ from $v^{\prime}$ to $v^{\prime\prime}$ (we know this path exists because there must be some path that realises these distances)
  \par\bigskip
  \noindent We obtain a walk from $v$ to $v^{\prime\prime}$ by concatenating $Q$ after $P$, problem is this is not a path, but erasing cycles from this walk creates a path. This only decreases the length. We found a path from $v$ to $v^{\prime\prime}$  which is shorter than the sum of the distances
  \par\bigskip
  \noindent Hence, $d_G(v,v^{\prime\prime})\leq d_G(v,v^{\prime})+d_G(v^{\prime},v^{\prime\prime})$
\end{prf}
\par\bigskip
\begin{theo}[Diameter]{thm:diameter}
  The \textit{diameter} of a weighted graph $G = (V,E)$ is the maximum distance between any 2 vertices:
  \begin{equation*}
    \begin{gathered}
    \text{diam}(G) = \text{max}\left\{d_G(v,v^{\prime})\;|\; v,v^{\prime}\in V\right\}
    \end{gathered}
  \end{equation*}
\end{theo}
\newpage
\subsection{Dijkstras Algorithm}\hfill\\\par
\noindent Given a weighted graph $G = (V,E)$, select an inital vertex $v_0$ and initialize a distance function $d(v_0,*)$ (where $*$ is the argument of our distance function and is some other vertex)  by:
\begin{equation*}
  \begin{gathered}
    d(v_0,v) = \begin{cases}0\quad v = v_0\\\infty\quad\text{otherwise}\end{cases}
  \end{gathered}
\end{equation*}
\par\bigskip
\noindent Define a set $Q$ of unvisited vertices. Initially, $Q = V$ 
\par\bigskip
\noindent Let $v_0$ be the currently visited vertex, and proceed as follows:
\par\bigskip
\noindent First, remove the current vertex $v$ from $Q$
\par\bigskip
\noindent Second, for all neighbours $v^{\prime}$ to $v$ in $Q$, check if $d(v_0,v)+w(\left\{v,v^{\prime}\right\})< d(v_0,v^{\prime})$
\par\bigskip
\noindent If yes, then going through $v$ is shorter than whatever path you have seen, so update $d(v_0,v^{\prime})$ to $d(v_0,v)+w(\left\{v,v^{\prime}\right\})$, otherwise keep $d(v_0,v^{\prime})$
\par\bigskip
\noindent Third, New current vertex is $v\in Q$ with the smallest value $d(v_0,v)$
\par\bigskip
\noindent Repeat first to third step until $Q$ is empty, and return $d(v_0,*)$
\par\bigskip
\noindent\textbf{Anmärkning:}\par
\noindent Interested in a concrete distance between vertices we can stop as soon as we hit it in step 3
\par\bigskip
\noindent\textbf{Anmärkning:}\par
\noindent Finding the path that realises the minimum distance, we can remember the previous vertex that we visited, and therefore we can trace back the steps and get a path
\par\bigskip
\noindent\textbf{Anmärkning:}\par
\noindent Runtime of $\mathcal{O}(\left|E\right|+\left|V\right|\log(\left|V\right|))$
\par\bigskip
\noindent\textbf{Example:}\par
Example 60 in lecture notes
