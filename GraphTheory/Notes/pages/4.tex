\section{Kirdhoffs Matrix-Tree theorem}\par
\begin{theo}[Complexity]{thm:complexity}
  LLet $G$ be a labelled graph. The \textit{complexity} of $G$, denoted by $t(G)$ is the number of spanning trees of $G$
\end{theo}
\par\bigskip
\noindent\textbf{Example:}\par
\noindent The complexity of a complete graph $K_n$ is $n^{n-2}$
\par\bigskip
\noindent From now on (this lecture), we assume that $G = (V,E)$ is a labelled ($V = \left\{1,2,\cdots,n\right\}$) finite and simple graph with $E = \left\{e_1,\cdots, e_n\right\}$
\par\bigskip
\begin{theo}[Ajacency matrix]{thm:adjmatr}
  The \textit{adjacency matrix} $A$ of $G$ is the $n\times n$ matrix having entries:
  \begin{equation*}
    \begin{gathered}
      A_{ij} = \begin{cases}1\quad \text{ if $i,j$ are ajacent}\\0\quad\text{ otherwise}\end{cases}
    \end{gathered}
  \end{equation*}
\end{theo}
\par\bigskip
\noindent\textbf{Example:}\par
%\begin{figure}[ht!]
    %\centering
    %\begin{tikzpicture}
      %\node[state](p0){$1$};
      %\node[state](p1){$2$};
      %\node[state](p2){$3$};
      %\node[state](p3){$4$};
      %\node[state](p4){$5$};
    %\end{tikzpicture}
    %\caption{}
%\end{figure}
\noindent has adjacency matrix:
\begin{equation*}
  \begin{gathered}
    A = \begin{pmatrix}0&1&1&1&0\\1&0&1&0&0\\1&1&0&0&1\\1&0&0&0&1\\0&0&1&1&0\end{pmatrix}
  \end{gathered}
\end{equation*}
\par\bigskip
\noindent\textbf{Anmärkning:}\par
\begin{itemize}
  \item $A$ is symmetric $\Rightarrow$ real eigenvalues
  \item $A_{i,i} = 0$ for all $i=1,\cdots,n$
  \item $tr(A) = \sum_{i=1}^{n}A_{i,i} = 0 = \sum_{i=1}^{n}\lambda_i$
\end{itemize}
\par\bigskip
\noindent We declare one part of the graph to be negative and the other to be positive. This is just a choice, and does not have an effect on the construction of the graph
\par\bigskip
\begin{theo}[Incidence matrix]{thm:incmatr}
  The \textit{incidence matrix} $D$ of $G$ with respect to a fixed orientation is the $n\times n$ matrix having entries:
  \begin{equation*}
    \begin{gathered}
      D_{i,j} = \begin{cases}1\quad \text{ if $i$ is the positive endpoint of $e_j$}\\-1\quad\text{ if $i$ is the negative endpoint of $e_j$}\\0\quad \text{ otherwise}\end{cases}
    \end{gathered}
  \end{equation*}
  \par\bigskip
  \noindent\textbf{Anmärkning:}\par
  \noindent $e_i\in E$
  \par\bigskip
  \noindent\textbf{Anmärkning:}\par
  \noindent We may choose a vertex to be negative for one edge, but it does not have to be fixed, for another edge that same vertex might be positive. This does not affect the graph. 
\end{theo}
\newpage
\noindent How do we relate these matrices to trees? Well, this is the idea of the next few lemmas:
\par\bigskip
\begin{lem}
  LLet $D$ be an incidence matrix to a graph $G$. Then the following statements are true:\par
  \begin{itemize}
    \item The sum of any column of $D$ is 0, thus the rank($D$)$\leq n-1$\par
    \item If $G$ is connected, then the rank$(D) = n-1$
    \item If $G$ has $c$ components, then rank$(D) = n-c$
  \end{itemize}
\end{lem}
\par\bigskip
\noindent This is good, because by looking at $D$ we can say something about if the graph is connected or not.
\par\bigskip
\begin{prf}
  DDenote by $r_i$ the $i$-th row vector of $D$:\par
  \begin{itemize}
    \item The sum of any column is 0 by the definition of $D$ (each row has 2 non-zero entries that are 1 or -1).\par
      This means that $\underbrace{\sum_{i=1}^{n}r_i}_{\text{linjärkomb.}} = 0\Rightarrow\text{rank}(D) \leq n-1$
      \par\bigskip
    \item Suppose there is a non-trivial linear combination of the form $\sum_{i=1}^{n}\alpha_i\cdot r_i = 0$ \par
      Consider the row $k$ for which $a_k\neq$\par
      In row $k$, there is a non-zero entry for every edge incident to the vertex $k$ \par
      For each of these columns where row $k$ has non-zero entries, there is a unique second entry in the same column (column $s$), but in a different row ($r_j$).\par
      This other entry $r_j$ will have an opposite sign\par
      We know $\alpha_k\cdot r_{k,s}+\alpha_jr_{j,s} = 0$\par
      This means that $\alpha_k = \alpha_j$ since they only differ by a sign
      \par\bigskip
      As a consequence of this, if $\alpha_k\neq0$ then $\alpha_j=\alpha_k$ for all $j$ adjacent to $k$\par
      Since $G$ is connected, this argument extends to all of $G$:s vertices (and thereby all of $G$), hence the linear combination $\sum_{i=0}^{n}\alpha_ir_i$ is a scalar multiple of $\sum_{i=1}^{n}r_i$
      \par\bigskip
      This gives that the rank$(D) = n-1$
      \par\bigskip
    \item This follows quite easily from proof from the above point.\par
      If $G$ has $c$ components, relabel the vertices and edges in such a way that $D$ takes the form of having edges and vertices of component 1 in the diagonal place 1,1.\par
       Then rank$(D) = n-c$ by applying previous point to each block
  \end{itemize}
  \par\bigskip
\end{prf}
\par\bigskip
\begin{lem}
  AAny square submatrix of an incidence matrix $D$ has determinant 0 or $\pm1$
\end{lem}
\par\bigskip
\begin{prf}
  PPure linear algebra. Left as an exercise to the reader 
\end{prf}
\par\bigskip
\noindent More intersting things that follows from Lemma 5.2 is if you pick a set of edges $S\subseteq E$. Denote by $D_s$ the submatrix containing exactly the columns that correspond exactly to the edges in $S$
\par\bigskip
\noindent Then $D_s$ is an incidence matrix to the spanning subgraph $(V,S)$
\par\bigskip
\noindent In particular, if $\left|S\right| = n-1$, then by second point of Lemma 5.1, rank$(D) = n-1$ iff $(V,S)$ is connecfted iff $(V,S)$ is a spanning tree 
\par\bigskip
\begin{lem}
  LLet $S\subseteq E$ with $\left|S\right| = n-1$\par
  \noindent Let $M$ denote any $(n-1)\times(n-1)$ submatrix of the $n\times(n-1)$ matrix $D_s$
  \par\bigskip
  \noindent Then $M$ is regular (invertible) iff $(V,S)$ is a spanning tree of $G$
\end{lem}
\par\bigskip
\begin{prf}
  WWe already know that the rank of $D_s = n-1$ iff $(V,S)$ is a spanning tree.\par
  \noindent In that case, if we just take $D_s$ and delete any row from it, it will create an invertible matrix, so if I start with a spanning tree, then $M$ is regular
  \par\bigskip
  \noindent Conversly, assume $M$ is regular. This means that $D_s$ contains at least $n-1$ linear independent rows and the same number of linear independent columns\par
  \noindent This means that rank$(D_s)\leq n-1$, but since this thing only has $n-1$ columns this must be equal to $n-1$\par
  \noindent $(V,S)$ is connected on $n-1$ edges, therefore it is a spanning tree.
  \par\bigskip
\end{prf}
\par\bigskip
\noindent We now have all the criteria to decide if a subgraph is a spanning tree
\par\bigskip
\begin{lem}
  LLet $A$ be the adjacency matrix of $G$ and let $D$ be an incidence matrix of $G$ (fixed labelling and ordering of edges as well as ordering of orientation)
  \par\bigskip
  \noindent Denote by $\Delta$ the $(n\times n)$ diagonal matrix with entries $\Delta_{i,i} = deg(i)$
  \par\bigskip
  \noindent Then $\Delta -A = DD^T$ and this matrix is the \textit{Laplacian matrix} of $G$
  \par\bigskip
  \noindent In particular, this product is independent from the orientation of $G$ (because the LHS does not rely on the orientation of the edges)
\end{lem}
\par\bigskip
\begin{prf}
  L$(DD^T)_{i,j}$ is the scalar product of the row vectors $r_i$ and $_j$ of $D$ (because of how matrix multiplication works)
  \par\bigskip
  \noindent If $i=j$, this means that $(DD^T)_{i,i} = \sum_{i=1}^{m}r_{i,s}^2 = \sum_{s=1}^{m}D_{i,s}^2 = \sum_{s=1}^{m}\mathbb{1}\left\{\text{edge $s$ is incident to vertex $i$}\right\} = deg(i) = (\Delta-A)_{i,i}$
  \par\bigskip
  \noindent If $i\neq j$, then:
  \begin{equation*}
    \begin{gathered}
      (DD^T)_{i,j} = \sum_{s=1}^{m}D_{i,s}D_{j,s} = \begin{cases}0\quad\text{ if $i,j$ non-adjacent}\\-1\quad\text{ if $i,j$ adjacent}\end{cases}\\
       = (\Delta-A)_{i,j}
    \end{gathered}
  \end{equation*}
  \par\bigskip
  \noindent This thing is non-zero iff $D_{i,s} = \pm 1$ and $D_{j,s} = \mp 1$. This only happens if $s$ is an edge connecting $i$ to $j$\par
  \noindent But between any given vertices, there can be at most one edge
\end{prf}
\par\bigskip
\begin{lem}
  ALet $Q = \Delta -A = DD^T$ the Laplacian of $G$.\par
  \noindent Denote by $J$ the $n\times n$ matrix with all entries $=1$
  \par\bigskip
  \noindent Then the adjoint matrix of the Laplacian $Q$ is a scalar multiple of $J$ 
\end{lem}
\par\bigskip
\begin{prf}
  OObserve:\par
  \begin{itemize}
    \item rank$(Q) = \text{rank}(D)$ (since $Q = DD^T$)
  \end{itemize}
  \par\bigskip
  \noindent If $G$ is disconnected, the rank$(Q)<n-1\Rightarrow$ all cofactors are 0
  \par\bigskip
  \noindent If $G$ is connected, then rank$(Q) = n-1$.
  \par\bigskip
  \noindent We know from linear algebra that $Q\cdot\text{adj}(Q) = \text{det}(Q)\cdot I_{n\times n} = 0$
  \par\bigskip
  \noindent This means that every column vector of adj$(Q)$ is in ker$(Q)$
  \par\bigskip
  \noindent But the dimension of the kernel is $n-\text{rank}(Q) = 1$ and $(1,\cdots,1)^T\in\text{ker}(Q)$ because:
  \begin{equation*}
    \begin{gathered}
      (\Delta-A)\cdot\begin{pmatrix}1\\1\\\vdots\\1\end{pmatrix} = deg(i)-\sum_{j\text{ adj. to } i}1 = deg(i)-deg(i) = 0
    \end{gathered}
  \end{equation*}
  \par\bigskip
  \noindent Therefore, $(1,\cdots,1)^T$ is a basis of the kernel
  \par\bigskip
  \noindent This means every column of adj$(Q)$ is a scalar multiple of $(1,\cdots,1)^t$
  \par\bigskip
  \noindent Since $Q^T = (\Delta-A)^T = \Delta^T-A^T = \Delta -A = Q$, also adj$(Q)^T = $ adj$(Q)$
  \par\bigskip
  \noindent But this means that every row adj$(Q)$ is a scalar multiple of $(1,\cdots,1)^T$
  \par\bigskip
  \noindent If every row and every column is a scalar multiple of that vector, then since every row and column intersect, adj$(Q)$ is a scalar multiple of $J$ 
\end{prf}
\par\bigskip
\begin{theo}[Kirchoff]{thm:kirchoff}
  Let $G$ be a connected finite simple graph with Laplacian matrix $Q$.
  \par\bigskip
  \noindent Then, adj$(Q) = t(G)\cdot J$ (equal to any cofactor of its Laplacian matrix)
  \par\bigskip
  \noindent Equivelantly, if $\lambda_1,\cdots,\lambda_{n-1}$ are the non-zero eigenvalues of $Q$, then $t(G) = \dfrac{1}{n}\lambda_1\cdot\cdots\cdot\lambda_{n-1}$ (if not connected, then there are more than one non-zero eigenvalues)
\end{theo}
\par\bigskip
\begin{theo}[Cauchy-Binet]{thm:cauchybinet}
  Assume you have two $n\times n$ matrices $A,B$ where $n\leq m$
  \par\bigskip
  \noindent Then, det($AB^T$) = $\sum_{s\subseteq\left\{1,\cdots,m\right\}}\text{det}(A_s)\text{det}(B^T_s)$ and $\left|s\right| = n$
\end{theo}
\newpage
\begin{prf}[Matrix-Tree theorem]{prf:matrixtree}
  We already know that all the cofactors are the same, so it is enough to evaluate one cofactor of $Q$
  \par\bigskip
  \noindent Let $\stackrel{D}{\sim}$ be the incident matrix with the last row deleted.\par
  \noindent This means that det$(\stackrel{D}{\sim}\stackrel{D}{\sim}^T)$ is a cofactor of $Q = DD^T$
  \par\bigskip
  \noindent By Cauchy-Binet we have det$(\stackrel{D}{\sim}\stackrel{D}{\sim}^T) = \sum_{S\subseteq E}(\text{det}\stackrel{D}{\sim}_s)^2$ where $\left|S\right| = n-1$
  \par\bigskip
  \noindent This means that it is a square $(n-1)\times(n-1)$, which by previous lemma has determinant 0 or 1.\par
  \noindent We have also seen that the determinant is 1 iff $(V,S)$ is a spanning tree. Therefore, the sum above is just the number of spanning trees ($t(G)$) 
\end{prf}
\par\bigskip
\noindent\textbf{Example:}\par
\noindent Assume it is the exam and you need to prove Cayleys formula but you have forgotten it and we want to find $t(K_n)$ 
\par\bigskip
\noindent We have $Q$ with $n-1$ on its diagonal and all the other edges are -1.
\par\bigskip
\noindent How do we find the eigenvalues? Its not recommended to find the characteristic polynomial, but we may be able to guess the eigenvalues (and verify using trace).
\par\bigskip
\noindent $Q\cdot(1,\cdots,1)^T = 0$ (sum of each row is 0) so $(1,\cdots,1)^t$ is eigenvector to eigenvalue 0\par
\noindent We try $Q\cdot\underbrace{(0,\cdots,1,-1,\cdots,0)^T}_{\text{eigenvector to eigenvalue $n$}} = (0,\cdots,n,-n,\cdots,0)^T$. These form $n-1$ linearly independent eigenvectors
\par\bigskip
\noindent Now we can use Kirchoff. By the Matrix-Tree theorem, $t(K_n) = \dfrac{1}{n}n^{n-1} = n^{n-2}$
