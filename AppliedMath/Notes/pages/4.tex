\par\bigskip
\section{Dynamical Systems}\par
\noindent In Dynamical Systems we have a set of all the possible status that a system can have, and a \textit{rule}  that tells us how the system evolves in time.
\par\bigskip
\noindent We can distinguish between discrete or continuous dynamical system. This can be due to the model itself, naturally we might be interested in a model where we are only interested in discrete time (height of someone), and sometimes it is more useful to have continuous cases (temperature).
\par\bigskip
\noindent Another case could be performing continuous operations on a computer, where we discretisise time when numerically computing integrals.
\par\bigskip
\noindent\textbf{Example:} We consider a particle that is moving with one degree of freedom (in one direction). One possible evolution can be described in the following way:
\begin{equation*}
  \begin{gathered}
    x_{n+1} = x_n-1
  \end{gathered}
\end{equation*}\par
\noindent Note that this is a discrete model. Let us look at a continuous model:
\begin{equation*}
  \begin{gathered}
    \hat{\hat{x}} = -1
  \end{gathered}
\end{equation*}
\par\bigskip
\noindent What is the difference between the discrete and the continuous case then? When the time is discrete, the system is described by some map $F:X\to X$ where $X$ is the space of the possible configurations. In the continuous case, we have a differential equation
\begin{equation*}
  \begin{gathered}
    \hat{x} = F(x,t)
  \end{gathered}
\end{equation*}
\par\bigskip
\noindent When the system does not depend on time, it is called an \textit{autonomous}  system.
\par\bigskip
\noindent Looking at how a system is defined, we are of course interested in a few points. Sometimes these points could be fixed, sometimes they might be singularities, limit points, stationary points/sets, etc.\par
\noindent We are also intersted in what happens around our point of interest, namely the neighborhood. Are there solutions that converge to some point/set in that neighborhood?
\par\bigskip
\noindent\textbf{Example:} A growth-decay model is defined as
\begin{equation*}
  \begin{gathered}
    x_{n+1} = (1+r)x_n
  \end{gathered}
\end{equation*}\par
\noindent We have a discrete mapping. In this particular case we can write the map in a recursive way:
\begin{equation*}
  \begin{gathered}
    x_{n+1} = (1+r)^{n+1}x_0
  \end{gathered}
\end{equation*}\par
\noindent Where $x_0$ is the start condition. Depending on the value of $r$, the quantity could be increasing (growth model) or decreasing (decay model). 
\par\bigskip
\begin{theo}[Fixed point]{}
  A \textit{fixed point} (or \textit{equilibrium solution}) is a point $x_*$ such that its iteration is stationary (can almost say that it is mapped to itself).
  \begin{equation*}
    \begin{gathered}
      x_{n+1} = f(x_n)\Rightarrow x_* = f(x_*)
    \end{gathered}
  \end{equation*}
\end{theo}
\par\bigskip
\begin{theo}[Periodic orbit]{}
  A \textit{periodic orbit} of period $K$ is a point $x_*$ such that after $k$ iterations of the map we are, again at $x_*$:
  \begin{equation*}
    \begin{gathered}
      f^K(x_*) = x_*
    \end{gathered}
  \end{equation*}
\end{theo}
\par\bigskip
\noindent\textbf{Example:}
\begin{equation*}
  \begin{gathered}
    x_{n+1} = (1+r)x_n
  \end{gathered}
\end{equation*}\par
\noindent A fixed point is $x_* = 0$
\par\bigskip
\noindent\textbf{Example:}\par
\noindent The map $x_{n+1} = -x_n$ has the origin as a fixed point and has a period orbit of period 2 since $f(x_n) = -x_n\Rightarrow f(f(x_n)) = f(-x_n) = x_n$, which is true $\forall x_n\neq0$ 
\par\bigskip
\begin{theo}[Asymptotic Stability]{}
  A fixed point is \textit{asymptotically stable} if $\forall y$ in the neighborhood of $x_*$,
  \begin{equation*}
    \begin{gathered}
      \lim_{n\to\infty}d(f^n(y), x_*)  = 0
    \end{gathered}
  \end{equation*}
\end{theo}
\par\bigskip
\noindent To study the stability, one could consider the linearisation of the probelem.
\par\bigskip
\noindent\textbf{Result:} Consider a fixed point for a map $x_{n+1} = f(x_n)$. Then, if the spectrum of the linearisation of the map evaluated at the fixed point $Df(x_*)$ (Jacobian) is contained in the unit circle, then the point is asymptotically stable. 
\par\bigskip
\noindent On the contrary, if there is an eigenvalue with absolute value greater than one, then the point is unstable. Recall from the lecture notes in ODE (P. 71):
\par\bigskip
\begin{center}
  \begin{tabular}{c|c|c}
    Egenvärden&Typ&Stabilitet\\\\
    \hline\\
    $0<\lambda_2<\lambda_1$&Nod&Instabil\\
    $\lambda_2<0<\lambda_1$&Sadelpunkt&Instabil\\
    $<\lambda_2<\lambda_1<0$&Nod&Asymptotiskt stabil\\\\
    \hline\\
    $\alpha>0, \beta\neq0$&Spiral&Instabil\\
    $\alpha=0, \beta\neq0$&Center&Stabil\\
    $\alpha<0, \beta\neq0$&Spiral&Asymptotiskt stabil\\\\
    \hline\\
    $\lambda_1=\lambda_2 > 0$&Nod (G.M = 2)&Instabil\\
    &Improper node (G.M $<$ 2)&Instabil\\\\
    \hline\\
    $\lambda_1=\lambda_2<0$&Nod (G.M = 2)&Asymptotiskt stabil\\
    &Improper nod (G.M $<$ 2)&Asymptotiskt stabil\\
  \end{tabular}
\end{center}
\par\bigskip
\noindent If we consider a point $x_*+y_n$ with $y_n$ sall, so that the point is close to the fixed point and then look at the iteration of this point is:
\begin{equation*}
  \begin{gathered}
    f(x_*+y_n) = f(x_*) + Df(x_*)y_n + \text{higher order terms}\\
    = x_*+y_{n+1} = x_*+Df(x_*)y_n + \text{higher order terms}
  \end{gathered}
\end{equation*}
\par\bigskip
\noindent At the linear approximation, $y_n$ satisfies $y_{n+1} = \underbrace{Df(x_*)}_{\text{Matrix}}y_n$
\par\bigskip
\noindent For example, in the 1-dimensional case:
\begin{equation*}
  \begin{gathered}
    y_{n+1} = \underbrace{f^{\prime}(x_*)}_{\text{$\lambda$}}y_n \Rightarrow \lambda y_n\\
    \Rightarrow y_{n+1} = \lambda^{n+1}y_0
  \end{gathered}
\end{equation*}
\par\bigskip
\noindent What does this formula tell us? Well, depending on $\lambda$ we can tell if our model is growing or decaying. Considering a point close to our $x_*$, then if $y_n$ is decaying then $x_*+y_n$ seems to converge to $x_*$. If it is exploding, then we do not have a stable point.
\par\bigskip
\noindent In the general $m$-dimensional case we deal with a problem in the form:
\begin{equation*}
  \begin{gathered}
    y_{n+1} = Ay_n = A^{n+1}y_0
  \end{gathered}
\end{equation*}\par
\noindent Suppose the matrix is diagonalisable and we can compute the eigenvalues/eigenvectors (and the eigenvectors form a basis of $\R^m$).\par
\noindent We can then express our point as a linear combination of our vectors:
\begin{equation*}
  \begin{gathered}
    y_0 = c_1v_1+\cdots+c_mv_m\\
    y_{n+1} = A^{n+1}y_0 = c_1\lambda_1^{n+1}v_1+\cdots+c_m\lambda_m^{n+1}v_m
  \end{gathered}
\end{equation*}\par
\noindent We can reorder the eigenvalues such that the first one is the biggest:
\begin{equation*}
  \begin{gathered}
    y_{n+1} = \lambda_1^{n+1}\left(c_1v_1+\cdots+\left(\dfrac{\lambda_m}{\lambda_1}\right)c_mv_m\right)\\
    \Rightarrow \left|\dfrac{\lambda_j}{\lambda_1}\right|<1
  \end{gathered}
\end{equation*}\par
\noindent The behaviour for $n\to\infty$ is given by the eigenvalue $\lambda_1$. The asymptotic is given by $c_1\lambda_1^{n+1}v_1$.
\par\bigskip
\noindent If $\left|\lambda_1\right|<1$, then the fixed point is stable. Otherwise, unstable.\par
\par\bigskip
\noindent\textbf{Exercise:} Find the fixed point and study the linear stability of the map 
\begin{equation*}
  \begin{gathered}
    x_{n+1} = 2´+x_n-x_n^2
  \end{gathered}
\end{equation*}\par
\noindent The function is given by $f(x) = 2+x-x^2$, then fixed point are given by:
\begin{equation*}
  \begin{gathered}
    f(x) = x\Lrarr 2-x^2 = 0\Lrarr x_* = \pm\sqrt{2}
  \end{gathered}
\end{equation*}
\par\bigskip
\noindent For the linear stability, we look at the value of $f^{\prime}(x_*)$:
\begin{equation*}
  \begin{gathered}
    f^{\prime}(x) = 1-2x\Rightarrow f^{\prime}(\pm\sqrt{2}) = 1\pm2\sqrt{2}
  \end{gathered}
\end{equation*}\par
\noindent Since $\left|1\pm2\sqrt{2}\right|>1$, both points of equilibrium are unstable. 
