\par\bigskip
\section{Dynamical Systems}\par
\noindent In Dynamical Systems we have a set of all the possible status that a system can have, and a \textit{rule}  that tells us how the system evolves in time.
\par\bigskip
\noindent We can distinguish between discrete or continuous dynamical system. This can be due to the model itself, naturally we might be interested in a model where we are only interested in discrete time (height of someone), and sometimes it is more useful to have continuous cases (temperature).
\par\bigskip
\noindent Another case could be performing continuous operations on a computer, where we discretisise time when numerically computing integrals.
\par\bigskip
\noindent\textbf{Example:} We consider a particle that is moving with one degree of freedom (in one direction). One possible evolution can be described in the following way:
\begin{equation*}
  \begin{gathered}
    x_{n+1} = x_n-1
  \end{gathered}
\end{equation*}\par
\noindent Note that this is a discrete model. Let us look at a continuous model:
\begin{equation*}
  \begin{gathered}
    \hat{\hat{x}} = -1
  \end{gathered}
\end{equation*}
\par\bigskip
\noindent What is the difference between the discrete and the continuous case then? When the time is discrete, the system is described by some map $F:X\to X$ where $X$ is the space of the possible configurations. In the continuous case, we have a differential equation
\begin{equation*}
  \begin{gathered}
    \hat{x} = F(x,t)
  \end{gathered}
\end{equation*}
\par\bigskip
\noindent When the system does not depend on time, it is called an \textit{autonomous}  system.
\par\bigskip
\noindent Looking at how a system is defined, we are of course interested in a few points. Sometimes these points could be fixed, sometimes they might be singularities, limit points, stationary points/sets, etc.\par
\noindent We are also intersted in what happens around our point of interest, namely the neighborhood. Are there solutions that converge to some point/set in that neighborhood?
\par\bigskip
\noindent\textbf{Example:} A growth-decay model is defined as
\begin{equation*}
  \begin{gathered}
    x_{n+1} = (1+r)x_n
  \end{gathered}
\end{equation*}\par
\noindent We have a discrete mapping. In this particular case we can write the map in a recursive way:
\begin{equation*}
  \begin{gathered}
    x_{n+1} = (1+r)^{n+1}x_0
  \end{gathered}
\end{equation*}\par
\noindent Where $x_0$ is the start condition. Depending on the value of $r$, the quantity could be increasing (growth model) or decreasing (decay model). 
\par\bigskip
\begin{theo}[Fixed point]{}
  A \textit{fixed point} (or \textit{equilibrium solution}) is a point $x_*$ such that its iteration is stationary (can almost say that it is mapped to itself).
  \begin{equation*}
    \begin{gathered}
      x_{n+1} = f(x_n)\Rightarrow x_* = f(x_*)
    \end{gathered}
  \end{equation*}
\end{theo}
\par\bigskip
\begin{theo}[Periodic orbit]{}
  A \textit{periodic orbit} of period $K$ is a point $x_*$ such that after $k$ iterations of the map we are, again at $x_*$:
  \begin{equation*}
    \begin{gathered}
      f^K(x_*) = x_*
    \end{gathered}
  \end{equation*}
\end{theo}
\par\bigskip
\noindent\textbf{Example:}
\begin{equation*}
  \begin{gathered}
    x_{n+1} = (1+r)x_n
  \end{gathered}
\end{equation*}\par
\noindent A fixed point is $x_* = 0$
\par\bigskip
\noindent\textbf{Example:}\par
\noindent The map $x_{n+1} = -x_n$ has the origin as a fixed point and has a period orbit of period 2 since $f(x_n) = -x_n\Rightarrow f(f(x_n)) = f(-x_n) = x_n$, which is true $\forall x_n\neq0$ 
\par\bigskip
\begin{theo}[Asymptotic Stability]{}
  A fixed point is \textit{asymptotically stable} if $\forall y$ in the neighborhood of $x_*$,
  \begin{equation*}
    \begin{gathered}
      \lim_{n\to\infty}d(f^n(y), x_*)  = 0
    \end{gathered}
  \end{equation*}
\end{theo}
\par\bigskip
\noindent To study the stability, one could consider the linearisation of the probelem.
\par\bigskip
\noindent\textbf{Result:} Consider a fixed point for a map $x_{n+1} = f(x_n)$. Then, if the spectrum of the linearisation of the map evaluated at the fixed point $Df(x_*)$ (Jacobian) is contained in the unit circle, then the point is asymptotically stable. 
\par\bigskip
\noindent On the contrary, if there is an eigenvalue with absolute value greater than one, then the point is unstable. Recall from the lecture notes in ODE (P. 71):
\par\bigskip
\begin{center}
  \begin{tabular}{c|c|c}
    Egenvärden&Typ&Stabilitet\\\\
    \hline\\
    $0<\lambda_2<\lambda_1$&Nod&Instabil\\
    $\lambda_2<0<\lambda_1$&Sadelpunkt&Instabil\\
    $<\lambda_2<\lambda_1<0$&Nod&Asymptotiskt stabil\\\\
    \hline\\
    $\alpha>0, \beta\neq0$&Spiral&Instabil\\
    $\alpha=0, \beta\neq0$&Center&Stabil\\
    $\alpha<0, \beta\neq0$&Spiral&Asymptotiskt stabil\\\\
    \hline\\
    $\lambda_1=\lambda_2 > 0$&Nod (G.M = 2)&Instabil\\
    &Improper node (G.M $<$ 2)&Instabil\\\\
    \hline\\
    $\lambda_1=\lambda_2<0$&Nod (G.M = 2)&Asymptotiskt stabil\\
    &Improper nod (G.M $<$ 2)&Asymptotiskt stabil\\
  \end{tabular}
\end{center}
\par\bigskip
\noindent If we consider a point $x_*+y_n$ with $y_n$ sall, so that the point is close to the fixed point and then look at the iteration of this point is:
\begin{equation*}
  \begin{gathered}
    f(x_*+y_n) = f(x_*) + Df(x_*)y_n + \text{higher order terms}\\
    = x_*+y_{n+1} = x_*+Df(x_*)y_n + \text{higher order terms}
  \end{gathered}
\end{equation*}
\par\bigskip
\noindent At the linear approximation, $y_n$ satisfies $y_{n+1} = \underbrace{Df(x_*)}_{\text{Matrix}}y_n$
\par\bigskip
\noindent For example, in the 1-dimensional case:
\begin{equation*}
  \begin{gathered}
    y_{n+1} = \underbrace{f^{\prime}(x_*)}_{\text{$\lambda$}}y_n \Rightarrow \lambda y_n\\
    \Rightarrow y_{n+1} = \lambda^{n+1}y_0
  \end{gathered}
\end{equation*}
\par\bigskip
\noindent What does this formula tell us? Well, depending on $\lambda$ we can tell if our model is growing or decaying. Considering a point close to our $x_*$, then if $y_n$ is decaying then $x_*+y_n$ seems to converge to $x_*$. If it is exploding, then we do not have a stable point.
\par\bigskip
\noindent In the general $m$-dimensional case we deal with a problem in the form:
\begin{equation*}
  \begin{gathered}
    y_{n+1} = Ay_n = A^{n+1}y_0
  \end{gathered}
\end{equation*}\par
\noindent Suppose the matrix is diagonalisable and we can compute the eigenvalues/eigenvectors (and the eigenvectors form a basis of $\R^m$).\par
\noindent We can then express our point as a linear combination of our vectors:
\begin{equation*}
  \begin{gathered}
    y_0 = c_1v_1+\cdots+c_mv_m\\
    y_{n+1} = A^{n+1}y_0 = c_1\lambda_1^{n+1}v_1+\cdots+c_m\lambda_m^{n+1}v_m
  \end{gathered}
\end{equation*}\par
\noindent We can reorder the eigenvalues such that the first one is the biggest:
\begin{equation*}
  \begin{gathered}
    y_{n+1} = \lambda_1^{n+1}\left(c_1v_1+\cdots+\left(\dfrac{\lambda_m}{\lambda_1}\right)c_mv_m\right)\\
    \Rightarrow \left|\dfrac{\lambda_j}{\lambda_1}\right|<1
  \end{gathered}
\end{equation*}\par
\noindent The behaviour for $n\to\infty$ is given by the eigenvalue $\lambda_1$. The asymptotic is given by $c_1\lambda_1^{n+1}v_1$.
\par\bigskip
\noindent If $\left|\lambda_1\right|<1$, then the fixed point is stable. Otherwise, unstable.\par
\par\bigskip
\noindent\textbf{Exercise:} Find the fixed point and study the linear stability of the map 
\begin{equation*}
  \begin{gathered}
    x_{n+1} = 2´+x_n-x_n^2
  \end{gathered}
\end{equation*}\par
\noindent The function is given by $f(x) = 2+x-x^2$, then fixed point are given by:
\begin{equation*}
  \begin{gathered}
    f(x) = x\Lrarr 2-x^2 = 0\Lrarr x_* = \pm\sqrt{2}
  \end{gathered}
\end{equation*}
\par\bigskip
\noindent For the linear stability, we look at the value of $f^{\prime}(x_*)$:
\begin{equation*}
  \begin{gathered}
    f^{\prime}(x) = 1-2x\Rightarrow f^{\prime}(\pm\sqrt{2}) = 1\pm2\sqrt{2}
  \end{gathered}
\end{equation*}\par
\noindent Since $\left|1\pm2\sqrt{2}\right|>1$, both points of equilibrium are unstable. 
\par\bigskip
\noindent\textbf{Example:} $u^{\prime} = f(u) = u^2(3-u)$.\par
\noindent The equilibrium points are:
\begin{equation*}
  \begin{gathered}
    f(u) = 0\Rightarrow\begin{cases}u_1=0\\u_2=3\end{cases}
  \end{gathered}
\end{equation*}
\par\bigskip
\noindent To study the behaviour of $u$, look at the sign of $u$:
\begin{equation*}
  \begin{gathered}
    f(u)>0\qquad u^{\prime}>0\Rightarrow u\text{ increasing}
    f(u)<0\qquad u^{\prime}<0\Rightarrow u\text{ decreasing}
  \end{gathered}
\end{equation*}
\par\bigskip
\noindent If $u<0$, then for $\lim_{t\to+\infty} u =0$\par
\noindent If $0<u<3$, then solution converges to 3 (follows the arrows in the phase-plane)\par
\noindent If $u>3$, then to 3.
\par\bigskip
\noindent We arrive at the following classifications:

\begin{tikzpicture}
  \def\xmax{2.0}
  \def\A{1.7}
  \def\a{0.85}
  \def\ang{60}
  \coordinate (O) at (0,0);
  \coordinate (X) at (\A,0);
  \coordinate (R) at (\ang:\A);
  \coordinate (R') at (0:\a);
  \draw[->,thick] (-\xmax,0) -- (\xmax+0.1,0) node[right=-1] {};
  \draw [<<<<-<<<<, thick, blue] (1,0) -- (-1,0);
  \fill[myred!50!black] (0,0) circle (0.06); %node[below right=-2] {$(x_1,0)$};
\end{tikzpicture} Unstable
\par\bigskip
\begin{tikzpicture}
  \def\xmax{2.0}
  \def\A{1.7}
  \def\a{0.85}
  \def\ang{60}
  \coordinate (O) at (0,0);
  \coordinate (X) at (\A,0);
  \coordinate (R) at (\ang:\A);
  \coordinate (R') at (0:\a);
  \draw[->,thick] (-\xmax,0) -- (\xmax+0.1,0) node[right=-1] {};
  \draw [>>>>-<<<<, thick, blue] (1,0) -- (-1,0);
  \fill[myred!50!black] (0,0) circle (0.06); %node[below right=-2] {$(x_1,0)$};
\end{tikzpicture} Asymptotically stable 
\par\bigskip
\noindent We can determine the type of stability through $f^{\prime}$. If $f^{\prime}<0$ then we can think of it as having "right"-facing arrows and the opposite for $f^{\prime}>0$
\par\bigskip
\noindent In case of asymptotical stability, is it just local? How does it look like in the global case?
\par\bigskip
\noindent In dimension $>1$, we can request an equilibrium point $x_*$ to be \textit{stable} , that is if we fix an $\varepsilon-$ball around $x_*$, then $\exists \delta>0$ such that all of the solutions with initial conditions in $B_{\delta}(x_*)\subseteq B_\varepsilon(x_*)\quad\forall t$ (see figure 2.7 on page 106)
\par\bigskip
\noindent\textbf{Anmärkning}: Asymptotically stable point $\Rightarrow$ stable solutions converge to $x_*$ as $t\to+\infty$
\par\bigskip
\subsection{Linear Systems}\hfill\\\par
\noindent Nice to study as tehy approximate behaviour of linear systems.
\par\bigskip
\noindent In this course, we will cover 2D systems $\mathbf{x^{\prime}} = A\mathbf{x}\quad\mathbf{x}\in\R^2, A\in\mathcal{M}_{2\times2}(\R)$
\par\bigskip
\noindent We look for a solution on the form $x = ve^{\lambda t}$:
\begin{equation*}
  \begin{gathered}
    v\lambda e^{\lambda}t = Ave^{\lambda t}\Lrarr v\lambda = Av\Rightarrow u\text{ is an eigenvector to $A$ with eigenvalue $\lambda$}
  \end{gathered}
\end{equation*}
\par\bigskip
\noindent Imagine $\lambda_1,\lambda_2\in\C$ are our eigenvalues. Then the solution can be written as a linear combination of the associated eigenvector:
\begin{equation*}
  \begin{gathered}
    c_1v_1e^{\lambda_1t}+c_2v_2e^{\lambda_2t} = x(t)\quad\text{(general solution)}
  \end{gathered}
\end{equation*}
\par\bigskip
\noindent Recall the classification of eigenvectors through the table above. 
\par\bigskip
\noindent If we can only find one eigenvalue, then the general solution is on the form $\mathbf{x}(t) = c_1v_1e^{\lambda t}+(w+v_1)c_2e^{\lambda t}$, where $w$ satisfies $(A-\lambda I)w = v$ 
\par\bigskip
\noindent If we have a new linear system $\mathbf{x^{\prime}} = F(x)\quad\mathbf{x}\in\R^2$, we can define $A = DF(x_*)$ where $x_*$ is an equilibrium point.\par
\noindent The system $\mathbf{x^{\prime}} = A\mathbf{x}$ is the linearised problem around the equilibrium $x_*$ 
\par\bigskip
\begin{theo}[]{}
  The origin $(0,0)$ is a critical point for $\mathbf{x^{\prime}} = A\mathbf{x}$ of the same type as  $x_*$ for $\mathbf{x^{\prime}} = F(\mathbf{x})$ when:\par
  \begin{itemize}
    \item$\lambda_j\in\R\qquad\lambda_1\cdot\lambda_2>0$ (same sign) (node)
    \item $\lambda_j\in\R\qquad\lambda_1\cdot\lambda_2<0$ (opposite sign) (sadlepoint)
    \item Re$(\lambda_j) = 0\Rightarrow$ Spiral
  \end{itemize}
\end{theo}
\par\bigskip
\noindent\textbf{Corollary:} If $(0,0)$ asymptotically stable for $\mathbf{x^{\prime}} = A\mathbf{x}$, then $x_*$ is asymptotically stable for $\mathbf{x^{\prime}} = F(\mathbf{x})$ 
\par\bigskip
\noindent\textbf{Example:} Study the stability of the linear system:
\begin{equation*}
  \begin{gathered}
    \begin{cases*}
      x^{\prime} = x+\mu y\\
      y^{\prime} = x-y
      \end{cases*}\Rightarrow A = \begin{pmatrix}1&\mu\\1&-1\end{pmatrix}\\
      \text{Eigenvalues: } (1-\lambda)(-1-\lambda)-\mu=0\Lrarr -1+\lambda^2-\mu = 0\Rightarrow \lambda = \pm\sqrt{\mu+1}
  \end{gathered}
\end{equation*}\par
\noindent Note that we have different equilibrium points depending on the value of $\mu$:
\begin{equation*}
  \begin{gathered}
    \mu>-1\Rightarrow\text{ real eigenvalues with opposite signs }\Rightarrow \text{ saddle}\\
    \mu<-1\Rightarrow\text{ complex eigenvalues with Re$(\lambda_j)=0$}\Rightarrow \text{ center}\\
    \mu=-1\Rightarrow\text{ critical value (\textit{bifurcation value})}
  \end{gathered}
\end{equation*}\par
\noindent At the bifurcation value, we have:
\begin{equation*}
  \begin{gathered}
    \begin{cases*}
      x^{\prime} = x-y\\
      y^{\prime} = x-y
    \end{cases*}
  \end{gathered}
\end{equation*}\par
\noindent In this specific case, not only the origin is equilibrium, in this case $x^{\prime} = y^{\prime} = 0$, and $y=  x$, every point in this line is an equilibrium point.
\par\bigskip
\noindent\textbf{Example:} Damped harmonic oscillator:
\begin{equation*}
  \begin{gathered}
    mx^{\prime\prime}+ax^{\prime}+kx=0\qquad m,a,k>0
  \end{gathered}
\end{equation*}\par
\noindent Note that for $a = 0$ we have a classical harmonical oscillator.
\par\bigskip
\noindent For the harmonic oscillator, the solutions are periodic oscillations aruond the equilibrium poisition. Do we have the same behaviour holding the damping term $y = x^{\prime}$?
\begin{equation*}
  \begin{gathered}
    \begin{rcases*}
      my^{\prime}+ay+kx=0\\
      y=x^{\prime}
    \end{rcases*}\Rightarrow 
    \begin{rcases*}
      y=x^{\prime}\\
      y^{\prime}  = \dfrac{-ay}{m} - \dfrac{k}{m}x
    \end{rcases*}
  \end{gathered}
\end{equation*}
\par\bigskip
\noindent Here, the orign is an equilibrium point. In matrix form, the system is expressed as:
\begin{equation*}
  \begin{gathered}
    \begin{pmatrix}x^{\prime}\\y^{\prime}\end{pmatrix} = A\begin{pmatrix}x\\y\end{pmatrix}\Rightarrow A= \begin{pmatrix}0&1\\-\dfrac{k}{m}&-\dfrac{a}{m}\end{pmatrix}
  \end{gathered}
\end{equation*}\par
\noindent The matrix $A$ has eigenvalues:
\begin{equation*}
  \begin{gathered}
    +\lambda\left(\dfrac{a}{m}+\lambda\right) + \dfrac{k}{m} = 0\Rightarrow \lambda^2+\dfrac{a}{m}\lambda+\dfrac{k}{m} = 0\\
    \Rightarrow \lambda = \dfrac{-a\pm\sqrt{a^2-4km}}{2m}
  \end{gathered}
\end{equation*}
\par\bigskip
\noindent We know for $a = 0\Rightarrow$ harmonic oscillation, we get the following eigenvalues: $\lambda=\dfrac{\pm\sqrt{-4km}}{2m}$\par
\noindent Note that we have pure imaginary roots,this implies we have perpetual oscillations around equilibrium.
\par\bigskip
\noindent Depending on the sign inside the square root, things can be a little different. It depends on $a^2-4km$:
\begin{equation*}
  \begin{gathered}
    a^2-4km<0\quad\text{ real part is negative}\Rightarrow\text{ asymptotically stable with spiral} 
  \end{gathered}
\end{equation*}
\par\bigskip
\noindent From a physical point of view, oscillation is getting smaller and smaller as $t\to+\infty$ \par
\noindent If we have a damping constant $a$ such that $a^2-4km\geq0$, then we have real eigenvalues who are both negative, so we are asymptotically stable without oscillation. It is a Node.
\par\bigskip
\noindent\textbf{Example, (Lotka-Volteria system):} 
\begin{equation*}
  \begin{gathered}
    x = x(t)\text{ (prey population)}\\
    y=y(t)\text{ (predator population)}
  \end{gathered}
\end{equation*}
\par\bigskip
\noindent The purpose of this system is to study the evolution of time and how they interract with each other, which can be described by a dynamical system:
\begin{equation*}
  \begin{gathered}
    \begin{rcases*}
      x^{\prime} = rx-axy\\
      y^{\prime} = -my+bxy
    \end{rcases*}
  \end{gathered}
\end{equation*}\par
\begin{itemize}
  \item $r = $ rate of increase for prey 
  \item $a = $ decrease in prey due to predator
  \item $m = $ mortality rate of predators
  \item $b = $ increase in the predator population due to preys killed ($b<a$, since each predator consumes more than 1 prey)
\end{itemize}
\pagebreak
\noindent The equilibrium points are given by: \par
$\begin{rcases*}x(r-ay) = 0\\y(-m+bx) = 0\end{rcases*}$
\begin{tikzpicture}
  \def\xmax{2.0}
  \def\A{1.7}
  \def\a{0.85}
  \def\ang{60}
  \coordinate (O) at (0,0);
  \coordinate (X) at (\A,0);
  \coordinate (R) at (\ang:\A);
  \coordinate (R') at (0:\a);
  \draw[->,thick] (-\xmax,0) -- (\xmax+0.1,0) node[right=-1] {};
  \draw[->,thick] (0,-\xmax) -- (0,\xmax+0.1) node[above=-1] {};
  \draw [<<<<->>>>, thick, blue] (1,0) -- (-1,0);
  \draw [>>>>-<<<<, thick, blue] (0,1) -- (0,-1);
  \fill[myred!50!black] (0,0) circle (0.06); %node[below right=-2] {$(x_1,0)$};
  \fill[mypurple!50!black] (1.5,0) circle (0.06) node[below] {$\dfrac{m}{b}$};
  \fill[mypurple!50!black] (0,1.5) circle (0.06) node[left] {$\dfrac{r}{a}$};
\end{tikzpicture} (1)
\par\bigskip
\noindent We see that one point is $(0,0)$, which is the point where botb prey and predator are extinct. Another point is $\left(\dfrac{m}{b}, \dfrac{r}{a}\right)$, which is the equilibrium of coexistance.
\par\bigskip
\noindent We study this non-linear system by looking at the stability of equilibrium:
\begin{equation*}
  \begin{gathered}
    A = \begin{pmatrix}r-ay&-ax\\by&-m+bx\end{pmatrix} = DF(\overline{x})\text{ where } F(\overline{x}) = \overline{x}^{\prime}
  \end{gathered}
\end{equation*}
\par\bigskip
\noindent In the point $(0,0)$:
\begin{equation*}
  \begin{gathered}
    \begin{pmatrix}r&0\\0&-m\end{pmatrix}
  \end{gathered}
\end{equation*}\par
\noindent We have real eigenvalues with opposite signs, ergo a saddle point with eigenvectors $\underbrace{\begin{pmatrix}1\\0\end{pmatrix}}_{\text{$r$}}, \underbrace{\begin{pmatrix}0\\1\end{pmatrix}}_{\text{$-m$}}$ (1)
\par\bigskip
\noindent In the point $\left(\dfrac{m}{b}, \dfrac{r}{a}\right)$:
\begin{equation*}
  \begin{gathered}
    \begin{pmatrix}0&\dfrac{-am}{b}\\\dfrac{br}{a}&0\end{pmatrix}
  \end{gathered}
\end{equation*}\par
\noindent Eigenvalues are given by $\lambda^2+mr = 0\Rightarrow=-mr$, which are pure imaginary and so we do not have enough information to conclude anything.
\par\bigskip
\noindent We can look at cases when $y^{\prime},x^{\prime} = 0$ to get an idea at the behaviour of the system: $x(r-ay) = 0\Rightarrow y= \dfrac{r}{a}\Rightarrow$ vector-field is vertical and arrows are determined by $y^{\prime} = y(-mx+bx)$ (2)

\begin{tikzpicture}
  \def\xmax{2.0}
  \def\A{1.7}
  \def\a{0.85}
  \def\ang{60}
  \coordinate (O) at (0,0);
  \coordinate (X) at (\A,0);
  \coordinate (R) at (\ang:\A);
  \coordinate (R') at (0:\a);
  \draw[->,thick] (-\xmax,0) -- (\xmax+0.1,0) node[right=-1] {};
  \draw[->,thick] (0,-\xmax) -- (0,\xmax+0.1) node[above=-1] {};
  \draw[-] (-\xmax,1) -- (\xmax+0.1,1) node[right=-1] {};
  \draw[-] (1,-\xmax) -- (1,\xmax+0.1) node[above=-1] {};
  \draw [->, thick, blue] (1.3,1.2);
  \draw [->, thick, blue] (1.5,1.2);
  \draw [->, thick, blue] (1.7,1.2);
  \draw [->, thick, blue] (0.9,1.3) -- (0.8,1.3);
  \draw [->, thick, blue] (0.9,1.5) -- (0.8,1.5);
  \draw [->, thick, blue] (0.9,1.7) -- (0.8,1.7);
  \draw [->, thick, blue] (0.3,0.9) -- (0.3, 0.8);
  \draw [->, thick, blue] (0.5,0.9) -- (0.5, 0.8);
  \draw [->, thick, blue] (0.7,0.9) -- (0.7, 0.8);
  \draw [->, thick, blue] (1.1,0.3) -- (1.2,0.3);
  \draw [->, thick, blue] (1.1,0.5) -- (1.2,0.5);
  \draw [->, thick, blue] (1.1,0.7) -- (1.2,0.7);
  \fill[myred!50!black] (0,0) circle (0.06); %node[below right=-2] {$(x_1,0)$};
  \fill[mypurple!50!black] (1.5,0) circle (0.06) node[below] {$\dfrac{m}{b}$};
  \fill[mypurple!50!black] (0,1.5) circle (0.06) node[left] {$\dfrac{r}{a}$};
\end{tikzpicture} (2)
\par\bigskip
\noindent Similarly, for $y^{\prime} = 0$, that is the line $x = \dfrac{m}{b}$, we try to express the solutions of the system by computing:
\begin{equation*}
  \begin{gathered}
    \dfrac{dy}{dx} = \dfrac{-my+bxy}{rx-axy}\Rightarrow (rx-axy)dy = (-my+bxy)dx = x(r-ay)dy = y(-m+bx)dx\\
    \Rightarrow \int \dfrac{r-ay}{y}dy = \int\dfrac{-m+bx}{x}dx\Rightarrow r\ln{\left(y\right)}-ay = -\ln{\left(x\right)}+bx+c\\
    \Rightarrow y^re^{-ay} = x^{-m}e^{bx}e^c\Rightarrow\text{ can't find closed form solution}
  \end{gathered}
\end{equation*}
\par\bigskip
\noindent We want to figure out if we have a closed orbit or if we have a spiral.\par
\noindent Looking at our expression, in the point $x = \dfrac{m}{b}$:
\begin{equation*}
  \begin{gathered}
    y^r = Ce^{ay}
  \end{gathered}
\end{equation*}\par
\noindent Where $C$ is all the constants. We cannot cross the line $x = \dfrac{m}{b}$  many times, ergo we have no spiral behaviour.\par
\noindent Therefore, every cycle is a coexistance state, but the population will not approach the equation state $\left(\dfrac{m}{b}, \dfrac{r}{a}\right)$ 
\par\bigskip
\noindent Some general results about the existance of clsoed orbits in non-linear systems:
\par\bigskip
\begin{theo}[Bendixson-Dulan]{}
  Given a system $\begin{rcases*}x^{\prime}=P(x,y)\\y^{\prime} = Q(x,y)\end{rcases*}$, if $P_x+Q_y$ has the same sign in some region, then there are no closed orbits in that region
\end{theo}
\par\bigskip
\noindent\textbf{Example:}
\begin{equation*}
  \begin{gathered}
    \begin{rcases*}
      x^{\prime} = x+x^3-2y\\
      y^{\prime} = -3x+y^5
    \end{rcases*}\quad P_x+Q_y = 1+3x^2+5y^4
  \end{gathered}
\end{equation*}\par
\noindent This is always positive for $(x,y)\in\R^2\Rightarrow $ no closed orbits in $\R^2$
\par\bigskip
\begin{theo}[]{}
  A closed orbit for the system $\begin{rcases*}x^{\prime}=P(x,y)\\y^{\prime} = Q(x,y)\end{rcases*}$ surrounds at least one equilibrium point. Of course, no equilibrium point implies no closed orbits. 
\end{theo}
\par\bigskip
\begin{theo}[Poincare-Bendixson]{}
  Let $R$ be a closed bounded region without critical points.
  \par\bigskip
  \noindent If there exists an orbit $\Gamma$  which is in $R$ at $t_0$ and $\Gamma(t)\in R\quad\forall t>t_0$, then $\Gamma$ is either a closed orbit, or it spirals towards a closed orbit for $t\to+\infty$
\end{theo}
\par\bigskip
\noindent In a plane, an orbit can:\par
\begin{itemize}
  \item Leave any bounded set as $t\to+\infty$
  \item Can be a closed orbit/equilibrium point 
  \item Approaches a closed orbit/equilibrium point
\end{itemize}
\par\bigskip
\noindent Note that we cannot have chaotic behaviour in the 2D case.
\par\bigskip
\noindent\textbf{Example:}
\begin{equation*}
  \begin{gathered}
    \begin{rcases*}
      x^{\prime} = -y-x(x^2+y^2-\mu)\\
      y^{\prime} = x-y(x^2+y^2-\mu)
    \end{rcases*}\quad\text{ Polar coordinates }\Rightarrow xx^{\prime}+yy^{\prime}=rr^{\prime}\qquad xy^{\prime}-yx^{\prime} = r^2\theta^{\prime}\\
    \begin{rcases*}
      x^{\prime} = r^{\prime}\cos(\theta)-r\sin(\theta)\theta^{\prime}\\
      y^{\prime} = r\sin(\theta)+r\cos(\theta)\theta^{\prime}
    \end{rcases*} xx^{\prime}+yy^{\prime}=r\cos(\theta(r^{\prime}\cos(\theta)-r\sin(\theta)\theta^{\prime})+r\sin(\theta)(r^{\prime}\sin(\theta)+r\cos(\theta)\theta^{\prime}))\\
    = rr^{\prime}\\
    x(-y-x(x^2+y^2-\mu)) + y(x-y(x^2+y^2-\mu)) = -x^2(r^2-\mu)-y^2(r^2-\mu) = -r^2(r^2-\mu) = rr^{\prime}\\
    \Rightarrow r^{\prime} = r(r^2-\mu)
  \end{gathered}
\end{equation*}\par
\noindent We do the same for $\theta$:
\begin{equation*}
  \begin{gathered}
    x(x-y(r^2-\mu))-y(-y-x(r^2-\mu)) = r^2\theta^{\prime}\\
    = x^2+y^2-xy(r^2-\mu)+xy(r^2-\mu) = r^2\theta^{\prime}\\
    \Rightarrow \underbrace{x^2+y^2}_{r^2} = r^2\theta^{\prime}\Rightarrow\theta^{\prime} = 1
  \end{gathered}
\end{equation*}\par
\noindent We arrive at:
\begin{equation*}
  \begin{gathered}
    \begin{rcases*}
      r^{\prime} = r(r^2-\mu)\\
      \theta^{\prime} = 1\quad\text{(constant angular velocity)}
    \end{rcases*}
  \end{gathered}
\end{equation*}
\par\bigskip
\noindent Equilibrium? $r^{\prime}=0\Rightarrow r=0\wedge r = \sqrt{\mu}$ (closed period orbit).\par
\noindent If inside the circle, that is $r<\sqrt{\mu}$, then $r^{\prime}<0$ and we are moving away from the origin. Else, opposite behaviour, except where $r$ becomes $<\sqrt{\mu}$, then it flips, ergo converging to the orbit.
