\section{Important notes from the book}\par
\subsection{Definitions/Theorems}\hfill\\\par
\begin{theo}[Sample/Stickprov]{thm:rsample}
  A sample $x_1,\cdots,x_n$ is of size $n$ and is an observation from the random variable $X = X_,1\cdots,X_n$ with distribution $F$
\end{theo}
\par\bigskip
\begin{theo}[Random Sample]{thm:rrsample}
  If the random variables $X_1,\cdots,X_n$ are independent, then the sample is a \textit{random sample}
\end{theo}
\par\bigskip
\begin{theo}[Estimate/Skattning]{thm:restimate}
  Given a sample from random variables with known distribution function but unknown "distribution function input", an \textit{estimate} $\theta^*(x)$ is a function of the sample attempting to decote the unknown input (parameter)
\end{theo}
\par\bigskip
\noindent\textbf{Anmärkning:}\par
\noindent The correct value one attempts to find is denoted by $\theta$
\par\bigskip
\begin{theo}[Estimator]{thm:restimator}
  The estimation observed in the previous theorem, is an observation from the \textit{estimator}; an observation of observed values of a random variable. The estimator is what the estimate observates, denoted by $\theta^*(X)$ 
\end{theo}
\par\bigskip
\begin{theo}[Bias]{thm:rbias}
  \begin{equation*}
    \begin{gathered}
      E(\theta^*(X))-\theta
    \end{gathered}
  \end{equation*}
\end{theo}
\par\bigskip
\begin{theo}[Random error]{thm:rranderr}
  \begin{equation*}
    \begin{gathered}
      \theta^*-E(\theta^*(X))
    \end{gathered}
  \end{equation*}
\end{theo}
\par\bigskip
\begin{theo}[Total error]{thm:rtotalerr}
  \begin{equation*}
    \begin{gathered}
      \theta^*-\theta = E(\theta^*(X))-\theta + \theta^*-E(\theta^*(X))
    \end{gathered}
  \end{equation*}
\end{theo}
\par\bigskip
\begin{theo}[Unbiased/Väntevärdesriktig]{thm:runbiased}
  \begin{equation*}
    \begin{gathered}
      E(\theta^*(X))-\theta = 0
    \end{gathered}
  \end{equation*}
\end{theo}
\newpage
\begin{theo}[Efficiency]{thm:reff}
  Suppose $\theta_1^*$ and $\theta_2^*$ are unbiased estimates of $\theta$ and
  \begin{equation*}
    \begin{gathered}
      V(\theta_1^*(X))\leq V(\theta_2^*(X))
    \end{gathered}
  \end{equation*}
  \par\bigskip
  \noindent Then $\theta_1^*$ is \textit{more efficient} than $\theta_2^*$
\end{theo}
\par\bigskip
\begin{theo}[Standard error/Medelfel]{thm:rstderr}
  Estimate of the standard deviation, which is $\sqrt{Var}$:
  \begin{equation*}
    \begin{gathered}
      D(\theta^*(X)) = d(\theta^*)
    \end{gathered}
  \end{equation*}
\end{theo}
\par\bigskip
  \begin{theo}[Mean squared error]{thm:rmserr}
    \begin{equation*}
      \begin{gathered}
        M(\theta^*) = E((\theta^*(X)-\theta)^2)
      \end{gathered}
    \end{equation*}
  \end{theo}
  \par\bigskip
  \noindent\textbf{Anmärkning:}\par
  \noindent Recall that $V(X) = E(X^2)-(E(X))^2\Lrarr E(X^2) = V(X)+(E(X))^2$\par
  \noindent Therefore, MSE can be written as $E((\theta^*(X)-\theta)^2) = V(\theta^*(X)-\theta)+\underbrace{(E(\theta^*(X)-\theta))^2}_{\text{bias$^2$}}$
  \par\bigskip
  \begin{theo}[Asymptotically unbiased/Asymptotiskt Väntevärdesriktig]{thm:rasymptconv}
    If the bias $B(\theta_n^*)$ goes to 0 as $n\to\infty$, then it is \textit{asymptotically unbiased}
    \par\bigskip
    \noindent (for all $\theta$ in the parameter-space)
  \end{theo}
  \par\bigskip
  \begin{theo}[Convergence/Konvergens]{thm:rconv}
    The estimator $\theta_n^*(X)$ \textit{converges} to $\theta$:\par
    \begin{itemize}
      \item\textbf{In probability}:\par
        \begin{itemize}
          \item If for every $\varepsilon>0$ $P(\left|\theta_n^*(X)-\theta\right|>0) = 0$ as $n\to\infty$\par
          \item Notice the comparison sign, we are saying "the probability that our estimate is off from the true value by a lot goes to zero"
        \end{itemize}\par
      \item\textbf{In square means}:\par
        \begin{itemize}
          \item If the mean squared error $M(\theta_n^*)\to0$ as $n\to\infty$
        \end{itemize}
    \end{itemize}
  \end{theo}
  \par\bigskip
  \noindent\textbf{Anmärkning:}\par
  \noindent If the estimator converges in square means, then it converges in probability
  \par\bigskip
  \begin{theo}[Consistent]{thm:rconsistent}
    The estimate $\theta_n^*$ is said to be \textit{consistent} if the estimator $\theta_n^*(X)$ converges in probability for all $\theta$
\end{theo}
\par\bigskip
\begin{theo}
  IIf the estimate $\theta_n^*$ is asymptotically unbiased and $V(\theta_n^*(X))\to0$ as $n\to\infty$ for all $\theta$, then our estimate is consistent
\end{theo}
\par\bigskip
\begin{prf}
  TThe mean square error goes to zero, by an earlier remark it therefore converges in probability and can be written in the following way:
  \begin{equation*}
    \begin{gathered}
      M(\theta_n^*) = V(\theta_n^*)+B^2(\theta_n^*)
    \end{gathered}
  \end{equation*}\par
 \noindent Since the estimate is unbiased, the bias = 0, and per the theorem, the variance goes to 0 as $n\to\infty$. Then, by theorem 6.13 it converges in square means and by the remark, it also converges in probability. 
\end{prf}
\par\bigskip
\subsection{Problems and Solutions}\hfill\\\par
\subsubsection{7.2.1}\hfill\\\par
\noindent We are observing events which have the same probability of happening. This is therefore a binomially distributed observation.
\par\bigskip
\noindent Since they have the same probability, one can estimate the probability as $\dfrac{\text{successfull attempts}}{\text{total attempts}} = \dfrac{4}{10} = 0.4$
\par\bigskip
\noindent In order to determine the expected value of our estimator, we look at the expected value for \textit{any} binomially distributed chain of events:
\begin{equation*}
  \begin{gathered}
    E(X) = \mu = np\\
  \end{gathered}
\end{equation*}
\par\bigskip
\noindent Since we are estimating $p^* = \dfrac{X}{n}$, we get the following for our expected value of our estimator:
\begin{equation*}
  \begin{gathered}
    E(p^*) = \dfrac{E(X)}{n} = p
  \end{gathered}
\end{equation*}
