\section{Konfidensintervall/Intervallskattning}\par
\noindent Instead of estimating the parameter, we estimate an interval in which the parameter is most likely to "be" in.
\par\bigskip
\noindent Attempts to answer the question of "was it random?"
\par\bigskip
\noindent\textbf{Recall:}\par
\noindent Let $Z\sim N(0,1)$. The $\alpha$ quantile $\lambda_\alpha$ is defined through $\alpha = P(Z>\lambda_\alpha)$
\par\bigskip
\noindent By symmetry, this gives $P(-\lambda_{\alpha/2}<Z<\lambda_{\alpha/2}) = 1-\alpha$
\par\bigskip
\noindent If $1-\alpha = 0.95$ then we have $\lambda_{\alpha/2} = \lambda_{0.025} = 1.96$ and:
\begin{equation*}
  \begin{gathered}
    P(-1.96<Z<1.96) = 0.95
  \end{gathered}
\end{equation*}
\par\bigskip
\noindent\textbf{Example:}\par
Slide 7 example 2\par
\noindent  We always start with an estimator:
\begin{equation*}
  \begin{gathered}
    \mu^*(X) = \dfrac{X_1+X_2+X_3+X_4+X_5}{5}\sim N(\mu,20)
  \end{gathered}
\end{equation*}
\par\bigskip
\noindent Then standardize it
