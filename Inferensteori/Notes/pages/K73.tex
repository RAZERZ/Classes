\section{Important notes from the book}\par
\subsection{Definitions/Theorems}\hfill\\\par
\begin{theo}[Method of moments/Momentmetoden]{thm:rmm}
  Let $x_1,\cdots,x_n$ random sample from $X$ with $E(X) = m(\theta)$\par
  \noindent If $\theta$ is one dimensional, the moment estimate $\theta = \theta^*$ solves equation $m(\theta) = \overline{x}$ 
\end{theo}
\par\bigskip
\begin{theo}
  LLet $x_1,\cdots,x_n$ random sample from $X$ with $E(X) = \theta$\par
  \noindent The estimate $\theta^* = \overline{x}$ is unbiased and if $\sigma^2 = V(X)<\infty$ then it is consistent as well 
\end{theo}
\par\bigskip
\begin{prf}
  A
  \begin{equation*}
    \begin{gathered}
      E(\overline{X}) = E\left(\dfrac{1}{n}\sum_{i=1}^{n}X_i\right) = \dfrac{1}{n}\sum_{i=1}^{n}E(X_i) = \dfrac{n}{n}E(X_i) = \theta\\
      V(\overline{X}) = V\left(\dfrac{1}{n}\sum_{i=1}^{n}X_i\right) = \dfrac{1}{n^2}\sum_{i=1}^{n}V(X_i) = n\dfrac{\sigma^2}{n^2} = \dfrac{\sigma^2}{n}
    \end{gathered}
  \end{equation*}\par
  \noindent By theorem 6.15, the estimate is unbiased (per def. in this case) and the variance goes to 0 as $n$ increases, therefore it is consistent. 
\end{prf}
\par\bigskip
\begin{theo}[Multivariate method of moments]{thm:rmmm}
  $\theta = (\theta_1,\theta_2)$, moment estimates solve the system:
  \begin{equation*}
    \begin{gathered}
      E(X) = m_1(\theta_1, \theta_2) = \overline{x}\\
      E(X^2) = V(X) + (E(X))^2 = m_2(\theta_1,\theta_2) = \dfrac{1}{n}\sum_{i=1}^{n}x_i^2
    \end{gathered}
  \end{equation*}
\end{theo}
\par\bigskip
\begin{theo}[Sample variance is unbiased]{thm:rsviu}
  Let $x_1,\cdots,x_n$ random sample from a random variable $X$ with variance $\sigma^2$ 
  \par\bigskip
  \noindent The sample variance $s^2$ is an unbiased estimation of $\sigma^2$ 
\end{theo}
\par\bigskip
\noindent\textbf{Anmärkning:}\par
\noindent If $E(X^4)<\infty$ then  $s^2$ is a consistent estimate of $\sigma^2$ 

\par\bigskip
\subsection{Problems and Solutions}\hfill\\\par
\subsubsection{7.2.5}\hfill\\\par
\par\bigskip
\subsubsection{7.2.6}\hfill\\\par
\par\bigskip
\subsubsection{7.2.7}\hfill\\\par
\par\bigskip
\subsubsection{7.2.8}\hfill\\\par
\par\bigskip
\subsubsection{7.2.10}\hfill\\\par
\par\bigskip
\subsubsection{7.2.12}\hfill\\\par
\par\bigskip
