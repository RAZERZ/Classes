\section{Conformal mappings}\par
\noindent Let $D$ be a domain in $\C$, $z_0\in D$.\par
\noindent Suppose $f:D\to\C$ is analytic with $f^{\prime}(z_0)\neq0$. Let $\gamma(t) = x(t)+iy(t)$ be a $C^1$-curve in $D$ through $z_0 = \gamma(0)$ with $\gamma^{\prime}(0)\neq0$. Then $(f\circ\gamma)(t) = f(\gamma(t))$ is a $C^1$-curve through $(f\circ\gamma)(0) = f(z_0)$.
\par\bigskip
\noindent Moreover,
\begin{equation*}
  \begin{gathered}
    (f\circ\gamma)^{\prime}(0) = \dfrac{d}{dt}f(\gamma(t))|_{_{t=0}} = \lim_{t\to0}\dfrac{f(\gamma(t))-f(\gamma(0))}{t}\\
    = \lim_{t\to0}\dfrac{f(\gamma(t))-f(\gamma(0))}{\gamma(t)-\gamma(0)}\cdot\dfrac{\gamma(t)-\gamma(0)}{t} = f^{\prime}(z_0)\gamma^{\prime}(0)
  \end{gathered}
\end{equation*}
\par\bigskip
\noindent From this, we can conclude $(f\circ\gamma)^{\prime}(0) = f^{\prime}(z_0)\gamma^{\prime}(0)$  is a tangent vector to $f\circ\gamma$ at $f(z_0)$
\par\bigskip
\noindent Note that arg$(f\circ\gamma)^{\prime}(0) =\text{arg}(f^{\prime}(z_0)+\text{arg}(\gamma^{\prime}(0)))$
\par\bigskip
\noindent If $\gamma_1$ and $\gamma_2$ are two $C^1$-curves which intersect at $z_0$, then the angle from $(f\circ\gamma_1)^{\prime}(0)$  to $(f\circ\gamma_2)^{\prime}(0)$ is the same as the angle from $\gamma_1^{\prime}(0)$  to $\gamma_2^{\prime}(0)$
\par\bigskip
\begin{theo}[Conformal $C^1$-mapping]{}
  A $C^1$-mapping $f:D\to\C$ is said to be \textit{conformal} at $z_0$ if it satisfies the above paragraph.
  \par\bigskip
  \noindent If $f$ maps $D$ bijectively onto $V$, and if $f$ is conformal at one point $z_0\in D$ , we call $f:D\to V$ a \textit{conformal mapping}
\end{theo}
\par\bigskip
\begin{theo}[]{}
  If $f$ is analytic at $z_0$ and $f^{\prime}(z_0)\neq0$, then $f$ is conformal at $z_0$
\end{theo}
\par\bigskip
\noindent\textbf{Anmärkning:}\par
\noindent One can in fact prove the converse of this theorem.
\newpage
\section{Stereographic projection}\par
\noindent Consider the unit sphere $S\in\R^3$.\par
\noindent Given any point $P = (x_1,x_2,x_3)\in S$ other than the north pole $N = (0,0,1)$, we draw the line through $N$ and $P$.\par
\noindent We define the \textit{stereographic projection} of $P$ to be the point $z = x+iy\in\C\sim (x,y,0)$, where the line intersects the plane $x_3 = 0$. Then the following holds:
\begin{equation*}
  \begin{gathered}
    (x,y,0) = (0,0,1)+t[(x_1,x_2,x_3)-(0,0,1)]
  \end{gathered}
\end{equation*}\par
\noindent Where $t$ is given by $1+t(x_3-1)=0\Lrarr t= \dfrac{1}{1-x_3}$. We arrive at the following:
\begin{equation*}
  \begin{gathered}
    z = x+iy = \dfrac{x_1+ix_2}{1-x_3}
  \end{gathered}
\end{equation*}
\par\bigskip
\noindent Conversely, given $z = x+iy\in\C\sim (x,y,0)$ teh line through $N$ and $z$ is given by:
\begin{equation*}
  \begin{gathered}
    (x_1,x_2,x_3) = (0,0,1)+t[(x,y,0)-(0,0,1)]\qquad t\in\R
  \end{gathered}
\end{equation*}
\par\bigskip
\textbf{Anmärkning:}\par
\noindent The line intersects $S$ when:
\begin{equation*}
  \begin{gathered}
    x_1^2+x_2^2+x_3^2 = 1\\
    \Lrarr (tx)^2+(ty)^2+(1-t)^2 = 1\\
    \Lrarr t^2(x^2+y^2+1)-2t=0\\
    \Lrarr t=0\vee t=\dfrac{2}{x^2+y^2+1} = \dfrac{2}{\left|z\right|^2+1}
  \end{gathered}
\end{equation*}
\par\bigskip
\noindent This corresponds to $P = N$ or:
\begin{equation*}
  \begin{gathered}
    P = \left(\dfrac{2x}{\left|z\right|^2+1},\dfrac{2y}{\left|z\right|^2+1},\dfrac{\left|z\right|^2-1}{\left|z\right|^2+1}\right)
  \end{gathered}
\end{equation*}
\par\bigskip
\noindent Thus, stereographic projections $s:S\backslash N\to\C$ define a bijection.
\par\bigskip
\noindent Letting $\widehat{C} = \C\cup\left\{\infty\right\}$ denote the \textit{extended complex plane} and define $s(N) = \infty$, then $s$ becomes a bijective map from $S$ onto $\widehat{\C}$
\par\bigskip
\begin{theo}[]{}
  Under stereographic projections, circles on $S$ correspond to circles and lines in $\C$
\end{theo}
\par\bigskip
\noindent\textbf{Anmärkning:}\par
\noindent We therefore call circles and lines in $\C$ "circles" in $\widehat{\C} = \C\cup\left\{\infty\right\}$, where lines are considered as "circles through $\infty$"
\newpage
\begin{prf}[]{}
  The general equation for a circle or line in the $z = x+iy$ plane is:
  \begin{equation*}
    \begin{gathered}
      A(x^2+y^2)+Cx+Dy+E = 0
    \end{gathered}
  \end{equation*}
  \par\bigskip
  \noindent Using $z = x+iy = \dfrac{x_1+ix_2}{1-x_3}$, we get:
  \begin{equation*}
    \begin{gathered}
      A\left(\left(\dfrac{x_1}{1-x_3}\right)^2+\left(\dfrac{x_2}{1-x_3}\right)^2\right)+\dfrac{Cx_1}{1-x_3}+\dfrac{Dx_2}{1-x_3}+E=0\\
      \Lrarr A(x_1^2+x_2^2)+Cx_1(1-x_3)+Dx_2(1-x_3)E(1-x_3)^2=0
    \end{gathered}
  \end{equation*}
  \par\bigskip
  \noindent Using $x_1^2+x_2^2+x_3^2=1$, we get:
  \begin{equation*}
    \begin{gathered}
      A(1-x_3^2)+Cx_1(1-x_3)+Dx_2(1-x_3)+E(1-x_3)^2 = 0
    \end{gathered}
  \end{equation*}
  \par\bigskip
  \noindent Dividing by $1-x_3$ yields:
  \begin{equation*}
    \begin{gathered}
      A(1+x_3)+Cx_1+Dx_2+E(1-x_3)=0\\
      \Lrarr Cx_1+Dx_2+(A-E)x_3+A+E=0
    \end{gathered}
  \end{equation*}
  \par\bigskip
  \noindent This is the equation for a plane in $\R^3$ , which intersects $S$ in a circle
\end{prf}
\par\bigskip
\section{Möbius transformations}
\par\bigskip
\begin{theo}[Moebius transformation]{}
  A \textit{Möbius transformation} is a mapping of the form:
  \begin{equation*}
    \begin{gathered}
      T(z) = \dfrac{az+b}{cz+d}\qquad a,b,c,d\in\C
    \end{gathered}
  \end{equation*}
  \par\bigskip
  \noindent Where $ad-bc\neq0$ ($T$ is not constant)
\end{theo}
\par\bigskip
\noindent\textbf{Anmärkning:}\par
\noindent If $c=0$, we let $T(\infty) = \infty$. Then $T:\widehat{\C}\to\widehat{\C}$ is bijective
\par\bigskip
\noindent If $c\neq0$, then:
\begin{equation*}
  \begin{gathered}
    T:\C\backslash\left\{-\dfrac{d}{c}\right\}\to\C\backslash\left\{\dfrac{a}{c}\right\}
  \end{gathered}
\end{equation*}\par
\noindent is a bijection. Letting $T\left(-\dfrac{d}{c}\right)=\infty$, and $T(\infty) = \dfrac{a}{c}$, we extend $T$  to a bijective map $T:\widehat{\C}\to\widehat{C}$
\par\bigskip
\noindent The inverse is found by solving:
\begin{equation*}
  \begin{gathered}
    w = T(z)
  \end{gathered}
\end{equation*}\par
\noindent which gives:
\begin{equation*}
  \begin{gathered}
    z = T^{-1}(w)=
    \begin{cases}
    \dfrac{-dw+b}{cw-a},\;\text{ if } w\neq\dfrac{a}{c}\; w\neq\infty\\
    \infty\;\text{ if } w=\dfrac{a}{c}\\
    \dfrac{-d}{c}\;\text{ if } w =\infty
  \end{cases}
  \end{gathered}
\end{equation*}
\par\bigskip
\noindent\textbf{Anmärkning:}\par
\noindent If we interpret $\dfrac{a}{c}$ and $-\dfrac{d}{c}$ as $\infty$, it also holds for $c=0$
\newpage
\noindent\textbf{Anmärkning:}\par
\begin{equation*}
  \begin{gathered}
    T^{\prime}(z) = \dfrac{d}{dt}\left(\dfrac{ax+b}{cz+d}\right) = \dfrac{a(cz+d)-(az+b)\cdot c}{(cz+d)^2}\\
    = \dfrac{ad-bc}{(xz+d)^2}\neq0
  \end{gathered}
\end{equation*}\par
\noindent Thus $T:\widehat{\C}\to\widehat{\C}$ is conformal
\par\bigskip
\noindent\textbf{Anmärkning:}\par
\noindent If:
\begin{equation*}
  \begin{gathered}
    T(z) = \dfrac{az+b}{cz+d}\qquad S(z) = \dfrac{\alpha z+\beta}{\gamma z+\delta}\\
    \Rightarrow (S\circ T)(z) = \dfrac{\alpha T(z)+\beta}{\gamma T(z)+\delta}\\
    = \dfrac{\alpha\left(\dfrac{az+b}{cz+d}\right)+\beta}{\gamma\left(\dfrac{az+b}{cz+d}\right)+\delta} = \dfrac{(\alpha a+\beta c)z+(\alpha b+\beta d)}{(\gamma a+\delta c)z+(\gamma b+\delta d)}
  \end{gathered}
\end{equation*}
\par\bigskip
\noindent This shows that compositions of Moebius transformations are Möbius transformations.
\par\bigskip
\noindent\textbf{Anmärkning:}\par
\begin{equation*}
  \begin{gathered}
    \begin{pmatrix}\alpha&\beta\\\gamma&\delta\end{pmatrix}\begin{pmatrix}a&b\\c&d\end{pmatrix} = \begin{pmatrix}\alpha a+\beta\delta&ab+\beta d\\\gamma a+\delta c &\gamma b+\delta d\end{pmatrix}
  \end{gathered}
\end{equation*}
\par\bigskip
\begin{lem}[]{}
  If a Moebius transformation $T$ has more than two fixed points in $\widehat{\C}$ ($z_0$ is a fixpoint if $T(z_0) = z_0$), then $T(z) = z$ $\forall z\in\widehat{\C}$
\end{lem}
\par\bigskip
\begin{prf}[]{}
  If $c=0$, then $T(z) = \dfrac{az+b}{d}$, so:
  \begin{equation*}
    \begin{gathered}
      T(z) = z\Lrarr \dfrac{az+b}{d}=z\Lrarr (a-d)z+b = 0
    \end{gathered}
  \end{equation*}\par
  \noindent So $T$ has at most one fixed point in $\C$ unless $a = d$ and $b=0$ $\Lrarr T(z) = z$ $\forall z\in\C$
  \par\bigskip
  \noindent So if $c=0$, $T$ has at most 2 fixed points in $\widehat{\C}$ ($T(\infty) = \infty$) unless $T(z) = z$  $\forall z\in\C$
  \par\bigskip
  \noindent If $c\neq0$, then:
  \begin{equation*}
    \begin{gathered}
      T(z) = z \Lrarr \dfrac{az+b}{cz+d} = z\\
      \Lrarr cz^2+(d-c)z-b=0
    \end{gathered}
  \end{equation*}
  \par\bigskip
  \noindent So $T$ has at most 2 fixed points in $\C$ (and $T(\infty) = \dfrac{a}{c}\neq\infty$) unless $c=0,a=d,b=0$\par
  \noindent This contradicts $c\neq0$
\end{prf}
\par\bigskip
\begin{theo}[]{}
  If $S,T$ are Möbius transformations such that $S(z_i) = T(z_i)$ at three different points $z_1,z_2,z_3\in \widehat{\C}$, then $S = T$
\end{theo}
\par\bigskip
\begin{prf}[]{}
  If $S(z_i) = T(z_i)$ for $i=1,2,3$, then the Moebius transformation $T^{-1}\circ S$ has at least 3 fixed points.\par
  \noindent By the previous lemma:
  \begin{equation*}
    \begin{gathered}
      T^{-1}\circ S(z) = z\quad\forall z\in\widehat{\C}
    \end{gathered}
  \end{equation*}\par
  \noindent i.e $S(z) = T(z)$ $\forall z\in\widehat{\C}$
\end{prf}
\par\bigskip
\noindent\textbf{Anmärkning:}\par
\noindent Particular cases of the Möbius transformation are:\par
\begin{itemize}
  \item $T(z) = z+b$ (\textit{translation})
  \item $T(z) = az = \left|a\right|e^{i\text{arg}(a)z}$ (\textit{rotation \& magnification})
  \item $T(z) = \dfrac{1}{z}$ (\textit{inversion})
\end{itemize}
\par\bigskip
\noindent\textbf{Anmärkning:}\par
\noindent If $c\neq0$:
\begin{equation*}
  \begin{gathered}
    T(z) = \dfrac{az+b}{cz+d} = \dfrac{\dfrac{a}{c}(cz+d)-\dfrac{ad}{c}+b}{cz+d} = \dfrac{a}{c}-\dfrac{ad-bc}{c^2}\dfrac{1}{z+\dfrac{d}{c}}
  \end{gathered}
\end{equation*}\par
\noindent This means that every Moebius transformation is a composition of translations, rotations, magnifications, and inversions. 
\par\bigskip
\begin{theo}[]{}
  Every Möbius transformation maps "circles" onto "circles"
\end{theo}
\par\bigskip
\noindent\textbf{Anmärkning:}\par
\noindent Recall that a "circle" in $\widehat{\C} = \C\cup\left\{\infty\right\}$ is a circle or line in $\C$ . A line in $\C$ is a "circle" through $\infty$ in $\widehat{\C}$
\par\bigskip
\begin{prf}[]{}
  It is easy to see that translations and rotations/magnifications map circles onto circles and line onto lines. This gives enough to prove that inversion:
  \begin{equation*}
    \begin{gathered}
      T(z) = \dfrac{1}{x+iy} = \dfrac{x-iy}{x^2+y^2} 0 \dfrac{x}{x^2+y^2}+i\dfrac{-y}{x^2+y^2} = u+iv
    \end{gathered}
  \end{equation*}\par
  \noindent maps circles onto circles
  \par\bigskip
  A circle in $\widehat{\C}$ has the equation:
  \begin{equation*}
    \begin{gathered}
      A(x^2+y^2)+Cx+Dy+E = 0\\
      \Lrarr A+C\dfrac{x}{x^2+y^2} + D\dfrac{y}{x^2+y^2}+E\dfrac{1}{x^2+y^2} = 0\\
      \Lrarr E(u^2+v^2) +Cu-Dv+A = 0
    \end{gathered}
  \end{equation*}
\end{prf}
\par\bigskip
\noindent Given a "circle" $C_z$ in the $z$-plane and a "circle" $C_w$ in the $w$-plane, can one find a Moebius transformation $T:\widehat{\C}\to\widehat{\C}$  such tath $T(C_z) = C_w$? Yes!
\newpage
\subsection{The cross-ratio}\hfill\\
\par\bigskip
\begin{theo}[Cross-ratio]{}
  Let $z_1,z_2,z_3\in\widehat{\C}$ be distinct and put:
  \begin{equation*}
    \begin{gathered}
      (z,z_1,z_2,z_3) = \dfrac{z-z_1}{z-z_3}\cdot\dfrac{z_2-z_3}{z_2-z_1}\in\widehat{\C}
    \end{gathered}
  \end{equation*}
  \par\bigskip
  \noindent If some of the $z_i$ is $\infty$, the right hand side should be interpret as:
  \begin{equation*}
    \begin{gathered}
      (z,z_1,z_2,z_3)= 
      \begin{cases}
        \dfrac{z_2-z_3}{z-z_3}\;\text{ if } z_1=\infty\\
        \dfrac{z-z_1}{z-z_3}\;\text{ if } z_2=\infty\\
        \dfrac{z-z_1}{z_2-z_1}\;\text{ if } z_3 = \infty
      \end{cases}
    \end{gathered}
  \end{equation*}
  \par\bigskip
  \noindent $(z,z_1,z_2,z_3)$ is called the \textit{cross-ratio} of the four points 
\end{theo}
\par\bigskip
\noindent\textbf{Anmärkning:}\par
\noindent $S(z) = (z,z_1,z_2,z_3)$ is a Möbius transformation such that:
\begin{equation*}
  \begin{gathered}
    S(z_1) = 0\qquad S(z_2) =1\qquad S(z_3) = \infty
  \end{gathered}
\end{equation*}\par
\noindent By an earlier remark, this is the unique Moebius transformation mapping $z_1,z_2,z_3$  to $0,1,\infty$
\par\bigskip
\begin{theo}[]{}
  Given a tripple $z_1,z_2,z_3\in\widehat{\C}$ of distinct points, and another tripple $w_1,w_2,w_3\in\widehat{\C}$ of distinct points, then there is a unique Möbius transformation $T$ such that $T(z_i) = w_i$
  \par\bigskip
  \noindent The mappings $w = T(z)$ is found by solving the cross-ratio equation:
  \begin{equation*}
    \begin{gathered}
      (w,w_1,w_2,w_3) = (z,z_1,z_2,z_3)
    \end{gathered}
  \end{equation*}
\end{theo}
\par\bigskip
\begin{prf}[]{}
  By an earlier remark, there is at most one such mapping. We now prove that there is exactly one by contradicting it.
  \par\bigskip
  \noindent Put $S(z) = (z,z_1,z_2,z_3)$, $U(w) = (w,w_1,w_2,w_3)$:
  \begin{equation*}
    \begin{gathered}
      \Rightarrow T(z) = (U^{-1}\circ S)(z) = U^{-1}(S(z))
    \end{gathered}
  \end{equation*}\par
  \noindent $U^{-1}(S(z))$ is a Moebius transformation such that:
  \begin{equation*}
    \begin{gathered}
      T(z_1) = U^{-1}(S(z_1)) = U^{-1}(0) = w_1\\
      T(z_2)= \cdots\\
      \vdots
    \end{gathered}
  \end{equation*}
  \par
  \noindent Then:
  \begin{equation*}
    \begin{gathered}
      w = T(z) \Lrarr w = U^{-1}(S(z))\Lrarr U(w) =S(z)\\
      \Lrarr (w,w_1,w_2,w_3) = (z,z_1,z_2,z_3)
    \end{gathered}
  \end{equation*}
\end{prf}
\par\bigskip
\noindent This theorem can be used to construct a $T$ as above, mappiong $C_z$ to $C_w$
\par\bigskip
\noindent Let $z_1,z_2,z_3$ be three distinct points on a circle $C_z$ in $\widehat{\C}$. Note that $C_z$ is \textit{oriented} by the order of these points, that is $C_z$ aquires an orientation by proceeding through $z_1,z_2,z_3$ in succession
\par\bigskip
\noindent Since a Möbius transformation is conformed, it maps the region to the left of $C_z$, oriented by $z_1,z_2,z_3$, to the region left of $C_w = T(C_z)$  oriented by $w_1,w_2,w_3$
\par\bigskip
\subsection{Symmetry-preserving property}\hfill\\\par
\noindent Two points $z_1$ and $z_2$ are said to be \textit{symmetric} with respect to a line $L$ if $L$ is the perpendicular bisector of the line-segment joining $z_1$ and $z_2$
\par\bigskip
\noindent This means that every circle or line through $z_1$  and $z_2$ intersects $L$ orthogonally
\par\bigskip
\begin{theo}[]{}
  Two points $z_1$ and $z_2$ are said to be \textit{symmetric} with respect to a circle $C$ if every circle of line through $z_1$ and $z_2$ intersects $C$ orthogonally
\end{theo}
\par\bigskip
\noindent In particular, the center $a$ of $C$ and $\infty$ are symmetric with respect to $C$
\par\bigskip
\begin{theo}[Symmetry principle]{}
  Let $C_z$ be a circle or line in the $z$-plane and $w = T(z)$ be any Moebius transformation. Then two points $z_1$ and $z_2$ are symmetric with respect to $C_z$ if and only if their images $w_1 = T(z_1)$ and $w_2 = T(z_2)$ are symmetric with respect to the image $C_w = T(C_z)$ under $T$.
\end{theo}
\par\bigskip
\begin{prf}[]{}
  "Two points are symmetric with respect to a given circle if and only if every circle containing the points intersects the given circle orthogonally" is a re-formulation of the theorem.
  \par\bigskip
  \noindent Möbius transformations preserve the class of circles, and they also preserve orthogonallity. Hence, they preserve the symmetric condition. 
\end{prf}
