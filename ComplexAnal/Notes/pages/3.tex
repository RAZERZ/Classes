\section{Topology of $\C$}
\par\bigskip
\begin{theo}[Open disc]{}
  The set $D_r(z_0) = \left\{z\in\C\;:\;\left|z-z_0\right|<r\right\}$ is called the \textit{open-disc} with center $z_0$ and radius $r$
\end{theo}
\par\bigskip
\noindent\textbf{Anmärkning:}\par
\noindent Since we have a strict inequality, it is open. If we had $\leq$, it would be a closed disc.
\par\bigskip
\begin{theo}[Open subset]{}
  A subset $M$ of $\C$ is called \textit{open} if for every $z_0\in M$ there exists an $r>0$ such that $D_r(z_0)\subseteq M$
\end{theo}
\par\bigskip
\begin{theo}[Interior point]{}
  A point $z_0\in M$ is called an \textit{interior-point} of $M$ if there exists an $r>0$ such that $D_r(z_0)\subseteq M$
\end{theo}
\par\bigskip
\begin{theo}[Boundary point]{}
  A point$z_0\in\C$ is called a \textit{boundary point} of $M$ if $\forall r>0$ it holds that:
  \begin{equation*}
    \begin{gathered}
      D_r(z_0)\cap M\neq\phi\quad\wedge \quad D_r(z_0)\cap M^c\neq\phi
    \end{gathered}
  \end{equation*}
\end{theo}
\par\bigskip
\noindent\textbf{Anmärkning:}\par
\noindent The set of all interior points of $M$ is denoted by int$(M)$ and the set of all boundary points of $M$  is denoted by $\partial M$
\par\bigskip
\noindent The following equivelances hold:\par
\begin{itemize}
  \item $M$ is closed $\Lrarr$ $\partial M\subseteq M$
  \item $M$ is open $\Lrarr$ $\partial M\subseteq M^c$
  \item $\C$ is clopen
  \item $\O$ is clopen
  \item The union of any collection of open subsets of $\C$ is open
  \item The intersection of any finite collection of open subsets of $\C$ is open
\end{itemize}
\par\bigskip
\begin{theo}[Closed set]{}
  We say that a set $X\subseteq\C$ is closed if its complement $X^c$ is open
\end{theo}
\par\bigskip
\begin{theo}[Polygonal path]{}
  A polygonal path $P$ (sometimes called piecewise linear curve) is a curve specified by a sequence of points $(A_1,A_2,\cdots,A_n)$.\par
  \noindent The curve itself consists of line segments connecting the consecutive points. 
\end{theo}
\par\bigskip
\begin{theo}[polygonal-path-connected open set]{}
  An open set $M$ is called \textit{polygonal-path-connected} if eveyr pair of points $z_1,z_2\in M$ can be connected by a polygonal path contained in $M$ 
\end{theo}
\par\bigskip
\noindent\textbf{Anmärkning:}\par
\noindent Some would call this just path-connected, or even just connected. This works in $\R^n$ (recall that $\C\cong\R^2$). Topologically speaking, polygonal-path-connectedness $\implies$ path-connectedness
\par\bigskip
\noindent\textbf{Anmärkning:}\par
\noindent A set $X$ is connected $\Lrarr$ the only subsets of $X$ which are clopen are $\O$ and $X$
\par\bigskip
\noindent\textbf{Anmärkning:}\par
\noindent One can assume the polygonal paths to have segments parallell to the ordinale ones.
\par\bigskip
\noindent\textbf{Anmärkning:}\par
\noindent An open connected set is called a \textit{domain}
\par\bigskip
\begin{theo}[]{}
  Suppose that $u(x,y)$ is a real-valued function defined in a domain $D\subseteq\R²$\par
  \noindent Also suppose that:
  \begin{equation*}
    \begin{gathered}
      \dfrac{\partial u}{\partial x} = \dfrac{\partial u}{\partial y} = 
    \end{gathered}
  \end{equation*}
  \par\bigskip
  \noindent in all of $D$. Then $u$ is contained in $D$
\end{theo}
\par\bigskip
\begin{theo}[Simply connected]{}
  A domain $D\subseteq\C$ is called \textit{simply connected} if ever closed curve in $D$ can be, within $D$, continously deformed to a point 
\end{theo}
\par\bigskip
\noindent\textbf{Anmärkning:}\par
\noindent Topologically speaking, $D$ is homeomorphic to a point. 
\par\bigskip
\begin{theo}[Non-connectedness]{}
  A set $A\subseteq\C$ is \textit{not connected} if there are open sets $U$ and $V$ such that:\par
  \begin{itemize}
    \item $A\subseteq U\cup V$
    \item $A\cap U\neq\O$ and $A\cap V\neq\O$
  \end{itemize}
\end{theo}
\newpage
\subsection{Limits and Continuity}\hfill\\
\par\bigskip
\begin{theo}[Complex limit]{}
  A sequence $\left\{z_n\right\}_{n=1}^\infty$ of complex numbers is said to have the limit $z_0$ (\textit{converges to} $z_0$) if for every given $\varepsilon>0$, there exists an integer $N\geq1$  such that
  \begin{equation*}
    \begin{gathered}
      \left|z_n-z_0\right|<\varepsilon\quad \forall n\geq N
    \end{gathered}
  \end{equation*}
  \par\bigskip
  \noindent We write this as:
  \begin{equation*}
    \begin{gathered}
      \lim_{n\to\infty}z_n = z_0
    \end{gathered}
  \end{equation*}
\end{theo}
\par\bigskip
\noindent\textbf{Anmärkning:}\par
\noindent Every cauchy sequence in $\C$ converges. 
\par\bigskip
\noindent\textbf{Anmärkning:}\par
\noindent $z_n\to z_0\Lrarr \text{Re}(z_n)\to\text{Re}(z_0)$ and Im$(z_n)\to\text{Im}(z_0)$\par
\noindent This follows from $\left|x\right|, \left|y\right|\leq\sqrt{x^2+y^2}\leq\left|x\right|+\left|y\right|$ 
\par\bigskip
\begin{theo}[]{}
  Let $f$ be a function defined ina  punctured neighborhood of $z_0$
  \par\bigskip
  \noindent We say that $f$ has the limit $w_0$ as $z\to z_0$, if for every given $\varepsilon>0$ there exists $\delta>0$ such that:
  \begin{equation*}
    \begin{gathered}
      0<\left|z-z_0\right|<\delta\implies \left|f(z)-w_0\right|<\varepsilon
    \end{gathered}
  \end{equation*}
  \par\bigskip
  \noindent We write this as:
  \begin{equation*}
    \begin{gathered}
      \lim_{z\to z_0}f(z) = w_0
    \end{gathered}
  \end{equation*}
\end{theo}
\par\bigskip
\noindent\textbf{Anmärkning:}\par
\noindent If a limit exists, it is unique.
\par\bigskip
\begin{theo}[]{}
  For $z = x+iy$, let:
  \begin{equation*}
    \begin{gathered}
      u(x,y) = \text{Re}(f(z))\qquad v(x,y) = \text{Im}(f(z))
    \end{gathered}
  \end{equation*}
  \par\bigskip
  \noindent Let $z_0 = x_0+iy_0$ and $w_0 = u_0+iv_0$\par
  \noindent Then the following holds:
  \begin{equation*}
    \begin{gathered}
      \lim_{z\to z_0} f(z) = w_0\Lrarr\begin{cases}\lim_{(x,y)\to(x_0,y_0)}u(x,y) = u_0\\\lim_{(x,y)\to(x_0,y_0)}v(x,y) = v_0\end{cases}
    \end{gathered}
  \end{equation*}
\end{theo}
\par\bigskip
\begin{theo}[Continous function]{}
  Let $f$ be a function defined in a neighborhood of $z_0$.
  \par\bigskip
  \noindent $f$ is said to be continous at $z_0$ if:
  \begin{equation*}
    \begin{gathered}
      \lim_{z\to z_0} f(z) = f(z_0)
    \end{gathered}
  \end{equation*}
  \par\bigskip
  \noindent A function $f$ is said to be \textit{continous on the (open) set} $M$ if it is continous at each point of $M$ 
\end{theo}
\newpage
\noindent\textbf{Anmärkning:}\par
\noindent The following statements are equivalent (for $f:A\to\C$):\par 
\begin{itemize}
  \item $f$ is continous
  \item The inverse image of every closed set is closed relative to $A$
  \item The inverse image of every open set is open relative to $A$
  \item The image set $f(A)$ is connected
\end{itemize}
\par\bigskip
\noindent Assume $\lim_{z\to z_0} f(z) = A$ and $\lim_{z\to z_0}g(z) = B$\par
\noindent The following properties from the real limit hold for the complex limit:\par
\begin{itemize}
  \item $\lim_{z\to z_0}(f(z)\pm g(z)) = A\pm B$
  \item $\lim_{z\to z_0} f(z)g(z) = AB$
  \item $\lim_{z\to z_0}\dfrac{f(z)}{g(z)} = \dfrac{A}{B}\qquad B\neq0$
\end{itemize}
\par\bigskip
\noindent\textbf{Anmärkning:}\par
\noindent If $f,g$ are continous at $z_0$, then so are $f\pm g$ and $fg$. The quotient is only continous if $g(z_0)\neq0$
\par\bigskip
\noindent\textbf{Anmärkning:}\par
\noindent Constant functions, polynomials, and rational functions (whenever the denominator is non-zero) are all continous in $\C$
\par\bigskip
\subsection{The complex derivative}\hfill\\\par
\noindent Analogous to the real case, we also have the following:
\par\bigskip
\begin{theo}[Differentiability]{}
  Let $f$ be a complex-valued function defined in a neighborhood of $z_0$.\par
  \noindent We say that $f$ is differentiable at $z_0$ if the limit:
  \begin{equation*}
    \begin{gathered}
      \lim_{\Delta z\to0}\dfrac{f(z_0+\Delta z)-f(z_0)}{\Delta z}
    \end{gathered}
  \end{equation*}\par
  \noindent exists.\par
  \noindent The limit is called the \textit{derivative} of $f$ at $z_0$, and is denoted by $f^{\prime}(z_0)$ or $\dfrac{df}{dz}(z_0)$
\end{theo}
\par\bigskip
\noindent\textbf{Anmärkning:}\par
\noindent Since $\Delta z$ is a complex unmber, it can approach $0$ from different directions. In order for the derivative to exist, the results must be independent of the direction of which $\Delta z$ approaches 0 (i.e, approaches 0 from all directions)
\par\bigskip
\noindent\textbf{Anmärkning:}\par
\noindent If $X$ is an open connceted set and $a,b\in X$, then there is a differntiable path $\gamma:[0,1]\to X$ with $\gamma(0) = a$ and $\gamma(1) = b$
\par\bigskip
\noindent\textbf{Example}:\par
\noindent The function $f(z) = \overline{z}$ is nowhere differentiable since:
\begin{equation*}
  \begin{gathered}
    \dfrac{f(z_0+\Delta z)-f(z_0)}{\Delta z} = \dfrac{\overline{z_0+\Delta z}-\overline{z_0}}{\Delta z} = \dfrac{\overline{\Delta z}}{\Delta z} = \dfrac{\overline{\Delta x+i\Delta y}}{\Delta x+i\Delta y}
  \end{gathered}
\end{equation*}\par
\noindent As $\Delta z\to0$ from the $x$-direction (real-line), the limit becomes $\dfrac{\overline{x}}{x} = 1$\par
\noindent However, as we approach from the $y$-direction (complex axis), the limit becomes $\dfrac{\overline{iy}}{iy} = \dfrac{-y}{y} = -1$\par
\noindent Since $x,y$ were chosen arbitrarily, this applies to all $x,y$. Since the limits did not match, it is not differentiable and at no point.
\par\bigskip
\noindent Of course, all the properties from the real case hold here as well.\par
\noindent Suppose $f,g$ are differentiable at $z$, then:\par
\begin{itemize}
  \item $(f\neq g)^{\prime}(z) = f^{\prime}(z)\neq g^{\prime}(z)$
  \item $(cf)^{\prime}(z) = cf^{\prime}(z)$
  \item $(fg)^{\prime}(z) = f^{\prime}(z)g(z)+f(z)g^{\prime}(z)$
  \item $(f\circ g)^{\prime}(z) = f^{\prime}(g(z))g^{\prime}(z)$
\end{itemize}
\newpage
\subsection{Analytic functions}\hfill\\
\par\bigskip
\begin{theo}[Analytic function]{}
  A complex-valued function $f$ is said to be \textit{analytic} in an open set $G$ if $f$ is differentiable at every point in $G$.
  \par\bigskip
  \noindent We say that $f$ is \textit{analytic at $z_0$}  if $f$ is differentiable in a neighborhood of $z_0$
\end{theo}
\par\bigskip
\noindent\textbf{Anmärkning:}\par
\noindent If $f$ is analytic in all of $\C$, then $f$ is said to be \textit{entire} (or \textit{holomorphic}).
\par\bigskip
\begin{theo}[]{}
  If an entire function $f(z)$ has a root at $w$, then:
  \begin{equation*}
    \begin{gathered}
      \lim_{z\to w}\dfrac{f(z)}{(z-w)}
    \end{gathered}
  \end{equation*}\par
  \noindent is an entire function.
\end{theo}
