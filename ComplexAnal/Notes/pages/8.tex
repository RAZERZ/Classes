\section{Independence of paths \& Cauchy's integral theorem}
\par\bigskip
\subsection{Independence of paths}\hfill\\
\par\bigskip
\begin{theo}[]{}
  Suppose that $f(z)$ is continous in a domain $D$ and that $f(z)$ has an antiderivative $F(z)$ in $D$, i.e $F^{\prime}(z) = f(z)$ $\forall z\in D$
  \par\bigskip
  \noindent Let $\Gamma $ be a contour in $D$ with initial point $z_I$ and terminal point $z_T$, then:
  \begin{equation*}
    \begin{gathered}
      \int_{\Gamma}f(z)dz = F(z_T)-F(z_T)
    \end{gathered}
  \end{equation*}
\end{theo}
\par\bigskip
\begin{prf}[]{}
  \begin{equation*}
    \begin{gathered}
      \int_{\Gamma}f(z) = \sum_{k}\int_{\gamma_k}f(z)dz = \sum_{k}\int_{\tau_{k-1}}^{\tau_k}f(z(t))z^{\prime}(t)dt
    \end{gathered}
  \end{equation*}\par
  \noindent where $z(t)$, $\tau_{k-1}\leq t\leq\tau_k$ is a parametrization of $\gamma_k$
  \par\bigskip
  \noindent Now:
  \begin{equation*}
    \begin{gathered}
      \dfrac{d}{dt}F(z(t)) = F^{\prime}(z(t))z^{\prime}(t) = f(z(t))z^{\prime}(t)
    \end{gathered}
  \end{equation*}\par
  \noindent So by Sats 11.66, we have:
  \begin{equation*}
    \begin{gathered}
      \int_{\tau_{k-1}}^{\tau_k}f(z(t))z^{\prime}(t)dt = F(z(\tau_k))-F(z(\tau_{k-1}))
    \end{gathered}
  \end{equation*}\par
  \noindent Sum over $k$
\end{prf}
\par\bigskip
\noindent\textbf{Anmärkning:}\par
\noindent If $f$ is continous in a domain $D$ and has an antiderivative in $D$, then:
\begin{equation*}
  \begin{gathered}
    \int_{\Gamma}f(z)dz = 0
  \end{gathered}
\end{equation*}\par
\noindent for every closed contour in $D$
\par\bigskip
\begin{theo}[]{}
  Let $f$ be continous in a domain $D$ , then the following are equivalent:\par
  \begin{itemize}
    \item $f$ has an antiderivative in $D$
    \item $\int_{\Gamma}f(z)dz = 0$ for every closed contour $\Gamma$ in $D$
    \item Contour integrals are independent of path in $D$\par
      \begin{itemize}
        \item If $\Gamma_1$ and $\Gamma_2$ are 2 contours with the same initial and terminal point, then $\int_{\Gamma_1}f(z) dz = \int_{\Gamma_2}f(z)dz$ 
      \end{itemize}
  \end{itemize}
\end{theo}
\par\bigskip
\begin{prf}[]{}
  \textbf{(I) }$\Rightarrow$\textbf{ (II)}: see anmärkning above.
  \par\bigskip
  \noindent\textbf{(II) }$\Rightarrow$\textbf{ (III)}: Given $\Gamma_1$ and $\Gamma_2$, let:
  \begin{equation*}
    \begin{gathered}
      \Gamma = \Gamma_1+(-\Gamma_2)\Rightarrow = 0 = \int_{\Gamma} = \int_{\Gamma_1}+\int_{-\Gamma_2} = \int_{\Gamma_1}-\int_{\Gamma_2}\\
      \Rightarrow \int_{\Gamma_1} = \int_{\Gamma_2}
    \end{gathered}
  \end{equation*}
  \par\bigskip
  \noindent\textbf{(III) }$\Rightarrow$\textbf{ (I)}: Fix $z_0\in D$. Since $D$ is a domain, $\Rightarrow$ for any $z\in D$ there is a polygonal path $\Gamma$ from $z_0$ to $z$
  \par\bigskip
  \noindent Define $F(z)$ = $\int_{\Gamma}f(s)ds$
  \par\bigskip
  \noindent $F(z)$ is well defined, i.e independed of the choice of $\Gamma$. By \textbf{(III)}.
  \par\bigskip
  \noindent We now show that $F^{\prime}(z) = f(z)$ $\forall z\in D$ by looking at a line segment $L$:
  \begin{equation*}
    \begin{gathered}
      \Rightarrow F(z+\Delta z)-F(z) = \int_{L}f(s)ds = \int_{L}f(z)ds +\int_{L}(f(s)-f(z))ds\\
      = f(z)\Delta z + \int_{L}(f(s)-f(z))ds
    \end{gathered}
  \end{equation*}\par
  \noindent i.e:
  \begin{equation*}
    \begin{gathered}
      \dfrac{F(z+\Delta z)-F(z)}{\Delta z} = f(z)+\dfrac{1}{\Delta z}\int_{L}(f(s)-f(z))ds
    \end{gathered}
  \end{equation*}\par
  \noindent By the ML-inequality:
  \begin{equation*}
    \begin{gathered}
      \left|\dfrac{1}{\Delta z}\int_{L}(f(s)-f(z))ds\right|\leq\dfrac{1}{\left|\Delta z\right|}\cdot\max_{s\in L}\left|f(s)-f(z)\right|\cdot\left|\Delta z\right|\stackrel{\Delta z\to0}{\to}0
    \end{gathered}
  \end{equation*}\par
  \noindent Thus:
  \begin{equation*}
    \begin{gathered}
      F^{\prime}(z) = \lim_{\Delta z\to0}\dfrac{F(z+\Delta z)-F(z)}{\Delta z} = f(z)
    \end{gathered}
  \end{equation*}
\end{prf}
\par\bigskip
\subsection{Cauchy's integral theorem}\hfill\\\par
\noindent Let $\Gamma$ bea  simple closed contour in $\C$ parametrized by $z = z(t)$ $a\leq t\leq b$:
\begin{equation*}
  \begin{gathered}
    \int_{\Gamma}f(z)dz = \int_{a}^{b}f(z(t))\dfrac{dz}{dt}dt\\
    = \int_{a}^{b}(u(x(t),y(t))+iv(x(t), y(t)))\left(\dfrac{dx}{dt}+i\dfrac{dy}{dt}\right)dt\\
    = \int_{a}^{b}\left(u(x(t),y(t))\dfrac{dx}{dt}-v(x(t),y(t))\dfrac{dy}{dt}\right)dt+i\int_{a}^{b}(v(x(t),y(t))+u(x(t),y(t))\dfrac{dy}{dt})dt
  \end{gathered}
\end{equation*}\par
\noindent This leads us to:
\begin{equation*}
  \begin{gathered}
    \int_{\Gamma}f(z)dz = \int_{\Gamma}(udx-vdy)+i\int_{\Gamma}(vdx+udy)
  \end{gathered}
\end{equation*}
\par\bigskip
\noindent We can use Greens theorem:
\par\bigskip
\begin{theo}[Greens theorem]{}
  Let $\overline{F}(x,y) = (F_1(x,y),F_2(x,y))$ be a $C^1$-vector field defined on a simply connected domain $D$, and let $\Gamma$ be a positively oriented simple closed contour in $D$
  \par\bigskip
  \noindent Then:
  \begin{equation*}
    \begin{gathered}
      \int_{\Gamma}(F_1dz+F_2dy) = \iint_{\Omega}\left(\dfrac{\partial F_2}{\partial x}-\dfrac{\partial F_1}{\partial y}\right)dxdy
    \end{gathered}
  \end{equation*}
  \par\bigskip
  \noindent Where $\Omega$ denotes the region interior to $\Gamma$
\end{theo}
\par\bigskip
\noindent Let's see if we can use this on the expression for $\int_{\Gamma}f(z)dz$:
\begin{equation*}
  \begin{gathered}
    \int_{\Gamma}f(z)dz = \int_{\Gamma}(udx-vdy)+i\int_{\Gamma}(vdx+udy) = \iint_{\Omega}\left(-\dfrac{\partial v}{\partial x}-\dfrac{\partial u}{\partial y}\right)dxdy + i\iint_{\Omega}\left(\dfrac{\partial u}{\partial x}-\dfrac{\partial v}{\partial y}\right)dxdy
  \end{gathered}
\end{equation*}
\par\bigskip
\noindent Now, if we suppose that $u,v\in C^1$ and assume that $f$ is analytic in $D$, then:
\begin{equation*}
  \begin{gathered}
    \int_{\Gamma}f(z)dz = 0
  \end{gathered}
\end{equation*}\par
\noindent In view of the Cauchy-Riemann equations
\par\bigskip
\noindent The following holds:
\par\bigskip
\begin{theo}[Cauchy's integral theorem]{}
  Suppose that $f$ is analytic in a simply connected domain $D$, and let $\Gamma$ be any closed contour in $D$, then:
  \begin{equation*}
    \begin{gathered}
      \int_{\Gamma}f(z)dz = 0
    \end{gathered}
  \end{equation*}
\end{theo}
\par\bigskip
\noindent\textbf{Anmärkning:}\par
\noindent The theorem generalizes our discussion in two ways.\par
\noindent First, $\Gamma$ can be any closed contour, it does not need to be simple.\par
\noindent Second, the assumption that $u,v\in C^2$ has been dropped. The fact that the second assumption is not necessary was first demonstrated by Edouard Goursat
\par\bigskip
\begin{theo}[Cauchy-Gourat theorem]{}
  If $f$ is analytic inside and on a simple closed contour, then:
  \begin{equation*}
    \begin{gathered}
      \int_{\Gamma}f(z)dz = 0
    \end{gathered}
  \end{equation*}
\end{theo}
\par\bigskip
\noindent Combined with the theorem of path Independence, we have the following:
\par\bigskip
\begin{theo}[]{}
  Suppose that $f$ is analytic in a simply connected domain. Then $f$ has an antiderivative, contour integrals are independent of path, and integrals over closed contours are 0
\end{theo}
