\section{Power series and Taylor series}
\par\bigskip
\subsection{Power series}\hfill\\
\par\bigskip
\begin{theo}[]{}
  A series of the form $\sum_{j=0}^{\infty}a_j(z-z_0)^j$ is called a \textit{power series}.\par
  \noindent The constans $a_j$ are called the \textit{coefficients} of the power series.
\end{theo}
\par\bigskip
\begin{theo}[]{}
  For any power series $\sum_{j=0}^{\infty}a_j(z-z_0)^j$, there exists an $R\in[0,+\infty]$ (dependent on $\left\{a_j\right\}$) such that:\par
  \begin{itemize}
    \item The seris converges absolutely for $\left|z-z_0\right|<R$
    \item The series converges uniformly in any closed subdisk $\left|z-z_0\right|\leq r<R$
    \item The series diverges for $\left|z-z_0\right|<R$
  \end{itemize}
  \par\bigskip
  \noindent The number $R$ is called the \textit{radius of convergence} of the series
\end{theo}
\par\bigskip
\noindent The proof depends on the following:
\par\bigskip
\begin{lem}[]{}
  Suppose that
  \begin{equation}
    \begin{gathered}
      \sum_{j=0}^{\infty}a_jw^j
    \end{gathered}
  \end{equation}
  converges at some point having modulus $r_0>0$.
  \par\bigskip
  \noindent Then the series $(4)$ converges absolutely and uniformly in any closed subdisk $|w|\leq r<r_0$
\end{lem}
\par\bigskip
\begin{prf}[]{}
  Suppose that the series $(4)$ converges at some $w_0$ with $\left|w_0\right| = r_0$
  \begin{equation*}
    \begin{gathered}
      \Rightarrow \left|a_jw_0^j\right|\to0\Rightarrow\exists M\text{ s.t } \left|a_jw_0^j\right| = \left|a_j\right|r_0^j\leq M\qquad\forall j\geq0
    \end{gathered}
  \end{equation*}
  \par\bigskip
  \noindent For $\left|w\right|\leq r$ it follows that:
  \begin{equation*}
    \begin{gathered}
      \left|a_jw^j\right| = \left|a_j\right|r_0^j\left(\dfrac{\left|w\right|}{r_0}\right)^j\leq M\left(\dfrac{r}{r_0}\right)^j
    \end{gathered}
  \end{equation*}
  \par\bigskip
  \noindent Letting $M_j = M\left(\dfrac{r}{r_0}\right)^j$ then the result follows by Weierstrass $M$-test
\end{prf}
\newpage
\begin{prf}[av Sats 17.2]{}
  Let $w = z-z_0$\par
  \noindent If the series $\sum_{j=0}^{\infty}a_jw^j$ converges only for $w = 0$, or it converges for all $w\in\C$, we are done.
  \par\bigskip
  \noindent Otherwise, let $R$ be the greatest $r$ such that $(4)$ converges for some $w$ with $\left|w\right| = r$.\par
  \noindent More precisely, let $M = \left\{r>0\;:\;(4)\text{ converges for some $w$ with $\left|w\right| = r$}\right\}$.\par
  \noindent Then, $M$ is non-empty and upper-bounded.
  \par\bigskip
  \noindent Let $R = \sup M$. For any $r<R$ there exists $r_1$ with $r<r_1\leq R$ such that $r_1\in M$.
  \par\bigskip
  \noindent By the lemma, $(4)$ converges absolutely and uniformly for $\left|w\right| = \left|z-z_0\right|\leq r$\par
  \noindent If $\left|w\right|>R$, then $(4)$ diverges.
\end{prf}
\par\bigskip
\noindent\textbf{Example:}\par
\noindent The geometric series $\sum_{j=0}^{\infty}z^j$ has radius of convergence $R = 1$..\par
\noindent The series does not converge if $\left|z\right| = 1$, since the terms do not tend to zero.
\par\bigskip
\noindent\textbf{Example:}\par
\noindent The power series $\sum_{j=0}^{\infty}\dfrac{z^j}{j^2}$ converges uniformly for for $\left|z\right|\leq 1$.\par
\noindent This follows from the Weierstrass $M$-test with $M_j = \dfrac{1}{j^2}$\par
\noindent It diverges if $\left|z\right|>1$ since $\dfrac{r^j}{j^2}\not\to0$ if $r>1$, hence $R = 1$
\par\bigskip
\noindent Are there any formulae for $R$?\par
\noindent In practise, the follwing is often useful:
\par\bigskip
\begin{theo}[]{}
  Consider a power series
  \begin{equation*}
    \begin{gathered}
      \sum_{j=0}^{\infty}a_j(z-z_0)^j
    \end{gathered}
  \end{equation*}
  \par\bigskip
  \begin{itemize}[leftmargin=*]
    \item\textit{Ratio test}: If
      \begin{equation*}
        \begin{gathered}
          L = \lim_{j\to\infty}\left|\dfrac{a_{j+1}}{a_j}\right|
        \end{gathered}
      \end{equation*}\par
      exists, then $R = \dfrac{1}{L}$
      \par\bigskip
    \item\textit{Root test}: If
      \begin{equation*}
        \begin{gathered}
          L = \lim_{j\to\infty}\sqrt[j]{\left|a_j\right|}
        \end{gathered}
      \end{equation*}\par
      exists, then $R = \dfrac{1}{L}$
  \end{itemize}
\end{theo}
\par\bigskip
\noindent\textbf{Anmärkning:}\par
\begin{itemize}
  \item In both cases it is understood that $R = +\infty$ if $L = 0$ and $R = 0$ if $L = +\infty$
  \item In fact, the following formula, due to Hadamard, is true for any power series:
    \begin{equation*}
      \begin{gathered}
        R = \dfrac{1}{\lim_{j\to\infty\sup\sqrt[j]{\left|a_j\right|}}}
      \end{gathered}
    \end{equation*}
\end{itemize}
\newpage
\begin{prf}[]{}
  Follows directly from the standard ratio and root test.
  \par\bigskip
  \noindent Let $c_j(z) = a_j(z-z_0)^j$:
  \begin{equation*}
    \begin{gathered}
      \Rightarrow \left|\dfrac{c_{j+1}(z)}{c_j(z)}\right| = \left|\dfrac{a_{j+1}}{a_j}\right|\left|z-z_0\right|\to L\left|z-z_0\right|\quad j\to\infty
    \end{gathered}
  \end{equation*}
  \par\bigskip
  \noindent So by the ratio test, if $L\left|z-z_0\right|<1$ the series converges, and if $L\left|z-z_0\right|>1$ the series diverges. Thus $R = \dfrac{1}{L}$
\end{prf}
\par\bigskip
\noindent\textbf{Example:}\par
\begin{itemize}[leftmargin=*]
  \item The series $\sum_{j=0}^{\infty}\dfrac{a_j}{j!}$ has $R = +\infty$, since $a_j = \dfrac{1}{j!}$ and so $\left|\dfrac{a_{j+1}}{a_j}\right| = \dfrac{1}{j+1}\stackrel{j\to\infty}{\to0}$
    \par\bigskip
  \item The series $\sum_{j=0}^{\infty}j!a^j$ has $R = 0$, since $a_j = j!$ and so $\left|\dfrac{a_{j+1}}{a_j}\right| = j+1\stackrel{j\to\infty}{\to+\infty}$
\end{itemize}
\par\bigskip
\noindent Since the partial sums of a power series are analytic (polynomials!), and converge uniformly in each closed subdisk of the "disks of convergence", the convergence theorems of Chapter 16 imply:
\par\bigskip
\begin{theo}[]{}
  Suppose $\sum_{k=0}^{\infty}a_k(z-z_0)^k$ has radius of convergence $R>0$.\par
  \noindent Then the function
  \begin{equation*}
    \begin{gathered}
      f(z) = \sum_{k=0}^{\infty}a_k(z-z_0)^k\qquad\left|z-z_0\right|<R
    \end{gathered}
  \end{equation*}\par
  \noindent is analytic. The derivatives of $f$ are obtained by termwise differentiation of the series:
  \begin{equation*}
    \begin{gathered}
      f^{\prime}(z) = \sum_{k=1}^{\infty}ka_k(z-z_0)^{k-1}\\
      f^{\prime}\prime(z) = \sum_{k=2}^{\infty}k(k-1)a_k(z-z_0)^{k-2}\\
      \vdots
    \end{gathered}
  \end{equation*}
  \par\bigskip
  \noindent In particular, the coefficients $a_k$ are given by
  \begin{equation*}
    \begin{gathered}
      a_k = \dfrac{1}{k!}f^{(k)}(z_0)
    \end{gathered}
  \end{equation*}
\end{theo}
\par\bigskip
\noindent\textbf{Anmärkning:}\par
\begin{equation*}
  \begin{gathered}
    f(z_0) = a_0\\
    f^{\prime}(z_0) = \sum_{k=1}^{\infty}ka_k(z-z_0)^{k-1}\Big|_{z = z_0} = 1\cdot a_1\\
    f^{\prime}\prime(z_0) = \sum_{k=2}^{\infty}k(k-1)a_k(z-z_0)^{k-2}\Big|_{z = z_0} = 2\cdot1\cdot a_2\\
    \vdots\\
    f^{(n)}(z_0)  =\sum_{k=n}^{\infty}k(k-1)\cdot\cdots\cdot(k-n+1)a_k(z-z_0)^{k-n}\Big|_{z= z_0} = n!a_n
  \end{gathered}
\end{equation*}
\par\bigskip
\noindent\textbf{Example:}\par
\noindent We have that:
\begin{equation*}
  \begin{gathered}
    \dfrac{1}{1-z} = \sum_{k=0}^{\infty}z^k\qquad \left|z\right|<1
  \end{gathered}
\end{equation*}\par
\noindent Since we may differentiate termwise:
\begin{equation*}
  \begin{gathered}
    \dfrac{1}{(1-z)^2} = \sum_{k=1}^{\infty}kz^{k-1}\qquad \left|z\right|<1
  \end{gathered}
\end{equation*}\par
\noindent Since we may also integrate termwise:
\begin{equation*}
  \begin{gathered}
    \int_{0}^{w}\dfrac{1}{1-z}dz = \sum_{k=0}^{\infty}\int_{0}^{w}z^kdz = \sum_{k=0}^{\infty}\dfrac{w^{k+1}}{k+1} = \sum_{n=1}^{\infty}\dfrac{w^n}{n}\qquad\left|w\right|<1
  \end{gathered}
\end{equation*}\par
\noindent But
\begin{equation*}
  \begin{gathered}
    \int_{0}^{w}\dfrac{1}{1-z}dz = [-\text{Log}(1-z)] = -\text{Log}(1-w)\\
    \Rightarrow\text{Log}(1-w) = -\sum_{n=1}^{\infty}\dfrac{w^n}{n}\qquad\left|w\right|<1
  \end{gathered}
\end{equation*}\par
\noindent Let $w = -z$:
\begin{equation*}
  \begin{gathered}
    \Rightarrow\text{Log}(1+z) = \sum_{n=1}^{\infty}\dfrac{(-1)^{n+1}}{n}z^n\qquad\left|z\right|<1
  \end{gathered}
\end{equation*}
\par\bigskip
\noindent \textbf{Example:}\par
\noindent Let $f(z) = \sum_{n=0}^{\infty}\dfrac{z^n}{n!}$\par
\noindent Since $R = +\infty$, $f$ is entire. One can might quickly notice that
\begin{equation*}
  \begin{gathered}
    f^{\prime}(z) = \sum_{n=1}^{\infty}\dfrac{z^{n-1}}{(n-1)!} = f(z)\qquad f(0) = 0\\
    \Rightarrow\dfrac{d}{dz}(f(z)e^{-z}) = \underbrace{f^{\prime}(z)}_{\text{$f(z)$}}e^{-z}-f(z)e^{-z} = 0\\
    \Rightarrow f(z)e^{-z} = c = \text{ constant}
  \end{gathered}
\end{equation*}\par
\noindent Let $z = 0\Rightarrow c=1\Rightarrow f(z) = e^z$
\begin{equation*}
  \begin{gathered}
    \Rightarrow e^z = \sum_{n=0}^{\infty}\dfrac{z^n}{n!}\qquad z\in\C
  \end{gathered}
\end{equation*}
\par\bigskip
\subsection{Taylor Series}\hfill\\
\par\bigskip
\noindent We have seen that power series are analytic inside the disk of convergence $\left\{\left|z-z_0\right|<R\right\}$
\par\bigskip
\noindent The following is an important converse:
\par\bigskip
\begin{theo}[Taylors theorem]{}
  Suppose that $f(z)$ is analytic for $\left|z-z_0\right|<R$
  \par\bigskip
  \noindent Then
  \begin{equation*}
    \begin{gathered}
      f(z) = \sum_{k=0}^{\infty}\dfrac{f^{(k)(z_0)}}{k!}(z-z_0)^k\qquad\left|z-z_0\right|<R
    \end{gathered}
  \end{equation*}
\end{theo}
\newpage
\begin{prf}[]{}
  Fix $z$ and let $r$ be such that $\left|z-z_0\right|<r<R$
  \par\bigskip
  \noindent By Cauchys integral formula
  \begin{equation*}
    \begin{gathered}
      f(z) = \dfrac{1}{2\pi i}\oint_{\left|\xi-z_0\right| = r}\dfrac{f(\xi)}{\xi-z}d\xi
    \end{gathered}
  \end{equation*}
  \par\bigskip
  \noindent Now,
  \begin{equation*}
    \begin{gathered}
      \dfrac{f(\xi)}{\xi-z} = \dfrac{f(\xi)}{\xi-z_0}\dfrac{1}{1-\dfrac{z-z_0}{\xi-z_0}} = \dfrac{f(\xi)}{\xi-z_0}\sum_{k=0}^{\infty}\left(\dfrac{z-z_0}{\xi-z_0}\right)^k\\
      = \sum_{k=0}^{\infty}\dfrac{f(\xi)(z-z_0)^k}{(\xi-z_0)^{k+1}}\qquad \left|\xi-z_0\right| = r
    \end{gathered}
  \end{equation*}
  \par\bigskip
  \noindent Where the convergence is uniform in the variable $\xi$ on $\left|\xi-z_0\right| = r$\par
  \noindent We can therefore interchange integration and summation
  \begin{equation*}
    \begin{gathered}
      \Rightarrow f(z) = \sum_{k=0}^{\infty}\dfrac{1}{2\pi i}\oint_{\left|\xi-z_0\right| = r}\dfrac{f(\xi)(z-z_0)^k}{(\xi-z_0)^{k+1}}d\xi\\
      = \sum_{k=0}^{\infty}\left(\dfrac{1}{2\pi i}\oint_{\left|\xi-z_0\right| = r}\dfrac{f(\xi)}{(\xi-z_0)^{k+1}}d\xi\right)(z-z_0)^k\\
      = \sum_{k=0}^{\infty}\dfrac{f^{(k)}(z_0)}{k!}(z-z_0)^k
    \end{gathered}
  \end{equation*}
  \par\bigskip
  \noindent Where we used to generalised Cauchys integral formula in the last step
\end{prf}
\par\bigskip
\noindent\textbf{Anmärkning:}\par
\noindent The radius of convergence of the Taylor series is the largest number $R$  such that $f(z)$ is (or can be extended to be) analytic on the disk $\left\{\left|z-z_0\right|<R\right\}$
\par\bigskip
\noindent\textbf{Example:}\par
\noindent Let $f(z) = e^z\Rightarrow f^{(k)}(0) = 1\qquad\forall k\geq0$
\begin{equation*}
  \begin{gathered}
    \Rightarrow e^z = \sum_{k=0}^{\infty}\dfrac{1}{k!}z^k\qquad z\in\C
  \end{gathered}
\end{equation*}
\par\bigskip
\noindent\textbf{Example:}\par
\noindent Let $f(z) = \sin(z)$\par
\noindent Then $f^{(2k)}(0) = 0,\quad f^{(2k+1)}(0) = (-1)^k$
\begin{equation*}
  \begin{gathered}
    \Rightarrow \sin(z) = \sum_{k=0}^{\infty}\dfrac{(-1)^k}{(2k+1)!}z^{2k+1}\qquad z\in\C
  \end{gathered}
\end{equation*}
\par\bigskip
\noindent We finally mention the following:
\par\bigskip
\begin{theo}[]{}
  If the power series $f(z) = \sum_{k=0}^{\infty}a_k(z-z_0)^k$ and $g(z) = \sum_{k=0}^{\infty}b_k(z-z_0)^k$ converge for $\left|z-z_0\right|<R$, then
  \begin{equation*}
    \begin{gathered}
      f(z)g(z) = \sum_{k=0}^{\infty}c_k(z-z_0)^k
    \end{gathered}
  \end{equation*}\par
  \noindent Where
  \begin{equation*}
    \begin{gathered}
      c_k = \sum_{j=0}^{k}a_{k-j}b_j
    \end{gathered}
  \end{equation*}
\end{theo}
\par\bigskip
\begin{prf}[]{}
  $f,g$ are analytic in $\left|z-z_0\right|<R\Rightarrow u(z) = f(z)g(z)$ is analytic in $\left|z-z_0\right|<R$:
  \begin{equation*}
    \begin{gathered}
      \Rightarrow u(z) = \sum_{b=0}^{\infty}\dfrac{u^{(k)}(z_0)}{k!}(z-z_0)^k\qquad\left|z-z_0\right|<R
    \end{gathered}
  \end{equation*}
  \par\bigskip
  \noindent Now, 
  \begin{equation*}
    \begin{gathered}
      u(z_0) = f(z_0)g(z_0)\\
      u^{\prime}(z_0) = f^{\prime}(z)g(z)+f(z)g^{\prime}(z)\Big|_{z = z_0} = f^{\prime}(z_0)g(z_0)+f(z_0)g^{\prime}(z_0)\\
      u^{\prime}\prime(z_0) = f^{\prime}\prime(z_0)g(z_0)+2f^{\prime}(z_0)g^{\prime}(z_0)+f(z_0)g^{\prime}\prime(z_0)\\
      u^{(k)}(z_0) = \sum_{j=0}^{k}\begin{pmatrix}k\\j\end{pmatrix}f^{(k-j)}(z_0)g^{(j)}(z_0)\\
      \Rightarrow \dfrac{u^{(k)}(z_0)}{k!} = \sum_{j=0}^{k}\dfrac{f^{(k-j)}(z_0)}{(k-j)!}\dfrac{g^{(j)}(z_0)}{j!}\\
      \Rightarrow \sum_{j=0}^{k}a_{k-j}b_j
    \end{gathered}
  \end{equation*}
\end{prf}
