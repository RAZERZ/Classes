\section{Multi-dimensional Models}
\begin{defo}[Multi Dimensional Model]{}
  A model $\begin{cases}
    dB_t = rB_tdt\\ dS_t^i = \mu_iS_t^idt+S_ti\sum_{j=1}^{n}\sigma_{ij}dW_t^j
    \end{cases}$ where $r,\mu_i,\sigma_{ij}$ are constants and $\begin{pmatrix}\sigma_{11}&\cdots&\sigma_{in}\\\vdots&\vdots&\vdots\\\sigma_{n1}&\cdots&\sigma_{nn}\end{pmatrix}$ is a non-singular matrix is a \textit{multi-dimensional }model
\end{defo}
\par\bigskip
\noindent\textbf{Remark:}\par
\noindent In the meta-theorem, $R=M=n$, so we expect the market to be arbitrage-free and complete.
\par\bigskip
\noindent The question becomes, what is the arbitrage-free price of a simple $T$-claim $\chi = \phi(S_T)$?
\par\bigskip
\noindent The idea is that we could construct a portfolio of $S^1,S^2,\cdots,S^n,\Pi(\chi)$ which is locally risk-free (no $dW$-terms). Then, to avoid arbitrage, the drift of the portfolio must be $r$. This will yield a PDE for the price.
\par\bigskip
\noindent Instead, we will take the following route. We \textit{guess} that the price is $\Pi_t(\chi) = F(t,S_t^1,\cdots,S_t^n)$ where $F(t,S_1,\cdots,S_n)$ satisfies
\begin{equation}
  \begin{gathered}
    \begin{cases}
      F_t+\dfrac{1}{2}\sum_{i,j=1}^{n}S_iS_jC_{ij}F_{s,S_j}+s\sum S_iF_{S_i}-rF=0\\
      F(T,S_1,\cdots,S_n) = \phi(S_1,\cdots,S_n)
    \end{cases}
  \end{gathered}
\end{equation}\par
\noindent where $C=\sigma\sigma^*$\par
\noindent To show that the guess is correct, we give a replication argument.
\par\bigskip
\begin{theo}[]{}
  To avoid arbitrage, the price of $\chi = \phi(S_T)$ has to be $F(t,S_t)$ where $F(t,s)$ is given by (6) above. Moreover, $\chi$ is replicated by $h = (h^B,h^1,\cdots,h^n)$ where
  \begin{equation*}
    \begin{gathered}
      \begin{cases}
        h_t^B = \dfrac{F(t,S_t)-\sum_{i=1}^{n}S_t^iF_{S_i}(t,S_t)}{B_t}\\
        h_t^i = F_{S_i}(t,S_t)\qquad(i=1,\cdots,n)
      \end{cases}
    \end{gathered}
  \end{equation*}
\end{theo}
\par\bigskip
\begin{prf}[]{}
  \begin{equation*}
    \begin{gathered}
      V_t^h = h_t^BB_t+\sum_{i=1}^{n}h_t^iS_t^i =F(t,S_t)
    \end{gathered}
  \end{equation*}\par
  \noindent So $V_T^h = F(T,S_T) = \phi(S_T) = \chi$ which is the correct terminal value.\par
  \noindent We have
  \begin{equation*}
    \begin{gathered}
      dV_t^h\stackrel{\text{Ito}}{=} F_tdt+\sum_{i=1}^{n}F_{S_i}dS_t^i+\dfrac{1}{2}\sum_{i,j=1}^{n}F_{S_i,S_j}(dS_t^i)(dS_t^j)\\
      = \left(F_t+\dfrac{1}{2}\sum_{i,j=1}^{n}S_t^iS_t^jC_{ij}F_{S_i,S_j}\right)dt+\sum_{i=1}^{n}F_{S_i}dS_t^i\\
      \stackrel{\text{(6)}}{=} \left(rF-r\sum_{i=1}^{n}S_t^jF_{S_i}\right)dt+\sum_{i=1}^{n}F_{S_i}dS_t^i\\
      =h_t^BdB_t+\sum_{i=1}^{n}h_t^idS_t^i
    \end{gathered}
  \end{equation*}
  \par\bigskip
  \noindent Thus $h$ is self-financing and it replicates $\chi$.\par
  \noindent  Any price different from $V_t^h=F(t,S_t)$ would lead to an arbitrage
\end{prf}
\par\bigskip
\begin{theo}[Risk Neutral Valuation]{}
  The prcing function has the representation 
  \begin{equation*}
    \begin{gathered}
      F(t,s) = \E_{t,s}^Q\left[\text{exp}\left\{-r(T-t)\right\}\phi(S_T)\right]
    \end{gathered}
  \end{equation*}\par
  \noindent Where the $Q$-dynamics of $S$ are $\begin{cases}
    dS_u^i=rS_u^idu+S_u^i\sum_{j=1}^{n}\sigma_{ij}dW_u^j\\S_t^i=S_i
  \end{cases}$
\end{theo}
\par\bigskip
\subsection{Reducing the state space}\hfill\\
\noindent Let $n=2$, and assume that $\phi(kS_1,kS_2) = k\phi(S_1,S_2)$ for $k>0$.\par
\noindent Then $\phi(S_1,S_2) = S_2\phi\left(\dfrac{S_1}{S_2},1\right)$
\par\bigskip
\noindent\textbf{Ansatz}:
\begin{equation*}
  \begin{gathered}
    F(t,S_1,S_2) = S_2G\left(t,\dfrac{S_1}{S_2}\right)
  \end{gathered}
\end{equation*}\par
\noindent For some function $G(t,z)$
\par\bigskip
\noindent The terminal condition $F(T,S_1,S_2) = \phi(S_1,S_2)$ translates into $G(T,z) = \phi(z,1)$\par
\noindent We now translate all derivatives in the BS-equation:
\begin{equation*}
  \begin{gathered}
    F_t+\dfrac{1}{2}S_1^2C_{11}F_{S_1S_1}+\dfrac{1}{2}S_2^2C_{22}F_{S_2S_2}+S_1S_2C_{12}F_{S_1S_2}+rS_1F_{S_1}+rS_2F_{S_2}-rF=0
  \end{gathered}
\end{equation*}\par
\noindent Into derivatives of $G:$
\begin{equation*}
  \begin{gathered}
    F_t = S_2G_t\qquad F_{S_1S_1} = \dfrac{1}{S_2}G_{zz}\\
    F_{S_1} = G_z\qquad F_{S_1S_2} = \dfrac{-S_1}{S_2^2}G_{zz}\\
    F_{S_2} = G-\dfrac{S_1}{S_2}G_z\qquad F_{S_2S_2} =\dfrac{S_1^2}{S_2^3}G_{zz}
  \end{gathered}
\end{equation*}\par
\noindent We get:
\begin{equation*}
  \begin{gathered}
    S_2G_t+\dfrac{1}{2}\dfrac{S_1^2}{2}C_{11}G_{zz}+\dfrac{1}{2}\dfrac{S_1^2}{S_2}C_{22}G_{zz}-\dfrac{S_1^2}{S_2}C_{12}G_{zz}+rS_1G_z+rS_2G-rS_1G_z-rS_2G=0
  \end{gathered}
\end{equation*}\par
\noindent which simplifies to
\begin{equation*}
  \begin{gathered}
    G_t+\dfrac{1}{2}\dfrac{S_1^2}{S_2^2}\left(C_{11}+C_{22}-2C_{12}\right)G_{zz} = 0
  \end{gathered}
\end{equation*}
\par\bigskip
\noindent Since the argument of $G$ and its derivatives is $\left(t,\dfrac{S_1}{S_2}\right)$, we have the following:
\par\bigskip
\begin{lem}[]{}
  Assume $\phi(kS_1,kS_2) = k\phi(S_1,S_2)$, then $F(t,S_1,S_2) = S_2G\left(t,\dfrac{S_1}{S_2}\right)$ where $G(t,z)$ solves
  \begin{equation*}
    \begin{gathered}
      \begin{cases}
        G_t+\dfrac{1}{2}\left(C_{11}+C_{22}-2C_{12}\right)z^2G_{zz}=0\\
        G(T,z) = \phi(z,1)
      \end{cases}
    \end{gathered}
  \end{equation*}
\end{lem}
\par\bigskip
\noindent\textbf{Example:}\par
\begin{equation*}
  \begin{gathered}
    \begin{cases}
      dS_t^1 = \mu_1S_t^1dt+\sigma_1S_t^1dW_t^1\\
      dS_t^2=\mu_2S_t^2dt+\sigma_2S_t^2dW_t^2\\
      dB_t = rB_tdt
    \end{cases}
  \end{gathered}
\end{equation*}\par
\noindent Let $\chi = \left(S_T^1-S_T^2\right)^+$. This is an \textit{exchange option}. It gives the right to exchange one share of $S^2$ for one share of $S^1$\par
\noindent We have $\phi(S_1,S_2) = (S_1-S_2)^+$ so $\phi(kS_1,kS_2) = k\phi(S_1,S_2)$\par
\noindent By our recipe, $F(t,S_1,S_2) = S_2G\left(t,\dfrac{S_1}{S_2}\right)$ where $G(t,z)$ solves
\begin{equation*}
  \begin{gathered}
    \begin{cases}
      G_t+\dfrac{1}{2}\left(\sigma_1^2+\sigma_2^2\right)z^2G_{zz}=0\\
      G(T,z) = (z-1)^+
    \end{cases}
  \end{gathered}
\end{equation*}\par
\noindent Using the BS-formula,  $G(t,z) = zN(d_1)-N(d_2)$ so
\begin{equation*}
  \begin{gathered}
    F(t,S_1,S_2) = S_2G\left(t,\dfrac{S_1}{S_2}\right) = S_1N(d_1)-S_2N(d_2)
  \end{gathered}
\end{equation*}\par
\noindent Where
\begin{equation*}
  \begin{gathered}
    \begin{cases}
      d_1 = \dfrac{\ln{\left(\dfrac{S_1}{S_2}\right)}+\dfrac{1}{2}(\sigma_1^2+\sigma_2^2)(T-t)}{\sqrt{\sigma_1^2+\sigma_2^2}\sqrt{T-t}}\\
      d_2 = d_1-\sqrt{(\sigma_1^2+\sigma_2^2)(T-t)}
    \end{cases}
  \end{gathered}
\end{equation*}
\par\bigskip
\noindent\textbf{Example:}\par
\noindent In the market $\begin{cases}
  dB_t = rB_tdt\\
  dS_t^1=\mu S_t^1dt+\sigma_1S_t^1dW_t^1\\
  dS_t^2=\mu_2S_t^2dt+\sigma_2S_t^2\left(\rho dW_t^1+\sqrt{1-\rho^2}dW_t^2\right)
\end{cases}$\par
\noindent Find the price at $t=0$ of the $T$-claim $\chi = \dfrac{(S_T^1)^2}{S_T^2}$
\par\bigskip
\noindent To answer this, notet that $\phi(S_1,S_2) = \dfrac{S_1^2}{S_2}$, to $\phi(kS_1,kS_2) = k\phi(S_1,S_2)$\par
\noindent Thus, $F(t,S_1,S_2) = S_2G\left(t,\dfrac{S_1}{S_2}\right)$ where
\begin{equation*}
  \begin{gathered}
    \begin{cases}
      G_t+\dfrac{1}{2}z^2\left(\sigma_1^2+\sigma_2^2-2\rho\sigma_1\sigma_2\right)G_{zz} = 0\\
      G(T,z) = z^2
    \end{cases}
  \end{gathered}
\end{equation*}par
\noindent Let $\sigma = \sqrt{\sigma_1^2+\sigma_2^2-2\rho\sigma_1\sigma_2}$, we have
\begin{equation*}
  \begin{gathered}
    G(0,z) = \E_{0,z}\left[Z_T^2\right]\qquad dZ_t = \sigma ZdW_t
  \end{gathered}
\end{equation*}\par
\noindent Let $Y_t = Z_t^2$, then 
\begin{equation*}
  \begin{gathered}
    dY_t = 2Z_tdZ_t+(dZ_t)^2=\sigma^2Y_tdt+2\sigma Y_tdW_t
  \end{gathered}
\end{equation*}\par
\noindent so $G(0,z) = \E\left[Z_T^2\right] = z^2\text{exp}\left\{\sigma^2T\right\}$
\par\bigskip
\noindent\textit{Answer:} $F(0,S_1,S_2) = S_2G\left(0,\dfrac{S_1}{S_2}\right) = \dfrac{S_1^2}{S_2}\text{exp}\left\{\left(\sigma_1^2+\sigma_2^2-2\rho\sigma_1\sigma_2\right)T\right\}$
\par\bigskip
\noindent\textbf{Example:}\par
\begin{equation*}
  \begin{gathered}
    \begin{cases}
      dS_t^1 = \mu_1S_t^1dt+\sigma_1S_t^1dW_t^1\\
      dS_t^2=\mu_2S_t^2dt+\sigma_2S_t^2dW_t^2\\
      dB_t = rB_tdt
    \end{cases}
  \end{gathered}
\end{equation*}\par
\noindent Here $dW^1dW^2  = \rho dt$. Let $\chi = \left(S_T^1-S_T^2\right)^+$.\par
\noindent By our recipe $F(t,S_1,S_2) = S_2G\left(t,\dfrac{S_1}{S_2}\right)$ where $G(t,z)$ satisfies
\begin{equation*}
  \begin{gathered}
    \begin{cases}
      G_t+\dfrac{1}{2}\left(\sigma_1^2+\sigma_2^2-2\rho\sigma_1\sigma_2\right)z^2G_{zz}=0\\
      G(T,z) = (z-1)^+
    \end{cases}
  \end{gathered}
\end{equation*}\par
\noindent Using the BS formula
\begin{equation*}
  \begin{gathered}
    G(t,z) = zN(d_1)-N(d_2)
  \end{gathered}
\end{equation*}\par
\noindent where 
\begin{equation*}
  \begin{gathered}
    \begin{cases}
      d_1 = \dfrac{\ln{\left(z\right)}+\overbrace{\dfrac{\sigma^2}{2}}^{\left(\sigma_1^2+\sigma_2^2-2\rho\sigma_1\sigma_2\right)}(T-t)}{\sigma\sqrt{T-t}}\\
      d_2 = \dfrac{\ln{\left(z\right)}-\dfrac{\sigma^2}{2}(T-t)}{\sigma\sqrt{T-t}}
    \end{cases}
  \end{gathered}
\end{equation*}\par
\noindent Thus, the pricing function $F$ is 
\begin{equation*}
  \begin{gathered}
    F(t,S_1,S_2) = S_2G\left(t,\dfrac{S_1}{S_2}\right) = S_2\left(\dfrac{S_1}{S_1}N(d_1)-N(d_2)\right)\\
    = S_1N(d_1)-S_2N(d_2)
  \end{gathered}
\end{equation*}\par
\noindent Where $d_1,d_2$ are now equal to
\begin{equation*}
  \begin{gathered}
    \begin{cases}
      d_1 = \dfrac{\ln{\left(\dfrac{S_1}{S_2}\right)}+\dfrac{\sigma^2}{2}(T-t)}{\sigma\sqrt{T-t}}\\
      d_2 = \dfrac{\ln{\left(\dfrac{S_1}{S_2}\right)}-\dfrac{\sigma^2}{2}(T-t)}{\sigma\sqrt{T-t}}
    \end{cases}
  \end{gathered}
\end{equation*}
\par\bigskip
\noindent\textbf{Remark:}\par
\noindent In general, the payoff function $\phi$ could be something like $\min\left\{S_1(T),S_2(T)\right\}$, then according to the recipe we should plug in for the terminal condition $\min\left\{z,1\right\} = \phi(z,1)$.\par
\noindent This is a linear function minus a call option, so it is solvable. For the linear function the one-dimensional BS PDE is easy to solve.
