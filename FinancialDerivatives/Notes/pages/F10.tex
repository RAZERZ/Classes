\subsection{Drift estimation}\hfill\\
Assume $X_t = \mu_t+\sigma W_t$ and we want a confidence interval for $\mu$. An estimare for $\mu$ is $\widehat{\mu} = \dfrac{X_t}{t}\in N\left(\mu,\dfrac{\sigma}{\sqrt{t}}\right)$ and a confidence interval is
\begin{equation*}
  \begin{gathered}
    \left(\widehat{\mu}-\dfrac{\sigma}{\sqrt{t}}\cdot1.96, \widehat{\mu}+\dfrac{\sigma}{\sqrt{t}}\cdot1.96\right)
  \end{gathered}
\end{equation*}\par
\noindent If one wants a certain precision $\Delta \mu$ so that $\P(\mu\in(\widehat{\mu}-\Delta\mu,\widehat{\mu}+\Delta\mu)) = 0.95$, one needs
\begin{equation*}
  \begin{gathered}
    \dfrac{2\sigma}{\sqrt{T}} =\Delta\mu\quad\Lrarr\quad t = \dfrac{4\sigma^2}{(\Delta\mu)^2}
  \end{gathered}
\end{equation*}\par
\noindent Plug in reasonable values $\begin{rcases}
  \sigma = 0.3\\\Delta\mu = 0.06
\end{rcases}\Rightarrow t=100$ years!
\par\bigskip
\noindent\textbf{Remark:}\par
\noindent When pricing options, the drift of the stock needs not be estimated (since under the pricing measure $Q$, the drift is $r$)
\par\bigskip
\section{Volatility}
In the BS-formula, $s,r,t$ are observable, $T,K$ are specified in the contract and $\sigma$ is not directly observable. All are needed.\par
\noindent There are 2 approaches, one using \textit{historic volatility} and one using \textit{implied volatility}.
\par\bigskip
\subsection{Historic volatility}\hfill\\
If $dS_t = \mu S_tdt + \sigma S_tdW_t$, then sample $S$ at $n+1$ time points and let 
\begin{equation*}
  \begin{gathered}
    \xi_i = \ln{\left(\dfrac{S_{ti}}{S_{t_{i-1}}}\right)} = \left(\mu-\dfrac{\sigma^2}{2}\right)\Delta t+\sigma(W_{t_i}-W_{t_{i-1}})\sim N\left((\mu-\dfrac{\sigma^2}{2})\Delta t, \sigma\sqrt{\Delta t}\right)
  \end{gathered}
\end{equation*}\par
\noindent An esimate of $\sigma^2$ is then $S^2 = \dfrac{\sum_{i=1}^{n}(\xi_i-\overline{\xi})^2}{(n-1)\Delta t}$ where $\overline{\xi} = \dfrac{1}{n}\sum_{i=1}^{n}\xi_i$
\par\bigskip
\subsection{Implied volatility}\hfill\\
\noindent Let $p$ be the price in the market of a certain call option (maturity $T$, with strike price $K$). Find $\sigma$ such that $p = \text{BS}(s,t,T,r,\sigma,K)$ where BS denotes the Black-Scholes formula\par
\noindent This $\sigma$ is called \textit{implied volatility}
\par\bigskip
\noindent\textbf{Remark:}\par
\noindent Recall that the BS-formula is increasing in $\sigma$
\par\bigskip
\noindent If gBm is the correct model (i.e option prices are calculated using the BS-formula), then the \textit{same} implied volatility would be obtained for different $K$ and $T$
\par\bigskip
\section{Completeness and Hedging}
\begin{defo}[]{}
  A $T$-claim $X$ can be \textit{replicated} if there exists a self-financing portfolio $h$ with $\P(V_T^h = X) = 1$.\par
  \noindent If every $T$-claim can be replicated then the market is \textit{complete}
\end{defo}
\par\bigskip
\begin{theo}[]{}
  Assume that a $T$-claim $X$ can be replicated using $h$. Then the only possible arbitrage-free price of $X$ is $\Pi_t(X) = V_t^h$
\end{theo}
\par\bigskip
\begin{prf}[]{}
  If for example $\Pi_t(X)<V_t^h$ for some $t$; sell the portfolio and buy the claim $\Rightarrow$ arbitrage
\end{prf}
\par\bigskip
\noindent We now specialize to the model
\begin{equation}
  \begin{gathered}
    \begin{cases}
      dB_t = rB_tdt\\
      dS_t = \mu(t,S_t)S_tdt + \sigma(t,S_t)S_tdW_t
    \end{cases}
  \end{gathered}
\end{equation}\par
\noindent with $\sigma(t,s)>0$
\par\bigskip
\begin{theo}[]{}
  The model (5) is complete
\end{theo}
\par\bigskip
\noindent We will prove a simpler result, namely that all \textit{simple} $T$-claims can be replicated.\par
\noindent Recall that the value $\Pi_t(X)$ of a simple $T$-claim $X = \phi(S_T)$ is $F(t,S_t)$ where $F(t,s)$ is the pricing function. Thus
\begin{equation*}
  \begin{gathered}
    d\Pi_t = F_tdt+F_sdS_t + \dfrac{1}{2}F_{ss}(dS_t)^2\\
    = \left(F_t + \dfrac{\sigma^2}{2}S_t^2F_{ss}\right)dt + F_sdS_t
  \end{gathered}
\end{equation*}\par
\noindent Moreover, a portfolio $h = (h^B, h^S)$ is self-financing if $dV_t^h = h_t^BdB_t+h_t^sdS_t$. Choose $h_t^S = F_s(t,S_t)$
\par\bigskip
\begin{theo}[]{}
  Let $X = \phi(S_T)$ and define $F(t,s)$ by 
  \begin{equation*}
    \begin{gathered}
      \begin{cases}
        F_t + \dfrac{\sigma^2S^2}{2}F_{ss}+rsF_s-rF = 0\\
        F(T,s)\phi(s)
      \end{cases}
    \end{gathered}
  \end{equation*}
  \par\bigskip
  \noindent Define $h = (h^B,h^S)$ by 
  \begin{equation*}
    \begin{gathered}
      \begin{cases}
        h_t^B = \dfrac{F(t,S_t)-S_tF_s(t,S_t)}{B_t}\\
        h_t^S = F_s(t,S_t)
      \end{cases}
    \end{gathered}
  \end{equation*}
  \par\bigskip
  \noindent Then $h$ replicates $X$ and $\Pi_t(X) = V_t^h = F(t,S_t)$
\end{theo}
\par\bigskip
\begin{prf}[]{}
  $V_t^h = h_t^BB_t + h_t^SS_t = F(t,S_t)$, so 
  \begin{equation*}
    \begin{gathered}
      dV_t^h = F_tdt + F_sdS_t + \dfrac{1}{2}F_{ss}(dS_t)^2\\
      = \left(F_t + \dfrac{\sigma^2}{2}S_t^2F_{ss}\right)dt + F_sdS_t\\
      \stackrel{\text{BS PDE}}{=} r(F-S_tF_s)dt + F_sdS_t = h_t^BdB_t+h_t^SdS_t
    \end{gathered}
  \end{equation*}
  \par\bigskip
  \noindent Thus $h$ is self-financing. Since $V_T^h = F(T,S_t)= \phi(S_T) = X$, $h$ replicates $X$.\par
  \noindent By no-arbitrage $\Pi_t(X) = V_t^h = F(t,S_t)$
\end{prf}
\par\bigskip
\noindent\textbf{Example:}\par
\noindent If $X = S_T$, then $F(t,s) = s$, so $h_t^S = F_s = 1$
\par\bigskip
\noindent\textbf{Example:}\par
\noindent For a call option (in the standard BS-model), $F(0,s) = sN(d_1) - K\text{exp}\left\{-rT\right\}N(d_2)$, thus
\begin{equation*}
  \begin{gathered}
    F_S(0,s) = N(d_1)+\dfrac{1}{\sqrt{2\pi}}\left(s\text{exp}\left\{-\dfrac{d_1^2}{2}\right\}-K\text{exp}\left\{-rT\right\}\text{exp}\left\{-\dfrac{d_2^2}{2}\right\}\right)\dfrac{\partial d_1}{\partial s}
  \end{gathered}
\end{equation*}
\par\bigskip
\noindent Moreoever, 
\begin{equation*}
  \begin{gathered}
    s\text{exp}\left\{-\dfrac{d_1^2}{2}\right\}-K\text{exp}\left\{-rT\right\}\text{exp}\left\{-\dfrac{d_2^2}{2}\right\} = \text{exp}\left\{-\dfrac{d^2}{2}\right\}\left(s-K\text{exp}\left\{-rT\right\}\text{exp}\left\{-\dfrac{\sigma^2T}{2}\right\}\text{exp}\left\{\sigma\sqrt{T}d_1\right\}\right) = 0
  \end{gathered}
\end{equation*}\par
\noindent so $F_s(0,s) = N(d_1)$
\par\bigskip
\noindent\textbf{Remark:}\par
\noindent The derivative $\Delta = F_s$ is called the \textit{delta}.\par
\noindent In a replicating portfolio one should hold $\Delta$ shares of $S$ at each time.
\par\bigskip
\noindent If the pricing function is convex in $S$, then in order to replicate it then $\Delta$ goes up then buy more stock. Conversely, sell off if the opposite.
\par\bigskip
\noindent\textbf{Example:}\par
\noindent For a call option in the standard BS-model
\begin{equation*}
  \begin{gathered}
    F(0,s) = sN(d_1)-K\text{exp}\left\{-rT\right\}N(d_2)
  \end{gathered}
\end{equation*}\par
\noindent Where $\begin{cases}
  d_1 = \dfrac{\ln{\left(\dfrac{s}{K}\right)}+(r+\dfrac{\sigma^2}{2})T}{\sigma\sqrt{T}}\\d_2 = \dfrac{\ln{\left(\dfrac{s}{K}\right)}+(r-\dfrac{\sigma^2}{2})T}{\sigma\sqrt{T}}
\end{cases}$\par
\noindent Thus
\begin{equation*}
  \begin{gathered}
    \Delta = F_s(0,s) = N(d_1)+s\varphi(d_1)\dfrac{1}{s\sigma\sqrt{T}}-K\text{exp}\left\{-rT\right\}\varphi(d_2)\dfrac{1}{s\sigma\sqrt{T}}\\
    = N(d_1) +\dfrac{1}{\sigma\sqrt{T}}\left(\varphi(d_1)-\dfrac{K}{s}\text{exp}\left\{-rT\right\}\varphi(d_2)\right)
  \end{gathered}
\end{equation*}\par
\noindent Where
\begin{equation*}
  \begin{gathered}
    N(x) = \int_{-\infty}^{x}\varphi(z)dz\\
    \varphi(z) = \dfrac{1}{\sqrt{2\pi}}\text{exp}\left\{-\dfrac{z^2}{2}\right\}
  \end{gathered}
\end{equation*}
\par\bigskip
\noindent The claim is that we are left with 0 on the second term, we check:
\begin{equation*}
  \begin{gathered}
    \sqrt{2\pi}\dfrac{\varphi(d_1)-\dfrac{K}{s}\text{exp}\left\{-rT\right\}\varphi(d_2)}{} \stackrel{d_2 = d_1-\sigma\sqrt{T}}{=} \text{exp}\left\{-\dfrac{d_1^2}{2}\right\}-\dfrac{K}{s}\text{exp}\left\{-rT\right\}\text{exp}\left\{-\dfrac{\left(d_1-\sigma\sqrt{T}\right)^2}{2}\right\}\\
    = \text{exp}\left\{-\dfrac{d_1^2}{2}\right\}\left(1-\dfrac{K}{s}\text{exp}\left\{-rT\right\}\text{exp}\left\{-\dfrac{\sigma^2T}{2}\right\}\text{exp}\left\{d_1\sigma\sqrt{T}\right\}\right)\\
    = \text{exp}\left\{-\dfrac{d_1^2}{2}\right\}\left(1-\dfrac{K}{s}\underbrace{\text{exp}\left\{-rT\right\}\text{exp}\left\{-\dfrac{\sigma^2T}{2}\right\}\text{exp}\left\{\ln{\left(\dfrac{s}{K}\right)}+(r+\sigma^2/2)T\right\}}_{\substack{\dfrac{s}{K}}}\right) = 0\\
    \Rightarrow N(d_1) +\dfrac{1}{\sigma\sqrt{T}}\left(\varphi(d_1)-\dfrac{K}{s}\text{exp}\left\{-rT\right\}\varphi(d_2)\right) = N(d_1)
  \end{gathered}
\end{equation*}
\par\bigskip
\noindent The $\Delta$ is simply the first derivative of the pricing function.
