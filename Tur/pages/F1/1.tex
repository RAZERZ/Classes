\section{Pedagogisk utbildning}

\subsection{Introövning}\hfill\\

\noindent Det är mer fördelaktigt om man tänker först, att brainstorma

\begin{itemize}
  \item Vad hjälpte dig att återupta det matematiska tänkandet
    \begin{itemize}
      \item Nice med fler härledningar, man kanske har glömt \textit{varför} något funkar. Ibland är en strikt logisk härledning mindre givande än kanske "viftande med handen".
      \item Göra provet för att inse att man är kanske inte bäst
      \item Tog hjälp av polare
      \item Våga fråga läraren, att upprepa att "det är okej att ställa frågor" gör att det blir skönt att ställa frågor. Samtidigt som man lyssnar på elever.
    \end{itemize}
  \item Vad hade kunnat göras bättre?
    \begin{itemize}
      \item Mer entusiasm, utgå från exemepel och ta fram verktygen för att kunna lösa dem.
      \item Menti och dylikt, pedagogiska knep
    \end{itemize}
  \item Hur vill du göra för att hjälpa din grupp?
    \begin{itemize}
      \item "Vad är det vi försöker göra?"
      \item "Känns det okej?"
      \item Pedagogiska hjälpmedel så att man kan få snabb respons (typ rött gult gröna kort där man kan läsa rummet).
    \end{itemize}
\end{itemize}
\par\bigskip
\noindent Man ska inte använda röst för att få tyst på studenter, annars pajar den.

\noindent Övningen som gjordes kallas för "THINK-PAIR-SHARE", tänk, sedan prata ihop er med grannen, och ta diskussion inför helklass.

\subsection{Hur integreras kunskap?}\hfill\\

\noindent Vi bygger kunskap med hjälp av saker vi redan vet. Detta kan användas genom att man tar problem som man känner i vardagen och sedan bygger vidare på det.

\begin{itemize}
  \item Assimilation: Ny info anpassas till existerande kunskapsstruktur utan att förändra den
  \item Ackomodation: Ny information påverkar den gamla strukturen. Den nya informationen leder till en obalans, som kräver omstrukturering (detta kräver mer tankekraft/energi).
\end{itemize}
\par\bigskip

\noindent Människor är ganska lata, vi föredrar assimilation. Därför är det viktgit att tänka "hur tänker personen jag försöker hjälpa? Finns det rum för missförstånd?". Matematik är logiskt, men i huvudet på folk är det inte alltid så. 

\subsection{Praktiska råd}\hfill\\

\begin{itemize}
  \item Jobba med att utnyttja vad de redan kan (diagnostik test)
  \item Börja med något du är säker på att de kan, typ $y = kx+m$ och sedan bygg vidare på det. 
  \item Uppmärksamma hur studenterna tänker tillsammans med dem
  \item Utgå från studenternas tänkade i dina förklaringar (man vet inte alltid varför det har blivit fel. så "går ner på deras nivå"). Bygger felet på slarv eller bygger det på bristande kunskap/konceptuellt missförstår? Fråga "hur långt har du kommit"/"vad tror du är problemet" så man kan starta på samma plats som dem
\end{itemize}
\par\bigskip
\noindent Exempel: $mk=3x+4k\Rightarrow m=3x+4$. Här har man glömt att ta bort $4k$ från \textit{båda leden}. Då kan man fråga sig om det beror på slarv eller om det beror på att man inte vet att det man gör på ena sidan måste man göra på den andra. Det kanske till och med blir visuellt problematiskt iochmed att man tycker det är svårt med bråk, så man "undermedvetet" förkastar den möjligheten och gör fel istället.
\par\bigskip
\noindent Man ska inte bara peka ut var det har blivit fel, man lär sig inte av att någon säger "här sket det sig", man måste ju lära ut hur man ska "debugga"! Samtidigt skall man inte fråga "hur tänkte du här?", ty det blir lite anklagande. Målet med dialogen är att man skall lösa problemet.

\subsection{Lärstrategier}\hfill\\

\noindent Djupinriktning: man har fokus på helheten. Man försöker söka mer principer, varför och hur de hänger samman och inte bara memorera. Försöker koppla ny kynskap till vardag och till tidigare kunskap, jobbar man så kan man producera man kan tänka nytt.
\par\bigskip
\noindent Ytinriktad inlärning: Man lär sig bara metoder och att räkna. Det kan bli svårt att ta med sig det ut till arbetslivet eller koppla samman saker.
\par\bigskip
\noindent När man tittar på hur man löser ett problem kan man göra det ytligt eller djupt, man ska våga ta sig an problem, att om man inte vet hur man ska göra så vågar man inte ta sig fram. Detta är även kopplat till motivationen \textit{att} lösa problemet. Tips på hur man får igång detta är att försöka lösa problemet med en vän och förklara \textit{varför} i varje steg.
\par\bigskip
\noindent Visualisera/ta upp exempel som binder samman olika områden, det brukar ge många "ahhh" stunder. Ju fler sätt man har att se på saker desto bättre.
\par\bigskip
\noindent Uppmuntra studenterna att fråga sig \textit{vad} uppgiften handlar om och vilket svar man kan förvänta sig (kanske positivt, eller något med $\pi$).\par
\noindent Uppmuntra studenterna att våga och ge dem riktning och att reflektera över sina lösningar. Vägled \textit{hur} man skall reflektera, exempelvis "är det här svaret rimligt?"
\par\bigskip
\noindent Lärare som försöker fundera över begreppsmässig förändring som fokuserar på studenternas lärande istället för sig själv. Hjälp studenterna utveckla och förändra sina begrepp (fånga upp var de är). Jobba med att engagera och utmana (hur kan vi "ha sönder" denna formel?). Detta istället för överföring av information.
\par\bigskip
\noindent Hur kan vi hjälpa studenterna för att få dem att förstå?

\begin{itemize}
  \item Hur kan vi försöka ha sönder denna sak/formel som vi arbetar med?
  \item Har man stenkoll på det djupa själv blir det lättare att se hurandra tänker fel
  \item Det är roligare att motbevisa än att visa.
\end{itemize}
\par\bigskip

\subsection{Vad underlättar djupinriktade lärstrategier?}\hfill\\

\begin{itemize}
  \item Motivation
  \item Utgångspunkt i förkunskaper och erfarenheter
  \item Fokus på begreppsförståelse
  \item Återkoppling (jämför varandras lösningar (oftast kan mam lösa samma problem på olika sätt vilket är viktigt att se))
  \item Varierad undervisning
  \item Egen aktivitet är viktigt, inte bara stå och prata men gotta ner sig med händerna
  \item Studenter klarar mer tillsammans än 1 och 1
\end{itemize}

\subsection{Varierad undervisning - vad är det?}\hfill\\

\noindent Handlar om att man ska kunna jobba med kanske ett visuellt uttryck (bilder och figurer), prata, beskriva, låta dem prata med varandra. Detta spelar roll för hur olika studenter tar till sig något. Om man har varierad undervisning, har man även varierad målgrupp, alltså når man ut till fler personer.
\par\bigskip
\noindent Har med att göra med \textit{representationer}, vad är det fösta du tänker på när du hör...
\par\bigskip
\noindent Vill vi räkna ut $\sin(1)$ kan man använda Taylor, men här är det inte helt klart att den har en geometrisk tolkning!
\par\bigskip
\noindent Anonymitet i frågor, dela ut lappar, se vad folk tycker är svårt! Studenter svarar sällan på en så öppen fråga som "är det något ni undrar?"
\par\bigskip
\noindent När man löser och använder sig av begrepp använder man oftast \textit{bilden} av begreppet istället för concept definition. Alltså måste man forma bilden så att den matchar definitionen, då har vi matchat definitionen. 




















