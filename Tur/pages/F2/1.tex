\section{Upplägg av lektion}

\noindent Anpassa koncept efter studenter. Klassikern (där det är föreläsning först och räkna sen) funkar om föreläsaren är bra. Är det en dålig föreläsare gynnar det bara de starka studenterna. Interaktiv kräver att man måste vara bra på det som föreläsare. Det kan gynna de som tycker att det är lite svårare. Man kanske får mindre gjort om man sitter i grupp. Att man sitter själv förutsätter att man kan det som föreläsaren går igenom. Viktigt att coacha \textit{hur} man ska arbeta i grupp \textit{om} man ska arbeta i grupp. Man skulle kunna dela ut roller i en grupp, då har man redan etablerat strukturen och det som är kvar för gruppen är att jobba.  
\par\bigskip
\noindent Informera om upplägg!
