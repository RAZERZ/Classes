\section{Pedagogisk utbildning}
\par\bigskip
\noindent Idag skall vi foksusera mer på hur man planerar ett pass och vi kommer få testa lite olika saker.
\par\bigskip
\subsection{Vad ska man tänka på när man planerar sin undervisnging}\hfill\\
\par\bigskip
\noindent Det första man ska tänka på är "vilka är det som är där?". Det finns säkert föreläsare som har missat denna punkt och har missat att ni läst en kurs. Det är viktigt för oss att fundera över vilka vi har framför, finns det något som vi vet om de som gör att vi kan fånga deras intresse? Vad har de lärt sig på gymnasiet? Andra erfarenheter? Förkunskaper.\par
\noindent Hur kan vi ta reda på mer? Vi diskuterar i grupp:
\par\bigskip
\begin{itemize}
  \item Hur kan vi ta reda på mer?
    \begin{itemize}
      \item Alla har läst minst matte 3
      \item Man kan fråga de "fyll i denna enket
      \item "Var kommer du ifrån"-leken 
      \item De har valt så förhoppningsvis har de redan intresse för teknologi/naturvetenskap
      \item En fördel med homogen klass (program)
      \item Förkunskapstestet ger en bild av gruppen som helhet
      \item Man kan ha en kahoot om hur de känner för begrepp och hur många år sedan de pluggade
      \item Viktigt att kolla om de går för fort. Räkna med att det kommer kännas stelt första gångerna.
      \item Visa att ni bryr er om, skapar trygghet och visar att man vill forma undervisningen för att det ska gå så bra som möjligt för dem
    \end{itemize}
\end{itemize}
\par\bigskip
\noindent Hur kan man skapa en sådan miljö där det är tryggt?\par
\noindent Kasta godis, gör misstag (extra poäng om man gör misstag så att de rättar).
\par\bigskip
\noindent Namnlekar kan funka om man inte har någon som inte har svenska som modersmål, eller har hörsvårigheter. Alternativt kan man ha pappersnamn.
\par\bigskip
\subsection{\textit{Vad} ska de lära sig?}\hfill\\
\par\bigskip
\noindent Fundera på syftet med proppen. Hur ska man balansera bekräftelse och utmaningar (att det blir lagom svårt men att de känner att de kan ta sig an det)\par
\noindent Vad ska ske vid de olika passen, vad händer om man måste lägga mer tid på något?\par
\noindent Vad är det mest centrala? Om de inte fixar att repetera allting, osv. Välj ut saker som de \textit{absolut} måste ha med sig och satsa på det.
\par\bigskip
\noindent Vad vill vi att de ska ha med sig från allra första passet?
\par\bigskip
\begin{itemize}
  \item De ska veta vad de kan och vad de borde kunna (metakognition)
  \item Försök locka de så att de kommer tillbaka, fokusera på stämningen (det är ju trots allt bråk vilket är lätt)
  \item En klar bild av vad syftet med proppen är, att det inte är obligatoriskt men att den finns där för de att förfoga
\end{itemize}
\par\bigskip
\noindent Det vi har fokuserat på nu är att första vad man behöver lära sig osv, men också matte grejset! Då kan man fokusera på enskildheter, men det finns vad och varför som är mer övergripande.
\par\bigskip
\noindent Om för ögonblicket struntar i studenternas känslor så finns det mål över vad de skall lära sig under proppen. Det enda syftet är inte bara matematiken. Allt de lär sig ska de redan ha sett, inget skall vara nytt. Det är därför viktigt att fokusera på just det. Det finns en massa extra-material vilket är jättebra (kluringar). Fokuset skall helt enkelt ligga på repetition.
\par\bigskip
\noindent Nästan alla som kommer från gymnasiet har några brister. Man kan inte allt, och även om man kan allt, och även om man har förstått allt, så har man inte automatiserat det. Det kan vara fördelaktigt att göra saker även fast det är lite enklare. Det är bra om man kan få de att känna att ju bättre de kan sakerna från gymansiet desto bättre kommer det gå för dem under universitetet.
\par\bigskip
\noindent Sammanhanget är viktigt, det är universitetsmatematik. Det är stor skillnad mellan gymnasiematte och universitetsmatematik. Om man vill ta in universitetsmatematik måste man kunna det från innan bättre. Tänk på hus, en bra grund gör att huset kan byggas högst. Om man måste tänka flera steg för att göra saker så blir det fler saker för arbetsminnet och då kan man inte ta in nya saker lika snabbt och effektivt. Få dem att första var de befinner sig just nu och hur det skiljer sig, lite kontext. 
\par\bigskip
\noindent Ett sätt att se till att de automatiserat är att se till att de har övat på det. Ett bra sät att öva på det är att koppla nya saker till varandra och till saker de redan vet.\par
\noindent Poängen med det är att det visar sig att det är så hjärnan fungerar. Om saker inte är kopplade till varandra uppenbarar de sig meningslösa. Kan man göra saker "tangible" så blir det tusen gånger lättare. Upprepa saker tills de kan det och gärna efter de kan det. 
\par\bigskip
\subsection{Hur starta så att de känner sig motiverade?}\hfill\\
\par\bigskip
\noindent Det är en sak vid första tillfället, och en annan vid andra. Målet är att få de att förstå att vi vill de väl och frid och fröjd. Men, det är inte bara det, de behöver känna sig motiverade att lära sig just det här, som de ska kunna för just det här blocket.\par
\noindent Övning:
\par\bigskip
\begin{itemize}
  \item Diskutera ett utvalt block i uppgiftshäftet
  \item Varför är det viktigt?
  \item Formulera (3 meningar) på ett sätt så att det blir/verkar intressant/viktigt 
\end{itemize}
\par\bigskip
\noindent Faktosiering är viktigt för att gränsvärden. Kvadratkomplettering är viktigt för att faktorisera kvadratekvationer. För att få förståelsen för pq \par
\noindent En av de viktigaste saker att veta är nollställen. Lättaste är via faktorisering eller kvadratkompelttering.
\par\bigskip
\noindent Det är inte alla som förstår varje moment är viktigt. Man behöver inte koppla allt till "det  är viktigt för att det kommer göra saker lättare", ibland är det bra att vara hård och säga "det är viktigt för att det finns överallt". Vet ni däremot att "jag vet att ni är kemister, i programmet kommer man göra allt till räta linjer" så lägger ni vikt vid räta linjer, vilket är bra. 
\par\bigskip
\noindent Oftast stannar studenten och säger "och så kom jag hit", tänk på våga derivera! Demåste våga sig att ta sig fram. Det är en annan sak däremot om de inte har någon som helst aning om hur de ska ta sig fram. Då handlar det om att slumpa sig fram. Vi vill gärna få det att bli "härma->förstå varför->lära sig". Det skall finnas ett värde av att ta med sig saker hem. Det skall finnas ett värde av att klara saker. 
\par\bigskip
\noindent Om vi nu antar att vi har kickat igång blocket med pepp-introt som vi gjorde. Då är nästa steg hur man fortsätter bollen rulla. Här finns det olika saker, men variera gärna aktiva moment och så att alla är aktivt. Fundera på hur man kan göra för att arbeta konstruktivt tillsammans. De får väldigt lite om de måste räcka upp handen hela tiden för att få återkoppling, vi måste se till att de hjälper varandra. Kolla på andras lösningar osv. Även här är det viktigt att etablera den där öppna miljön så att de kan prata om allt. Blanda gärna grupperna så att om man hamnar med någon som är duktig så slipper de bli "lata"
\par\bigskip
\noindent Anknyt till tidigare kunskap, hur kan jag förmedla tankarna bakom uppläggen (handlar om \textit{varför} vi vill att de ska byta runt). Sedan handlar det också om att vi kommer stå där framme och förklarar saker.
\par\bigskip
\noindent Huvudfokus bör vara att kunna stegvis konkret och förklara i detalj hur man löser ett problem. Allt det andra som vi har gått igenom slängs ner i fyrsiån om man inte tar hänsyn till detta. Vilka problem kan man stöta på? På vilket sätt kan man lägga upp undervisningen på ett sådant sätt att dessa problem som man stöter på inte blir omöjliga?
\par\bigskip
\noindent Vi löser i grupp $2x+y=1$. Vi säger att det är en rätlinje, men det är det ju inte! Vi måste införa koordinatsystem med riktningar. 
\par\bigskip
\noindent Efter att studenterna har automatiserat metoderna så kan man börja filosofera med kopplingar mellan det de precis lärt sig, och annat. 
\par\bigskip
\noindent Gruppuppgiter med tvärgruppsredovisning är en bra metod för att få igång diskussion samt att man lär sig bra saker när man ska börja förklara saker till andra.
\par\bigskip
\subsection{Hur avslutar man en lektion/block?}\hfill\\
\par\bigskip
\noindent Vad tar studenterna med sig från dagen/blocket? Hur vet de vad de behöver jobba mer med? Vad inspirerar dem att tänka vidare? Ibland säger man att man ska tala om på slutet av varje lektion "idag har vi jobbat med det här och det jag vill att ni tar med er är det här och det här", istället kanske man skall fråga "vad tar \textbf{ni} med er från idag?". 
