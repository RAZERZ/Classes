\section{Ordinära differentialekvationer}
\par\bigskip
\noindent Man brukar dela in ODE:er i 2 grupper beroende på randvillkorlet.\par
\noindent Explicit: $y^{(k)}(t)=f(t,y, y^{\prime}, y^{\prime\prime},\cdots,y^{(k-1)})$
\par\bigskip
\noindent Den första gruppen är \textit{Begynnelsevärdesproblem} (IVP).\par
\noindent Exempel på första ordningens IVP:
\begin{equation*}
  \begin{gathered}
    \begin{cases*}
      y^{\prime}(t)=f(t,y(t)) = 3y+t\qquad t>t_0\\
      y(t_0) = y_0
    \end{cases*}
  \end{gathered}
\end{equation*}
\par\bigskip
\noindent Andra gruppen är \textit{Randvärdesproblem}\par
\noindent Exempel:
\begin{equation*}
  \begin{gathered}
    \begin{cases*}
      y^{\prime\prime}=f(x,y,y^{\prime})\qquad a\leq x\leq b\\
      y(a) = \alpha\\
      y(b) = \beta
    \end{cases*}
  \end{gathered}
\end{equation*}
\par\bigskip
\noindent Randvärdesproblem tas inte upp i denna kurs, däremot kommer det i Berv3/Berv för PDE.
\par\bigskip
\noindent Det som är viktigt att notera (för att få en unik lösning) är att antalet villkor som krävs är lika med ordningen på ekvationen.
\par\bigskip
\subsection{Speciella egenskaper för att klassificera ODE:er}\hfill\\
\par\bigskip
\noindent En ODE kallas \textit{linjär }om den är linjär i $y$ och alla dess derivator. Exempelvis $y^{\prime} = 3y+t$
\par\bigskip
\noindent \textit{Variabla koefficienter}, exempelvis $y^{\prime} = \sin(t)y+t$
\par\bigskip
\noindent En icke-linjär ODE kallas för \textit{ickelinjär}, exempelvis $y^{\prime} = y^2+t$
\par\bigskip
\noindent \textit{System av ekvationen}, SIR-modellen för epidemi. Antalet villkor för system av första ordningens ODE:er = antalet ekvationer.
\par\bigskip
\noindent Om alla villkor ges i samma punkt så har vi begynnelsevärdesproblem, annars randvärdesproblem.
\par\bigskip
\noindent Högre ordningens ekvationer, exempelvis:
\begin{equation*}
  \begin{gathered}
    y^{\prime\prime}=f(t,y,y^{\prime})\qquad t\geq t_0\\
    y(t_0) = y_0\\
    y^{\prime}(t_0) = z_0
  \end{gathered}
\end{equation*}
\par\bigskip
\subsection{Omskrivning till förta ordningens ODE}\hfill\\
\par\bigskip
\noindent Givet en ODE av ordning $n$:
\begin{equation*}
  \begin{gathered}
    u^{(n)} = f(t, u, u^{\prime}, u^{\prime\prime},\cdots, u^{(n-1)})
  \end{gathered}
\end{equation*}
\par\bigskip
\noindent För att skriva om den till system av första ordningen introducerar vi en vektor $y$ med följande komponenter:
\begin{equation*}
  \begin{gathered}
    y_1 = u,\qquad y_2 = u^{\prime},\qquad y_3 = u^{\prime\prime},\qquad\cdots, \qquad y_n = u^{(n-1)}
  \end{gathered}
\end{equation*}
\par\bigskip
\noindent Vilket ger ett system av första ordningen. Vi behöver HL för varje $y_i$. Vi kikar närmare på $y_1^{\prime}$, vilket blir $y_2$:
\begin{equation*}
  \begin{gathered}
    \begin{cases*}
        y_1^{\prime} = y_2\\
        y_2^{\prime} = y_3\\
        \vdots\\
        y_{n-1}^{\prime} = y_n\\
        y_n = f(t,u, u^{\prime}, u^{\prime\prime},\cdots,u^{(n-1)}) = f(t,y_1, y_2,\cdots, y_n)
    \end{cases*}
  \end{gathered}
\end{equation*}
\par\bigskip
\noindent Kom ihåg! Begynnelsevärden.
\par\bigskip
\noindent Vad gör man om man har ett system med flera differentialekvationer? Jo, då skriver man om de till första ordningen och lägger ihop i en enda lång vektor.\par
\noindent Vid system kan varje ekvation med högre ordningens derivator skrivas om till första ordningen. Därför är vanligtvis mjukvara för ODE utformad för system av första ordningen.
\par\bigskip
\subsection{Begynnelsevärdesproblem i SciPy}\hfill\\
\par\bigskip
\noindent Vi ska ta fram en slags guide på givet ett problem, hur man löser det i SciPy.
\par\bigskip
\noindent Det första man ska göra är att skriva om på standardformen:
\begin{equation*}
  \begin{gathered}
    \begin{cases*}
      y^{\prime} = f(t,y)\qquad a\leq t\leq b\\
      y(a) = \alpha
    \end{cases*}
  \end{gathered}
\end{equation*}
\par\bigskip
\noindent där $y,f, \alpha$ är vektorvärda.
\par\bigskip
\noindent Skriv om till system av 1:a ordningen och specifiera position i vektorn $Y$ för varje komponenter i lösningen.\par\bigskip
\noindent Exempel, $\begin{pmatrix}S\\I\\R\end{pmatrix} = \begin{pmatrix}y_1\\y_2\\y_3\end{pmatrix}$. Man kan givetvis byta runt på raderna, men var konsekvent. Byter man på en rad på ena sidan av ledet måste man göra det på det andra.
\par\bigskip
\noindent Ställ upp begynnelsevärden för $Y$.
\par\bigskip
\noindent Skriv python funktion för HL, för $f(t,y)$. 
\par\bigskip
\noindent Från SciPy.integrate, används solve\_ivp(func,(a,b), $\alpha$.
\par\bigskip
\noindent Exempel: Omloppsbana
\begin{equation*}
  \begin{gathered}
    \begin{rcases*}
      x^{\prime\prime}=-GMy/r^3\\
      y^{\prime\prime}=-GMy/r^3\qquad t>0
    \end{rcases*}
  \end{gathered}
\end{equation*}\par
\noindent där $r=\sqrt{x^2+y^2}$ och $GM = 1$\par
\begin{equation*}
  \begin{gathered}
    \begin{rcases*}
      x(0)=1-e \text{ $e$ är excentritet}\\
      y(0)=0\\
      x^{\prime}(0)=0\\
      y^{\prime}(0)=\sqrt{\dfrac{1+e}{1-e}}
    \end{rcases*}
  \end{gathered}
\end{equation*}
\par
\noindent Där $0\leq e<1$
\par\bigskip
\noindent Som vi såg tidigare var första steget att skriva om på standardform. Vi kör båda ekvationerna samtidigt:
\begin{equation*}
  \begin{gathered}
    u_1 = x,\qquad u_2 = x^{\prime},\qquad u_3 = y,\qquad u_4 = y^{\prime}\\
    \begin{cases*}
      u_1^{\prime} = u_2\\
      u_2^{\prime} = -GMu_1/r^3\\
      u_3^{\prime} = u_4\\
      u_4^{\prime} = -GMu_3/r^3\\
      r = sqrt{u_1^2+u_3^2}
    \end{cases*}
  \end{gathered}
\end{equation*}
\par\bigskip
\noindent Vi påminner om begynnelsevärden:
\begin{equation*}
  \begin{gathered}
    u(0)= \begin{pmatrix}1-e\\0\\0\\\sqrt{\dfrac{1+e}{1-e}}\end{pmatrix}
  \end{gathered}
\end{equation*}
\par\bigskip
\noindent Då har vi gjort första steget. Andra steget var att skriva en python-funktion för $f(t,u)$:
\par\bigskip

\begin{verbatim}
import numpy as np
from scipy.integrate import solve_ivp
import matplotlib.pyplot as plt

def fun(t,u):
    GM = 1
    r = np.sqrt(u[0]**2+u[2]**2)
    return np.array([u[1],
                     -GM*u[0]/r**3,
                     u[3],
                     -GM*u[2]/r**3])

\end{verbatim}
\par\bigskip
\noindent Då kommer nästa steg, sätt begynnelsevillkor och anropa solve\_ivp:
\par\bigskip

\begin{verbatim}
tspan = (0, 2*np.pi) # en period
ecc = 0
u0 = np.array([1-ecc, 0, 0, np.sqrt((1+ecc)/(1-ecc))])
SOL = solve_ivp(fun, tspan, u0, max_step=0.01)

plt.plot(SOL.y[0,:], SOL.y[2,:])
plt.grid()
plt.axis('equal')
plt.xlabel('x')
plt.ylabel('y')
plt.show()
\end{verbatim}
\par\bigskip
\noindent SOL är ett objekt med bland annat SOL.t som är en vektor med $n+1$ tidpunkter, SOL.y som är den numeriska lösningen i de här tidpunkterna (vilket i vårat fall är 4 vektorer, en vektor för varje komponent) alltså $4$x$(n+1)$-matris.
\par\bigskip
\noindent Sista steget är att plotta lösningen. $x$-axeln kommer ha funktionen $x$ och samma med $y$-axeln, alltså inte variabeln $t$.
\par\bigskip
\subsection{Eulers metod (explicit Euler)(Euler framåt)}\hfill\\
\par\bigskip
\noindent Givet följande differentialekvation:
\begin{equation*}
  \begin{gathered}
    \begin{rcases*}
      y^{\prime}(t)=f(t,y(t))\qquad a\leq t\leq b\\
      y(a)=\alpha
    \end{rcases*}
  \end{gathered}
\end{equation*}
\par\bigskip

\begin{figure}[ht]
    \centering
    \incfig{euler}
    \caption{Euler}
    \label{fig:euler}
\end{figure}
\par\bigskip
\noindent Generellt: $t_{i+1}=t_i+h$ där $h=\dfrac{b-a}{n}$
\begin{equation*}
  \begin{gathered}
    \begin{rcases*}
      t_{i+1}=t_i+h$ \text{ där } $h=\dfrac{b-a}{n}\\
      y_{i+1}=y_i + hf(t_i,y_i)\\
      t_0=a,\qquad y_0 = \alpha
    \end{rcases*}
  \end{gathered}
\end{equation*}\par
\noindent Detta är någon slags geometrisk motivering till metoden, men den går att härleda analytiskt via en taylorserie:
\begin{equation*}
  \begin{gathered}
    y(t+h)=y(t)+hy^{\prime}(t)+\dfrac{h^2}{2}y^{\prime}(t)+\cdots\\
    \underbrace{y(t)=hf(t,y(t))}_{\text{Euelersteg}}+\underbrace{O(h^2)}_{\text{Lokalt trunkeringsfel}}
  \end{gathered}
\end{equation*}\par
\noindent Det lokala trunkeringsfelet är bara i 1 steg, det är bland annat därför den kallas för \textit{lokal}.
\par\bigskip
\subsection{Diskretiseringsfel - Eulers metod}\hfill\\
\par\bigskip
\noindent Vi talade om det \textit{lokala} diskretiseringsfelet, vilket är felet i \textit{ett} steg. 
