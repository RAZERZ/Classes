\section{Förtydligande/Från boken}

\subsection{Integraler}\hfill\\

\noindent Ibland går det inte att lösa integraler analytiskt och då kan vi använda numeriska metoder för att lösa dem. Vi kan göra detta givet hela funktionen, eller så kan vi göra detta givet punkter och där vi senare använder olika metoder för att approximera grafen så att vi kan integrera den. Det finns 2 typer av beräkningsalgoritmer:

\begin{itemize}
  \item Öppna
    \begin{itemize}
      \item En väldigt kraftfull typ av öppen integrering är så kallad \textit{Gaussisk kvadratur}. Här uppskattas integralen genom att evaluera den i icke-ekvidistanta punkter och genom att ansätta en "vikt" vid varje punkt som talar om hur mycket den punkten bidrar.
    \end{itemize}
  \item Stängda
    \begin{itemize}
      \item Evaluerar function i ändpunkterna av det givna intervallet. Den mest grundläggande algoritmen för detta kallas för \textit{Romberg} integrering vilket är baserad på trapetsidén som man senare har förfinat till Simpsons formula. Romberg integrering är alltså i princip Simpsons, men med lite mer logisk koppling mellan trapetsidén och Simpsons.
    \end{itemize}
\end{itemize}

\subsection{Riemann integralen}\hfill\\
\noindent Vi vet från envariabelanalysen att integralens definition ser ut på följande sätt:


\begin{equation*}
  \begin{gathered}
    I=\int_{a}^{b}f(x)dx=\lim_{\Delta x\to0}\sum_{i=0}^{n}f(x_i)\Delta x\\
    \Delta x = \dfrac{b-a}{n}
  \end{gathered}
\end{equation*}
\par\bigskip
\noindent Detta bildar våra rektanglar som vi sedan summerar deras area över ett intervall. Ett lite bättre sätt att approximera integralen hade varit att om vi istället för att betrakta rektanglar, att vi betraktar trapetser, det vill säga man "ansluter" $f(x_i)$ och $f(x_{i+1})$. Arean av en trapets ges av $A=\dfrac{a+b}{2}\cdot h$. Om vi låter höjden vara $\Delta x$ och $a$ resp. $b$ ges av funktionsvärderna får vi istället $\dfrac{f(x_i)+f(x_{i+1})}{2}\cdot\Delta x=\Delta A$, varpå trapetsformeln kommer ifrån. 
\par\bigskip
\noindent Det visar sig att felet som uppstår i trapetsens uppskattning är proportionell mot $\Delta x^3f^{\prime\prime}(\xi)$, där $\xi$ är någon okänd punkt i intervallet. Denna metod för att räkna fram felet används sällan, ty man behöver finna andra-derivatan och inte nog med det så måste man även hitta när uttrycket antas sitt största värde (man utgår alltid från det värsta fallet, det vill säga att man har det högsta felet på intervallet).
\par\bigskip

\subsection{Relationen mellan Simpsons och Trapetsapprox.}\hfill\\

\noindent Eftersom både trapets och Simpsons evaluerar funktionen på precis samma mängd av punkter, bör det finnas något form av samband. De använder även samma geometriska härledning för att komma fram till formlerna, en där man använder parabler, och den andra använder vi trapetser. Både geometriskt och algebraiskt måste det finnas någon koppling!
\par\bigskip
\noindent Simpsons behöver som absolut minst 3 datapunkter som input. Låt oss kalla dessa $f_1, f_2, f_3$. Vad händer om vi stoppar in samma datapunkter i trapetsen istället, kan vi härleda något algebraiskt uttryck/samband mellan de två?


\begin{equation*}
  \begin{gathered}
    S_k=T_k+\dfrac{T_k-T_{k-1}}{3}
  \end{gathered}
\end{equation*}
\pagebreak

\subsection{Analytiskt bevis för feltermen i trapetsmetoden}\hfill\\

\noindent Antag att vi har en funktion $f(t)$, vars integral $I =\int_{a}^{b}f(t)dt$ vi vill approximera.
\par\bigskip
\noindent Vi påstår att med hjälp av trapetsmetoden kommer feltermen vara $\dfrac{M(b-a)^2}{4n}$ där $M$ är derivatans övre begränsning och $n$ antal steglängder.
\par\bigskip
\noindent Vi påminner oss om att steglängden $\neq$ antal steglängder och att steglängden $h = \dfrac{b-a}{n}$.
\par\bigskip
\noindent Feltermen uttrycks av $|I-T(h)|$
\par\bigskip
\noindent Vi börjar!
\par\bigskip

\begin{lem}[Indeldning i delintervall]{lem:stepint}
  \noindent Låt varje steglängd betecknas med intervallet $[x_i, x_{i+1}]$, där mittpunkten ges av $c_i=\dfrac{x_{i+1}+x_{i}}{2}$. Detta betyder att $h = x_{i+1}-x_i$. Då uttrycks feltermen för varje delintervall av:

  \begin{equation*}
    \begin{gathered}
      \left|I-\dfrac{h}{2}(f(x_{i+1})+f(x_i)\right| = \left|-(I-\dfrac{h}{2}(f(x_{i+1})+f(x_i))\right| = \left|\dfrac{h}{2}(f(x_{i+1})+f(x_i)-I\right|\\
      \left|\dfrac{h}{2}(f(x_{i+1})+f(x_i)-\int_{x_i}^{x_{i+1}}f(t)dt\right|
    \end{gathered}
  \end{equation*}
  \par\bigskip
  \noindent Och för alla delintervall är det summan av alla delintervall:

  \begin{equation*}
    \begin{gathered}
      \sum_{i=0}^{n}\left|\dfrac{h}{2}(f(x_{i+1})+f(x_i)-\int_{x_i}^{x_{i+1}}f(t)dt\right|
    \end{gathered}
  \end{equation*}

\end{lem}
\par\bigskip

\begin{lem}[Felterm uttryckt i integral]{lem:errint}
  \noindent Vi påminner oss om partial integrering, $\int uv^{\prime}= uv - \int u^{\prime}v$
  \par\bigskip
  \noindent Vi vill försöka flytta in feltermen så att den ser ut som att den är partialt integrerad:

  \begin{equation*}
    \begin{gathered}
      \begin{rcases*}
        (t-c_i) = u(t)\\
        f^{\prime}(t) = v^{\prime}(t)
      \end{rcases*}\\\\
      \dfrac{h}{2}(f(x_{i+1})+f(x_i) = \dfrac{x_{i+1}-x_i}{2}f(x_{i+1})+\dfrac{x_{i+1}-x_i}{2}f(x_i)\\
      (x_{i+1}-c_i)f(x_{i+1})+(c_i-x_i)f(x_i) = (x_{i+1}-c_i)f(x_{i+1})+(-(c_i-x_i))f(x_i)\\
      (x_{i+1}-c_i)f(x_{i+1})-(c_i-x_i)f(x_i)\\
      \text{Feltermen ges alltså av } \left((x_{i+1}-c_i)f(x_{i+1})-(c_i-x_i)f(x_i)\right) - \int_{x_i}^{x_{i+1}}f(t)dt\\
      \Lrarr \int_{x_i}^{x_{i+1}}(t-c_i)f^{\prime}(t)dt
    \end{gathered}
  \end{equation*}
 
\end{lem}
\pagebreak

\begin{prf}[Felterm med första derivata]{prf:errone}

 \par\bigskip
  \noindent Lemma 3.1 och 3.2 ger att vi kan använda faktumet att $f^{\prime}(t)$ hade en övre begränsning på $M$:
  \par\bigskip


  \begin{equation*}
    \begin{gathered}
      \int_{x_i}^{x_{i+1}}(t-c_i)f^{\prime}(t)dt \leq\left|M\int_{x_i}^{x_{i+1}}(t-c_i)dt\right| = M\int_{x_i}^{x_{i+1}}\left|t-c_i\right|dt\\
      \text{Låt } u = t-c_i\text{, } \int_{x_i}^{x_{i+1}}\left|t-c_i\right| = \int_{x_i}^{x_{i+1}}\left|u\right|\\
      \begin{rcases*}
        f = \left|u\right|\\
        f^{\prime} = \dfrac{\left|u\right|}{u}
        g^{\prime} = 1
      \end{rcases*}\\
      \int_{x_i}^{x_{i+1}}\left|u\right| = \left|\left|u\right|u\right|_{x_i}^{x_{i+1}} - \int_{x_i}^{x_{i+1}}\dfrac{\left|u\right|}{u}\cdot u\\
      \int_{x_i}^{x_{i+1}}\left|u\right|= \left|u\left|u\right|\right|_{x_i}^{x_{i+1}}-\int_{x_i}^{x_{i+1}}\left|u\right|\\
      2\int_{x_i}^{x_{i+1}}\left|u\right| = \left|u\left|u\right|\right|_{x_i}^{x_{i+1}}\\
      \Lrarr\int\left|u\right| = \dfrac{u\left|u\right|}{2}\\
      \text{Bort med $u$-sub:} \int_{x_i}^{x_{i+1}}\left|t-c_i\right| = \dfrac{(x_{i+1}-c_i)\left|x_{i+1}\right|}{2}- \dfrac{(x_{i}-c_i)\left|x_{i}-c_i\right|}{2}\\
      \Lrarr \dfrac{(x_{i+1}-c_i)\left|x_{i+1}\right|}{2}- \dfrac{(-(c_i-x_{i}))\left|-(c_i-x_i)\right|}{2} = \dfrac{(x_{i+1}-c_i)\left|x_{i+1}\right|}{2}+\dfrac{(c_i-x_{i})\left|c_i-x_i\right|}{2}\\
      \begin{rcases*}
        x_{i+1}-c_i\\
        c_i-x_i
      \end{rcases*}=\dfrac{x_{i+1}-x_i}{2}\\\\
      \dfrac{2\cdot\left(\dfrac{x_{i+1}-x_i}{2}\right)\left|\dfrac{x_{i+1}-x_i}{2}\right|}{2} = \left(\dfrac{x_{i+1}-x_i}{2}\right)\left|\dfrac{x_{i+1}-x_i}{2}\right| = \dfrac{1}{4}(x_{i+1}-x_i)^2\\
      \text{Nu kan vi skriva tillbaks $M$ framför: } \dfrac{M}{4}(x_{i+1}-x_i)^2\\
      \text{Notera att } (x_{i+1}-x_i) \text{ är ju steglängden, dvs } h:\\
      \dfrac{Mh^2}{4} \geq \dfrac{f^{\prime}(t)h^2}{4}
    \end{gathered}
  \end{equation*}

 \end{prf}
\pagebreak
\noindent Vi vill nu visa att om andraderivatan är så kan vi använda begränsningen för att få en ännu finare felterm.
\begin{lem}[Partialintegrering av feltermintegral för delintervall]{thm:perrstepint}
   \noindent Vi vill partialintegrera, men vi kommer inte göra det enkelt för oss utan vi ansätter $(t-c_i)$ som den "redan deriverade" termen:
   \par\bigskip

   \begin{equation*}
     \begin{gathered}
       \int_{x_i}^{x_{i+1}}(t-c_i)f^{\prime}(t)dt\Lrarr \left|\dfrac{(t-c_i)^2}{2}f^{\prime}(t)\right|_{x_i}^{x_{i+1}}-\int_{x_i}^{x_{i+1}}\dfrac{(t-c_i)^2}{2}f^{\prime\prime}(t)dt\\
     \end{gathered}
   \end{equation*}
   \par\bigskip
   \noindent Vi påminner oss om:

   \begin{equation*}
     \begin{gathered}
       x_{i+1}-c_i = c_i-x_i = \dfrac{x_{i+1}-x_i}{2} = \dfrac{h}{2}
     \end{gathered}
   \end{equation*}
   \par\bigskip
   \noindent Insättning av detta i första partialintegrerade termen ger:


   \begin{equation*}
     \begin{gathered}
       \dfrac{\left(\dfrac{x_{i+1}-x_i}{2}\right)^2}{2}f^{\prime}(x_{i+1})-\dfrac{\left(x_i-c_i\right)^2}{2}f^{\prime}(x_i) = \dfrac{\left(\dfrac{x_{i+1}-x_i}{2}\right)^2}{2}f^{\prime}(x_{i+1})-\dfrac{\left(-(c_i-x_i)\right)^2}{2}f^{\prime}(x_i)\\
       \Lrarr \dfrac{\left(\dfrac{x_{i+1}-x_i}{2}\right)^2}{2}f^{\prime}(x_{i+1})-\dfrac{\left(\dfrac{x_{i+1}-x_i}{2}\right)^2}{2}f^{\prime}(x_i)\\
       \Lrarr \dfrac{\left(\dfrac{x_{i+1}-x_i}{2}\right)^2}{2}\left(f^{\prime}(x_{i+1})-f^{\prime}(x_i)\right) = \dfrac{\left(\dfrac{h}{2}\right)^2}{2}\\
       = \dfrac{h^2}{8}\left(f^{\prime}(x_{i+1})-f^{\prime}(x_i)\right) = \dfrac{1}{2}\left(\dfrac{h}{2}\right)^2\left(f^{\prime}(x_{i+1})-f^{\prime}(x_i)\right)\\
     \end{gathered}
   \end{equation*}
   \par\bigskip
   \noindent Men detta är bara följande integral:

   \begin{equation*}
     \begin{gathered}
       \dfrac{1}{2}\int_{x_i}^{x_{i+1}}f^{\prime\prime}(t)dt =\int_{x_i}^{x_{i+1}}\dfrac{1}{2}\left(\dfrac{h^2}{2}\right)^2f^{\prime\prime}(t)dt
     \end{gathered}
   \end{equation*}
   \par\bigskip
   \noindent Då har vi alltså:


   \begin{equation*}
     \begin{gathered}
       \int_{x_i}^{x_{i+1}}\dfrac{1}{2}\left(\dfrac{h^2}{2}\right)^2f^{\prime\prime}(t)dt -\int_{x_i}^{x_{i+1}}\dfrac{(t-c_i)^2}{2}f^{\prime\prime}(t)dt\\
     = \int_{x_i}^{x_{i+1}}\dfrac{1}{2}\left(\dfrac{h}{2}\right)^2f^{\prime\prime}(t)-\dfrac{(t-c_i)^2}{2}f^{\prime\prime}(t)dt\\
     = \dfrac{1}{2}\int_{x_i}^{x_{i+1}}f^{\prime\prime}(t)\left(\left(\dfrac{h}{2}\right)^2-(t-c_i)^2\right)
     \end{gathered}
   \end{equation*}
 
\end{lem}
\pagebreak
 \begin{prf}[Felterm med andraderivata]{prf:errtwo}
  \noindent Från Lemma 3.3 påminner oss om att vi hade en begränsning $M$ på $f^{\prime\prime}(t)$:
   \begin{equation*}
     \begin{gathered}
       \dfrac{1}{2}M\int_{x_i}^{x_{i+1}}\left(\left(\dfrac{h}{2}\right)^2-(t-c_i)^2\right)dt
     \end{gathered}
   \end{equation*}
   \par\bigskip
   \noindent Vi vill lösa för integralen:

   \begin{equation*}
     \begin{gathered}
       \int_{x_i}^{x_{i+1}}\left(\left(\dfrac{h}{2}\right)^2-(t-c_i)^2\right)dt = \int_{x_i}^{x_{i+1}}\left(\dfrac{h}{2}\right)^2dt-\int_{x_i}^{x_{i+1}}(t-c_i)^2dt\\
       = \left(\dfrac{h}{2}\right)^2\left[x_{i+1}-x_{i}\right]-\left|\dfrac{(t-c_i)^3}{3}\right|_{x_i}^{x_{i+1}}\\
       = \left(\dfrac{h}{2}\right)^2\left[x_{i+1}-x_i\right]-\left(\dfrac{(x_{i+1}-c_i)^3}{3}-\dfrac{(x_i-c_i)^3}{3}\right)\\
       = \left(\dfrac{h}{2}\right)^2\left[x_{i+1}-x_i\right]-\left(\dfrac{\left(\dfrac{h}{2}\right)^3}{3}-\dfrac{(-(c_i-x_i))^3}{3}\right)\\
       = \left(\dfrac{h}{2}\right)^2\left[x_{i+1}-x_i\right]-\left(\dfrac{\left(\dfrac{h}{2}\right)^3}{3}-\dfrac{-\left(\dfrac{h}{2}\right)^3}{3}\right)\\
       = \left(\dfrac{h}{2}\right)^2\left[x_{i+1}-x_i\right]-\left(\dfrac{h^3}{24}-\dfrac{-h^3}{24}\right)\\
       = \left(\dfrac{h}{2}\right)^2\left[x_{i+1}-x_i\right]-2\left(\dfrac{h^3}{24}\right)\\
       = h\left(\dfrac{h}{2}\right)^2-\left(\dfrac{h^3}{12}\right) = \dfrac{h^3}{4}-\dfrac{h^3}{12} = \dfrac{h^3}{6}
     \end{gathered}
   \end{equation*}
   \par\bigskip
   \noindent Nu när integralen är löst kan vi sätta fram koefficienterna:


   \begin{equation*}
     \begin{gathered}
       \dfrac{1}{2}M\cdot\dfrac{h^3}{6} = \dfrac{Mh^3}{12}\geq\dfrac{h^3}{12}f^{\prime\prime}(t)
     \end{gathered}
   \end{equation*}
\end{prf}




































