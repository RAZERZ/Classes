\section{Repetition/Gamla tentor}
\par\bigskip
\subsection{ODE:er}\hfill\\
\par\bigskip
\noindent Hur skall man göra om en metod är konsistent samt bestämma nogrannhetsordning?\par
\noindent Vi introducerar en alternativ definition av det lokala trunkeringsfelet som förenklar analysen.
\par\bigskip
\noindent \textbf{Enstegsmetoder} är något vi tagit upp i kursen och skrivs på den generella formen:
\begin{equation*}
  \begin{gathered}
    y_{i+1} = y_i + h\varphi(t_i,y_i, y_{i+1}, h)\text{ (Implicit metod ty vi har $y_{i+1}$)}
  \end{gathered}
\end{equation*}
\par\bigskip
\noindent \textbf{Lokalt trunkeringsfel}, felet i ett steg som anges som avstånd mellan exakta lösning och den numeriska metoden. Det globala ges genom att ta flera steg.
\par\bigskip
\noindent En alternativ definition av trunkeringsfelet är att den exakta lösningen insätts i metoden, varpå det lokala trunkeringsfelet ges av resttermen $\tau_i = \left|u_i(t_{i+1})-(\underbrace{u_{i}(t_i)}_{\text{=$y_i$}}+h\varphi(t_i,\underbrace{u_i(t_i)}_{\text{=$y$}}, \underbrace{u_i(t_{i+1})}_{\text{$\neq y_{i+1}$}}, h))\right|$\par
\noindent För explicita metoder ($\varphi$ oberoende av $y_{i+1}$) är lokala trunkeringsfelet = $\tau_i$\par
\noindent För implicta metoder med $\varphi(t_i,y_i,y_{i+1},h)$ Lipschitz-kontinuerlig i $y_{i+1}$ med konstant $c$ gäller att lokala trunkeringsfelet är mindre eller lika med $\dfrac{1}{1-hc}\tau_i$ vilket innebär att om vi kan visa att $\tau_i = O(h^{p+1})$ så gäller också $O(h^{p+1})$
\par\bigskip
\noindent Hur gör man om detta kommer på tentamen?
\begin{itemize}
  \item Steg 1: Sätt in exakt lösning $y(t)$ i metoden, detta ger $\tau_i = y(t_i+h)-(y(t_i)+h\varphi(t_iy(t_i),y(t_i+h),h))$
  \item Taylorutveckla $\Rightarrow$ målet är att få ett lokalt trunkeringsfel som funktion av $h$
  \item $\tau_i = O(h^{p+1})\Rightarrow$ ordning $p$
  \item $\lim_{h\to 0}\dfrac{\tau_i}{h}=0\Rightarrow$ konsistens
\end{itemize}
\par\bigskip
\noindent Låt oss tillämpa detta på Euler framåt/bakåtm där $y^{\prime} = f(t,y)$:
\begin{equation*}
  \begin{gathered}
    y_{i+1} = y_i+hf(t_i,y_i)\\
  \tau_i = y(t_i+h)-y(t_i)-h \underbrace{f(t_i,y(t_i))
}_{\text{$y^{\prime}(t_i)$}}
\end{gathered}
\end{equation*}\par
\noindent Taylorutveckling ger:
\begin{equation*}
  \begin{gathered}
    \tau_i = y(t_i)+hy^{\prime}(t_i)+O(h^2) -y(t_i)-hy^{\prime}(t_i) = O(h^2)
  \end{gathered}
\end{equation*}\par
\noindent Ordningen ges då av $2-1 = 1$
\par\bigskip
\begin{equation*}
  \begin{gathered}
    \lim_{h\to 0}\dfrac{\tau_i}{h} = 0
  \end{gathered}
\end{equation*}
\par\bigskip
\noindent För Euler bakåt:
\begin{equation*}
  \begin{gathered}
    y_{i+1} = y_i+hf(t_i,y_{i+1})
  \end{gathered}
\end{equation*}\par
\noindent Vi sätter in exakta lösningen:
\begin{equation*}
  \begin{gathered}
    \tau_i = y(t_{i+1})-y(t_i)-h\underbrace{f(t_{i+1},y_{i+1})}_{\text{$y^{\prime}(t_{i+1})$}}
  \end{gathered}
\end{equation*}\par
\noindent Taylorutveckling ger av $y(t_i)$ (för att hamna i $y(t_{i+1})$):
\begin{equation*}
  \begin{gathered}
    y(t_i) = y(t_{i+1} + \underbrace{(t_i-t_{i+1})}_{\text{$=-h$}}) = y(t_{i+1}-h) = y(t_{i+1})-hy^{\prime}(t_{i+1})+O(h^2)\\
    \tau_i = y(t_{i+1})-(y(t_{i+1})-hy^{\prime}(t_{i+1})+O(h^2)+hy^{\prime}(t_{i+1}))\\
    \Lrarr \tau_i = O(h^2)
  \end{gathered}
\end{equation*}\par
\noindent Ordningen ges då av $2-1=1$ och den blir därmed konsistent.
