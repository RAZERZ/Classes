\section{Tolkning av sannolikheter}
\par\bigskip
\noindent Om vi tar exemplet att singla slant. Vad betyder det att sannolikheten är $\dfrac{1}{2}$?\par
\noindent Man kan tolka det som att "det finns 2 fall, och båda har lika stor chans att inträffa"\par
\noindent Eller en mer data-inriktad tolkning, det vill säga om man singlar slant 100ggr, kommer ungefär hälften av kasten resultera i krona eller klave.
\par\bigskip
\noindent Det finns däremot tolkningar via Kolmogorovs axiom, det vill säga:
\begin{itemize}
  \item $P(A) = p$  betyder att $A$ utgör $p$ enheter av utfallsrummet $\Omega$
  \item Om vi upprepat slumpar ett $x\in\Omega$ så kommer tillslut $x\in A$  inträffa med frekvens $p$ (\textbf{stora talens lag})
\end{itemize}
\par\bigskip
\subsection{Sannolikhetsmåttet $P$}\hfill\\
\par\noindent Uppfyller följande:
\par\bigskip
\begin{itemize}
  \item $P(A^c) = 1-P(A)$
  \item $P(\O) = P(\Omega^c) = 1-P(\Omega) = 1-1 = 0$
  \item $A,B$ disjunkta gäller $P(A\cup B) = P(A\cup B \cup\O\cdots)$ (ty axiomet säger att vi skall ha oändliga disjunkta par, vi kan därför fylla ut med oändligt många tomma mängder) $\Rightarrow P(A)+P(B)$
  \item $P(B\setminus A) = P(B)-P(A\cap B)$
  \item Om $A\subseteq B$ så gäller $A\cap B = A$ och $P(B\setminus A) = P(B)-P(A)$
  \item Om $P(B\setminus A)\geq0$ så $A\subseteq B\Rightarrow P(A)\leq P(B)$
  \item $P(A\cup B) = P(A)+P(B)-\underbrace{P(A\cap B)}_{\text{$\geq0$}}$
  \item $P(A\cup B) \leq P(A)+P(B)$ (Booles olikhet)
\end{itemize}
\par\bigskip

\begin{theo}
  OOm $A_1\subseteq A_2\subseteq\cdots\subseteq\Omega$ så gäller
  \begin{equation*}
    \begin{gathered}
      P(\bigcup_{i=1}^{\infty}A_i) = \lim_{n\to\infty}P(A_n)
    \end{gathered}
  \end{equation*}
  \par\bigskip
  \noindent Kallas även för att sannolikhetsmåttet är kontinuerligt ovanifrån
\end{theo}
\par\bigskip
\begin{prf}[Bevis av föregående sats]{prf:prev}
  \begin{equation*}
    \begin{gathered}
      \underbrace{A_1}_{\text{$B_1$}}, \underbrace{A_2\setminus A_1}_{\text{$B_2$}},\underbrace{A_3\setminus A_2}_{\text{$B_3$}}\cdots \underbrace{A_{n+1}\setminus A_n}_{\text{$B_{n+1}$}}
    \end{gathered}
  \end{equation*}
  \par\bigskip
  \noindent $B_i$ är disjunkta, och följande gäller:
  \begin{equation*}
    \begin{gathered}
      \bigcup_{i=1}^{\infty}A_i = \bigcup_{i=1}^{\infty}B_i\\
      P\left(\bigcup_{i=1}^{\infty}A_i\right) = P\left(\bigcup_{i=1}^{\infty}B_i\right) = \sum_{i=1}^{\infty}P(B_i) = \lim_{n\to\infty}(P(B_1)+P(B_2)+P(B_3)+\cdots+P(B_n))\\
      \Lrarr \lim_{n\to\infty}(P(A_1)+(P(A_2)-P(A_1))+(P(A_3)-P(A_2))+\cdots+(P(A_n)-P(A_{n-1})))\\
      \Lrarr\lim_{n\to\infty}P(A_n)
    \end{gathered}
  \end{equation*}
\end{prf}
\newpage
\begin{theo}
  LLåt $A_3\subseteq A_2\subseteq A_1\cdots\subseteq\Omega$:
  \begin{equation*}
    \begin{gathered}
      P\left(\bigcap_{i=1}^{\infty}A_i\right) = \lim_{n\to\infty}P(A_n)
    \end{gathered}
  \end{equation*}
\end{theo}
\par\bigskip
\begin{lem}[De morgans lagar]{lem:demorgan}
  \begin{itemize}
    \item $\left(\bigcup_{i=1}^{\infty}A_i\right)^c = \bigcap_{i=1}^{\infty}A_i^c$
    \item $\left(\bigcap_{i=1}^{\infty}A_i\right)^c=\bigcup_{i=1}^{\infty}A_i^c$
  \end{itemize}
\end{lem}
\par\bigskip
\begin{prf}[Bevis av Lemma]{prf:lem}
  \begin{equation*}
    \begin{gathered}
      x\in\left(\bigcup_{i=1}^{\infty}A_i\right)^c\Lrarr x\notin \bigcup_{i=1}^{\infty}A_i\\
      \Lrarr x\notin A_i\quad \forall i\\
      \Lrarr x\in A_i^c\quad\forall i\\
      \Lrarr x\in\bigcap_{i=1}^{\infty}A_i^c
    \end{gathered}
  \end{equation*}
\end{prf}
\par\bigskip
\begin{prf}[Bebis av sats]{prf:prefv}
  Vi har $A_1^c\subseteq A_2^c\subseteq A_3^c\subseteq\cdots$:
  \begin{equation*}
    \begin{gathered}
      P\left(\bigcap_{i=1}^{\infty}A_i\right) = 1-P\left(\left(\bigcap_{i=1}^{\infty}A_i\right)^c\right) = 1-P\left(\bigcup_{i=1}^{\infty}A_i^c\right)\\
      \Rightarrow 1-\lim_{n\to\infty}P(A_i^c) = \lim_{n\to\infty}(1-P(A_i^c)) = \lim_{n\to\infty}P(A_n)
    \end{gathered}
  \end{equation*}
\end{prf}
\newpage
\section{Betingade sannolikheten $P(A|B)$}
\par\bigskip
\begin{theo}[Betingade sannolikheten]{thm:condprob}
  \begin{equation*}
    \begin{gathered}
      P(A|B) = \dfrac{P(A\cap B)}{P(B)} = \text{Sannolikheten för $A$ givet $B$ förutsatt att $P(B)>0$}
    \end{gathered}
  \end{equation*}
\end{theo}
\par\bigskip
\noindent\textbf{Exempel:}
\par\bigskip
\noindent Låt $\Omega = \{1,2,3,4,\cdots\}$, $P(\{n\})=\dfrac{1}{2^n}$
\par\bigskip
\noindent Detta sade vi kunde representera antalet slantsinglingar som krävs för att landa på krona (eller klave)\par
\noindent Säg nu att vi sätter det här $B = $ första försöket landar på klave = $\{1\}^c = \{2,3,4,5,\cdots\}$
\par\bigskip
\noindent Vi förväntar oss att $P(1|B) = 0$ ($B$ gäller, alltså att vi har fått klave på första försöket, men då gäller det att det inte finns någon chans att vi får krona på första försöket)
\par\bigskip
\noindent Med motiveringen över gäller $P(2|B) = \dfrac{1}{2}$ och följande:
\begin{equation*}
  \begin{gathered}
    P(n|B) = \dfrac{1}{2^{n-1}} =\dfrac{P(\{n\}\cap B)}{P(B)} = \dfrac{P(\{n\})}{1/2} = 2P(n)=2\cdot\dfrac{1}{2^n} = \dfrac{1}{2^{n-1}}
  \end{gathered}
\end{equation*}
\par\bigskip
\noindent Vi kan definiera ett sannolikhetsmått $Q:2^B\to\R$ (för något $B\in\Omega$) och $Q(A) = \dfrac{P(A)}{P(B)}$ = betingade sannolikheten\par
\noindent Mer generellt kan vi definiera $Q:2^\Omega\to\R$ genom $Q(A) = \dfrac{P(A\cap B)}{P(B)}$ (med andra ord, den betingade sannolikheten)
\par\bigskip
\noindent För att visa att $Q$ är ett sannolikhetsmått måste vi visa att den uppfyller Kolmogorovs axiom:
\begin{itemize}
  \item $P(A)\geq0\quad \forall A\in2^{\Omega}$
  \item $P(\Omega) = 1$
  \item $P\left(\bigcup_{i=1}^{\infty}A_i\right) = \sum_{i=0}^{\infty}P(A_i)$ om $A_i$ är parvis disjunkta
\end{itemize}
\par\bigskip
\noindent Detta kommer inte vara så svårt, om vi visar det för $Q:2^\Omega\to\R$ så har vi visat det för $Q:2^B\to\R$.\par
\noindent\textbf{Vi visar första axiomet:} 
\begin{equation*}
  \begin{gathered}
    Q(A) = \dfrac{P(A\cap B)}{P(B)}\geq0\quad\forall A\in2^\Omega\\
  \end{gathered}
\end{equation*}
\par\bigskip
\noindent\textbf{Andra axiomet:} 
\begin{equation*}
  \begin{gathered}
    Q(\Omega) = \dfrac{P(\Omega\cap B)}{P(B) = \dfrac{P(B)}{P(B)}} = 1
  \end{gathered}
\end{equation*}
\par\bigskip
\noindent\textbf{Tredje axiomet:}\par
\noindent Antag $A_1, A_2,\cdots$ disjunkta. Då är $B\cap A_1, B\cap A_2,\cdots$ också disjunkta.\par
\noindent Vi vill räkna följande:
\begin{equation*}
  \begin{gathered}
    Q\left(\bigcup_{i=1}^{\infty}A_i\right) = \dfrac{P\left(\left(\bigcup_{i=1}^{\infty}A_i\right)\cap B\right)}{P(B)}
  \end{gathered}
\end{equation*}\par\bigskip
\noindent Notera:
\begin{equation*}
  \begin{gathered}
    \left(\bigcup_{i=1}^{\infty}A_i\right)\cap B = \bigcup_{i=1}^{\infty}(A_i\cap B)\text{ ty följande:}\\
    x\in \left(\bigcup_{i=1}^{\infty}A_i\right)\cap B\Rightarrow x\in A_i \text{ för något $i$ och $x\in B$}\\
    \Lrarr x\in A_i\cap B\text{ för något $i$}\\
    \Lrarr x\in \bigcup_{i=1}^{\infty}(A_i\cap B)
  \end{gathered}
\end{equation*}
\par\bigskip
\noindent Vi får då:
\begin{equation*}
  \begin{gathered}
    Q\left(\bigcup_{i=1}^{\infty}A_i\right) = \dfrac{P(\bigcup_{i=11}^{\infty}A_i\cap B)}{P(B)} = \dfrac{\sum_{i=1}^{\infty}P(A_i\cap B)}{P(B)} = \underbrace{\dfrac{\sum_{i=1}^{\infty}P(A_i\cap B)}{P(B)}}_{\text{$Q(A_i)$}} = \sum_{i=1}^{\infty}Q(A_i)
  \end{gathered}
\end{equation*}
\par\bigskip
\noindent Nu följer det till exempel att:
\begin{itemize}
  \item $P(A^c|B) = 1-P(A|B)$
  \item $P(A\cup C| B) = P(A|B)+P(C|B)-P(A\cap C |B)$
  \item Om $A\cap B\subseteq A_2\cap B\subseteq A_2\cap B\subseteq\cdots$ så gäller $P\left(\bigcup_{i=1}^{\infty}A_i|B\right) = \lim_{n\to\infty}P(A_n|B)$
\end{itemize}
\par\bigskip
\subsection{Oberoende utsagor}\hfill\\
\par
\noindent Antag att $P(A) >0$ och $P(B)>0$. Vi säger att $A$ och $B$ är \textbf{oberoende} om $P(A|B) = P(A)$ och $P(B|A) = P(B)$
\par\bigskip
\noindent\textbf{Exempel:}\par
\noindent Singla slant 2ggr, $\Omega = \{kr,kl\}^2$.\par
\noindent Vi ansätter $A= $ första försöket ger krona $= \{(kr,kr),(kr,kl)\}$\par
\noindent Vi ansätter $B = $ andra försöket ger krona $ = \{(kl,kr),(kr,kr)\}$
\par\bigskip
\noindent Vi får då följande:
\begin{equation*}
  \begin{gathered}
    P(A) = \dfrac{1}{2} = P(B)\qquad P(A\cap B) = P(kr,kr) = \dfrac{1}{4}\\\\
    P(A|B) = \dfrac{P(A\cap B)}{P(B)} = \dfrac{1/4}{1/2} = \dfrac{1}{2} = P(A)\\\\
    P(B|A) = \dfrac{P(B\cap A)}{P(A)} = \dfrac{1/4}{1/2} = \dfrac{1}{2}=P(B)\\\\
    \Rightarrow\text{$A$ och $B$ är oberoende}
  \end{gathered}
\end{equation*}
\par\bigskip
\noindent\textbf{Exempel:}\par
\noindent Låt $\Omega = $ Uppsalas vuxna befolkning.\par
\noindent Låt $A = \{\text{Man}\}\qquad B = \{\text{Bruna ögon}\}\qquad C = \{\text{Över 170cm}\}$
\par\bigskip
\noindent Avgör vilka som är oberoende 
