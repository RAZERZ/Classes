\section{Räkna gamla tentor - 2021-12-22}\par
\subsection{Uppgift 1}\hfill\\\par
\noindent Sannolikheten att alla har samma färg är kombinatorik. Antingen har vi plockat 4st vita, 4st blå eller 4st gröna. På hur många sätt kan man välja 4st gröna? Jo $\begin{pmatrix}8\\4\end{pmatrix}$st sätt att välja 4 gröna, $\begin{pmatrix}6\\4\end{pmatrix}$ 4st blå, osv osv.\par
\noindent Totalt har vi $\begin{pmatrix}8\\4$$\end{pmatrix}+\begin{pmatrix}6\\4\end{pmatrix}+\begin{pmatrix}4\\4\end{pmatrix}$ och totalt finns det 18st val får vi att sannolikheten är:
\begin{equation*}
  \begin{gathered}
    \dfrac{\begin{pmatrix}8\\4$$\end{pmatrix}+\begin{pmatrix}6\\4\end{pmatrix}+\begin{pmatrix}4\\4\end{pmatrix}}{\begin{pmatrix}18\\4\end{pmatrix}}
  \end{gathered}
\end{equation*}
\par\bigskip
\noindent Sannolikheten att det finns minst en boll av alla färger bland de 4 plockade. Detta kan löses kombinatoriskt, eller så kan vi lösa det sannolikhetsteoretiskt.\par
\noindent Låt $A=$ minst en grön boll, $B = $ minst en blå boll, $C=$ minst en vit\par
\noindent Vi söker då $P(A\cap B\cap C)$. Notera att $P(A\cup B\cup C) = 1$.\par
\noindent $P(A\cup B) = 1-P(\text{alla vita}) = 1- \dfrac{\begin{pmatrix}4\\4\end{pmatrix}}{\begin{pmatrix}18\\4\end{pmatrix}}$\\
\noindent $P(A\cup C) = 1-P(\text{alla blå}) = 1-\dfrac{\begin{pmatrix}6\\4\end{pmatrix}}{\begin{pmatrix}18\\4\end{pmatrix}}$\par
\noindent $P(B\cup C) = 1-\dfrac{\begin{pmatrix}8\\4\end{pmatrix}}{\begin{pmatrix}18\\4\end{pmatrix}}$\par
\noindent $P(A) = 1-P(\text{alla blå och vit}) = 1-\dfrac{\begin{pmatrix}10\\4\end{pmatrix}}{\begin{pmatrix}18\\4\end{pmatrix}}$\par
\noindent $P(B) = 1-\dfrac{\begin{pmatrix}12\\4\end{pmatrix}}{\begin{pmatrix}18\\4\end{pmatrix}}$\par
\noindent $P(C) = 1-\dfrac{\begin{pmatrix}14\\4\end{pmatrix}}{\begin{pmatrix}18\\4\end{pmatrix}}$\par
\noindent $\overbrace{P(A\cup B\cup C)}^{\text{=1}} = P(A)+P(B)+P(C)-P(A\cap B)-P(A\cap C)-P(B\cap)+P(A\cap B\cap C)$\par
\noindent $P(A\cap B) = P(A)+P(B)-P(A\cup B) = 1-\dfrac{\begin{pmatrix}10\\4\end{pmatrix}}{\begin{pmatrix}18\\4\end{pmatrix}}+1-\dfrac{\begin{pmatrix}12\\4\end{pmatrix}}{\begin{pmatrix}18\\4\end{pmatrix}}-1+\dfrac{\begin{pmatrix}4\\4\end{pmatrix}}{\begin{pmatrix}18\\4\end{pmatrix}}$\par
\noindent $P(A\cap C)$ och $P(B\cap C)$ räknas ut på samma sätt.\par
\noindent Lös ut $P(A\cap B\cap C)$ och stoppa in värdena.
\par\bigskip
\subsection{Uppgift 2}\hfill\\\par
\noindent  Axiomen är triviala. Att $F$ inträffar när $E$ inträffar, så betyder det att $E\subseteq F$ \par
\noindent $F = \underbrace{E\cup (F\backslash E)}_{\text{disjunkta}}$, då kan vi summera sannolikheterna:
\begin{equation*}
  \begin{gathered}
    P(F) = P(E)+\underbrace{P(F\backslash E)}_{\text{$\geq0$}}\geq P(E)+0 = P(E)
  \end{gathered}
\end{equation*}
\par\bigskip
\subsection{Uppgift 3}\hfill\\\par
\noindent $X\sim Bin(n,p) = E(X) = np, Var(X) = np(1-p)$ (finns i formelsamlingen)
\begin{equation*}
  \begin{gathered}
    E(Z) = 2E(X) + E(Y) = 2\cdot5\cdot0.1+10\cdot0.2 = 3\\
    Var(Z) \stackrel{ober.}{=} Var(2X)+Var(Y) = 4Var(X)+Var(Y)=4\cdot5\cdot0.1\cdot(1-0.1)+10\cdot0.2\cdot0.8 = 3.4
  \end{gathered}
\end{equation*}
\par\bigskip
\noindent För att bestämma sannolikheten att $Z=2$ kollar vi på vad det betyder att $Z=2$:
\begin{equation*}
  \begin{gathered}
    Z=2 \Lrarr \left\{(0,2),(1,0)\right\}
  \end{gathered}
\end{equation*}\par
\noindent Detta är disjunkta händelser:
\begin{equation*}
  \begin{gathered}
    P(X=0\& Y=2)+P(X=1\&Y=0) = P(X=0)P(Y=2)+P(X=1)P(Y=0)\\
    0.9^5\cdot\begin{pmatrix}10\\2\end{pmatrix}0.2^2(1-0.2)^{10-2}+5\cdot0.1\cdot0.9^4\cdot0.8^{10}
  \end{gathered}
\end{equation*}
\par\bigskip
\subsection{Uppgift 4}\hfill\\\par
\noindent Sannolikheten att en slumpmässig planta ger en grön avkomma.\par
\noindent Vi vet att $P(A) = 1/9$, $P(B)=P(C) = 4/9$ samt\par
\noindent $P(\text{Grön}|A) = 1$, $P(\text{Grön}|B) = 3/4$\par
\noindent  $P(\text{Grön}|C) = 9/16$\par
\noindent Vi använder lagen om total sannolikhet för att hitta $P(\text{Grön})$:
\begin{equation*}
  \begin{gathered}
    P(\text{Grön}) = P(\text{Grön}|A)P(A) + P(\text{Grön}|B)P(B)+ P(\text{Grön}|C)P(C)\\
    = \dfrac{25}{36}
  \end{gathered}
\end{equation*}
\par\bigskip
\noindent Om avkomman är grön, vad är då sannolikheten att den kommer från $A,B,C$?\par
\noindent Vi söker alltså $P(A|\text{Grön})$ osv. Vi kan vända på betigningen:
\begin{equation*}
  \begin{gathered}
    P(A|\text{Grön}) = \dfrac{P(\text{Grön}|A)P(A)}{P(\text{Grön})}\\
    = \dfrac{1\cdot1/9}{25/36} = \dfrac{4}{25}
  \end{gathered}
\end{equation*}
\par\bigskip
\subsection{Uppgift 5}\hfill\\\par
\noindent Centrala gränsvärdessatsen för att visa:
\begin{equation*}
  \begin{gathered}
    \lim_{n\to\infty} e^{-n}\sum_{k=1}^{n}\dfrac{n^k}{k!} = 0.5
  \end{gathered}
\end{equation*}\par
\noindent Notera, detta ser lite ut som:
\begin{equation*}
  \begin{gathered}
    X_n\sim Po(n)\Rightarrow P(X_n=k) = \dfrac{e^{-n}n^k}{k!}
  \end{gathered}
\end{equation*}\par
\noindent Det är som att vi summerar $Po(n)$, vi får då:
\begin{equation*}
  \begin{gathered}
    \lim_{n\to\infty} e^{-n}\sum_{k=1}^{n}\dfrac{n^k}{k!} = P(1\leq X_n\leq n)
  \end{gathered}
\end{equation*}\par
\noindent Om $Y_1,Y_2,\cdots\sim Po(1)$ (och oberoende), då kommer:
\begin{equation*}
  \begin{gathered}
    \sum_{k=1}^{n}Y_k \sim Po(n)
  \end{gathered}
\end{equation*}\par
\noindent Om vi byter ut $k=1$ mot 0:
\begin{equation*}
  \begin{gathered}
    \underbrace{e^{-n}\sum_{k=0}^{n}\dfrac{n^k}{k!}}_{\text{$f_X$ för $X\sim Po(n)$}} = P(\overbrace{Y_1+\cdots+Y_n}^{\text{$n\cdot\overline{Y_n}$}}\leq n)
  \end{gathered}
\end{equation*}\par
\noindent För en poisson $n$ fördlening gäller att $E(X_n) = n$ och $Var(X_n)= n$:
\begin{equation*}
  \begin{gathered}
    = P(\overline{Y_n}\leq 1) = P(\underbrace{\overline{Y_n}-1}_{\text{$E(\overline{Y_n})$}}\leq 0)\\
    = P\left(\dfrac{\overline{Y_n}-n}{D(\overline{Y_n})\leq0}\right)\\
    \stackrel{CGS}{\Rightarrow}\Phi(0) = 0.5\\
    \lim_{n\to\infty}e^{-n}\sum_{k=0}^{n}\dfrac{n^k}{k!} = 0.5\\
    e^{-n}\sum_{k=1}^{n}\dfrac{n^k}{k!} = \left(e^{-n}\sum_{k=0}^{n}\dfrac{n^k}{k!}\right)-e^n\to 0.5
  \end{gathered}
\end{equation*}
\par\bigskip
\subsection{Uppgift 7}\hfill\\\par
\noindent Lösning till $a$:
\begin{equation*}
  \begin{gathered}
    \psi_{X_1+X_2}(t)\stackrel{ober.}{=} = \psi_{X_1}(t)\psi_{X_2}(t) = (1-2t)^{-n_1/2}(1-2t)^{-n_2/2}\\
    = (1-2t)^{-(n_1+n_2)/2}
  \end{gathered}
\end{equation*}
\par\bigskip
\noindent Kom ihåg att $E(X^k)= \psi^{(k)})(0)$, vi får:
\begin{equation*}
  \begin{gathered}
    E(X) = \dfrac{d}{dt}(1-2t)^{-n/2}|_{_t=0} = \left(\dfrac{-n}{2}\right)(1-2t)^{-\dfrac{n}{2}-1}\cdot(-2)\\
    = n(1-2\cdot0)^{\dfrac{-n}{2}-1} = n\\
    E(X^2) = \psi_X^{\prime\prime}(0)= \dfrac{d}{dt}n(1-2)^{-n/2-1}|_{_t=0}\\
    \Rightarrow n(n+2)\\
    Var(X)=E(X^2)-(E(X))^2 = n(n+2)-n^2 = n^2+2n-n^2=2n
  \end{gathered}
\end{equation*}
\par\bigskip
\subsection{Uppgift 8}\hfill\\\par
\noindent Finns 2 metoder, man kan räkna ut fördelningsfunktionen och ta gränsvärdet och försöka lösa det, eller så tar man gränsvärdet av momentgenererande funktionen och försöker identifiera den med en fördelning\par
\noindent Vi vill hitta $\lim_{n\to\infty}F_{X_n}(t) = \lim_{n\to\infty}P(X_n\leq t)$:
\begin{equation*}
  \begin{gathered}
    t\in\N
    P(X_n\leq t) = 1-P(X_n> t) = 1-\left(1-\dfrac{\lambda}{\lambda+n}\right)^{t+1}\\
    P(X_n\leq t) = 0,\; t<0\\
    t>0\Rightarrow P(X_n/n\leq t) = 1-\left(1-\dfrac{\lambda}{\lambda+n}\right)^{\left\lfloor t\right\rfloor+1}\\
    P\left(\dfrac{X_n}{n}\leq t\right) = 1-\left(1-\dfrac{\lambda}{\lambda+n}\right)^{\left\lfloor nt\right\rfloor+1}\\
    \Rightarrow\lim_{n\to\infty}\left(1-\dfrac{\lambda}{\lambda+n}\right)^{\left\lfloor nt\right\rfloor+1}\\
    \Rightarrow\lim_{m\to\infty}\left(1-\dfrac{\lambda}{m}\right)^{mt-m\lambda}\qquad m = \lambda+n\\
    = \lim_{m\to\infty}\underbrace{\left(1-\dfrac{\lambda}{m}\right)^{mt}}_{\text{$e^{-\lambda t}$}}\cdot\underbrace{\left(1-\dfrac{\lambda}{m}\right)^{-\lambda t}}_{\text{1}}
  \end{gathered}
\end{equation*}
\par\bigskip
\noindent\textbf{Anmärkning:}\par
\noindent $\left\lfloor t\right\rfloor$ = största heltal $\leq t$
