\section{Sammanfattning K2}
\subsection{Betingning}\hfill\\\par
\noindent Givetvis kan faktumet att en annan händelse har inträffat påverka sannolikheten att en annan händelse inträffar, detta kallas för \textit{betingning}, där man undersöker sannolikheten för att en händelse $A$ inträffar, givet att en händelse $B$ inträffar.\par
\noindent Uttallas även $A$ betingat $B$ och skrivs $P(A|B)$
\par\bigskip
\noindent\textbf{Exempel:}\par
\noindent Antag att vi har en kortlek (52 kort, 4st av dessa 52 är ess osv) och vi ska dra två kort från en kortlek. \par
\noindent Låt $A$ = händelsen att vi drar ett ess vid första draget och $B$ = händelsen att vi drar ett ess vid andra draget, vad är då $P(B|A)$?
\par\bigskip
\noindent Om $A$ har inträffat har vi inte längre 52 kort, utan 51 (vi har nämligen dragit ett) och vi har inte längre 4 ess, utan 3, alltså har vi en chans på $\dfrac{3}{51}$ givet att $A$ har inträffat, vilket vi skriver på följande: $P(B|A) = \dfrac{3}{51}$
\par\bigskip
\begin{theo}[Betingad sannolikhet]{thm:igown}
  Antag $P(A)>0$. Den \textit{betingade sannolikheten} för händelsen $B$ givet att händelsen $A$ har inträffat skrivs $P(B|A)$ och definieras som
  \begin{equation*}
    \begin{gathered}
      P(B|A) = \dfrac{P(B\cap A)}{P(A)}
    \end{gathered}
  \end{equation*}
\end{theo}
