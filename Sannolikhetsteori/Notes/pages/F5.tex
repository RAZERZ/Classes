\section{Sammanfattning K2}
\subsection{Komplement och additionssatsen}\hfill\\\par
\noindent Om $A$ och $B$ är godtyckliga händelser i utfallsrummet $\Omega$ så gäller följande:\par
\begin{itemize}
  \item $P(A^c)=1-P(A)$
  \item $P(\O)=0$
  \item $P(A\cup B) = P(A)+P(B)-P(A\cap B)$
\end{itemize}\par
\subsection{Sannolikhet på utfallsrum}\hfill\\\par
\noindent Vi vill definiera en funktion som tar varje händelse i vårt utfallsrum och tilldelar den ett värde mellan 0 till 1 som talar om hur sannolikt det är att denna händelse inträffar.\par
\noindent Denna reellvärda funktion $P:\Omega\to\R$, kallar vi för \textit{sannolikhetsfunktionen} på utfallsrummet.\par
\noindent Vi kan däremot inte kalla funktionen för ett sannolikhetsfunktion om den inte uppfyller följande axiom:
\par
\begin{itemize}
  \item $0\leq P\leq1$
  \item $P(\Omega) = 1$
  \item Om $A\cap B = \O$ så gäller $P(A\cup B) = P(A)+P(B)$
\end{itemize}\par
\noindent Dessa axiom, kallas för \textit{Kolmogorovs axiom}.
\par\bigskip
\noindent En funktion $P$ som är definierad på delmängder till utfallsrummet $\Omega$ som också uppfyller Kolmogorovs axiom kallas för ett \textit{sannolikhetsmått} på $\Omega$
\par\bigskip
\noindent Ur detta ska vi se vad som händer om vi definierar betingning som en sannolikhetsfunktion, uppfyller den axiomen?
\par\bigskip
\noindent Vi fixerar en händelse $C$ i vårt utfallsrum och defnierar en funktion $Q(A) = P(A|C) = \dfrac{P(A\cap C)}{P(C)}$ där $P(C)>0$, vi verifierar om detta är ett sannolikhetsmått på $\Omega$ genom att kolla om axiomen uppfylls:
\par\bigskip
\noindent\textbf{Första axiomet}: Detta följer ur att $P(C)>0$ och att $P\in[0,1]$. Då kan inte bråket hamna utanför intervallet
\par\bigskip
\noindent\textbf{Andra axiomet}: Vi testar att stoppa in hela $\Omega$ i funktionen:
\begin{equation*}
  \begin{gathered}
    Q(\Omega) = P(\Omega|C) = \dfrac{P(\Omega\cap C)}{P(C)} = \dfrac{P(C)}{P(C)}=1
  \end{gathered}
\end{equation*}
\par\bigskip
\noindent\textbf{Tredje axiomet:} Här kommer vi nog behöva använda lite mängdlära, specifikt saker från komplement och additionssatsen samt distributiva lagar.\par
\noindent Givet att $A\cap B=\O$ vill vi visa att detta betyder att $Q(A\cup B) = Q(A)+Q(C)$ 
\begin{equation*}
  \begin{gathered}
    Q(A\cup B) = P(A\cup B|C)= \dfrac{P\overbrace{((A\cup B)\cap C)}^{\text{$=P((A\cap C)\cup(B\cap C))$}}}{P(C)}\\
    =\dfrac{P((A\cap C)\cup(B\cap C))}{P(C)}\stackrel{add.}{=}\dfrac{P(A\cap C)+P(B\cap C)-\overbrace{P\overbrace{(A\cap B)}^{\text{$=\O$}}}^{\text{=0}}}{P(C)}\\
    = Q(A)+Q(C)
  \end{gathered}
\end{equation*}
\par\bigskip
\noindent Från detta, följer faktiskt följande:
\begin{itemize}
  \item $P(A^c|C) = 1-P(A|C)$
  \item $P(A\cup B|C) = P(A|C)+P(B|C)-P(A\cap B|C)$
|C\end{itemize}

\subsection{Betingning}\hfill\\\par
\noindent Givetvis kan faktumet att en annan händelse har inträffat påverka sannolikheten att en annan händelse inträffar, detta kallas för \textit{betingning}, där man undersöker sannolikheten för att en händelse $A$ inträffar, givet att en händelse $B$ inträffar.\par
\noindent Uttallas även $A$ betingat $B$ och skrivs $P(A|B)$
\par\bigskip
\noindent\textbf{Exempel:}\par
\noindent Antag att vi har en kortlek (52 kort, 4st av dessa 52 är ess osv) och vi ska dra två kort från en kortlek. \par
\noindent Låt $A$ = händelsen att vi drar ett ess vid första draget och $B$ = händelsen att vi drar ett ess vid andra draget, vad är då $P(B|A)$?
\par\bigskip
\noindent Om $A$ har inträffat har vi inte längre 52 kort, utan 51 (vi har nämligen dragit ett) och vi har inte längre 4 ess, utan 3, alltså har vi en chans på $\dfrac{3}{51}$ givet att $A$ har inträffat, vilket vi skriver på följande: $P(B|A) = \dfrac{3}{51}$
\par\bigskip
\begin{theo}[Betingad sannolikhet]{thm:igown}
  Antag $P(A)>0$. Den \textit{betingade sannolikheten} för händelsen $B$ givet att händelsen $A$ har inträffat skrivs $P(B|A)$ och definieras som
  \begin{equation*}
    \begin{gathered}
      P(B|A) = \dfrac{P(B\cap A)}{P(A)}
    \end{gathered}
  \end{equation*}
\end{theo}
\par\bigskip
\noindent Från detta följer det givetvis att $P(B\cap A) = P(A)P(B|A)$.\par
\noindent Coolt faktum! Eftersom snitt-operatorn är kommutativ, så innebär det faktiskt följande: $P(A|B) = P(B|A)$
\par\bigskip
\noindent Låt oss undersöka vad som händer som vi betraktar $P(A\cap B\cap C)$:
\begin{equation*}
  \begin{gathered}
    P(A\cap B\cap C) = P(\underbrace{(A\cap B)}_{\text{$=Q$}}\cap C) = P(C\cap\underbrace{(A\cap B)}_{\text{$=Q$}})\\
    \Rightarrow P(C|Q) = \dfrac{P(C\cap Q)}{P(Q)} = P(C|A\cap B) = \dfrac{P(C\cap A\cap B)}{P(A\cap B)}\\
    \Rightarrow P(Q|C) = \dfrac{P(Q\cap C)}{P(C)} = P(A\cap B|C) = \dfrac{P(A\cap B\cap C)}{P(C)}\\\\
    \Rightarrow P(C|A\cap B) = P(A\cap B|C)
  \end{gathered}
\end{equation*}
\par\bigskip
\subsection{Oberoende}\hfill\\\par
\noindent Med betingning har vi undersökt hur sannolikheten påverkas av andra händelser, exempelvis hur sannolikheten att dra ett ess påverkas av att dra ett annat kort. När man studerar slumpexperiment är det ofta av intresse att veta om händelserna beror av varandra eller inte, eftersom de kan möjligen påverka slutsatserna av detta slumpexperiment.
\par\bigskip
\noindent Informellt säger vi att två händelser är \textit{oberoende} om de inte har med varandra att göra.
\par\bigskip
\noindent\textbf{Exempel:}\par
\noindent Låt $L=$ att vinna på lotto en viss dag, $R=$ att det regnar i Stockholm samma dag\par
\noindent Eftersom dessa händelser inte har något med varandra att göra, så säger vi att dessa är \textit{oberoende}. Det vi formellt vill formulera, är att sannolikheten för att $L$ inträffar är densamma även om $R$ inträffar (och vice versa).\par
\noindent Använder vi notationen från betingning, så uttrycker vi det på följande sätt:
\begin{equation*}
  \begin{gathered}
    P(L|R) = P(L)\qquad P(R|L) = P(R)
  \end{gathered}
\end{equation*}\par
\noindent Det är faktiskt så vi definierar oberoende:
\par\bigskip
\begin{theo}[Oberoende händelser]{thm:disjpoiwog}
  Två händelser $A$ och $B$ sägs vara \textit{oberoende} om:\par
  $P(A|B) = P(A)$ förutsatt att $P(B)>0$\par
  $P(B|A)=P(B)$ förutsatt att $P(A)>0$
\end{theo}
\par\bigskip
\noindent\textbf{Anmärkning:}\par
\noindent Vi sade tidigare att betingade händelser kommuterar ($P(A|B)=P(B|A)$), detta gäller även här förutsatt att sannolikheten för vardera händelser är $>0$, men från detta följer det ju att $P(B)=P(A)$. Från detta följer det då att det räcker att verifiera att $P(A|B)=P(A)$ för att visa att både $A$ och $B$ är obereonde!
\par\bigskip
\noindent Låt oss undersöka vidare, eftersom vi vet hur vi kan uttrycka $P(A|B)$, så bör vi kunna hitta ett uttryck för $P(A\cap B)$ förutsatt att $A$ och $B$ är obereonde:
\begin{equation*}
  \begin{gathered}
    P(A|B) = \dfrac{P(A\cap B)}{P(B)} = P(A) \Lrarr P(A\cap B) = P(A)P(B)
  \end{gathered}
\end{equation*}
\par\bigskip
\noindent Intressant! Givetvis antas $P(A)$ och $P(B)$ vara $>0$
\par\bigskip
\noindent\textbf{Svagare obereonde:}\par
\noindent En svagare variant av oberoende är att titta på par av oberoende händelser i utfallsrummet. Att händelser är parvis oberoende innebär \textbf{inte} att mängden av dessa händelser är fullständigt oberoende, man måste nämligen undersöka alla par och se till att även de är oberoende.\par\bigskip
\noindent Mer formellt säger vi att en mängd händelser $\left\{A_1,\cdots\right\}$ sägs vara \textit{parvis oberoende} om för alla par $(i,j)$ (där $i\neq j $), gäller att $P(A_i\cap A_j)=P(A_i)P(A_j)$\par
\noindent Mängden sägs vara \textit{fullständigt oberoende} om det för alla $k\geq2$ och alla delmängder $\left\{A_{i_1},\cdots, A_{i_k}\right\}$ med\par\noindent $i_1<\cdots<i_k$, gäller att $P(A_{i_1}\cap\cdots\cap A_{i_k}) = P(A_{i_1})\cdots P(A_{i_k})$
\par\bigskip
\noindent\textbf{Exempel:}\par
\noindent Antag att vi har två normala tärningar (6 sidor), en röd och en svart. Vi låter $A$ vara händelsen att vi slår ett udda tal på den röda tärningen, och $B$ vara händelsen att vi slår ett udda tal på den svarta tärningen. Låt nu $C$ var händelsen att summan av den röda och svarta tärningen är udda.\par
\noindent\textit{Avgör om händelserna är parvis och eller fullständigt oberoende.}
\par\bigskip
\noindent Det lättaste är att avgöra om händelserna är fullständigt oberoende, så vi kollar det först. Då vill vi kolla $P(A\cap B\cap C)$, vilket översatt till ord blir "sannolikheten att $A,B,C$ inträffar". Att slå udda på båda tärningar är inte osannolikt, men tyvärr kan då inte summan bli udda eftersom udda+udda = jämnt. Alltså måste $P(A\cap B\cap C) = 0$.\par
\noindent Om det skulle vara så att händelserna är fullständigt oberoende, så skulle även $P(A\cap B\cap C) = 0 = P(A)P(B)P(C)$, men eftersom $P(A), P(B)$ och $P(C)$ har sannolikhet $>0$, så motsäger detta att sannolikheten $=0$, alltså är de ej fullständigt oberoende.
\par\bigskip
\noindent Vi undersöker nu om de är parvis oberoende. $A$ och $B$ är oberoende eftersom resultatet från $A$ inte påverkar $B$ alls. Det gäller nu att undersöka om $(A,C)$ samt $(B,C)$ är oberoende, men enligt anmärkningen ovan gäller det faktiskt bara att undersöka om $A$ och $C$ är oberoende, så följer det att $B$ och $C$ är oberoende (eftersom $A$ och $C$ är obereonde).\par
\noindent Vi vill kolla att $P(C|A) = P(C)$. Givet att $A$ är ett udda tal, så måste alltså vi slå ett jämnt tal från $B$ för att $C$ ska gälla. Att slå ett jämnt tal har sannolikheten $\dfrac{1}{2}$, alltså är $P(C|A) = \dfrac{1}{2}$. Vi måste nu visa att $P(C) = \dfrac{1}{2}$:\par
\noindent Betrakta alla slagningar som par, vi får då $(1,1),(1,2),\cdots,(6,6)$. Det är $6\cdot6=36$st par. Hur många av dessa par har ett udda och ett jämnt tal? Rimligtvis 18 av de! Alltså är sannolikheten att $C$ inträffar $\dfrac{18}{36} = \dfrac{1}{2}$\par
\noindent Detta var ju dock precis $P(C|A)$, alltså har vi visat att $P(C|A) = P(C)$ vilket betyder att händelserna parvis är oberoende.
\par\bigskip
\subsection{Lagen om total sannolikhet}\hfill\\\par
\noindent Premisserna går ut på att det ibland är lättare att beräkna en betingad sannolikhet än att direkt räkna sannolikheten.\par
\noindent Målet är att hitta en "sluten formel" för att räkna $P(B)$ betingat andra händelser i utfallsrummet. Vi undersöker:\par
\begin{figure}[ht]
    \centering
    \incfig{initialt}
    \caption{Initialt}
    \label{fig:initialt}
\end{figure}
\par\bigskip
