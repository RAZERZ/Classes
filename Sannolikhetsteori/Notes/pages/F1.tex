\section{Repetition - (K2.1)}
\subsection{Mängdlära}\hfill\\
\par
\noindent \textbf{Tips för hela kursen!} Rita venndiagram\par
$\Omega$ är vår grundmängd/utfallsrum\par
\noindent $x\in\Omega$: $x$ är ett element/utfall i $\Omega$\par
\noindent $A\subseteq\Omega$:$A$ är en delmängd/händelse till $\Omega$\par
\noindent $2^{\Omega} = \{A:A\subseteq\Omega\}$, kallas även för potensmängden
\par\bigskip
\subsection{Begrepp}\hfill\\
\par
\noindent Om $A$ och $B$ är disjunkta säger vi att de är \textbf{oförenliga}, dvs $A\cap B = \O$\par\par
\noindent $A,B$ och $C$ är disjunkta om $A\cap B = \O$ och $A\cap C = \O$ och $B\cap C=\O$\par
\noindent $\lambda\subseteq 2^{\Omega}$ är disjunkta om $A\cap B = \O$ för alla $A,B\in\lambda$\par
\noindent Sannolikhetsrum = ($\Omega, P$)
\par\bigskip
\section{Regler för sannolikheter - (K2.2)}
\par\bigskip
\subsection{Kolmogorovs Axiom}\hfill\\
\par
Ett \textbf{sannolikhetsmått} är en funktion $P:2^{\Omega}\to\R$ som uppfyller:
\begin{itemize}
  \item $P(A)\geq0\quad \forall A\in2^{\Omega}$
  \item $P(\Omega) = 1$
  \item $P\left(\bigcup_{i=1}^{\infty}A_i\right) = \sum_{i=0}^{\infty}P(A_i)$ om $A_i$ är parvis disjunkta
\end{itemize}
\par\bigskip
\noindent\textbf{Exempel:}
\par\bigskip
\noindent Singla slant är det klassiska exemplet, där har vi 2 möjliga utfall (krona eller klave).\par
\noindent Utfallsrummet $\Omega$ är mängden $\{krona, klave\}$\par
\noindent Ett rimligt antagande är att sannolikheten att landa på krona är $\dfrac{1}{2}$ och samma för klave, dvs $P(\{krona\}) = \dfrac{1}{2}$ och $P(\{klave\}) = \dfrac{1}{2}$\par\bigskip
\noindent $P(\Omega) = 1$, $P(\O) = 0$
\par\bigskip
\noindent\textbf{Exempel:}
\par\bigskip
\noindent Singla slant 2 gånger\par
\noindent Utfallsrummet bör rimligtvis vara kopplad till föregående exempel:
\par\bigskip
\noindent$\Omega = \{kr,kl\}\text{x}\{kr,kl\} = \{(kr,kr),(kr,kl),(kl,kr),(kl,kl)\}$
\par\bigskip
\noindent $P(\{x\})=\dfrac{1}{4}$, $P(\text{minst en krona}) = P\left(\{(kr,kr),(kr,kl),(kl,kr)\}\right) = \dfrac{1}{4}+\dfrac{1}{4}+\dfrac{1}{4}+\dfrac{1}{4}=\dfrac{3}{4}$
\par\bigskip
\noindent\textbf{Exempel:}
\par\bigskip
\noindent Singla slant $n$ gånger: $\Omega = \{kr,kl\}^n$\par
\noindent $P(\{x\}) = \dfrac{1}{2^n}\quad \forall x\in\Omega,\quad P(A) = \sum_{x\in A}\dfrac{1}{2^n}$\par
\noindent $P(\text{exakt $k$st krona}) = \sum_{x\text{$x$ innehåller $k$ kronor}}\dfrac{1}{2^n} = \begin{pmatrix}n\\k\end{pmatrix}\left(\dfrac{1}{2^n}\right)$
\par\bigskip
\noindent\textbf{Exempel:}
\par\bigskip
\noindent Tärningskast är återigen ett klassiskt exempel, då är $\Omega = \{1,2,3,4,5,6\}$
\par\bigskip
\noindent Är det en normal tärning så är sannolikheten för varje kast $\dfrac{1}{6}$, $P(\{x\}) = \dfrac{1}{6}$
\newpage
\noindent Antag att jag har en riggad tärning sådant att ettan är ombytt till en sexa. Då kommer följande gälla:\par
\noindent $P(\{1\}) = 0$ och $P(\{6\}) = \dfrac{1}{3}$
\par\bigskip
\noindent Sannolikheter ska man tänka som proportioner, som associerar en vikt till varje delmängd
\par\bigskip
\noindent\textbf{Exempel:}
\par\bigskip
\noindent Låt $\Omega = \N_+$, $P(\{n\}) = \dfrac{1}{2^n}$
\par\bigskip
\noindent Eftersom $\sum_{n=1}^{\infty}\dfrac{1}{2^n}=1$ gäller det att $P(\Omega) = 1$
\par\bigskip
\noindent Kopplar vi detta exempel till verkligheten så kan detta vara "hur stor är sannolikheten att slinga krona $n$ gånger" eller "sannolikheten att slinga krona för första gången på $n$:te slinglen"
\par\bigskip
\noindent\textbf{Exempel:}
\par\bigskip
\noindent Vad är sannolikheten att tärningen hamnar på en sexa på $n$:te slinglen?\par
\noindent Jo, $P(\{x\}) = \underbrace{\left(\dfrac{5}{6}\right)^{n-1}}_{\text{alla andra siffror}}\cdot\dfrac{1}{6}$
\par\bigskip
\noindent\textbf{Exempel:}
\par\bigskip
\noindent Slumpa ett reellt tal mellan $0$ och $1$:\par
\noindent $\Omega = (0,1)\subseteq\R$, då är $P(A) = $ längden på intervallet $A$ $= 1$
\par\bigskip
\noindent Notera att det inte spelar roll om det är ett öppet eller slutet intervall
\par\bigskip
\noindent Vill man räkna ut unionen av sannolikheten summerar man sannolikheterna:
\begin{equation*}
  \begin{gathered}
    P\left(\left[\dfrac{1}{2},\dfrac{1}{4}\right]\cup\left(\dfrac{3}{4},\dfrac{7}{8}\right)\right) = \dfrac{1}{4}+\dfrac{1}{8} = \dfrac{3}{8}
  \end{gathered}
\end{equation*}
\par\bigskip
\noindent Vad är då sannolikheten att vi slumpar ett rationellt tal mellan $(0,1)$? Vi får inte glömma att $\Q$ är uppräknelig:
\begin{equation*}
  \begin{gathered}
    P\left(\Q\cap(0,1)\right) = P\left(\bigcup_{q\in\Q\cap(0,1)}\{q\}\right) = \sum_{q\in\Q\cap(0,1)}P(\{q\}) = 0
  \end{gathered}
\end{equation*}
\par\bigskip
\noindent Hur ser $P(\text{irrationellt tal})$ ut?
\begin{equation*}
  \begin{gathered}
    P(\Q^c\cap(0,1))\\
    \underbrace{(\Q\cap(0,1))\cup(\Q^c\cap(0,1))}_{\text{disjunkta}}=\Omega\\
    1 = P(\Omega) = \underbrace{P(\Q\cap(0,1))}_{\text{=0}}+P(\Q^c\cap(0,1))\Rightarrow P(\text{irrationellt tal}) = 1-0=1
  \end{gathered}
\end{equation*}
\par\bigskip
\noindent\textbf{Exempel:}
\par\bigskip
\noindent Ta en Riemann-integrerbar funktion $f:[0,1]\to\R$ så $\int_{0}^{1}f(x)dx=1$.
\par\bigskip
\noindent Vi sätter $P(A) = \int_{A}f(x)dx$
\par\bigskip
\noindent\textbf{Exempel:}
\par\bigskip
\noindent Tag enhetskvadraten $\Omega = [0,1]^2$, $P(A) =$ arean. Slumpa ett tal i kvadraten
\newpage
\begin{theo}[Diskreta Sannolikhetsrum]{thm:discroom}
  Sannolikhetsrummet $(\Omega, P)$ kallas för \textbf{diskret} om det finns en uppräknelig delmängd $A\subseteq\Omega$ så att:
  \begin{equation*}
    \begin{gathered}
      P(B) = \sum_{x\in B\cap A}P(\{x\})
    \end{gathered}
  \end{equation*}
  \par\bigskip
  \noindent Alternativ beskrivning:
  \begin{equation*}
    \begin{gathered}
      \exists A\subseteq\Omega:\\
      \sum_{x\in A}P(\{x\}) = 1
    \end{gathered}
  \end{equation*}
\end{theo}
\par\bigskip
\begin{theo}[Kontinuerliga Sannolikhetsrum]{thm:controom}
  Icke-diskreta sannoliketsrum (förutom blandade osv, men vi kommer inte arbeta med dessa ändå)
\end{theo}
\par\bigskip
\subsection{$A^c$}\hfill\\
\par
\noindent Med komplementet menar vi $x\in A^c\Lrarr x\in A$ där ($x\in\Omega,\quad A\subseteq\Omega$)
\par\bigskip
\noindent $P\left(A\cup A^c\right) = P(\Omega) = 1 = P(A)+P(A^c)\Rightarrow P(A^c)=1-P(A)$
\par\bigskip
\subsection{B-A}\hfill\\\par

\begin{equation*}
  \begin{gathered}
    x\in B\setminus A\Rightarrow x\in B \wedge x\notin A\\
    \Rightarrow x\in B \wedge x\in A^c\\
    x\in B\cap A^c\\
    P(B) = P(A\cap B)+P(B\setminus A)\Rightarrow P(B\setminus A) = P(B)-P(A\cap B)\\
    \Rightarrow P(A\cup B) = P(A)+P(B)-P(A\cap B)
  \end{gathered}
\end{equation*}
