\section{$\sigma$-algebras \& Measure spaces}
\subsection{$\sigma$-algebras}\hfill\\\par
\begin{defo}[$\sigma$-algebra]{}
  A collection of subsets $\Sigma$ of a set $S$ is called a $\sigma$\textit{-algebra} if:\par
  \begin{itemize}
    \item $\varnothing\in\Sigma$
    \item Is an algebra:\par
      \begin{itemize}
        \item Closed under complements such that for $A\in\Sigma\Rightarrow A^c = S\backslash A\in\Sigma$
      \item Closed under unions such that $A,B\in\Sigma\Rightarrow A\cup B\in\Sigma$
      \end{itemize}
    \item Closed under countably infinite unions $A_i\in\Sigma$ for $i\in\N$, then $\bigcup_{i=1}^{\infty}A_i\in\Sigma$
  \end{itemize}
\end{defo}
\par\bigskip
\noindent\textbf{Example:}\par
\noindent $\Sigma = \left\{\varnothing, S\right\}$ is a $\sigma$-algebra on any set $S$.\par
\noindent Another example is $\mathcal{P}(S)$, which denotes the powerset.\par
\noindent Another example is $S = \N$, then $\Sigma = \left\{\varnothing,\N, \left\{2k:k\in\N\right\},\left\{2k+1:k\in\N\right\}\right\}$
\par\bigskip
\noindent\textbf{Remark:}\par
\noindent There exists many equivalent definitions of a $\sigma$-algebra. For example, instead of the first axiom of $\varnothing\in\Sigma$, an equivalent definition could be "$\Sigma$ is non-empty", since then $\exists A\in\Sigma\Rightarrow A^c\in\Sigma\Rightarrow A\cup A^c = S \in\Sigma\Rightarrow (A\cup A^c)^c = \varnothing\in\Sigma$
\par\bigskip
\noindent\textbf{Remark:}\par
\noindent Closed under unions $\Rightarrow$ closed under finite unions since $A_1,\cdots, A_n\in\Sigma\Rightarrow A_1\cup A_2\in\Sigma, A_1\cup A_2\cup A_3 = \underbrace{(A_1\cup A_2)}_{\in\Sigma}\cup A_3$, thus by induction $A_1\cup\cdots\cup A_n\in\Sigma$
\par\bigskip
\noindent This does \textit{not} imply $\Sigma$ is closed under countable unions.
\par\bigskip
\noindent\textbf{Counter-example:}\par
\noindent Consider $S = [0,1)\subseteq\R$. Let $\Sigma$ be all finite unions of disjoint sets on the form $[a,b)$ such that $0\leq a\leq b<1$ (if $a=b\Rightarrow \varnothing$).\par
\noindent First and all algebra axioms are fulfilled, but the last one is not since we cvan consider $A_n = \left[\dfrac{1}{n}, 1\right)$. Then $\bigcup_{i=2}^{\infty} = (0,1)\not\in\Sigma$
\par\bigskip
\noindent An algebra $\Sigma$ is an algebra in an algebraic sense.\par
\noindent The symmetric difference $A\triangle B = (A\backslash B)\cup (B\backslash A)$. This behaves like "+" on $\Sigma$ and intersections behave like multiplication.\par
\noindent Just like one would expect from an algebra, the multiplication is distributive over addition, eg. $C\cap(A\triangle B) = (C\cap A)\triangle (C\cap B)$
\par\bigskip
\subsection{Measures}\hfill\\
\noindent Let $\Sigma$ be a $\sigma$-algebra on $S$, and let $\mu_0$ be a function from $\Sigma_0$ to $[0,\infty] = [0,\infty)\cup\left\{\infty\right\}$, essentially a function that assigns some value to subsets of $\Sigma$.\par
\noindent Intuitively, a measure should increase if we measure something bigger.
\par\bigskip
\begin{defo}[Additive and $\sigma$-additive measures]{}
  A measure $\mu_0$ is called \textit{additive} if $\mu_0(A\cup B) = \mu_0(A)+\mu_0(B)$ where $A, B$ are disjoint sets.
  \par\bigskip
  \noindent A measure $\mu_0$ is called \textit{$\sigma$-additive} if this holds for ocuntable unions, i.e if $A_n$ are pairwise disjoint, then $\mu_0\left(\bigcup_{n=1}^{\infty}A_n\right) = \sum_{n=1}^{\infty}\mu_0(A_n)$
\end{defo}
\par\bigskip
\noindent\textbf{Remark:}\par
\noindent We say that $\mu_0$ is a measure if $\mu_0$ is $\sigma$-additive and $\mu_0(\varnothing) =0$
\par\bigskip
\noindent\textbf{Example:}\par
\noindent $S = \left\{1,2,\cdots, 6\right\}$, $\Sigma = \mathcal{P}(S)$ and set $\mu_0(A) = \dfrac{1}{6}\left|A\right|$. Note here that $\mu_0(S) = 1$
\par\bigskip
\begin{defo}[Probability measures]{}
  All measures that sum up to 1 are called \textit{probability measures}
\end{defo}
\par\bigskip
\noindent\textbf{Example:}\par
\noindent $S = \N$, $\Sigma = \mathcal{P}(S)$ and set $\mu_0(A\in\Sigma) = \left|A\right|$. Here $\mu_0(S) = \infty$
\par\bigskip
\noindent\textbf{Example:}\par
\noindent $S = \N$, $\Sigma = \mathcal{P}(S)$ and set $\mu_0(A\in\Sigma) = \begin{cases*}0\quad\text{if}\quad \left|A\right|<\infty\\\infty\quad\text{if}\quad \left|A\right| = \infty\end{cases*}$\par
\noindent This is an example of an additive but not $\sigma$-additive measure, since if $A_n = \left\{n\right\}$, then $\mu_0\left(\bigcup_{n=1}^{\infty}A_n\right) = \infty$, but $\sum_{n=1}^{\infty}\mu_0(A_n) = -1$
\par\bigskip
\subsection{Measure spaces}\hfill\\
\begin{defo}[Measure space triplet]{}
  A \textit{measure space} is a triplet $(S,\Sigma, \mu)$ where $S$ is some set, $\Sigma$ is a $\sigma$-algebra over $S$, and $\mu$ is a $\sigma$-additive function $\mu:\Sigma\to[0,\infty]$ such that $\mu(\varnothing) = 0$
\end{defo}
\par\bigskip
\begin{defo}[Probability space]{}
  If $\mu(S) = 1$, then the triplet is called a \textit{probability space}.
\end{defo}
\par\bigskip
\noindent\textbf{Example:} (finite measure space)\par
\noindent Let $S = \left\{s_1,\cdots,s_k\right\}$ where $k\in\N$ be a set of outcomes. We also associate probabilities $p_1,\cdots,p_k$ to each $s_1,\cdots,s_k$ such that $\sum_i p_i = 1$. Let $\mu(A) = \sum_{s_i\in A}p_i$ $\forall A\subseteq S$. If we let $\Sigma = \mathcal{P}(S)$, then $(S,\Sigma,\mu)$ is a measure and a probability space.
\par\bigskip
\noindent\textbf{Example:} (Lebesgue measure)\par
\noindent Let $S = \R$, $\Sigma = \mathcal{B}(\R)$ be the Borel $\sigma$-algebra (smallest $\sigma$-algebra that makes open sets measureable, note that $\mathcal{B}(\R)\neq\mathcal{P}(\R)$) and let $\mu$ be something measuring length on finite unions of disjoint open intervals $A = (a_1,b_1)\cup\cdots\cup (a_n,b_n)$ such that $\mu(A) = \left|b_1-a_1\right|+\cdots+\left|b_n-a_n\right|$
\par\bigskip
\noindent This $\mu$ is called the Lebesgue measure ($\mathcal{L}$)\par
\noindent Restricting $S$ to $[0,1]$, then we have a probability measure
\begin{equation*}
  \begin{gathered}
    \mu = \mathcal{L}\mid_{[0,1]}(A) = \mathcal{L}(A\cap[0,1])\Rightarrow ([0,1], \mathcal{B}([0,1],\mathcal{L}\mid_{[0,1]}))\quad\text{is a probability measure}
  \end{gathered}
\end{equation*}\par
\noindent This is a formulation of uniform random numbers in $[0,1]$
\par\bigskip
\subsection{Properties of measures}\hfill\\
\noindent For a measure space, we have the following properties:\par
\begin{enumerate}[leftmargin=*]
  \item $\mu(A\cup B)\leq \mu(A)+\mu(B)$
  \item $\mu\left(\bigcup A_i\right)\leq \sum\mu(A_i)$
  \item $\mu\left(\bigcup_{i=1}^{\infty}A_i\right) = \sum_{i=1}^{\infty}\mu(A_i)-\mu(A_1\cap A_2)-\cdots-\mu(A_{n-1}\cap A_n) + \mu(A_1\cap A_2\cap A_3)\cdots + (-1)^{n+1}\mu(A_1\cap A_2\cdots\cap A_n)$
\end{enumerate}
\par\bigskip
\noindent Note that for the first two points, we have previously assumed that $A,B$ were disjoint. This would be the case for "joint" sets.
\par\bigskip
\begin{prf}[]{}
  Consider $\mu(A) = \mu(A\backslash B\cup (A\cap B)) = \mu(A\backslash B) + \mu(A\cap B)$ and proceed.
\end{prf}
\par\bigskip
\noindent\textbf{Remark:}\par
\noindent For point 4, check \href{https://math.stackexchange.com/questions/1487456/a-closed-formula-for-the-measure-of-the-union-of-n-sets}{Math Stackexchange}
\par\bigskip
\noindent The idea is if we can consider some set that is measurable, we want to be able to say something about the compositions of those measurable sets so the idea is we include their subsets in the $\sigma$-algebra (in the space we set up) as well as keeping it closed in an algebraic sense.
\par\bigskip
\subsection{Monoticity of measure}\hfill\\
\noindent Let $(A_i)$ be a sequence of increasing sets in $\Sigma$ such that $\varnothing\subseteq A_1\subseteq\cdots\subseteq S$. Then:
\begin{equation*}
  \begin{gathered}
    \mu(A_i) = \mu(A_i\backslash A_{i-1}\cup (A_i\cap A_{i-1})) = \mu(A_i\backslash A_{i-1}\cup A_{i-1}) = \mu(A_i\backslash A_{i-1})+\mu(A_{i-1})\geq\mu(A_{i-1})
  \end{gathered}
\end{equation*}\par
\noindent Thus, by induction, $\mu(A_1)\leq\mu(A_2)\leq\cdots$ and by monotone convergence the limit $\lim_{i\to\infty}\mu(A_i)$ exists in the extended positive real line.
\par\bigskip
\noindent Writing $A = \bigcup_{i=1}^{\infty}A_i$, we have $\mu(A) = \lim_{i\to\infty}\mu(A_i)$, this because:
\begin{equation*}
  \begin{gathered}
    A = A_1\cup(A_2\backslash A_1)\cup(A_3\backslash A_2)\cup\cdots\\
    \mu(A) = \mu(A_1)+\mu(A_2\backslash A_1)+\mu(A_3\backslash A_2)+\cdots = \lim_{n\to\infty}\sum_{i=1}^{n}\mu(A_i\backslash A_{i-1})
  \end{gathered}
\end{equation*}\par
\noindent where $A_0 = \varnothing = \lim_{n\to\infty}\mu(A_n)$
\par\bigskip
\noindent A similar result holds for decreasing sets, i.e $S\supseteq A_1\supseteq A_2\cdots\supseteq \varnothing$\par
\noindent We do the limit as $A = \bigcap_{i=1}^{\infty}A_i$ and by monotone convergence $\mu(A) = \lim_{i\to\infty}\mu(A_i)$ with similar proof.
\par\bigskip
\noindent\textbf{Remark:}\par
\noindent The last set in the decreasing sets does not necessarily have to be the empty set, recall that we are dealing with intersections instead of unions.
