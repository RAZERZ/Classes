\section{Integration}
\noindent Let $(S,\Sigma, \mu)$ be a measure-space. Given some measurable function $f\in m\Sigma$ where $f:S\to\R$, our goal is to define the integral of this function with respect to our measure $\mu$\par
\noindent We do it in 3 steps, and then the idea is to reduce integration problems to these 3 cases and proceed. Later we will develop some tools to ease in this conversion.
\par\bigskip
\begin{enumerate}[leftmargin=*]
  \item Define integration of an indicator function 
    \begin{equation*}
      \begin{gathered}
        I_A(s) = \begin{cases}1\quad\text{ if } s\in A\\0\quad\text{ else}\end{cases}\qquad \int I_A(s)d\mu = \mu(A)\quad\forall A\in\Sigma
      \end{gathered}
    \end{equation*}
    \par\bigskip
  \item Define the integral for finite linear combinations of characteristic functions $f(s) = \sum_{k=1}^{n}a_kI_{A_k}(s)$ where $a_k\in\R$ (sometimes we say they are non-negative, but this is not always the case) and $A_k\in\Sigma$\par
    \noindent These functions are called \textit{step-functions}. We set the integral:
    \begin{equation*}
      \begin{gathered}
        \int f(s)d\mu = \sum_{k=1}^{n}a_k\int I_{A_k}d\mu = \sum_{k=1}^{n}a_k\mu(A_k)
      \end{gathered}
    \end{equation*}\par
    \noindent Note that if $A_k$  overlap, they are double-counted. This is the desired functionality
    \par\bigskip
  \item For $f\in m\Sigma^+$, we define:
    \begin{equation*}
      \begin{gathered}
        \int f d\mu = \sup\left\{\int gd\mu:g\leq f\text{ $f,g$ are non-negative step-functions}\right\}
      \end{gathered}
    \end{equation*}
    \par\bigskip
  \item We extend this to all measurable functions $f$ by defining the positive $\&$ negative parts of a functions by 
    \begin{equation*}
      \begin{gathered}
        f^+(s) = \begin{cases}f(s)\quad\text{if $f(s)>0$}\\0\quad\text{else}\end{cases}\\
        f^-(s) = \begin{cases}-f(s)\quad\text{if $f(s)<0$}\\0\quad\text{else}\end{cases}
      \end{gathered}
    \end{equation*}\par
    \noindent Note here that both are non-negative and $f = f^+-f^-$. All operations utilized preserve measurability, thus we define
    \begin{equation*}
      \begin{gathered}
        \int f d\mu = \int f^+ d\mu-\int f^- d\mu
      \end{gathered}
    \end{equation*}
\end{enumerate}
\par\bigskip
\subsection{Properties of the integral}\hfill\\
\begin{enumerate}[leftmargin=*]
  \item\textbf{Linearity:} $\int af+bg d\mu = a\int f d\mu+b\int g d\mu\quad \forall a,b\in \R$ and $f,g\in m\Sigma$
    \par\bigskip
  \item\textbf{Monotonicity:} If $f\leq g\quad\forall s$, then $\int f d\mu\leq \int g d\mu$ (note here that the $\forall$ is really just a "for almost all" sign, since $\mu(\varnothing)=0$)
    \par\bigskip
  \item\textbf{Triangle-inequality:}
    \begin{equation*}
      \begin{gathered}
        \left|\int f d\mu\right| = \left|\int f^+ d\mu-\int f^- d\mu\right|\leq\left|\int f^+ d\mu\right|+\left|\int f^- d\mu\right| = \int\left|f\right|d\mu
      \end{gathered}
    \end{equation*}
\end{enumerate}
\par\bigskip
\noindent\textbf{Remark:}\par
\noindent If $\mu$ is a Lebesgue measure on $\R$ (in general in $\R^n$), then $\int d\mu$ is standard Lebesgue integration. If both the Riemann and Lebesgue integral exist (and are equal) then it is just standard Riemann integration
\par\bigskip
\noindent\textbf{Example:}\par
\noindent Let $\mu$ be the counting measure on integers. Consider $(\R,\mathcal{B}\left(\R\right),\mu)$\par
\noindent Then $\int f d\mu$ for any $f\in m\mathcal{B}\left(\R\right) = \sum_{k=1}^{\infty}f(k)\mu(\left\{k\right\}) = \sum_{k=1}^{\infty}f(k)$\par
\noindent Note, this sees very little of the space and does not work with Riemann integration.
\par\bigskip
\noindent We can restrict the domain just like we would with "regular integration". Let $A\in\Sigma$, then
\begin{equation*}
  \begin{gathered}
    \int_A fd\mu = \int fI_{A}d\mu
  \end{gathered}
\end{equation*}
\par\bigskip
\begin{defo}[Integrable]{}
  We say that $f\in m\Sigma$ is \textit{integrable} with respect to the measure $\mu$ if $\int f^+ d\mu$ and $\int f^- d\mu$ exist and are finite. Otherwise, we say that the integral is \textit{undefined}
  \par\bigskip
  \noindent The class of integrable functions is denoted $L^1(S,\Sigma,\mu)$
\end{defo}
\par\bigskip
\noindent\textbf{Note:} 
\noindent If $f(s) = \pm\infty$ and $f\in L^1$, we must have $\mu(\left\{s\right\}) = 0$: Moreoever,
\begin{equation*}
  \begin{gathered}
  \mu(\left\{s:f(s)=\pm\infty\right\}) = 0
  \end{gathered}
\end{equation*}
\par\bigskip
\begin{lem}[]{}
  If $f$ is a non-negative integrable function and $\int f d\mu = 0$, we claim that the measure of all points $s$ such that $f$ is positive is zero for almost every $s$ (i.e $\mu(\left\{s:f(s)=\pm\infty\right\}) = 0$)
\end{lem}
\par\bigskip
\begin{prf}[]{}
  The proof strategy here will be to dissect this using unions/intersections and their boundedness.\par
  \noindent Note that
  \begin{equation*}
    \begin{gathered}
    \bigcup_{n\in\N}\left\{s:f(s)>\dfrac{1}{n}\right\} = \left\{s:f(s)>0\right\}
    \end{gathered}
  \end{equation*}\par
  \noindent If $\mu(A_n) = 0\quad\forall n$, then the function is almost surely 0\par
  \noindent If this is not the case, then $\exists k\in\N$ such that $\mu(A_k)>0$, but then we can take a function that is bounded above by $f$ like $\dfrac{1}{k}I_k\leq f$.\par
  \noindent By monotonicity, 
  \begin{equation*}
    \begin{gathered}
      \int f(s)d\mu\geq \int\dfrac{1}{k}I_{A_k}d\mu = \dfrac{1}{k}\mu(A_k)>0
    \end{gathered}
  \end{equation*}\par
  \noindent By assumption, $\int f d\mu = 0$ which is a contradiction, so $\mu(I_{A_k}) = 0$
\end{prf}
\par\bigskip
\noindent One question that we shall try to explore now is what happens with limits of sequences of functions and their integrals?\par
\noindent Well, first we may see them as a sequence of expectation of random variables. By intution one might jump to the conclusion that
\begin{equation*}
  \begin{gathered}
    \lim_{n\to\infty}\int f_n d\mu = \int\lim_{n\to\infty}f_n d\mu
  \end{gathered}
\end{equation*}\par
\noindent But this is not the case, for example if $f_n = I_{[n,n+1)}$ and $\mu$ is the Lebesgue measure we have
\begin{equation*}
  \begin{gathered}
    \int f_n(s) d\mu = \int_{[n,n+1)}(x)dx = 1
  \end{gathered}
\end{equation*}\par
\noindent In particular, $\lim_{n\to\infty}\int f_n(s)d\mu = 1$\par
\noindent However, for a fixed $x$ we have $\lim_{n\to\infty}f_n(x) = 0$ and
\begin{equation*}
  \begin{gathered}
    \int\lim_{n\to\infty}f_n d\mu = 0d\mu = 0 
  \end{gathered}
\end{equation*}
\par\bigskip
\noindent There are circumstances where this equality holds however, and the goal is somewhat to discover \textit{when} this is.
\par\bigskip
\begin{theo}[Monotone Convergance Theorem]{}
  Let $f_n$ be a sequence of non-negative measurable funcitons $f_n\in m\Sigma^+$ such that $f_n\to f$ pointwise (i.e $f_n(x)$ is non-decreasing in $n$ and $f_n(x)\to f(x)$ as $n\to\infty$). Then
  \begin{equation*}
    \begin{gathered}
      \mu(f_n)\to \mu(f)\Rightarrow \lim_{n\to\infty}\int f_n d\mu\to\int f d\mu = \int\lim_{n\to\infty}f_n d\mu
    \end{gathered}
  \end{equation*}
\end{theo}
\noindent We can always approximate measurable functions using sequences of step functions using
\begin{equation*}
  \begin{gathered}
    \alpha^{(r)}(x) = \begin{cases}0\quad\text{if } x=0\\(i-1)2^{-r}\quad\text{ if } (i-r)2^{-r}\leq x\leq i2^{-r}\leq r\\r\quad\text{ for } x>r\end{cases}
  \end{gathered}
\end{equation*}\par
\noindent Note here that we do not mean $i\in\C$\par
\noindent This is basically discretisation of $y = x$ up to $r$. It is non-decreasing (monotonly) in $r$, and always less than y = x. \par
\noindent By setting $f^{(r)}(x) = \alpha^{(r)}(f(x))\quad f\in m\Sigma^+$, we get a function that has all the properties that we wished for (i.e $f_n\to f$ and $f_n$ are step functions).\par
\noindent We can use this to base our proofs on since we can then start to construct proofs in the following way:\par
\begin{enumerate}[leftmargin=*]
  \item Prove the property for indictor functions
  \item Extend to linear combination of step functions (by proven linearity)
  \item Extend to $f\in m\Sigma^+$ by monotonicity (exchange limits)
  \item Extend to $f\in m\Sigma$ by splitting $f = f^+-f^-$
\end{enumerate}
\par\bigskip
\begin{lem}[]{}
  Suppose $f,g\in L^1$ and $\in m\Sigma$ and $f=g$ for almost all $s\in S$. Then
  \begin{equation*}
    \begin{gathered}
      \int f d\mu = \int g d\mu
    \end{gathered}
  \end{equation*}
\end{lem}
\par\bigskip
\begin{prf}[(Sketch)]{}
  Consider $f-g$. We want to show that $\int f-g d\mu = 0$\par
  \noindent If $f,g$ are indictor functions, this is trivially true (since they must have the same step-functions)\par
  \noindent$\Rightarrow$ Take finite linear combinations and consider monotonicity through step functions (step of 0 is 0) and split into $f^+$ and $f^-$
\end{prf}
\par\bigskip
\begin{lem}[Fatous Lemma v2]{}
  Suppose $f_n$ is a sequence of non-negative measurable functions ($\in m\Sigma^+$). Then
  \begin{equation*}
    \begin{gathered}
      \mu(\lim_{n\to\infty}\inf f_n) \leq \lim_{n\to\infty}\inf\mu(f_n)
    \end{gathered}
  \end{equation*}
\end{lem}
\par\bigskip
\begin{prf}[]{}
  Consider $g_k = \inf_{n\geq k} f_n$ (increasing). Then, $\lim_{k\to\infty} g_k$ exists and is equal to $\lim_{k\to\infty}\inf f_n$\par
  \noindent Since $g_k$ is monotonly increasing, we have that $g_k\to\lim_{n\to\infty}\inf f_n$, we can use the Monotone Conversion Theorem (MCT) to take the limit out
  \begin{equation*}
    \begin{gathered}
      \mu(\lim_{k\to\infty}g_k) = \mu(\lim_{n\to\infty}\inf f_n) = \lim_{k\to\infty}\mu(g_k)
    \end{gathered}
  \end{equation*}\par
  \noindent We have $g_k\leq f_n\quad\forall n\geq k$, we can use the monotonicity
  \begin{equation*}
    \begin{gathered}
      \mu(g_k)\leq \mu(f_n)\quad\forall n\geq k \Rightarrow \mu(g_k)\leq \inf_{n\geq k}\mu(f_n)
    \end{gathered}
  \end{equation*}\par
  \noindent Substitution yields
  \begin{equation*}
    \begin{gathered}
      \mu(\lim\inf f_n)\leq \lim_{k\to\infty}\inf_{n\geq k} \mu(f_n) = \lim_{n\to\infty}\inf\mu(f_n)
    \end{gathered}
  \end{equation*}
\end{prf}
\par\bigskip
\noindent\textbf{Corollary:}\par
\noindent We can say something about $\lim\sup$, suppose $f_n\leq g$ for $f_n,g\in m\Sigma^+$, then
\begin{equation*}
  \begin{gathered}
    \mu(\lim_{n\to\infty}\sup f_n)\geq \lim_{n\to\infty}\sup \mu(f_n)
  \end{gathered}
\end{equation*}
\par\bigskip
\begin{prf}[]{}
  Apply Fatous lemma with some $h_n = g-f_n$. The sign flips things arround and yields what we want
\end{prf}
\par\bigskip
\begin{theo}[Dominated Convergance Theorem]{}
  Let $f_n$ be a sequence of measurable functions $\&$ assume $\left|f_n\right|\leq g$ for some $g\in L^1$. If $f_n\to f$ pointwise, then:\par
  \begin{itemize}
    \item $\mu(\left|f_n-f\right|) = \int \left|f_n-f\right|d\mu\to0$
    \item $\mu(f_n) = \int f_n d\mu\to\int f d\mu = \mu(f)$
  \end{itemize}
\end{theo}
\par\bigskip
\begin{prf}[Dominated Convergance Theorem]{}
  We have
  \begin{equation*}
    \begin{gathered}
      \left|f_n-f\right|\leq \left|f_n\right|+\left|f\right|\leq 2g
    \end{gathered}
  \end{equation*}\par
  \noindent Since $\left|f_n\right|\leq g$ and $\left|f_n\right|\to \left|f_n\right|$ pointwise.\par
  \noindent By reverse Fatous lemma
  \begin{equation*}
    \begin{gathered}
      \lim_{n\to\infty}\sup\mu\left(\left|f_n-f\right|\right)\leq\mu\left(\lim_{n\to\infty\sup}\left|f_n-f\right|\right)\Rightarrow \lim_{n\to\infty}\sup\mu\left(\left|f_n-f\right|\right)=0
    \end{gathered}
  \end{equation*}
  \par\bigskip
  \noindent It goes to 0 $\forall n$, so $\lim_{n\to\infty}\inf\mu\left(\left|f_n-f\right|\right)\leq0$, so $\lim_{n\to\infty}\mu\left(\left|f_n-f\right|\right)=0$\par
  \noindent So $\left|\mu(f_n)-\mu(f)\right|$, by linearity of the integral:
  \begin{equation*}
    \begin{gathered}
      \left|\mu(f_n-f)\right|\leq\mu\left(\left|f_n-f\right|\right)\to0\text{ as } n\to\infty\\
      \Rightarrow \mu(f_n)\to\mu(f)
    \end{gathered}
  \end{equation*}
\end{prf}
\par\bigskip
\begin{lem}[Scheffes lemma]{}
  Suppose $f_n,f$ are non-negative and $f_n\to f$ for almost any $s\in S$, then
  \begin{equation*}
    \begin{gathered}
      \mu(f_n)\to \mu(f)\text{ as } n\to\infty \Lrarr \mu(\left|f_n-f\right|)\to0\text{ as } n\to\infty
    \end{gathered}
  \end{equation*}
\end{lem}
\par\bigskip
\noindent Recall how we defined $\int f d\mu$. We started with indicator functions, then linear combinations of them, then we defined for non-negative functions by taking supremum. Then for negative functions, we looked at their positive parts and their negative part.\par
\noindent THe idea now is we want to measure one function with respect to another function.
