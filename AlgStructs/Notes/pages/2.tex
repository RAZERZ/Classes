\section{Algorithm Analysis}
\subsection{Time complexity}\hfill\\
\par\bigskip
\noindent It takes a lot longer \textit{time} for an algorithm to solve a given problem.\par
\begin{itemize}
  \item Size of the input
  \item Value of the input (could be the total number of bits in the 2 integers)
\end{itemize}
\par\bigskip
\noindent The \textit{input size}  depends on the problem being studied:\par
\begin{itemize}
  \item The number of items in the input such as the size of the array being sorted
  \item Could be something else (bits)
  \item Could be described by more than one number (graph algorithms are expressed in terms of vertices and edges in input graph) 
\end{itemize}
\par\bigskip
\noindent\textbf{Find runtime}\par
\textit{Experimentation}: Run a clock, run algorithm, stop clock\par
Issues arise since it depends on the input, programming language, environment, etc. We need to ensure that we test on "difficult" examples
\par\bigskip
\textit{Mathematically}: Using reasoning \& logic to give an estimation of the algorithm runtinme in terms of the size of the input.\par
Does not depend on OS, CPU, etc. 
\par\bigskip
\textbf{Does complexity matter?}\par
\noindent Recall the traveling salesperson problem (TSP). Assume we have an algorithm than enumerates all routes and chooses the shortest one. This is a natural occurring problem and can be adapted to a bunch of other industries.
\par\bigskip
\noindent For $n$ cities to visit, there are $n!$ possible routes.
\par\bigskip
\noindent\textbf{Performing Time Complexity Analysis}\par
\noindent Based on an abstract model of computation (\textit{Random Access Machine} (mental model ofa computer))\par
Instructions are executed one after nother (no concurrency)\par
ELementary instructions can be perfomed in constant time (does not depend on the size of arguments)\par
\begin{itemize}
  \item Arithmetical operations
  \item Assignment operations, access to array elements
  \item Control operations such as branching (if), loops
\end{itemize}
\par\bigskip
\begin{theo}[Runtime of algorithm]{}
  The runtime (running time)
  \begin{equation*}
    \begin{gathered}
      \sum_{\text{all eLementary operations}}(\text{cost of operations})\cdot(\text{times operations is executed})
    \end{gathered}
  \end{equation*}
\end{theo}
