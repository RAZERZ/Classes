\section{Discriminant Analysis and Classification}
\begin{figure}[ht!]
    \centering
    \begin{tikzpicture}
      \node[state](p0){ECM};
      \node[state, below of=p0, yshift=-2cm, xshift=-2cm](p1){Max Likelihood};
      \node[state, right of=p1, xshift=2cm](p2){min TPM};
      \node[state, right of=p2, xshift=2cm](p3){Naive Bayes};
      \node[state, below of=p3, yshift=-2cm](p4){Gaussian DA};
      \node[state, left of=p4, xshift=-2cm](p5){$\Sigma_1 =\Sigma_2 = \text{ LDA}$};
      \node[state, below of=p5](p6){$\Sigma_1\neq\Sigma_2=\text{ QDA}$};
      \path[-stealth] (p0) edge[] node{} (p1);
      \path[-stealth] (p0) edge[] node{} (p2);
      \path[-stealth] (p0) edge[] node{} (p3);
      \path[-stealth] (p3) edge[] node{} (p4);
    \end{tikzpicture}
    \caption{}
\end{figure}
\subsection{Slide 3 - Motivation}\hfill\\\par
\noindent Discrimination here means to separate subjects.\par
\noindent \textit{Use labeled observations to build a classification rule}, means using previous data.
\par\bigskip
\noindent Classes are well defined, that is they are fixed in some sense. Having two classes such as cat and trying to classify a picture of a horse will only lead to the picture of the horse being classified into either cat or dog. It will not "create a new class" just because the data does not match. It will find which class it matches the best in, and sort into there.
\par\bigskip
\subsection{Slide 4 - Two-class problem}\hfill\\\par
\noindent Dividing $\Omega$ into two disjoint sets $R_1$ and $R_2$ is the discrimination step.
\par\bigskip
\noindent \textit{Probabilistic = } vague group belonging\par
\noindent\textit{Deterministic = } you are either in group $A$ or group $B$
\par\bigskip
\subsection{Slide 6 - Classification Table}\hfill\\\par
\noindent $m_{ij}$ where $j = $ observed, and $i = $ predicted
\par\bigskip
\noindent There is another thing we can compute, accuracy = $\dfrac{m_{11}+m_{22}}{m_{11}+m_{12}+m_{21}+m_{22}}$
\par\bigskip
\noindent Note that using absolute rates, ie $\dfrac{m_{12}+m_{21}}{m_{11}+m_{22}}$ does not take into account he cost of misclassification, it cares more about maximizing $m_{12}+m_{21}$ 
\par\bigskip
\subsection{Slide 7 - F-Score of Binary Classification}\hfill\\\par
\noindent If F-score $>0.5$, then we are goood (DIY rule of thumb)
\par\bigskip
\subsection{Slide 8 - Cost of Misclassification}\hfill\\\par
\noindent ECM $ = \underbrace{c(2|1)}_{\text{cost of misclas.}}\underbrace{P(2|1)}_{\text{prob. of misclas.}}p_1+\cdots$
\par\bigskip
\noindent Bayesian means "I have some knowledge, I get new data, I update my previous knowledge"
\par\bigskip
\subsection{Slide 9 - Minimizing ECM}\hfill\\\par
\noindent Intuitive idea: If it costs a lot to misclass into class 1, then we "want" to shit our mistake-making into the other class (ie shift such that whenever we misclass it is a higher probability that we misclass into the "less costly" class).
\par\bigskip
\subsection{Slide 10 - An Example Using ECM}\hfill\\\par
\begin{equation*}
  \begin{gathered}
    \pi_1: X\sim N(0,\Sigma)\qquad p_1 = 0.8\qquad c(2|1) = 5\\
    \pi_2: X\sim N(\mu,\Sigma)\qquad p_2 = 0.2\qquad c(1|2) = 10
  \end{gathered}
\end{equation*}\par
\noindent We now have everything we need to compute:
\begin{equation*}
  \begin{gathered}
    \dfrac{c(1|2)p_2}{c(2|1)p_1} = \dfrac{10\cdot 0.2}{5\cdot 0.8} = \dfrac{1}{2}\\
    R_1: \dfrac{f_1(x)}{f_2(x)} = \dfrac{\dfrac{1}{(2\pi)^{p/2}\left|\Sigma\right|^{1/2}}\exp{\left(-\dfrac{1}{2}x^T\Sigma^{-1}x\right)}}{\dfrac{1}{(2\pi)^{p/2}\left|\Sigma\right|^{1/2}}\exp{\left(-\dfrac{1}{2}(x-\mu)^T\Sigma^{-1}(x-\mu)\right)}}>\dfrac{1}{2}
  \end{gathered}
\end{equation*}
\par\bigskip
\noindent The cost is subject to your own judhement, can be biased to you/the one who decides what the cost of misclassification is. 
\par\bigskip
\subsection{Slide 13 - Special Case III: Highest Posterior Probability}\hfill\\\par
\noindent Assigning class depending on posterior probability.\par
\begin{itemize}
  \item Prior: $P(\pi_i)$
  \item Posterior: $P(\pi_i\underbrace{x}_{\text{data}})$
\end{itemize}\par
\noindent After reading mail (spam not spam mail problem), I may update my knowledge of the probabilities of belonging to each class.\par
\noindent Naive Bayes does this, after updating probabilities it adds into the class with highest probability.
\par\bigskip
\subsection{Slide 14 - Niave Bayes: Gaussian Discriminant Analysis}\hfill\\\par
\noindent If we do not know something, either estimate or use your own judgement.
\par\bigskip
\noindent Log-likelihood given by:
\begin{equation*}
  \begin{gathered}
    \ell = \sum_{j}^{n}Z_j\left[\ln{\left(\phi\right)}-\dfrac{p}{2}\ln{\left(2\pi\right)}-\dfrac{1}{2}\ln{\left(\left|\Sigma\right|\right)}-\dfrac{1}{2}(x_j-\mu_1)^T\Sigma^{-1}(x_j-\mu_1)\right]+\\\left(1-Z_j\right)\left[\ln{\left(1-\phi\right)}-\dfrac{p}{2}\ln{\left(2\pi\right)}-\dfrac{1}{2}\ln{\left(\left|\Sigma\right|\right)}-\dfrac{1}{2}(x_j-\mu_2)\Sigma^{-1}(x_j-\mu_2)\right]
  \end{gathered}
\end{equation*}\par
\noindent Where $\widehat{\phi} = \dfrac{\text{numbers of 1}}{n} = \widehat{p_1}\Rightarrow \widehat{p_2} = 1-\phi = \dfrac{\text{number of 2}}{n}$
\par\bigskip
\noindent MLE: In order to optimize (find $\widehat{\mu_1}$), we can use the optimization lemma from chapter 4 (slide 24):
\begin{equation*}
  \begin{gathered}
    \begin{rcases*}
      \widehat{\mu_1} = \dfrac{\Sigma \overbrace{Z_j}^{\text{if in group 2, $Z = 0$, so no contrib.}}x_j}{\Sigma Z_j = \dfrac{1}{n}\text{ part}}
    \end{rcases*}\text{ average of group 1}\\
    \begin{rcases*}
      \widehat{\mu_2} = \dfrac{\Sigma(1-Z_j)x_j}{\Sigma(1-Z_j)}
    \end{rcases*} \text{ average of group 2}
  \end{gathered}
\end{equation*}
\par\bigskip
\noindent Only thing we do not know is $\Sigma$, but we use the same lemma from chapter 4.
\par\bigskip
\noindent\textbf{Anmärkning:}\par
\noindent Quadratics are the same trace!
\begin{equation*}
  \begin{gathered}
    (x_j-\mu_1)^T\Sigma^{-1}(x_j-\mu_1) = \text{tr}\left(\Sigma^{-1}(x_j-\mu_1)(x_j-\mu_1)^T\right)\\
    \Rightarrow -\dfrac{1}{2}\left[\Sigma Z_j+\Sigma(1-Z_j)\right]\ln{\left(\left|\Sigma\right|\right)}-\dfrac{1}{2}\Sigma\underbrace{\text{tr}\left(\Sigma^{-1}Z_j(x_j-\mu_1)(x_j-\mu_1)^T + (1-Z_j)(x_2-\mu_2)(x_2-\mu_2)^T\right)}_{\underbrace{=-\dfrac{1}{2}\text{tr}\left(\Sigma\Sigma^{-1}\left(Z_j(x_j-\mu_1)\cdots\right)\right)}_{=A}}\\
    q = \Sigma Z_j+\Sigma (1-Z_j)
  \end{gathered}
\end{equation*}
\par\bigskip

\begin{equation*}
  \begin{gathered}
    P(Z = 1) = \widehat{p_1}\qquad X|Z = 1\sim N(\widehat{\mu_1},\widehat{\Sigma})
    P(Z = 0) = \widehat{p_2}\qquad X|Z = 0\sim N(\widehat{\mu_2},\widehat{\Sigma})\\
    \Rightarrow P(Z|x_0) = \dfrac{P(x_0|Z)P(Z)}{\sum_{Z}P(x_0|Z)P(Z)}
  \end{gathered}
\end{equation*}
\par\bigskip
\noindent Given you have cancer, you will observe $X$. Logistic regression is opposite, given $X$ determine if the patient has cancer. 
\par\bigskip
\subsection{Slide 15 - Gaussian Discriminant Analysis: Classification}\hfill\\\par
\noindent The equivalence is same as slide 16, by taking the logarithm
\par\bigskip
\noindent\textbf{Anmärkning:}\par
\noindent Gaussian $\Rightarrow$ only for normally distributed
\par\bigskip
\subsection{Slide 16 - Gaussian Discriminant Analysis: Decision Boundary}\hfill\\\par
\noindent Changing $\geq$ to $=$ yields your \textit{decision boundary}
\par\bigskip
\subsection{Slide 17 - Fishers Linear Discriminant Analysis}\hfill\\\par
\noindent Assuming $\Sigma_1 = \Sigma_2 = \Sigma$ yields linearity. Gaussian discriminant analysis is a type of ECM. Fisher ECM works through projections instead.
\par\bigskip
\subsection{Slide 18 - MANOVA-Like Idea}\hfill\\\par
\noindent $W = $ within group variation\par
\noindent $B = $ between group variation.
\par\bigskip
\noindent Want things in the "within" set to be as close to each other, ie small $a^TWa$ and big $a^TBa$ 
\par\bigskip
\subsection{Slide 19 - Within Versus Between Variation}\hfill\\\par
\noindent Note that we have not made any assumptions on the distribution.
\par\bigskip
\subsection{Slide 20 - Fishers LDA}\hfill\\\par
\noindent Idea is the same as Guassiam, instead of $\Sigma$, we use $W$ and we have "mean - mean"$\Sigma$ (or $W$)
\par\bigskip
\subsection{Slide 22 - Case I: $\Sigma_1 = \Sigma_2 = \Sigma$}\hfill\\\par
\noindent The $-\dfrac{1}{2}(\mu_1-\mu_2)^T$ looks similar to $\dfrac{1}{2}(\overline{x_1}-\overline{x_2})^T$
\par\bigskip
\subsection{Slide 24 - Connection to Fishers LDA}\hfill\\\par
\noindent Normality yields robust results, but it can be from other distributions and not necessarily normal.
\par\bigskip
\subsection{Slide 26 - Case II: $\Sigma_1\neq\Sigma_2$}\hfill\\\par
\noindent QDA allows $\Sigma$ to be different, contrary to LDA where it has to be the same.
\par\bigskip
\subsection{Slide 27 - Logistic Model For Two Populations}\hfill\\\par
\noindent Probabilistic model of $X$ to momdel probability
\par\bigskip
\subsection{Slide 28 - Maximum Likelihood Estimator}\hfill\\\par
\noindent For Bernoulli! No close method for finding $\alpha$ and $\beta$. We can either guess or use some numerical method to find it/approximate it.
\par\bigskip
\subsection{Slide 29 - Penalized Logistic Regression}\hfill\\\par
\noindent Think of $\alpha$ like the intercept and $\beta$ like the slope coefficients \par
\noindent Maximizing likelihood function is the same as minimizing the likelihood function after you multiply it by -1.
\par\bigskip
\noindent\underline{SPLINE} approach (yields continuous and differential function)\par
\noindent Recall noise in data, what can happen to complicated models? Well they would end up modelling the noise in their attempt to fit the curve to the data-points.
\par\bigskip
\noindent\underline{Doupple descent} phenomenon
\par\bigskip
\noindent\textbf{Example:}
\begin{equation*}
  \begin{gathered}
    -\ell(\alpha,\beta)+\lambda[\left|\beta_1\right|+\left|\beta_2\right|]\Lrarr \max\left(\ell(\alpha,\beta)\right) \text{ such that } \left|\beta_1\right|+\left|\beta_2\right| = t
  \end{gathered}
\end{equation*}\par
\noindent Here $t$ is some number. Tuning $t$ tunes $\lambda$. The above example is an example of LASSO. Below is an example of Ridge:
\begin{equation*}
  \begin{gathered}
    \beta_1^2+\beta_2^2
  \end{gathered}
\end{equation*}\par
\noindent One way to tune $\lambda$ is by cross validation. 3-fold cross validation works in the following way:\par
\begin{itemize}
  \item Find a sequence of lambdas beforehand, testing will yield which $\lambda$ to pick ($\underbrace{\lambda = 0,\cdots,\lambda_{\text{max}}}_{\text{somewhere in here}}$)
  \item Split data into 3 disjoint sets
  \item Use the first 2 parts to predict 3
  \item Tune accordingly
  \item Repeat previous 2 steps $\begin{pmatrix}3\\2\end{pmatrix}$ times
\end{itemize}
\par\bigskip
\noindent\textbf{Anmärkning:}\par
\noindent 10-fold method is the most popular. 
\par\bigskip
\subsection{Slide 30 - Limitation}\hfill\\\par
\noindent Kind of like how Newtons method works to find 0:es of a function
\par\bigskip
\subsection{Slide 31 - Euclidean Inner Product}\hfill\\\par
\noindent We replace the estimate $\widehat{\beta}$ with $\Sigma d_jx_j$ 
\par\bigskip
\subsection{Slide 32 - New Features}\hfill\\\par
\noindent Here \textit{features}  = covariates. $\delta(x)$ includes "old" features:
\begin{equation*}
  \begin{gathered}
    \begin{bmatrix}x_1\\x_2\end{bmatrix}\stackrel{\delta}{\Rightarrow}\delta\left(\begin{bmatrix}x_1\\x_2\end{bmatrix}\right) = \begin{bmatrix}x_1\\x_2\\x_1^2\\x_2x_1\\x_2^2\\x_1^3\end{bmatrix}
  \end{gathered}
\end{equation*}
\par\bigskip
\subsection{Slide 33 - Kernel Function and Kernel Matrix}\hfill\\\par
\noindent The output of the function $\kappa$  is a scalar.
\par\bigskip
\noindent $\delta^T(x)\delta(z) = $ Euclidean inner product of $\delta(x)$ and $\delta(z)$
\par\bigskip
\subsection{Slide 38 - Motivation: Margin}\hfill\\\par
\noindent Idea of a \textit{support vector machine}; if we have multiple ways to pick our decision line, this will give us the best lines.
\par\bigskip
\noindent $Y $ here is some binary set.
\par\bigskip
\subsection{Slide 40 - Hinge Loss: Brief Intro}\hfill\\\par
\noindent Term after hinge loss is the Ridge penalty
\par\bigskip
\subsection{Slide 41 - Hinge Loss Vs Log-Likelihood Loss}\hfill\\\par
\noindent You always lose some with logistic regression. With Hinge loss we will lose more because of how the curve is. 
\par\bigskip
\noindent\textbf{Anmärkning:}\par
\noindent In the R code, $\gamma$ is the inverse of $\Sigma$
\par\bigskip
\subsection{Slide 45 - Best Partition}\hfill\\\par
\noindent $\widehat{p}_{mk}$ is the proportion of say $x_1$ in that region.
\par\bigskip
\noindent\textit{Purity} is about how red is in the black area etc. A very pure area is homogenous and has a small Gini index.
\par\bigskip
\subsection{Slide 47 - A Tree Versus A Forest}\hfill\\\par
\noindent $B = $ the number of trees in the forest.
\par\bigskip
\noindent We cannot mimick the population due to biases, but we can see the data as our population instead and pick out a subset of our data and pretend we are picking out a subset of our population 
\par\bigskip
\noindent Line 3 is the bootstrapping method. 
\par\bigskip
\subsection{Slide 56 - Linear Discriminants}\hfill\\\par
\noindent $x_0$ is our new variable $\Rightarrow$ still linear
