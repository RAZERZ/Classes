\section{Factor Analysis}
\subsection{Slide 3 - Latent Variable Modelling}\hfill\\\par
\noindent LVM = factor analysis. Find values such as personality using a proxy. Personality is the factor/latent variable.
\par\bigskip
\noindent In PCA, we had a bunch of values and we wanted to simplify them and keep them as concise as possible while still retaining as much of the infomration as possible. In factor analysis, we go the other direction, we have some "simplified" data set and we want to draw more conclusions from this.
\par\bigskip
\subsection{Slide 4 - The Model}\hfill\\\par
\noindent $i = i$th question in questionaire
\begin{equation*}
  \begin{gathered}
    \underbrace{X_i}_{\text{Math-score/what you know}} = \mu_i + \ell_{i1}\underbrace{F_1}_{\text{Algebra ability/what you want to predict}}+\cdots+\varepsilon \rightarrow \text{ math ability still there, but $\varepsilon$ may be lack of sleep. If the $\varepsilon$ is convoluted, then you have not captured everything in your questionaire}
  \end{gathered}
\end{equation*}
\par\bigskip
\noindent Depending on application, sometimes we need to find $\ell$.
\par\bigskip
\noindent Tasks is usually lower than factors.
\par\bigskip
\noindent In multiple regression we had $X = X\beta + \varepsilon$, but this is for all people/subjects, while $X = \mu+\ell F+\varepsilon$  is for 1 person/subject, like in multivariate multiple regression model, difference is $Y_i = \beta^T\underbrace{X_i}_{\text{observed}}+\varepsilon$ 
\par\bigskip
\noindent A regressor is values you do not really observe such as IQ, but you still want to build a model using observations.
\par\bigskip
\subsection{Slide 5 - Scale Indeterminacy}\hfill\\\par
\noindent We can redefine scale, continuing with the IQ example, there really is nothing stopping us from saying that the IQ scale should lay inbetween [-1,1] by just compressing the Gaussian.
\par\bigskip
\subsection{Slide 7 - Model Implied Covariance Matrix}\hfill\\
\begin{equation*}
  \begin{gathered}
    \text{Cov}\left(LF+e\right) = \text{Cov}\left(LF,LF\right)+\underbrace{\text{Cov}\left(LF,e\right)}_{L\text{Var}\left(F\right)L^T+\text{Var}\left(e\right) = LL^T+\psi = \text{Cov}\left(X\right)}+\underbrace{\text{Cov}\left(e,LF\right)}_{\text{$=0$}}+\text{Cov}\left(e,e\right)
  \end{gathered}
\end{equation*}
\par\bigskip
\noindent The reason for the name model implied covariance matrix, is because it is implied by the setup of the model. 
\par\bigskip
\noindent\textbf{Example:}
\begin{equation*}
  \begin{gathered}
    \underbrace{\begin{bmatrix}l_{11}&l_{12}\\l_{21}&l_{22}\\l_{31}&l_{32}\end{bmatrix}}_{\text{$L$}}\underbrace{\begin{bmatrix}l_{11}&l_{21}&l_{31}\\l_{12}&l_{22}&l_{32}\end{bmatrix}}_{\text{$L^T$}} + \begin{bmatrix}\psi_1&\quad&\quad\\\quad&\psi_2&\quad\\\quad&\quad&\psi_3\end{bmatrix}\\
    = \begin{bmatrix}l_{11}^2+l_{12}^2+\psi_1&0&0\\0&l_{21}^2+l_{22}^2+\psi_2&0\\0&0&l_{31}^2+l_{32}^2+\psi_3\end{bmatrix}
  \end{gathered}
\end{equation*}
\par\bigskip
\noindent Here  $l_{11}^2+l_{12}^2+\psi_1 = \text{Var}\left(X_1\right)$, $l_{21}^2+l_{22}^2+\psi_2 = \text{Var}\left(X_2\right)$, $l_{31}^2+l_{32}^2+\psi_3 = \text{Var}\left(X_3\right)$
\noindent Want communality over uniqueness
\par\bigskip
\subsection{Slide 8 - Existence of Decomposition}\hfill\\
\par\bigskip
\noindent Note that the invalid decomposition is invalid due to negative $\psi$ , $\text{Var}\left(e\right)$ not as assumption.
\par\bigskip
\noindent\textbf{Anmärkning:} PCA always doable, not the same with factor analysis. 
\par\bigskip
\subsection{Slide 9 - Indeterminacy}\hfill\\\par
\noindent Rotation (multiplication by diagonal matrix $T$) is invariant $\Rightarrow$ indeterminate
\par\bigskip
\subsection{Slide 10 - Scale Invariant}\hfill\\\par
\noindent\textbf{Curiosity:} What happens if $c\mu$ or $cz$?
\par\bigskip
\subsection{Slide 11 - Popular Estimation Methods}\hfill\\\par
\noindent\textbf{Curiosity:} What if only one person in our sample has $p+1\times 1$?
\par\bigskip
\subsection{Slide 12 - Spectral Decomposition}\hfill\\\par
\noindent Recurring small eigenvalues yields a possibility to find $\psi$ such that $\psi$ is diagonal.
\par\bigskip
\subsection{Slide 14 - Determine Number of Factors}\hfill\\\par
\noindent A factor/factors = things we want to test, while $m$ is the number of things we want to test. Say for example we use the multivariate analysis exam as an example. It might test our knowledge in multivariate analysis ($m=1$), but might also in addition test our R knowledge ($m=2$) 
\par\bigskip
\noindent Kaiser criterion sucks.
\par\bigskip
\subsection{Inbetween lectures}\hfill\\
\noindent We want the RMSEA index $<0.1$. Or rank of the two sample test $>3$ = bad?
\par\bigskip
\subsection{Slide 23 - Versus PCA}\hfill\\\par
\noindent Notice that they are very similar. I can do PCA to solve FA.
\par\bigskip
\subsection{Slide 24 - Orthogonal Rotation}\hfill\\\par
\noindent Multiplication will not change anything.
\par\bigskip
\subsection{Slide 26 - Oblique Rotation}\hfill\\\par
\noindent All factors cannot allways be orthogonal. For example, say we test English knowledge, then spelling will have some correlation to writing $\rightarrow$ oblique.
\par\bigskip
\noindent If we have the following correlation matrix $\begin{bmatrix}1&0.9\\0.9&1\end{bmatrix}$, then 0.9 (since it is too close to 1), might be the same so we could reduce 1 factor.
\par\bigskip
\subsection{Slide 28 - Bartlett Score}\hfill\\\par
\noindent A factor score is the estimate of $F$
