\section{Introduction}
\par\bigskip
\noindent Analysis dealing with simultaneous measurements on many variables. 
\par\bigskip
\noindent We may want to do some statistical analysis on not only salary, but factor in things such as gender, wether or not one has been to uni etc.
\par\bigskip
\noindent One should always stride to use as much information as possible, you want to remove any chance to miss a pattern.
\par\bigskip
\noindent In general, if you arrive to a conclusion, think of why/what caused this and factor everything in your data and analysis. 
\par\bigskip
\subsection{MANOVA}\hfill\\\par
\noindent MANOVA is a method to measure if a data-set shares a similar mean. For example, with different flower types we may want to check if "does sweden has a similar income as norwegian citizens", comparing the sample from sweden to norwegian. We will get different numbers but that is something that we take into analysis.
\par\bigskip
\subsection{Regressionanalysis}\hfill\\\par
\noindent Allows us to predict a variable $y$ from an observation $x$. $x= $ bmi, while $y$ is your blood pressure. 
