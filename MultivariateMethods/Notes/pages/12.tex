\section{Old exam - training exam}
\noindent\textbf{Anmärkning:} There are certain questions that have come in the homeworks, see respective homework solution. 
\par\bigskip

\begin{enumerate}[label=\arabic*., leftmargin=*]
  \setcounter{enumi}{3}
  \item
  \begin{enumerate}[label=\alph*),leftmargin=*]
  \item The test problem that has been considered here is if the mean is $\begin{bmatrix}60\\29\\60\\1000\end{bmatrix}$. This is a one sample $T^2$ test where we assume our data is i.i.d and normal.
    \par\bigskip
  \item We know that we can express the $T^2$ distribution using the $F$ distribution throught the follwing relationship:
    \begin{equation*}
      \begin{gathered}
        T^2(p,n-1)\\sigmaim \dfrac{(n-1)p}{n-p}F_{p,n-p}\qquad 
        \begin{rcases*}
          n-p = 199\\p=4
        \end{rcases*}n = 403
      \end{gathered}
    \end{equation*}
    \par\bigskip
  \item Yes they can if we look invidually for each $X_i$ , no if they are joint.
    \par\bigskip
  \item We are not testing any interaction between the data points, only if they have the same value. The MANOVA model is therefore specified through:
    \begin{equation*}
      \begin{gathered}
        X_{ij} = \mathbf{\mu}+\alpha_i+\varepsilon_{ij}\qquad H_0:\alpha_i = 0\quad\forall i
      \end{gathered}
    \end{equation*}
    \par\bigskip
  \item Our assumptions correspond to the normal MANOVA assumptions, that is $\varepsilon_{ij}$ is multivariate normal, data is independent across stations.
    \par\bigskip
  \item We can use 2 methods, either the elbow method (subjective) or we fix an acceptable percentage of explained variance such as 95\%.
    \par\bigskip
  \item We sort our $n$ eigenvalues, and then compute the $k$th component by: $\dfrac{\sum_{j=1}^{k}\lambda_j}{\sum_{i=1}^{n}\lambda_i}$\par
    \noindent In our case, this yields $\dfrac{3}{3+0.6+0.3+0.1} = 75\%$
    \par\bigskip
  \item
    \begin{equation*}
      \begin{gathered}
        X = \mu+LF+\varepsilon\qquad \text{Cov}\left(X\right) = LL^T+\Psi
      \end{gathered}
    \end{equation*}
    \par\bigskip
    \noindent Under rotation, this becomes:
    \begin{equation*}
      \begin{gathered}
        L\mapsto LT
        \Rightarrow (LT)(LT)^T+\Psi = L^TT^TL^T+\Psi
      \end{gathered}
    \end{equation*}
    \par\bigskip
    \noindent We can see here that the matrix $T$ did not disappear, so the communalitites after rotations are not the same as the communalitites before rotation they need to be rotated as well. 
    \par\bigskip
  \item PCA is dimension reduction while FA is environmental/causal stuff.
  \end{enumerate}
  \par\bigskip
  \item 
  \begin{enumerate}[label=\alph*),leftmargin=*]
    \item $Y$-axis groups distance, so if three clusters are desired then we pick those top 3 highest $Y$ values $\Rightarrow 29,6,12$
      \par\bigskip
    \item Using the max CH-index, we get 3 clusters.
      \par\bigskip
    \item In $k$-means, the data needs to be standardized and the nstart is too low otherwise it would be unstable.
  \end{enumerate}
\end{enumerate}
