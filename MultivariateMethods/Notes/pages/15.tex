\section{Clustering}
Using 1 model to describe your data is not good, clustering and modelling per cluster = much better
\par\bigskip
\subsection{Slide 3 - Applications of Multivariate Analysis: Clustering}\hfill\\
\noindent We don't know which group it comes from, as opposed to classification.\par
\noindent The first step is to guess how many groups we have to classify our data into.
\par\bigskip
\subsection{Slide 5 - Distance for Items/Observations}\hfill\\
Need to define a metric of how "different" something is.
\par\bigskip
\subsection{Slide 6 - Clustering Algorithms}\hfill\\
\begin{enumerate}[leftmargin=*]
  \item Forward search (start chaos (lots of groups) $\rightarrow$ purify)
    \par\bigskip
  \item Backwards (start with 1 group $\rightarrow$ chop into pieces)
\end{enumerate}
\par\bigskip
\subsection{Slide 7 - Linkage Methods}\hfill\\
\noindent With forward search we will always end up with 1 cluster. You will be needing to trim the search to your liking. It is exploratory.
\par\bigskip
\subsection{Slide 9 - An example}\hfill\\
\noindent Note that the matrix is symmetric since metrics are symmetric.\par
\noindent Distance between cluster (12) and (3) and (4) is smallest.
\par\bigskip
\noindent\textbf{Anmärkning:}\par
\noindent For average linkage, we only look for distance between 2 clusters.
\par\bigskip
\subsection{Slide 16 - Dendrogram}\hfill\\
\noindent Height tells us ordering of joining clusters
\par\bigskip
\subsection{Slide 18 - Wards Hierarchical Clustering Method}\hfill\\
\noindent At every step you loose something. This tries to account for sum of squares.\par
\noindent Large sum of squares = distance is large
\par\bigskip
\subsection{Slide 20 - Pros and Cons}\hfill\\
\noindent Monotone in the sense $x\mapsto x^2, x\mapsto x^3$
\par\bigskip
\subsection{Slide 27 - CH Index and Gap Statistic}\hfill\\
\noindent $W(K)$, take euclidean distance to center and sum. Minimized when each point in its own cluster.
\par\bigskip
\noindent\textbf{Idea:} \textit{Difference of clusters} \par
\noindent Collect distances in vector with closest first, second closest second etc...\par
\noindent Make $B$ large, make $W$ small.
\par\bigskip
\noindent Gap Statistic: If random cluster is similar, you're not doing well. Our model should beat a purely random model.
\par\bigskip
\noindent $a_i$ small, $b_i$ large (= sparse points)
\par\bigskip
\subsection{Slide 29 - Parametric Approach: Mixing Distribution}\hfill\\
\noindent Complete is best scenario
\par\bigskip
\subsection{Slide 30 - EM Algorithm}\hfill\\
$\theta = \left\{p_k,\mu_k,\Sigma_k,\forall k\right\}$
\par\bigskip
\subsection{Slide 31 - E step}\hfill\\
\noindent Need to know expected value of $log(\underbrace{f(\overbrace{X}^{\text{given}},\overbrace{Z}^{\text{unknown}},\theta)}_{\text{complete distribution}}|X;\underbrace{\widehat{\theta}}_{\text{guessed}})$
\par\bigskip
\noindent Expectation is for $Z|X$
\begin{equation*}
  \begin{gathered}
    \E(log(\cdots)) = \sum_{k=1}^{K}\log(f(X,Z,;\theta))P(Z=k|X;\widehat{\theta}^{(t)})
  \end{gathered}
\end{equation*}\par
\noindent By Bayes: 
\begin{equation*}
  \begin{gathered}
    P(Z=k|X,\widehat{\theta}^{(t)}) = \dfrac{f(X|Z=k,\widehat{\theta}^{(t)})\widehat{P}_k^{(t)}}{\sum_{k=1}^{K}f(X|Z=k,\widehat{\theta}^{(t)})\widehat{P}_k^{(t)}}
  \end{gathered}
\end{equation*}\par
\noindent With normal assumption:
\begin{equation*}
  \begin{gathered}
    \Rightarrow \sum_{k=1}^{K}\left[\overbrace{\log(p_k)}^{Z=k}\underbrace{-\dfrac{q}{2}\log(2\pi)-\dfrac{1}{2}\log(\left|\Sigma_k\right|-\dfrac{1}{2}(x-\mu_k)^T\Sigma_k^{-1}(x-\mu_k))}_{X|Z}\right]P(Z=k|X;\widehat{\theta}^{(t)})
  \end{gathered}
\end{equation*}\par
\noindent Because we have a sample of $N$ observations, we get:
\begin{equation*}
  \begin{gathered}
    \sum_{j=1}^{N}\sum_{k=1}^{K}\log(f(x_j,Z;\theta))P(Z_j=k|x_j;\widehat{\theta}^{(t)}) = Q
  \end{gathered}
\end{equation*}\par
\noindent In order to do M step (maximize Q):
\begin{equation*}
  \begin{gathered}
    \dfrac{\partial Q}{\partial p_k} = \sum_{j=1}^{N}\left[\dfrac{1}{p_k}P(Z_j=k|x_j,\widehat{\theta}^{(t)})-\dfrac{1}{p_k}P(Z_j=k|x_j;\widehat{\theta}^{(t)})\right] = 0\\
    \Rightarrow p_k = \dfrac{\sum_{j=1}^{N}P(Z_j=k|x_j;\widehat{\theta}^{(t)})P_k}{\sum_{j=1}^{N}P(Z_j=K|x_j;\widehat{\theta}^{(t)})}\\
    \Rightarrow \dfrac{\sum_{k=1}^{K-1}\sum_{j=1}^{N}P(Z_j=k|x_j;\widehat{\theta}^{(t)})P_k}{\sum_{j=1}^{N}P(Z_j=K|x_j,\widehat{\theta}^{(t)})} = 1\\
    \Rightarrow \widehat{p}_k^{(t+1)} = \dfrac{\sum_{j=1}^{N}P(Z_j=k|x_j;\widehat{\theta}^{(t)})}{\sum_{k=1}^{K}\sum_{j=1}^{N}P(Z_j=k|x_j;\widehat{\theta}^{(t)})}\leftarrow\text{ updated value at $p_k$}
  \end{gathered}
\end{equation*}
\par\bigskip
\noindent Next parameter is $\mu_k$, but that depends on the term $-\dfrac{1}{2}(x_j-\mu_k)^T\Sigma_k^{-1}(x_j-\mu_k)\Rightarrow$ essentially want to maximize the following:
\begin{equation*}
  \begin{gathered}
    \sum_{j=1}^{N}(x_j-\mu_k)^T\Sigma_k^{-1}(x_j-\mu_k)P(Z_j=k|x;\widehat{\theta}^{(t)})
  \end{gathered}
\end{equation*}\par
\noindent Similar to Naive Bayes. We can use the lemma from Chapter 4. This yields to something complicated, but there exists a closed form expression of $\mu_k^{(t+1)}$ 
\par\bigskip
\noindent Next parameter is $\Sigma_k$, we want to maximize:
\begin{equation*}
  \begin{gathered}
    -\dfrac{N}{2}\log(\left|Z_k\right|)-\dfrac{1}{2}\sum_{j=1}^{N}(x_j-\mu_k)^T\Sigma_k^{-1}(x_j-\mu_k)P(Z_j=k|x_j;\widehat{\theta}^{(t)})
  \end{gathered}
\end{equation*}\par
\noindent to get $\Sigma_k$
\par\bigskip
\subsection{Slide 32 - M step}\hfill\\
\noindent Need to determine which cluster its from, but that is already done! Since the $P(Z=k\cdots)$ part gives probability of being in cluster $k$
\par\bigskip
\subsection{Slide 33 - EM Estimator}\hfill\\
\noindent We update using this $P(Z=k\cdots)$ probability. Like slide 15 in Gaussian DA.
\par\bigskip
\subsection{Slide 34 - Gaussian Mixture}\hfill\\
\noindent Assume $Z$ is multinomial: $P(Z_j=k) = p_k$\par
$x_j|Z_j=k\sim N_q(\mu_k,\Sigma_k)\leftarrow$ conditional only even though marginals are not normal. 
