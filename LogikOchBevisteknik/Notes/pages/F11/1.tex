\section{Semantik för 1:a ordningens logik}
\par\bigskip
\noindent Idag skall vi tolka slutna termer som ett element i strukturen. Vi ska även betrakta slutna formler och tolka dessa i strukturen.
\par\bigskip
\noindent Vi skall titta på en särskilld signatur $\sigma = <\bar{c_1},\cdots, \bar{c_n};\bar{F_1},\cdots, \bar{F_m};\bar{R_1},\cdots,\bar{R_k}>$\par\noindent av typ $<0,\cdots,0;s_1,\cdots,s_m;r_1,\cdots,r_k>$. Har vi signaturen samt typen så har vi "språket", vi vet i någon mening vad vi kan göra vi vet hur alla symboler får användas. Då kan vi bygga upp våra termer och formler (induktivt).
\par\bigskip
\noindent Vill vi tolka dessa måste vi välja en struktur som passar till detta språk. Låt därför \textgoth{A} vara en $\sigma$-struktur:
\begin{equation*}
  \begin{gathered}
    \textgoth{A} = <A; c_1\cdots,c_n;F_1,\cdots,F_m;R_1,\cdots,R_k> \text{ ($A$ är strutkurens universum)}
  \end{gathered}
\end{equation*}\par
\noindent Se på det som att $\sigma$ är definitionen av en ring, medan \textgoth{A} är en ring exempelvis $\Z$
\par\bigskip
\noindent Vi skall tolka \textbf{slutna termer }som element i $A$ och \textbf{slutna formler} som sanna eller falska i strukturen \textgoth{A}. Notera, enbart slutna, alltså om det existerar en fri variabel gör vi ingenting.
\par\bigskip
\noindent \textit{Vad behöver vi?}\par
\begin{itemize}
  \item Namn på varenda element (för att tolka kvantorer)
    \begin{itemize}
      \item Ett namn kommer representera ett element i strutkurens universum. Från definitionen av de naturliga talen är det behändigt att kunna säga "10" istället för "9 iterationer av $S$(0)"
    \end{itemize}
\end{itemize}
\par\bigskip
\noindent Vi utvecklar därmed \textit{det utvidgade språket} $L(\textgoth{A})$ med signatur $\sigma_{\textgoth{A}}$.\par
\noindent Det språket har nya konstantsymboler, vilket betyder att för varje element $a\in A$ finns det en symbol $\bar{a}$.
\par\bigskip
\noindent Exempel: $\textgoth{n} = <\N,0,S,<>$ med $\sigma = <\bar{0}, \bar{s}, \bar{<}>$ av typ $<0;1;2>$. Utvidgade språket $\sigma_{\textgoth{n}} = \sigma\cup\{\bar{n}:n\in\N\}$
\par\bigskip
\subsection{Tolkning av slutna termer i \textgoth{A}}\hfill\\
\par\bigskip
\noindent Om $t$ är en sluten term, så ska vi definiera elementet $t^{\textgoth{A}}\in A$. Man kan säga att vi ska göra en funktion "slutna-termer $\to A$" så att $t\mapsto t^{\textgoth{A}}$ 
\par\bigskip
\begin{itemize}
  \item Bas:
    \begin{itemize}
      \item $(\bar{c_i})^{\textgoth{A}} = c_1$
      \item $a\in A$\qquad $(\bar{a})^{\textgoth{A}} = a$
    \end{itemize}
  \item Induktionssteg:
    \begin{itemize}
      \item Om $\bar{F}$ är $k$-ställig funktionssymbol, och $t_1,\cdots,t_k$ är slutna termer så:
        \begin{itemize}
          \item $\bar{(F(t_1,\cdots,t_k))^{\textgoth{A}}}= F(t_1^{\textgoth{A}},\cdots,t_k^{\textgoth{A}})\in A$\qquad $\{t_1^{\textgoth{A}},\cdots, t_k^{\textgoth{A}}\}\in A$
        \end{itemize}
    \end{itemize}
\end{itemize}
\par\bigskip
\noindent Exempel: Betrakta $\textgoth{n} = <\N,0,S,<>$ med $\sigma = <\bar{0}, \bar{s}, \bar{<}>$ av typ $<0;1;2>$\par
\noindent $\sigma_{\textgoth{n}} = \{\bar{n}:n\in\N\}$:
\par\bigskip
\begin{itemize}
  \item $(\bar{0})^{\textgoth{n}}=0$
  \item $(\bar{3})^{\textgoth{n}}=3$
  \item $\bar{S}(\bar{S}(\bar{0}))^{\textgoth{n}} = S((\bar{S}(\bar{0}))^{\textgoth{n}}) = S(S(\bar{0})^{\textgoth{n}}) = S(0)+1 = 0+1+1 = 2$
\end{itemize}
\par\bigskip
\noindent Man kan ha en symbol som pekar på samma sak. 2+3 pekar på 5, men det gör även 5.
\par\bigskip
\subsection{Tolkning av slutna formler}\hfill\\
\par\bigskip
\noindent Från tidigare har vi även sagt att det kallas för sats. I vanlig matematik kan det betyda en sann sak, exempelvis "enligt sats 69", men det kallas egentligen teorem.
\par\bigskip
\noindent Slutna formler ($\tau$) skall vi tolka som sanna eller falska i strukturen (\textgoth{A}).
\par\bigskip
\noindent Notation: $\textgoth{A}\vDash\tau$, utläses som "\textgoth{A} är en modell för $\tau$" eller "$\tau$ är sann i $\textgoth{A}$".\par
\noindent Vi kan även skriva $\textgoth{A}\nvDash\tau$ som uttalas "$\tau$  är falsk i \textgoth{A}" eller även \textgoth{A} är inte en modell för $\tau$.
\par\bigskip
\noindent Observera: Vi tolkar endast \textit{slutna formler}. Vi kan konstruera en slags funktion, som vi gjorde tidigare, "slutna-formler$\to\{\text{sann i }\textgoth{A}\text{, falsk i } \textgoth{A}\}$"\par
\noindent Detta gör vi induktivt:
\par\bigskip
\begin{itemize}
  \item Bas:
    \begin{itemize}
      \item $\textgoth{A}\nvDash\perp$
      \item $t_1,t_2$ slutna termer. $\textgoth{A}\vDash t_1\dot{=}t_2 \Lrarr t_1^{\textgoth{A}}=t_2^{\textgoth{A}}$
      \item $\bar{R}$ är en $k$-ställigt relationssymbol, $t_1,\cdots, t_k$ slutna termer: $\textgoth{A}\vDash\bar{R}(t_1,\cdots, t_k) \Lrarr (t_1^{\textgoth{A},\cdots, t_k^{\textgoth{A}}})\in R$
    \end{itemize}
  \item Induktionssteg (konnektiv):
    \begin{itemize}
      \item För $\neg\varphi$: $\textgoth{A}\vDash\neg\varphi\Lrarr\textgoth{A}\nvDash\varphi$
      \item För $\varphi\vee\psi$: $\textgoth{A}\vDash\varphi\vee\psi\Lrarr \textgoth{A}\vDash\varphi$ eller $\textgoth{A}\vDash\psi$
      \item För $\varphi\wedge\psi$: $\textgoth{A}\vDash\varphi\wedge\psi\Lrarr \textgoth{A}\vDash\varphi$ och $\textgoth{A}\vDash\psi$
      \item $\varphi\rightarrow\psi$: $\textgoth{A}\nvDash\varphi\rightarrow\psi\Lrarr \textgoth{A}\vDash\varphi$ och $\textgoth{A}\nvDash\psi$
      \item För $\varphi\leftrightarrow\psi$: $\textgoth{A}\vDash\varphi\leftrightarrow\psi\Lrarr\left(\textgoth{A}\vDash\varphi\text{ och } \textgoth{A}\vDash\psi\right)\text{ eller }\left(\textgoth{A}\nvDash\varphi\text{ och }\textgoth{A}\nvDash\psi\right)$
    \end{itemize}
  \item Induktionssteg (kvantorer):
    \begin{itemize}
      \item $\tau$ är $\exists x\varphi$.\qquad $\tau$ sluten $\Rightarrow$ den enda variabeln som kan förekomma fritt i $\varphi$ är $x$. Skriver $\varphi(x)$.
      \item $\textgoth{A}\vDash\exists x\varphi(x)\Lrarr$ det finns element $a\in A$ så att $\textgoth{A}\vDash\underbrace{\varphi[\bar{a}/x]}_{\text{sluten formel}}$
      \item $\textgoth{A}\vDash\underbrace{\forall x\varphi(x)}_{\text{sluten}}\Lrarr$ för varje element $a\in A$ gäller $\textgoth{A}\vDash\underbrace{\varphi[\bar{a}/x]}_{\text{sluten}}$
    \end{itemize}
\end{itemize}
\par\bigskip
\noindent Anmärkning: Skriver ofta $\varphi(\bar{a})$ när vi menar $\varphi[\bar{a}/x]$
\par\bigskip
\noindent Exempel: Betrakta $\textgoth{n} = <\N,0,S,<>$ med $\sigma = <\bar{0}, \bar{s}, \bar{<}>$ av typ $<0;1;2>$ där $S(n) = n+1$ och < är mindre än.\par
\noindent Vi påstår att $\textgoth{n}\vDash\underbrace{\bar{<}(\bar{0},\bar{S}(\bar{0}))}_{\text{$\tau$}}$. Notera att det är en sluten formel med en relationssymbol. Kolla på den induktiva definitionen så får vi att det betyder:
\begin{equation*}
  \begin{gathered}
    \bar{0}^{\textgoth{n}}<(\bar{S}(\bar{0}))^{\textgoth{n}}\\
    \Lrarr 0 < S(0)\Lrarr 0<0+1 \Rightarrow 0<1 \text{ (sannt)}
  \end{gathered}
\end{equation*}\par
\noindent Alltså gäller att $\textgoth{n}\vDash\tau$
\par\bigskip
\noindent Exempel: Samma strutkur och signatur från tidigare exempel. Vi påstår att $\textgoth{n}\vDash\exists x\bar{<}(x,\bar{S}(\bar{0}))$.\par
\noindent Detta gäller omm det finns element $n\in\N$ så att $\textgoth{n}\vDash\bar{<}(\bar{n},\bar{S}(\bar{0}))$ (vi gjorde substitutionen direkt utan att skriva in subtitutionsoperatorn). Vi kan tolka det som:\par
\noindent Det finns ett element $n\in\N$ så att $(\bar{n})^{\textgoth{n}}<(\bar{S}(\bar{0}))^{\textgoth{n}}\Lrarr n< 1$. Detta påstående stämmer, tag $n=0$. Alltså $\textgoth{n}\vDash\tau$ 
