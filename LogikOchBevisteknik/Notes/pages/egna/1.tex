\section{Förtydligande}

\begin{theo}[Disjunktion]{thm:disjunk}
  $A$ eller $B$. Uttrycker att minst en av $A$ och $B$ är fallet. Betecknas $A\vee B$ där $A, B$ kallas \textit{disjunktionsled} eller \textit{disjunkter}.
  \par\bigskip
  \noindent Det finns 2 typer av disjunktion, \textit{uteslutande} och \textit{icke-uteslutande}. Uteslutande disjunktion är helt enkelt A eller B (men inte båda) och icke-uteslutande är motsatt.
\end{theo}
\par\bigskip

\noindent Disjunktionen är så kallad \textit{inklusiv}, det vill säga det är helt okej att både $A$ och $B$ är sann. Det finns en så kallad \textit{exklusiv} disjunktion vilket kommer lite senare (kanske $A\wedge B$?). 
\par\bigskip

\begin{theo}[Implikation]{thm:impl}
  Om $A$ så $B$ uttrycker att $B$ är fallet givet att $A$ är det. Detta betecknas $A\rightarrow B$. Här är $A$ \textit{antecendenten} eller även \textit{förledet} och $B$ är \textit{konsekventen} eller \textit{efterledet}. En implikation kallas också ivland en \textit{materiell implikation}, \textit{konditionalsats}, \textit{villkorssats}.
\end{theo}
\par\bigskip

\begin{theo}[Ekvivalens]{thm:eqv}
  $A$ omm $B$ yttrycker konjunktionen av två implikationer, det vill säga om $A$ så $B$ oh om $B$ så $A$. Skrivs $A\leftrightarrow B$ och kallas även för \textit{materiella ekvivalenser} eller \textit{bikonditionalsatser}.
\end{theo}
\par\bigskip

\begin{theo}[Molekyler och Atomär]{thm:molat}
  En sats är \textit{molekylär} om den är uppbyggd av en eller två andra satser med hjälp av ett konnektiv. I motsatt fall är satsen \textit{atomär}. En molekylär sats innehåller minst ett konnektiv.
\end{theo}
\par\bigskip

\begin{theo}[n-ställig satsoperator]{thm:noperator}
  En \textit{n-ställig satsoperator} är en operator som "tar in" $n$ variabler. Exempelvis är negationen $\neg$ en 1-ställig operator, medan $\wedge$ är en 2-ställig operator.
\end{theo}
\par\bigskip

\begin{theo}[Huvudoperator]{thm:headop}
  En \textit{huvudoperator} är den operator som har tillämpats sist i uppbyggnaden av satsen. Exempelvis, i $(A\rightarrow(B\vee\neg A))$ är $\rightarrow$ huvudoperatorn. Däremot är $\vee$ huvudoperatorn i delformen $(B\vee\neg A)$.
\end{theo}
\par\bigskip
\newpage

\subsection{Sanningstabeller}\hfill\\

\noindent För negation $\neg$:
\par\bigskip
\noindent $\neg$A är sann $\Lrarr$ A är falsk $\Lrarr$ A inte är sann.
\par\bigskip

\noindent\begin{tabular}{|c|c|}
  \hline
  A&$\neg$A\\
  \hline
  S&F\\
  \hline
  F&S\\\hline
\end{tabular}
\par\bigskip

\noindent För Konjunktion:
\par\bigskip
\noindent A$\wedge$B är sann $\Lrarr$ A är sann och B är sann.
\par\bigskip

\noindent\begin{tabular}{|c|c|c|}
  \hline
  A & B & A $\wedge$ B\\
  \hline
  S&S&S\\
  \hline
  S&F&F\\
  \hline
  F&S&F\\
  \hline
  F&F&F\\
  \hline
\end{tabular}
\par\bigskip

\noindent För disjunktion:
\par\bigskip
\noindent A$\vee$B är sann $\Lrarr$ A är sann eller B är sann $\Lrarr$ minst en av A och B är sann.
\par\bigskip

\noindent\begin{tabular}{|c|c|c|}
  \hline
  A&B&A$\vee$B\\
  \hline
  S&S&S\\
  \hline
  S&F&S\\
  \hline
  F&S&S\\
  \hline
  F&F&F\\
  \hline
\end{tabular}
\par\bigskip

\noindent För implikation:
\par\bigskip
\noindent A$\rightarrow$B är sann $\Lrarr$ om A är sann så B är sann $\Lrarr$ A är falsk eller B är sann. (Tips, vad är sista sanningsvärdet, dvs sanningsvärdet på B?)
\par\bigskip

\noindent\begin{tabular}{|c|c|c|}
  \hline
  A&B&A$\rightarrow$B\\
  \hline
  S&S&S\\
  \hline
  S&F&F\\
  \hline
  F&S&S\\
  \hline
  F&F&S\\
  \hline
\end{tabular}
\par\bigskip

\noindent För ekvivalens:
\par\bigskip
\noindent A$\leftrightarrow$B är sann $\Lrarr$ om A är sann så är B saan och om B är sann så är A sann. $\Lrarr$ A$\rightarrow$B är sann och B$\rightarrow$A är sann $\Lrarr$ A och B har samma sanningsvärde.
\par\bigskip

\noindent\begin{tabular}{|c|c|c|}
  \hline
  A&B&A$\leftrightarrow$B\\
  \hline
  S&S&S\\
  \hline
  S&F&F\\
  \hline
  F&S&F\\
  \hline
  F&F&S\\
  \hline
\end{tabular}
\par\bigskip

\begin{theo}[Satsparametrar]{thm:params}
  Antalet satsparametrar är antalet "variabler" i vår utsaga. Exvis, i $A\rightarrow(B\wedge C\leftrightarrow\neg A)$ har 3st satsparametrar.
\end{theo}








