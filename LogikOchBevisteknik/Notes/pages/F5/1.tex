\section{Logisk ekvivalens och konsekvens}

\subsection{Logisk ekvivalens}\hfill\\

\begin{theo}[Logisk ekvivalens]{thm:logeqv}
  En relation som 2 formler har till varandra, de kan kallas för \textit{logiskt ekvivalenta}.
  \par\bigskip
  \noindent Låt $\sigma$ vara en struktur, $\varphi$ och $\psi$ i LP($\sigma$). $\varphi$ och $\psi$ \textit{kallas logiskt ekvivalenta} om $\vDash\varphi\leftrightarrow\psi$
  \par\bigskip
  \noindent Notation: $\varphi$ eq $\psi$, alltså samma sanningsvärde.
\end{theo}

\subsection{Exempel}\hfill\\

\noindent $p\rightarrow$ eq $\neg p\vee q$. Här får man rita upp sanningstabellen:

\begin{center}
  \begin{tabular}{|c|c|c|c|}
    \hline
    $p$&$q$&$p\rightarrow q$&$\neg p\vee q$\\
    \hline
    1&1&1&1\\
    \hline
    1&0&0&0\\
    \hline
    0&1&1&1\\
    \hline
    0&0&1&1\\
    \hline
  \end{tabular}
\end{center}

\begin{theo}[Eq]{thm:eq}
  Eq är en ekvivalensrelation
\end{theo}
\par\bigskip

\begin{prf}[Eq]{prf:eq}
  \begin{itemize}
    \item Reflexiv ($\varphi$ eq $\varphi$)
    \item Symetrisk ($\varphi$ eq $\psi \Lrarr\psi$ eq $\varphi$)
    \item Transitiv ($\varphi$ eq $\psi$ och $\psi$ eq $\xi\Rightarrow$ $\varphi$ eq $\xi$)
  \end{itemize}
\end{prf}
\par\bigskip

\subsection{Några viktiga ekvivalenser}\hfill\\

\begin{equation*}
  \begin{gathered}
    \begin{rcases*}
      p\vee(q\wedge r)$ eq $(p\vee q)\wedge(p\vee r)\\
      p\wedge(q\vee r)$ eq $(p\wedge q)\vee(p\wedge r)
    \end{rcases*}=\text{Distributiva lagar}
  \end{gathered}
\end{equation*}
\par\bigskip

\begin{equation*}
  \begin{gathered}
    \begin{rcases*}
      p\vee(q\vee r)$ eq $(p\vee q)\vee r\\
      p\wedge(q\wedge r)$ eq $(p\wedge q)\wedge r
    \end{rcases*}=\text{Associativa lagar}
  \end{gathered}
\end{equation*}
\par\bigskip

\begin{equation*}
  \begin{gathered}
    \begin{rcases*}
      p\vee q \text{ eq } q\vee p\\
      p\wedge q \text{ eq } q\wedge p
    \end{rcases*}=\text{Kommutativa lagar}
  \end{gathered}
\end{equation*}
\par\bigskip

\begin{equation*}
  \begin{gathered}
    \begin{rcases*}
      \neg(p\wedge q) \text{ eq } \neg p \vee\neg q\\
      \neg(p\vee q) \text{ eq } \neg(p)\wedge \neg(q)
    \end{rcases*}=\text{de Morgans lagar}
  \end{gathered}
\end{equation*}
\par\bigskip

\begin{equation*}
  \begin{gathered}
    \begin{rcases*}
      p\vee p \text{ eq } p\\
      p\wedge p \text{ eq } p
    \end{rcases*}=\text{Idempotenslagar}
  \end{gathered}
\end{equation*}
\par\bigskip

\begin{equation*}
  \begin{gathered}
    \begin{rcases*}
      \neg\neg p \text{ eq } p
    \end{rcases*}=\text{Lagen om dubbel-negation}
  \end{gathered}
\end{equation*}
\par\bigskip

\subsection{Logisk konsekvens}\hfill\\

\begin{theo}[Logisk konsekvens]{thm:logcons}
  Låt $\Gamma$ vara en mängd av formler i LP($\sigma$) och låt $\varphi$ en formel i $\Gamma$. $\varphi$ är en \textit{logisk konsekvens} av $\Gamma$, skrivet $\Gamma\vDash\varphi$, om $\varphi$ är sann i varje modell för $\Gamma$.
  \par\bigskip
  \noindent Dvs, för varje $\sigma$-struktur gäller: Om $A\vDash\gamma$ för varje $\gamma\in\Gamma$, så kräver vi att $\varphi$ (den logiska konsekvensen) också ska vara sann. Här betyder $\vDash$ "när hälst varje struktur i $\Gamma$".
\end{theo}

\subsection{Exempel}\hfill\\
\noindent Visa att $\{p_1\rightarrow p_2, p_1\}\vDash p_2$ (visa att $p_2$ är en logisk konsekvens av mängden).
\par\bigskip
\noindent Vi visar detta genom att låta $A$ vara en modell för $p_1\rightarrow p_2$ och $p_1$, alltså $A$ är en struktur där de två är sanna. Detta betyder att $A$ är en $\sigma$-struktur, dvs $A^*(p_1\rightarrow p_2)=1$ och $A^*(p_1)=1$. Det vi måste visa är att $A^*(p_2)=1$. Detta kan vi göra genom att anta att $A^*(p_2)=0$, vi vill få en motsägelse. Då blir $A^*(p_1\rightarrow p_2)=0$, men detta mostäger antagandet att $A^*(p_1\rightarrow p_2)=1$, alltså $A^*(p_2)=1$
\par\bigskip

\noindent Om inte $\Gamma\vDash\varphi$ gäller, så skriver man $\Gamma\nvDash\varphi$, dvs "$\varphi$ är inte en logisk konsekvens av $\Gamma$" $\Lrarr\neg$($\varphi$ sann i varje modell för $\Gamma$) $\Lrarr$ det finns någon (minst 1) modell för $\Gamma$ i vilken $\varphi$ är falsk.
\par\bigskip

\subsection{Exempel}\hfill\\
$\{p_1\rightarrow p_2, p_2\}\nvDash p_1$. Vi måste ge en motexempelstruktur $A$, så att $A^*(p_1\rightarrow p_2)=A^*(p_2)=1$ och $A^*(p_1)=0$. Kom ihåg att struktur betyder tilldeling av sanningsvärde till satssymbolerna.
\par\bigskip
\noindent Ta tex följande:
\begin{itemize}
  \item $A(p_1)=0$
  \item $A(p_2)=1$
\end{itemize}
\par\bigskip
\noindent Nu har vi hittat en struktur som gör VL sann och HL falsk.
\par\bigskip

\subsection{Algoritm för att avgöra $\Gamma\vDash\varphi$}\hfill\\

\noindent Använd sanningsvärdestabeller för alla formler i $\Gamma$ och för $\varphi$. För de rader där alla i $\Gamma$ är sanna, kolla om $\varphi$ är sann.
\par\bigskip
\subsection{Exempel}\hfill\\

\noindent Avgör om följande gäller $\neg A\vee \neg B$, $B\vee C\vDash(A\wedge C)\rightarrow B$:
\par\bigskip

\begin{center}
  \begin{tabular}{|c|c|c|c|c|c|c|}
    \hline
    $A$&$B$&$C$&$\neg A\vee\neg B$&$B\vee C$&$A\wedge C\rightarrow B$\\
    \hline
    0&0&0&1&0&1\\
    \hline
    0&0&1&1&1&1\\
    \hline
    0&1&0&1&1&1\\
    \hline
    0&1&1&1&1&1\\
    \hline
    1&0&0&1&1&1\\
    \hline
    1&0&1&1&0&0\\
    \hline
    1&1&1&0&1&1\\
    \hline
  \end{tabular}
\end{center}
\par\bigskip
\noindent Rad 3 säger ju att VL = 1 men HL = 0, alltså finns modell för $\Gamma$ så att $(A\wedge C)\rightarrow B$ är falsk, alltså $\Gamma\nvDash(A\wedge C)\rightarrow B$







