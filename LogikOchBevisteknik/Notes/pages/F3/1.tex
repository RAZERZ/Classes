\section{Föreläsning - Naturlig deduktion forts.}

\noindent Exempel: Visa att för alla heltal $n$ gäller att $n^2+n$ är jämn.
\par\bigskip

\begin{prf}[Exempel]{prf:ex}
  Ett heltal är antingen jämn \textit{eller} udda, så vi gör en falluppdelning:
  \begin{itemize}
    \item Fall 1 (jämn):
      \begin{itemize}
        \item $n=2k \Rightarrow (2k)^2+2k=4k^2+2k=2(2k^2+k)$ alltså jämn
      \end{itemize}
    \item Fall 2 (udda):
      \begin{itemize}
        \item $n=2k+1\Rightarrow (2k+1)^2+2k+1=4k^2+4k+1+2k+1=2(2k^2+3k+1)$ alltså jämn.
      \end{itemize}
  \end{itemize}
\end{prf}
\par\bigskip

\noindent Exempel: $A\vee B\vdash B\vee A$ (sanningstabellen för dessa är likadana). Vi skall producera ett bevisträd som har en premiss $A\vee B$ där slutsatsen är $B\vee A$
\par\bigskip

\subsection{Botten ($\perp$)}\hfill\\

\noindent Reglerna här är inte riktigt "intro/elimination" utan det har med motsägelser osv att göra.\par\bigskip

\noindent Antag att vi vill visa $A$, då har vi premissen $\neg A$ så att vi kommer fram till $\perp$. Då kan vi dra slutsatsen $A$. Detta kallas för RAA = Reductio ad absurdum. Då får vi stryka $\neg A$
\par\bigskip

\begin{prf}[Exempel: Det finns oändligt många primtal]{thm:prim}
  Vi antar motsatsen (att det finns ändligt många primtal) och försöker visa motsägelse.
  \par\bigskip
  \noindent Låt $A$ = det finns oändligt många primtal. Då antar vi $\neg A$ och visar $\perp$, och sen drar slutsatsen $A$.
\end{prf}
\par\bigskip

\noindent Bevisträd är inte entydiga.
