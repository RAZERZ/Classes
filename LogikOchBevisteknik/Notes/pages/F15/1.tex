\section{Fullständighet i 1:a ordningens logikk}
\par\bigskip
\subsection{Repetition från satslogiken}\hfill\\
\par\bigskip
\noindent Vi påminner oss om att given en mängd av satser $\Gamma$ så kallas den \textit{konsistent} om $\Gamma\nvdash\perp$ och inkonsistent om $\Gamma\vdash\perp$ 
\par\bigskip
\noindent För att visa detta i predikatlogiken så behöver vi visa att det inte finns ett bevisträd för detta. För att kunna göra detta behöver vi ha \textit{Fullständighetssatsen}
\par\bigskip
\subsection{Sundhetssatsen}\hfill\\\par
\noindent Låt $\sigma$ vara en signatur och $\Gamma$ en mängd satser i LR($\sigma$) och $\varphi$ en sats i LR($\sigma$)\par
\noindent Då gäller $\Gamma\vdash\varphi\Rightarrow\Gamma\vDash\varphi$\par
\noindent Dvs om det finns bevis i naturlig deduktion av $\varphi$ från $\Gamma$, så är $\varphi$ en logisk konsekvens av $\Gamma$
\par\bigskip
\begin{prf}[Sundhetssatsen]{prf:sund}
  \noindent Detta är bara en skiss.\par
  \noindent Visa att alla regler är sunda (bevarar sanning)\par
  \noindent Exempelvis, betrakta $\forall$-intro regeln som säger att givet ett $\varphi(x)$ får vi skriva $\forall \varphi(x)$ förutsatt att $x$ ej förekommer fritt i premisser ovanför $\varphi(x)$. Vi vill visa att $\Gamma\vDash\forall x \varphi(x)$, låt $\textgoth{A}$ vara en modell för $\Gamma$, detta fööljer från att $\Gamma\vDash\varphi(x)$
\end{prf}
\par\bigskip
\subsection{Adekvathetssatsen}\hfill\\\par
\noindent Låt $\sigma$ vara en signatur och $\Gamma$ en mängd satser i LR($\sigma$) och $\varphi$ en sats i LR($\sigma$)\par
\noindent Då gäller att $\Gamma\vDash\varphi\Rightarrow\Gamma\vdash\varphi$\par
\noindent Dvs, om $\varphi$ är logisk konsekvens av $\Gamma$ så finns bevissträd i naturlig deduktion av $\varphi$ från premisser i $\Gamma$
\par\bigskip
\noindent Varje gång vi har en logisk konsekvens måste vi visa att det går att visa med naturlig deduktion.
\par\bigskip
\begin{prf}[Adekvathetssatsen]{prf:adequ}
  \noindent Vi kommer skissa detta genom en så kallad sökmetod för att försöka avgöra om $\Gamma\vDash\varphi$\par
  \noindent Att leta efter ett motexempel i logisk konsekvens handlar om att hitta att allting i $\Gamma$ sann men $\varphi$ falsk. Detta gör man genom att stycka upp formlerna och det komemr bli som ett träd. \par\bigskip
  \noindent Om $\Gamma\nvDash\varphi$ så kommer metoden hjälpa att hitta motexempel. Däremot, om $\Gamma\vDash\varphi$ så kommer metoden hjälpa oss att förstå att motexempel saknas samt hjälpa att inse naturlig deduktionsbevis finns\par\bigskip
  \noindent Antag $\Gamma\vDash\varphi$, gör ett söktträd (även kallat sekventträd). Alla vägar kommer få $X$. Man visar sedan att träd där alla vägar får $X$ kan omvandla till bevissträd i naturlig deduktion för $\Gamma\vdash\varphi$ 
\end{prf}
\par\bigskip
\begin{theo}[Följdsats från Fullständighetssatsen i predikatlogiken]{thm:aplem}
  \noindent Låt $\Gamma$ vara en mängd satser.\par
  \noindent $\underbrace{\Gamma \text{konsistent}}_{\text{$\Gamma\nvdash\perp$}}\Rightarrow\underbrace{\Gamma\text{ satisfierbar}}_{\text{$\Gamma$ har en modell}}$
\end{theo}
\newpage
\begin{prf}[$\Rightarrow$]{prf:rightarrow}
  $\Gamma\nvdash\perp\Lrarr\text{ inte }\left(\Gamma\vdash\perp\right)\underbrace{\Lrarr}_{\text{Fullständighetssatsen}}\text{ inte }\left(\Gamma\vDash\perp\right)$
  \par\bigskip

  \noindent$\Lrarr$ det finns struktur $\textgoth{A}$ så att $\textgoth{A}\vDash\Gamma$ och $\underbrace{\textgoth{A}\nvDash\perp}_{\text{gäller alltid}}$
  \par\bigskip
  \noindent$\Gamma$ har en modell
\end{prf}
\par\bigskip
\begin{theo}[Kompakthetssatsen]{thm:compactnessthm}
  Låt $\Gamma$ var en mängd av satser. Då gäller:
  \begin{itemize}
    \item $\Gamma$ har en modell $\Lrarr$ varje ändlig delmängd av $\Gamma$ har en modell
  \end{itemize}
\end{theo}
\par\bigskip
\begin{prf}[$\Rightarrow$]{prf:rarr}
  Varje modell för $\Gamma$ är också en modell för dess delmängder
\end{prf}
\par\bigskip
\begin{prf}[$\Leftarrow$]{prf:larr}
  Bevisas kontrapositivt. Antag att $\Gamma$ saknar modell, visa att då gäller inte denna sida av pilen.
  \par\bigskip
  \noindent Från Sats 18.1 vet vi då att $\Gamma$ är inkonsistent, dvs $\Gamma\vdash\perp$\par
  \noindent Låt $D$ vara ett bevissträd som visar $\perp$. Vi vet att dessa inte är entydiga, så låt därför $\Gamma_0$ vara infimum av $\Gamma$ sådant att bevissträdet är så litet som möjligt.\par
  \noindent Då vet vi att $\Gamma_0\vdash\perp$ och $\Gamma_0$ är ändlig och $\Gamma_0\subseteq\Gamma$.
  \par\bigskip
  \noindent Från detta vet vi även att $\Gamma_0\vDash\perp$, men $\perp$ är inte sann i någon modell (per definition), alltså saknar $\Gamma_0$ modell\par
  \noindent Alltså finns ändlig delmängd av $\Gamma$ så att $\Gamma_0$ saknar modell.
\end{prf}
\par\bigskip
\subsection{Tillämpning av Kompakthetssatsen}\hfill\\
\par\bigskip
\noindent Antag $\Gamma$ har godtyckligt stora ändliga modeller.
\par\bigskip
\noindent Då har $\Gamma$ en oändlig modell (från Kompakthetssatsen)
\par\bigskip
\noindent\textbf{Bevisskiss}
\par\bigskip
\noindent $\lambda_n$: det finns minst $n$ element (detta får vi skriva så fort vi har tillgång till likhetsstecknet).
\par\bigskip
\noindent Ansätt $\Gamma^* = \Gamma\cup\{\lambda_n:n\geq 1\}\qquad$ Observera, varje modell för $\Gamma^*$ måste vara oändlig
\par\bigskip
\noindent Låt $\Gamma_0$ vara ändlig delmängd av $\Gamma^*$. $\Gamma_0$ innehåller ändligt många av $\lambda$ formlerna.\par
\noindent Låt $k$ så att $\lambda_n\in\Gamma_0\Rightarrow n<k\qquad$ $k$ agerar supremum\par
\noindent Enligt antagande finns det modell för $\Gamma$ som har minst $k$ element.\par
\noindent Säg $\textgoth{A}$ är en modell. Då är $\textgoth{A}\vDash\Gamma$ och $\textgoth{A}$ har minst $k$ element. Detta medför att $\textgoth{A}$ är en modell för $\Gamma_0$ (från Kompakthetssatsen), alltså varje ändlig delmängd av $\Gamma^*$ har en modell, och från Kompakthetssatsen har $\Gamma^*$ har en modell, säg $\textgoth{A}^*\Rightarrow\textgoth{A}^*$ är en modell för $\Gamma$ och $\textgoth{A}^*$ är oändlig 
