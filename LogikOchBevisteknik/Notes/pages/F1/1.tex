\section{Föreläsning - Introduktion till Satslogik}

\subsection{Historia}
\par\bigskip

  Vad är ett matematiskt bevis?\par
  Vad får användas i bevis?\par
  Är matematiken motsägelsefri (Konsistent = motsägelsefritt)?

\subsection{Vad behövs för ett matematiskt bevis?}
\par\bigskip

\begin{itemize}
  \item Ett påstående (även kallad utsaga) (ex.vis $\sqrt{2}$ är irrationellt), dessa har sanningsvärde \textit{sant} eller \textit{falskt}
  \item (Giltigt) Argument (resonemang)
\end{itemize}

\subsection{Exempel}

\begin{itemize}
  \item Påstående: \textit{Varje kvadrat är en rektangel}
  \item Påstående: \textit{Det finns en fyrhörning som inte är en rektangel}
\end{itemize}
---------------------------------------------------------

Alltså finns en fyrhörning som inte är kvadrat
\par\bigskip


\noindent Detta är syftet med kursen, låt oss nu göra det mer abstrakt.
(Låt x vara form, K(x) kvadrat, R(x) rektangel, F(x) fyrhörning)
Då blir påståenden:

\begin{itemize}
  \item $\forall x (K(x) \Rightarrow R(x))$
  \item $\exists x (F(x) \wedge \neg R(x))$
\end{itemize}
---------------------------------------------------------\par
$\exists x (F(x) \wedge\neg(K(x)))$

\begin{figure}[ht]
    \centering
    \incfig{grafisk-tolkning}
    \caption{Grafisk tolkning}
    \label{fig:grafisk-tolkning}
\end{figure}

\begin{theo}[Logiskt giltighet (Predikat-logik = 1:a ordningens logik)]{thm:logiskGilt}
  En följd av logiska steg kallas för \textit{logiskt giltig} om det gäller $\forall$ tolkningar av $K,R,F$ (även vovvisar)
\end{theo}
\newpage

\subsection{Satslogik}
\par\bigskip

Vi behöver:
\begin{itemize}
  \item Ett skriftligt språk (mängd av teckensträngar som betyder utsagor/satser)
  \item Regler/Formella bevis (Hur får man dra slutsatser av nya teckensträngar), \textit{syntax}
  \item Tolka teckensträngarna i en "verklighet" (sant eller falskt)
\end{itemize}


\subsection{Predikatlogik (1:a ordningens logik)}
Skiljer sig från satslogik i och med att vi inte hanterar satser utan predikaten.

\begin{itemize}
  \item Språk behövs, liknande med teckenmängd men vi kan även hantera elementen
  \item Formella bevis består av teckensträngsmanipulation, relation mellan teckensträngar\par(Exvis $A_1, A_2, A_3 \vdash B$ (A bevisar B))
  \item Vad betyder det att något är sant eller falskt? Om det går att visa B utan något så är det sant. 
    \par\bigskip

    \begin{theo}[Definiera sanning i struktur]{thm:truth}
      $\vdash B$ = Sant om det inte krävs något för att visa B:s sanningsvärde.\par\bigskip

      $A_1, A_2, A_3 \vDash B$ det vill säga om alla A krav är uppfyllda så gäller B
    \end{theo}
    \par\bigskip

  \item Samband mellan $\vdash$ (formell bevisbarhet) och $\vDash$ (hur man tolkar något som sant eller falskt)
    \par\bigskip


    \begin{theo}[Sundhetssatsen]{thm:sund}
      Detta säger att bevissystemet är sunt, allt man visar är sunt
      \par\bigskip
      $A_1, A_2, A_3 \vdash B \Rightarrow A_1, A_2, A_3 \vDash B$
    \end{theo}
    \par\bigskip


    \begin{theo}[Adekvathet]{thm:adekvat}
      Om det är så att B är sann så fort alla A är sanna så finns det ett bevis, vi kommer kunna påstå att det finns ett formellt bevis för samma sak.
      \par\bigskip
      $A_1, A_2, A_3 \vDash B \Rightarrow A_1, A_2, A_3 \vdash B$
    \end{theo}
    \par\bigskip

    \begin{theo}[Fullständighet]{thm:full}
      Slår vi ihop dessa 2 (Sundhetssatsen och Adekvathet) får vi fullständighet
      \par\bigskip
      $A_1, A_2, A_3 \vdash B \Lrarr A_1, A_2, A_3 \vDash B$
    \end{theo}

\end{itemize}

\newpage

