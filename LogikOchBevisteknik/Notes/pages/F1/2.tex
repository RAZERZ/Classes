\section{Satslogik (Propositional calculus)}
\subsection{Språk (LP)}

\begin{itemize}
  \item Satssymboler: $\sigma = \{p_0, p_1\cdots,p_n\}$ (en \textit{satslogisk signatur}, även kallas \textit{språkets signatur})
    \par\bigskip
    LP($\sigma$) kallas för en \textit{dialekt} av LP
  \item Alfabet i LP($\sigma$):
    \begin{itemize}
      \item Alla symboler i $\sigma$
      \item Konnektiver: $\Lrarr, \Leftarrow, \Rightarrow, \vee, \wedge, \neg, \perp$
      \item Paranteser: (, )
    \end{itemize}
\end{itemize}
\par\bigskip

\noindent Viktigt att notera, $\perp$ är konnektiv men även sats, så den skapar inga nya formler.
\par\bigskip


\begin{theo}[Mängden av formler i LP($\sigma$)]{thm:setofChars}
  Mängden definieras rekursivt (likt naturliga talen).
  \par\bigskip
  \begin{itemize}
    \item Basfall:
      \begin{itemize}
        \item Alla tecken i $\sigma$ är en formel
        \item $\perp$ är en formel
        \item Dessa kallas för \textit{Atomära} formler
      \end{itemize}
    \item Induktion:
      \begin{itemize}
        \item Om $\varphi$ är en formel, så är ($\neg\varphi$) en formel
      \end{itemize}
    \item Låt $\Box$ vara konnektiv:
      \begin{itemize}
        \item Induktion $\Box$:
          \begin{itemize}
            \item Om $\varphi_1$ och $\varphi_2$ är formler, så är ($\varphi_1\Box\varphi_2$) en formel
          \end{itemize}
      \end{itemize}
  \end{itemize}
\end{theo}
\par\bigskip


\subsection{Exempel}
\par\bigskip

Några exempel på formler i LP($\sigma$) där $\sigma = \{p,q,r\}$:
\begin{itemize}
  \item $\perp, r, (p\Rightarrow(\neg r)), (p\vee(\neg p))$
\end{itemize}
\par\bigskip

Några som \textit{inte} är formler:

\begin{itemize}
  \item $(p\wedge, p\vee q\wedge r$ (inga paranteser!)
\end{itemize}
