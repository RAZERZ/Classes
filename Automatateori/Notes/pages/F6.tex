\section{Slutenhetsegenskaper hos reguljära språk}\par
\noindent Vi vet att operationerna på språken gör att vi får ett reguljärt språk, det följer ur definitionen av ett reguljärt språk.\par
\noindent Men! Det visar sig att om man har ett reguljärt språk så kommer komplementet vara reguljärt, det följer \textit{inte} från definitionen!
\par\bigskip
\noindent Antag att $L_1$ och $L_2$ är reguljära språk. Från definitionen reguljära språk, följer det att även $L_1\cup L_2$, $L_1L_2$, $L_1^*$ och $L_1^+$ är reguljära.\par
\noindent Det finns dock fler slutenhetsegenskaper:
\par\bigskip
\begin{theo}
  OOm $L_1$ och $L_2$ är reguljära, så är $\bar{L_1}$ (på samma sätt för $L_2$), och $L_1\cap L_2$, $L_1^{\text{rev}} = \left\{w^{\text{rev}}: w\in L_1\right\}$, Prefix($L_1$), Suffix($L_1$).\par
  \noindent Alla dessa är reguljära.
\end{theo}
\par\bigskip
\noindent\textbf{Bevisskiss}:\par
\noindent Vi tittar först på komplementet. Om $L_1$ är reguljärt, så finns en DFA $M_1$ som accepterar $L_1$. Vi kan därför konstruera en DFA med starttillstånd $M_1$ som accepterar komplementet.\par
\noindent Vi påminner oss om att om $w\in L_1\Lrarr M_1$ accepterar $w\Lrarr w$ driver $M_1$ från $s$ till ett accepterande tillstånd.\par
\noindent Låt $M_1^{\prime}$ vara DFA:n som fås från $M_1$ genom att göra alla $M_1$:s accepterande tillstånd till icke-accepterande och göra alla $M_1$:s icke-accepterande tillstånd till accepterande. Då accepteras $\bar{L_1}$ av $M_1^{\prime}$ och därmed är $\bar{L_1}$ reguljärt.
\par\bigskip
\noindent\textbf{Bevisskiss}:\par
\noindent Vi tittar nu på snittet. Antag att $L_1$ och $L_2$ är reguljära. Enligt det vi visade i bevisskiss ovan så är $\bar{L_1}$ och $\bar{L_2}$ reguljära. Enligt definitionen av reguljära språk, så är $\bar{L_1}\cup\bar{L_2}$ och då är även komplementet av unionen också reguljärt. Men, snittet $L_1\cap L_2$ är ju komplementet till unionen.
\par\bigskip
\noindent\textbf{Anmärkning:}\par
\noindent Snittet av 2 reguljära uttryck är \textit{inte} ett reguljärt uttryck, men snittet av 2 reguljära språk är reguljära.
\newpage
\subsection{Konstruktion av en "snitt"-DFA}\hfill\\\par
\noindent Givet input: Två DFA:er $M_1$ resp. $M_2$:\par
\noindent $M_1$:
\par
\noindent $M_2$
\par\bigskip
\noindent En DFA som accepterar $L(M_1)\cap L(M_2)$ med ordnade par från vardera DFA. Starttillståndet blir tillståndet med starttillstånd från vardera DFA. Lägg till alla kombinationer, och ta bort de isolerade tillstånden. I vårat fall får vi:

\begin{figure}[ht!]
    \centering
    \begin{tikzpicture}
      \node[state](p0){$3,8$};
      \node[state, below left of=p0, xshift=-2cm, yshift=-1cm](p1){$1,5$};
      \node[state, initial right, initial text=, right of=p1, xshift=4cm](p2){$4$};
      \node[state, below of=p1, xshift=.5cm, yshift=-1cm](p3){$2$};
      \node[state, accepting left, below of=p0, yshift=-3cm](p4){$7$};
      \node[state, below of=p2, yshift=-.75cm, xshift=.5cm](p5){$6$};
      \node[below of=p4, yshift=-1cm, xshift=3cm](x){};
      \path[-stealth] (p1) edge[loop above] node{$a,b$} (p1);
      \path[-stealth] (p4) edge[below, bend right] node{$b$} (p5);
      \path[-stealth] (p5) edge[above, bend right] node{$a$} (p4);
      \path[-stealth] (p2) edge[above, bend right] node{$a$} (p0);
      \path[-stealth] (p4) edge[left] node{$a$} (p0);
      \path[-stealth] (p5) edge[above] node{$b$} (p1);
      \path[-stealth] (p3) edge[left] node{$a$} (p1);
      \path[-stealth] (p2) edge[above] node{$b$} (p1);
      \path[-stealth] (p0) edge[above] node{$b$} (p5);
      \path[-stealth] (p0) edge[above, left] node{$a$} (p3);
      %\path[-stealth] (p3) edge[below, bend right] node{$b$} (p2);
      \draw[-stealth] (p3) edge[bend right] (x);
      \path[-stealth] (x) edge[right, bend right] node{$b$} (p2);
    \end{tikzpicture}
    \caption{}
\end{figure}
