\section{Determiniska finita (ändliga) automater [DFA]}
\par\bigskip
\noindent Förr i tiden programmerades datorer genom att man hade "punch-cards", det vill säga kort av styvt papper som hade "hål" som agerade som dagens transistorer gör.\par
\noindent DFA:er påminner lite om hur dessa fungerar, det vill säga att vi ska tänka oss en slags textremsa som inmatning:
\begin{center}
  \begin{tabular}{|c|c|c|c|c|c|}
    \hline
    $a$&$b$&$a$&$b$&$b$&$\cdots$\\
    \hline
  \end{tabular}
\end{center}
\par\bigskip
\noindent Där ändligt många inledande rutor innehåller tecken från en sådan textremsa med alfabet $\Sigma$.
\par\bigskip
\noindent Vi har också en "kontrollmekanism", det vill säga själva DFA:n som vi kan tänka oss har ett "läshuvud" som avläser en ruta i taget (det är viktigt att precisera, DFA:er läser EN ruta i taget).\par
\noindent DFA:n befinner sig alltid i ett tillstånd, av ändliga antal möjliga. När den avläser en ruta så övergår den till ett nytttillstånd samt flyttar läshuvudet ett steg till höger.\par
\noindent Det nya tillståndet beror (endast) på det tidigare tillståndet samt det just avlästna tecknet (i föregående ruta).
\par\bigskip
\noindent De två "viktigaste" tillstånd kallas för \textbf{starttillståndet} och \textbf{accepterande tillståndet}.\par
\noindent Vi antar att en DFA alltid befinner sig i starttillståndet då den sätts igång.\par
\noindent Om en DFA befinner sig i ett accepterande tillstånd då en sträng har avlästs (det vill säga, läshuvudet befinner sig på den första blanka rutan till höger om strängen) så säger vi att DFA:n \textbf{accepterar }strängen. Om strängen inte accepteras, så avvisas strängen. 
\par\bigskip
\subsection{Grafisk beskrivning av en DFA}\hfill\\\par
\noindent Vi väljer att representera tillstånden med hjälp av noder, och tillståndsövergångar med pilar:\par
\begin{figure}[ht]
    \centering
    \begin{tikzpicture}
      \node[state](p){$p$};
      \node[state, right of=p, xshift=1cm](q){$q$};
      \path[-stealth] (p) edge[above, bend left] node{$a$} (q);
    \end{tikzpicture}
    \caption{}
\end{figure}
\par\bigskip
\noindent Figur 1 betyder att om DFA:n befinner sig i tillstånd $p$ och avläser $a$ på textremsan så övergår DFA:n till tillstånd $q$ (och flyttar därmed läshuvudet ett steg till höger).
\par\bigskip
\noindent\textbf{Anmärkning:}\par
\noindent För varje tillstånd och varje tecken från input-alfabetet skall det finnas \textit{preci} en utgående pil som bär detta tecken.
\par\bigskip
\noindent Starttillstånd markeras med en pil mot sig:
\begin{figure}[ht]
    \centering
    \begin{tikzpicture}
      \node[state, initial, initial text=](){};
    \end{tikzpicture}
    \caption{}
\end{figure}
\par\bigskip
\noindent Accepterande tillstånd markeras med en pil från sig \textit{eller} dubbelrand:
\begin{figure}[ht]
    \centering
    \begin{tikzpicture}
      \node[state, accepting right](p){};
      \node[state, accepting, right of=p, xshift=2cm](){};
    \end{tikzpicture}
    \caption{}
\end{figure}
\newpage
\noindent\textbf{Exempel:}
\begin{figure}[ht]
    \centering
    \begin{tikzpicture}
      \node[state, initial, initial text=, accepting above](p0){$p_0$};
      \node[state, accepting left, below left of=p0, yshift=-.5cm](p1){$p_1$};
      \node[state, below right of=p1, yshift=-.5cm](p2){$p_2$};
      \path[-stealth] (p0) edge[loop right] node{$b$} (p0);
      \path[-stealth] (p0) edge[left, bend right] node{$a$} (p1);
      \path[-stealth] (p1) edge[right, bend right] node{$b$} (p0);
      \path[-stealth] (p1) edge[left, bend right] node{$a$} (p2);
      \path[-stealth] (p2) edge[loop right] node{$a,b$} (p2);
    \end{tikzpicture}
    \caption{}
\end{figure}
\par\bigskip
\noindent Kom ihåg att en sträng accepteras om vi befinner oss i ett accepterande tillstånd då hela strängen är avläst.\par
\noindent Vilka strängar accepterar av ovanstående DFA? ($(b\cup ab)^*(\varepsilon\cup a)$)
\par\bigskip
\begin{theo}
  AAntag att $M$ är en DFA.\par
  \noindent Mängden av strängar som $M$ accepterar kallas för \textbf{$M$:s språk} och betecknas $L(M)$
\end{theo}
\par\bigskip
\begin{theo}[Formell definition av DFA]{thm:dfa}
  En \textbf{DFA} är en 5-tupel $M = (Q,\Sigma,\delta,s,F)$ där:\par
  \begin{itemize}
    \item $Q$ är en ändlig mängd (tillståndsmängden)
    \item $\Sigma$ är en ändlig mängd (inputalfabet)
    \item $\delta$ är en funktion (övergångsfunktion) från $Q$x$\Sigma\to Q$
    \item $s\in Q$ ($s$ är Starttillstånd)
    \item $F\subseteq Q$ (accepterande tillstånden)
  \end{itemize}
\end{theo}
\par\bigskip
\noindent För att formellt kunna definiera \textit{acceptans} av en sträng behöver vi en \textbf{utvidgad övergångsfunktion}. Denna kan vi löst definiera på följande vis:
\par\bigskip
\begin{theo}[Utvidgad övergångsfunktion]{thm:transfunc}
  En funktion $T(\text{nuvarande tillstånd}, \text{nuvarande symbol på läshuvu})\to\text{nästa tillstånd}$
\end{theo}
\newpage
\section{Icke-Determiniska Finita (åndliga) automater [NFA]}
\par\bigskip
\noindent En NFA är en generalisering av en DFA på följande sätt:\par
\begin{itemize}
  \item En NFA kan ha flera Starttillstånd
  \item En NFA tillåts kunna läsa flera tecken på en gång, dvs vi kan tillståndsövergångar på formen:
\end{itemize}
\begin{figure}[ht]
    \centering
    \begin{tikzpicture}
      \node[state](p){$p$};
      \node[state, right of=p, xshift=2cm](q){$q$};
      \path[-stealth] (p) edge[above, bend left] node{$abb$} (q);
    \end{tikzpicture}
    \caption{}
\end{figure}\par
\begin{itemize}
  \item En NFA kan göra en tillståndsövergång utan att läsa något (läser tecknet $\varepsilon$)
  \item En NFA kan ha flera valmöjligheter i en given situation (icke-determinism), dvs vi kan ha följande situation:
\end{itemize}
\begin{figure}[ht]
    \centering
    \begin{tikzpicture}
      \node[state](p0){$p_0$};
      \node[state, above right of=p0, xshift=1cm](p1){$p_1$};
      \node[state, below left of=p0, yshift=-.5cm, xshift=-1cm](p2){$p_2$};
      \node[state, below right of=p0, yshift=-.5cm, xshift=1cm](p3){$p_3$};
      \path[-stealth] (p0) edge[above left, bend left] node{$a$} (p1);
      \path[-stealth] (p0) edge[below, bend left] node{$a$} (p2);
      \path[-stealth] (p0) edge[below, bend right] node{$abb$} (p3);
    \end{tikzpicture}
    \caption{}
\end{figure}\par
\begin{itemize}
  \item Det följer att en NFA (i allmänhet) kan "hänga sig"
\end{itemize}
\par\bigskip
\noindent\textbf{Exempel: (På en NFA)}\par
\begin{figure}[ht]
    \centering
    \begin{tikzpicture}
      \node[state, initial above, initial text=](p0){$p_0$};
      \node[state, initial above, initial text=, right of=p0, xshift=2cm](p1){$p_1$};
      \node[state, accepting below, below of=p0, xshift=1.5cm, yshift=-1cm](p2){$p_2$};
      \path[-stealth] (p0) edge[loop left] node{$ab$} (p0);
      \path[-stealth] (p0) edge[above, bend left] node{$\varepsilon$} (p1);
      \path[-stealth] (p1) edge[below] node{$\varepsilon$} (p0);
      \path[-stealth] (p1) edge[loop right] node{$bb$} (p1);
      \path[-stealth] (p0) edge[above] node{$a$} (p2);
      \path[-stealth] (p2) edge[above, bend left] node{$a$} (p0);
      \path[-stealth] (p1) edge[above] node{$b$} (p2);
      \path[-stealth] (p2) edge[above, bend right] node{$a$} (p1);
    \end{tikzpicture}
    \caption{}
\end{figure}
\newpage
\begin{theo}[NFA Acceptans]{thm:nfaacc}
  En NFA accepterar en sträng $w$ om vi kan börja i något starttillstånd och gå från tillstånd till tillstånd genom att avläsa en del av strängen $w$ och slutligen hamna i ett accepterande tillstånd.\par
  \noindent Det vill säga, om $w$ kan delas upp i delsträngar $w = v_1v_2\cdots v_n$ där vi tillåter $v_k = \varepsilon$ \textit{och} det finns tillstånd $p_0, p_1,\cdots, p_n$ där $p_0$ är ett starttillstånd och $p_n$ är ett accepterande tillstånd samt följande övergångar:
\end{theo}
\begin{figure}[ht]
    \centering
    \begin{tikzpicture}
      \node[state](p0){$p_1$};
      \node[state, right of=p0, xshift=.5cm](p1){$p_1$};
      \node[state, below of=p0, yshift=-.5cm](p2){$p_1$};
      \node[state, below of=p1, yshift=-.5cm](p3){$p_2$};
      \node[state, below of=p2, yshift=-.5cm](p4){$p_{n-1}$};
      \node[state, below of=p3, yshift=-.5cm](p5){$p_n$};
      \path[-stealth] (p0) edge[above, bend left] node{$v_1$} (p1);
      \path[-stealth] (p2) edge[above, bend left] node{$v_2$} (p3);
      \path[-stealth] (p4) edge[above, bend left] node{$v_n$} (p5);
    \end{tikzpicture}
    \caption{}
\end{figure}
\par\bigskip
\noindent\textbf{Exempel:}\par
\noindent Betrakta föregående exempel och strängarna:
\begin{equation*}
  \begin{gathered}
    aaa\\
    aaaa\\
    aabbaa\\
    abbba
  \end{gathered}
\end{equation*}\par
\noindent Vilka strängar accepteras av NFA:n?
\par\bigskip
\begin{theo}
  MMängden av strängar som accepteras av en NFA $M$ kallas för $M$:s språk, och betecknas precis som i DFA fallet med $L(M)$
\end{theo}
\newpage
\subsection{Omvandling av en NFA till en DFA som accepterar samma språk}\hfill\\\par
\noindent Betrakta följande NFA:
\begin{figure}[ht]
    \centering
    \begin{tikzpicture}
      \node[state, initial above, initial text=](p0){$1$};
      \node[state, accepting below, left of=p0, xshift=-1cm](p1){$2$};
      \node[state, accepting below, right of=p0, xshift=1cm](p2){$3$};
      \path[-stealth] (p0) edge[below, bend right] node{$\varepsilon$} (p1);
      \path[-stealth] (p0) edge[above, bend left] node{$\varepsilon$} (p2);
      \path[-stealth] (p1) edge[loop above] node{$b$} (p1);
      \path[-stealth] (p2) edge[loop above] node{$a$} (p2);
      \path[-stealth] (p2) edge[loop right] node{$aba$} (p2);
    \end{tikzpicture}
    \caption{}
\end{figure}\par
\noindent Vi konstruerar först en "icke-glupsk" NFA $M^{\prime}$ som avlöser högst ett tecken i taget (försöker gå från generaliseringen tillbaks till en DFA) och accepterar samma strängar som $M$.\par
\noindent $M^{\prime} = $
\begin{figure}[ht]
    \centering
    \begin{tikzpicture}
      \node[state, initial above, initial text=](p0){$1$};
      \node[state, accepting below, left of=p0, xshift=-1cm](p1){$2$};
      \node[state, accepting below, right of=p0, xshift=1cm](p2){$3$};
      \node[state, below right of=p2, yshift=-.25cm, xshift=.5cm](p3){$4$};
      \node[state, right of=p3, xshift=.5cm](p4){$5$};
      \path[-stealth] (p0) edge[below, bend right] node{$\varepsilon$} (p1);
      \path[-stealth] (p0) edge[above, bend left] node{$\varepsilon$} (p2);
      \path[-stealth] (p1) edge[loop above] node{$b$} (p1);
      \path[-stealth] (p2) edge[loop above] node{$a$} (p2);
      \path[-stealth] (p2) edge[left, bend right] node{$a$} (p3);
      \path[-stealth] (p3) edge[below, bend right] node{$b$} (p4);
      \path[-stealth] (p4) edge[above, bend right] node{$a$} (p2);
    \end{tikzpicture}
    \caption{}
\end{figure}\par
\noindent Sedan använder vi den så kallade "delmängdsalgoritmen" för att konstruera en DFA $M^{\prime\prime}$ som accepterar samma strängar som $M^{\prime}$.\par
\noindent $M^{\prime\prime}$:s starttillstånd är mängden av alla tillstånd i $M^{\prime}$ som kan nås uitan att avläsa några tecken alls.\par\bigskip
\noindent I vårat fall når vi naturligtvis noden 1 när vi startar, men genom att använda den tomma strängen kan vi nå nod 2 och 3, alltså blir denna mängd $\left\{1,2,3\right\}$.\par
\noindent Notera att de element från vårat alfabet som vi använder är $a$ och $b$, därför kommer vi nu följa var vi hamnar med just dessa två bokstäver.\par\bigskip
\noindent Vi ställer oss frågan, vilka noder som nås från $\left\{1,2,3\right\}$ genom att \textit{bara} avläsa $a$. Börjar vi i nod 1 kan vi ta $\varepsilon$ till 3, loopa runt till 3 igen, och sedan nå nod 4 genom att avläsa $a$. Vi kan inte nå nod 5 och avläsa $a$ för att gå tillbaks till nod 3, ty det kräver att vi avläser $b$. Mängden här blir alltså $\left\{3,4\right\}$.\par\bigskip
\noindent Vårat diagram ser just nu ut på följande:
\begin{figure}[ht]
    \centering
    \begin{tikzpicture}
      \node[state, initial above, initial text=](p0){$1,2,3$};
      \node[state, right of=p0,xshift=1cm](p1){$3,4$};
      \path[-stealth] (p0) edge[above] node{$a$} (p1);
    \end{tikzpicture}
    \caption{}
\end{figure}
\newpage
\noindent Vi undersöker nu villka tillstånd vi kan nå genom att avläsa $b$. Vi kan avläsa $\varepsilon$ och komma till nod 2, varpå vi kan avläsa $b$ för att återigen komma till nod 2. Mängden blir därför $\left\{2\right\}$ och vårat diagram ser ut på följande sätt:\par
\begin{figure}[ht]
    \centering
    \begin{tikzpicture}
      \node[state, initial above, initial text=](p0){$1,2,3$};
      \node[state, left of=p0, xshift=-1cm](p2){$2$};
      \node[state, right of=p0,xshift=1cm](p1){$3,4$};
      \path[-stealth] (p0) edge[above] node{$a$} (p1);
      \path[-stealth] (p0) edge[above] node{$b$} (p2);
    \end{tikzpicture}
    \caption{}
\end{figure}\par
\noindent Vi utför denna frågeställning rekursivt tills dess att mängden vi når är $\O$, det vill säga nu när vi är på nod 2 så frågar vi oss "vilka noder når vi härifrån om vi bara använder $a$" och samma sak för $b$.\par
\noindent Vi får då följande diagram:
\begin{figure}[ht]
    \centering
    \begin{tikzpicture}
      \node[state, initial above, initial text=](p0){$1,2,3$};
      \node[state, left of=p0, xshift=-1cm](p2){$2$};
      \node[state, right of=p0,xshift=1cm](p1){$3,4$};
      \node[state, below of=p1, yshift=-1cm](p3){$5$};
      \node[state, below of=p2, yshift=-1cm](p4){$\O$};
      \path[-stealth] (p0) edge[above] node{$a$} (p1);
      \path[-stealth] (p0) edge[above] node{$b$} (p2);
      \path[-stealth] (p2) edge[loop above] node{$b$} (p2);
      \path[-stealth] (p1) edge[loop right] node{$a$} (p1);
      \path[-stealth] (p1) edge[right] node{$b$} (p3);
      \path[-stealth] (p2) edge[right] node{$a$} (p4);
    \end{tikzpicture}
    \caption{}
\end{figure}\par
\noindent Givetvis fortsätter vi med $\O$ och $5$:
\begin{figure}[ht]
    \centering
    \begin{tikzpicture}
      \node[state, initial above, initial text=](p0){$1,2,3$};
      \node[state, left of=p0, xshift=-1cm](p2){$2$};
      \node[state, right of=p0,xshift=1cm](p1){$3,4$};
      \node[state, below of=p1, yshift=-1cm](p3){$5$};
      \node[state, below of=p2, yshift=-1cm](p4){$\O$};
      \node[state, below of=p3, yshift=-1cm](p5){$3$};
      \path[-stealth] (p0) edge[above] node{$a$} (p1);
      \path[-stealth] (p0) edge[above] node{$b$} (p2);
      \path[-stealth] (p2) edge[loop above] node{$b$} (p2);
      \path[-stealth] (p1) edge[loop right] node{$a$} (p1);
      \path[-stealth] (p1) edge[right] node{$b$} (p3);
      \path[-stealth] (p2) edge[right] node{$a$} (p4);
      \path[-stealth] (p4) edge[loop below] node{$a,b$} (p4);
      \path[-stealth] (p3) edge[above] node{$b$} (p4);
      \path[-stealth] (p3) edge[left] node{$a$} (p5);
    \end{tikzpicture}
    \caption{}
\end{figure}
\newpage
\noindent Forstätter vi med $3$ får vi slutligen:
\begin{figure}[ht]
    \centering
    \begin{tikzpicture}
      \node[state, initial above, initial text=](p0){$1,2,3$};
      \node[state, left of=p0, xshift=-1cm](p2){$2$};
      \node[state, right of=p0,xshift=1cm](p1){$3,4$};
      \node[state, below of=p1, yshift=-1cm](p3){$5$};
      \node[state, below of=p2, yshift=-1cm](p4){$\O$};
      \node[state, accepting below, below of=p3, yshift=-1cm](p5){$3$};
      \path[-stealth] (p0) edge[above] node{$a$} (p1);
      \path[-stealth] (p0) edge[above] node{$b$} (p2);
      \path[-stealth] (p2) edge[loop above] node{$b$} (p2);
      \path[-stealth] (p1) edge[loop right] node{$a$} (p1);
      \path[-stealth] (p1) edge[right] node{$b$} (p3);
      \path[-stealth] (p2) edge[right] node{$a$} (p4);
      \path[-stealth] (p4) edge[loop below] node{$a,b$} (p4);
      \path[-stealth] (p3) edge[above] node{$b$} (p4);
      \path[-stealth] (p3) edge[left] node{$a$} (p5);
      \path[-stealth] (p5) edge[below] node{$b$} (p4);
      \path[-stealth] (p5) edge[right, bend right] node{$a$} (p1);
    \end{tikzpicture}
    \caption{}
\end{figure}
\par\bigskip
\noindent Eftersom inga nya tillstånd (noder) har tillkommit, så är vi klara. Vi kan fråga oss varför denna algoritm alltid terminerar, och då får vi inte glömma att tillståndsmängden bara har ändligt många delmängder.\par
\noindent Notera att de tillstånd i $M^{\prime\prime}$ som innehåller något accepterande tilltånd från $M^{\prime}$ blir de accepterande tillstånden för $M^{\prime\prime}$ (det vill säga, i $M^{\prime}$ (Figure 10) hade nod 3 accepterande tillstånd, så de accepterande tillstånden i $M^{\prime\prime}$ blir de noder med enbart mängden 3).
\par\bigskip
\noindent Varför accepterar $M^{\prime\prime}$ då samma strängar som $M^{\prime}$?\par
\noindent Jo, enligt konstruktion för en sträng $w$ så gäller att:
\begin{itemize}
  \item Det finns en avläsningsväg av $w$ igenom $M^{\prime}$ så att man till slut hamnar i ett accepterande tillstånd $\Lrarr$ Det finns en (unik) avläsningsväg av $w$ igenom $M^{\prime\prime}$ så att man till slut hamnar i ett accepterande tillstånd.
\end{itemize}
