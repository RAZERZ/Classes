\section{Determiniska finita (ändliga) automater [DFA]}
\par\bigskip
\noindent Förr i tiden programmerades datorer genom att man hade "punch-cards", det vill säga kort av styvt papper som hade "hål" som agerade som dagens transistorer gör.\par
\noindent DFA:er påminner lite om hur dessa fungerar, det vill säga att vi ska tänka oss en slags textremsa som inmatning:
\begin{center}
  \begin{tabular}{|c|c|c|c|c|c|}
    \hline
    $a$&$b$&$a$&$b$&$b$&$\cdots$\\
    \hline
  \end{tabular}
\end{center}
\par\bigskip
\noindent Där ändligt många inledande rutor innehåller tecken från en sådan textremsa med alfabet $\Sigma$.
\par\bigskip
\noindent Vi har också en "kontrollmekanism", det vill säga själva DFA:n som vi kan tänka oss har ett "läshuvud" som avläser en ruta i taget (det är viktigt att precisera, DFA:er läser EN ruta i taget).\par
\noindent DFA:n befinner sig alltid i ett tillstånd, av ändliga antal möjliga. När den avläser en ruta så övergår den till ett nytttillstånd samt flyttar läshuvudet ett steg till höger.\par
\noindent Det nya tillståndet beror (endast) på det tidigare tillståndet samt det just avlästna tecknet (i föregående ruta).
\par\bigskip
\noindent De två "viktigaste" tillstånd kallas för \textbf{starttillståndet} och \textbf{accepterande tillståndet}.\par
\noindent Vi antar att en DFA alltid befinner sig i starttillståndet då den sätts igång.\par
\noindent Om en DFA befinner sig i ett accepterande tillstånd då en sträng har avlästs (det vill säga, läshuvudet befinner sig på den första blanka rutan till höger om strängen) så säger vi att DFA:n \textbf{accepterar }strängen. Om strängen inte accepteras, så avvisas strängen. 
\par\bigskip
\subsection{Grafisk beskrivning av en DFA}\hfill\\\par
\noindent Vi väljer att representera tillstånden med hjälp av noder, och tillståndsövergångar med pilar:\par
\begin{figure}[ht]
    \centering
    \begin{tikzpicture}
      \node[state](p){$p$};
      \node[state, right of=p, xshift=1cm](q){$q$};
      \path[-stealth] (p) edge[above, bend left] node{$a$} (q);
    \end{tikzpicture}
    \caption{}
\end{figure}
\par\bigskip
\noindent Figur 1 betyder att om DFA:n befinner sig i tillstånd $p$ och avläser $a$ på textremsan så övergår DFA:n till tillstånd $q$ (och flyttar därmed läshuvudet ett steg till höger).
\par\bigskip
\noindent\textbf{Anmärkning:}\par
\noindent För varje tillstånd och varje tecken från input-alfabetet skall det finnas \textit{preci} en utgående pil som bär detta tecken.
\par\bigskip
\noindent Starttillstånd markeras med en pil mot sig:
\begin{figure}[ht]
    \centering
    \begin{tikzpicture}
      \node[state, initial, initial text=](){};
    \end{tikzpicture}
    \caption{}
\end{figure}
\par\bigskip
\noindent Accepterande tillstånd markeras med en pil från sig \textit{eller} dubbelrand:
\begin{figure}[ht]
    \centering
    \begin{tikzpicture}
      \node[state, accepting right](p){};
      \node[state, accepting, right of=p, xshift=2cm](){};
    \end{tikzpicture}
    \caption{}
\end{figure}
\newpage
\noindent\textbf{Exempel:}
\begin{figure}[ht]
    \centering
    \begin{tikzpicture}
      \node[state, initial, initial text=, accepting above](p0){$p_0$};
      \node[state, accepting left, below left of=p0, yshift=-.5cm](p1){$p_1$};
      \node[state, below right of=p1, yshift=-.5cm](p2){$p_2$};
      \path[-stealth] (p0) edge[loop right] node{$b$} (p0);
      \path[-stealth] (p0) edge[left, bend right] node{$a$} (p1);
      \path[-stealth] (p1) edge[right, bend right] node{$b$} (p0);
      \path[-stealth] (p1) edge[left, bend right] node{$a$} (p2);
      \path[-stealth] (p2) edge[loop right] node{$a,b$} (p2);
    \end{tikzpicture}
    \caption{}
\end{figure}
\par\bigskip
\noindent Kom ihåg att en sträng accepteras om vi befinner oss i ett accepterande tillstånd då hela strängen är avläst.\par
\noindent Vilka strängar accepterar av ovanstående DFA? ($(b\cup ab)^*(\varepsilon\cup a)$)
\par\bigskip
\begin{theo}
  AAntag att $M$ är en DFA.\par
  \noindent Mängden av strängar som $M$ accepterar kallas för \textbf{$M$:s språk} och betecknas $L(M)$
\end{theo}
\par\bigskip
\begin{theo}[Formell definition av DFA]{thm:dfa}
  En \textbf{DFA} är en 5-tupel $M = (Q,\Sigma,\delta,s,F)$ där:\par
  \begin{itemize}
    \item $Q$ är en ändlig mängd (tillståndsmängden)
    \item $\Sigma$ är en ändlig mängd (inputalfabet)
    \item $\delta$ är en funktion (övergångsfunktion) från $Q$x$\Sigma\to Q$
    \item $s\in Q$ ($s$ är Starttillstånd)
    \item $F\subseteq Q$ (accepterande tillstånden)
  \end{itemize}
\end{theo}
\par\bigskip
\noindent För att formellt kunna definiera \textit{acceptans} av en sträng behöver vi en \textbf{utvidgad övergångsfunktion}. Denna kan vi löst definiera på följande vis:
\par\bigskip
\begin{theo}[Utvidgad övergångsfunktion]{thm:transfunc}
  En funktion $T(\text{nuvarande tillstånd}, \text{nuvarande symbol på läshuvu})\to\text{nästa tillstånd}$
\end{theo}
\newpage
\section{Icke-Determiniska Finita (åndliga) automater [NFA]}
\par\bigskip
\noindent En NFA är en generalisering av en DFA på följande sätt:\par
\begin{itemize}
  \item En NFA kan ha flera Starttillstånd
  \item En NFA tillåts kunna läsa flera tecken på en gång, dvs vi kan tillståndsövergångar på formen:
\end{itemize}
\begin{figure}[ht]
    \centering
    \begin{tikzpicture}
      \node[state](p){$p$};
      \node[state, right of=p, xshift=2cm](q){$q$};
      \path[-stealth] (p) edge[above, bend left] node{$abb$} (q);
    \end{tikzpicture}
    \caption{}
\end{figure}
