\section{Analys av reguljära uttryck}\par
\noindent Om $L$ är ett reguljärt uttryck $\Lrarr$ $L$ accpeteras av någon NFA $\Lrarr$ $L$ accpeteras av någon DFA. Detta väcker frågan, vad är den minsta DFA:n vi kan skapa givet ett reguljärt uttryck?
\par\bigskip
\subsection{Minimering av DFA}\hfill\\\par
\begin{theo}
  EEn DFA $M$ är \textit{minimal} om det inte finns en DFA som accepterar samma som $M$ och har färre tillstånd än $M$
\end{theo}
\par\bigskip
\begin{theo}
  LLåt $M$ vara en DFA och $w$ en sträng ($w$ är en sträng över $M$:s alfabet).\par
  \noindent Vi säger att $w$ \textit{driver} DFA:n $M$ från ett tillstnd $p$ till ett annat $q$ om när $M$ startas i tillstånd $P$ med $w$ på tapen så stannar $M$ i tillstånd $q$ 
\end{theo}
\par\bigskip
\noindent\textbf{Exempel:}\par
\begin{figure}[ht!]
    \centering
    \begin{tikzpicture}
      \node[state, initial above, initial text=](p0){$p_0$};
      \node[state, below left of=p0, xshift=-1cm, yshift=-1cm](p1){$p_1$};
      \node[state, right of=p1, xshift=1cm](p2){$p_2$};
      \path[-stealth] (p0) edge[above] node{$a$} (p1);
      \path[-stealth] (p1) edge (p0);
      \path[-stealth] (p1) edge[below] node{$b$} (p2);
      \path[-stealth] (p2) edge (p1);
      \path[-stealth] (p0) edge[left, bend right] node{$b$} (p2);
      \path[-stealth] (p2) edge[right, bend right] node{$a$} (p0);
    \end{tikzpicture}
    \caption{}
\end{figure}\par
\noindent Här driver strängen $aaa$ $M$ från $p_2$ till $p_0$. Strängen $ab$ driver $M$ från $p_2$ till $p_2$
\par\bigskip
\noindent Om ingen sträng driver $M$ från starttillstånd till $q$ så kallas $q$ för \textit{isolerat} tillstånd.
\par\bigskip
\noindent\textbf{Exempel:}
\begin{figure}[ht!]
    \centering
    \begin{tikzpicture}
      \node[state, initial above, initial text=](p0){$p_0$};
      \node[state, below left of=p0, yshift=-1cm, xshift=-1cm](p1){$p_1$};
      \node[state, right of=p0, xshift=1cm](p2){$p_2$};
      \node[state, below of=p2, yshift=-1cm](p3){$p_3$};
      \path[-stealth] (p0) edge[left] node{$a$} (p1);
      \path[-stealth] (p1) edge (p0);
      \path[-stealth] (p1) edge[loop left] node{$b$} (p1);
      \path[-stealth] (p0) edge[loop below] node{$b$} (p0);
      \path[-stealth] (p2) edge[above, bend right] node{$b$} (p0);
      \path[-stealth] (p2) edge[right] node{$a$} (p3);
      \path[-stealth] (p3) edge (p2);
      \path[-stealth] (p3) edge[loop below] node{$b$} (p3);
    \end{tikzpicture}
    \caption{}
\end{figure}\par
\noindent Notera här att $p_2$ och $p_3$ är isolerade tillstånd! Då kan vi lika gärna plocka bort de utan att det påverkar vilka strängar som accepteras.
\newpage
\begin{theo}[Reduktion av ett tillstånd (om möjligt)]{thm:reduct}
  Låt $M$ vara en DFA (utan isolerade tillstånd) och antag att $p$ och $q$ är olika tillstånd i $M$.\par
  \noindent Antag att följande villkor gäller för varje sträng $w$ (även $\varepsilon$):
  \begin{itemize}
    \item $w$ driver $M$ från $p$ till ett accepterande tillstånd om och endast om $w$ driver $M$ från $q$ till ett accepterande tillstånd
  \end{itemize}\par
  \noindent Låt nu $M^{\prime}$ vara som $M$ utom att $q$ tas bort, och varje övergång i Figure 52 ersätts med Figure 53.\par
  \noindent Då gäller $L(M^{\prime}) = L(M)$ och $M^{\prime}$ har färre tillstånd än $M$
\end{theo}\par
\begin{figure}[ht!]
    \centering
    \begin{tikzpicture}
      \node[state](p0){$r$};
      \node[state, right of=p0, xshift=1cm](p1){$q$};
      \path[-stealth] (p0) edge[above] node{$\sigma$} (p1);
    \end{tikzpicture}
    \caption{}
\end{figure}
\begin{figure}[ht!]
    \centering
    \begin{tikzpicture}
      \node[state](p0){$r$};
      \node[state, right of=p0, xshift=1cm](p1){$p$};
      \path[-stealth] (p0) edge[above] node{$\sigma$} (p1);
    \end{tikzpicture}
    \caption{}
\end{figure}\par
\noindent Detta påminner lite om tillståndeliminering, men vi vill inte skapa reguljära uttryck utan det är ok att bara koppla bort pilarna från en nod till en annan.
\par\bigskip
\begin{theo}[Särskiljandealgoritmen för minimering av en DFA]{thm:sfmaedj}
\end{theo}
\par\bigskip
\noindent\textbf{Exempel:}\par
\noindent Låt $M$ vara följande DFA: (\textbf{KOPIERA FRÅN ANTECKNINGAR})
\par\bigskip
\noindent Här är det enklare att skriva en tabell över tillståndsövergångarna:
\par\bigskip
\begin{center}
  \begin{tabular}{c|c|c|c|c|c|c|c}
    &1&2&3&4&5&6&7\\
    \hline\\
    $a$&3&3&3&3&3&3&7\\
    $a$&1&1&6&4&4&5&7
  \end{tabular}
\end{center}
\par\bigskip
\begin{center}
  \begin{tabular}{c|c}
    Nivå&Sönderdelningar\\
    \hline\\
    1&$\left\{7\right\}\left\{1,2,3,4,5,6\right\}$ (initial \& accept. tillst)\\
    2&$\left\{7\right\}\left\{1,2,3,4,5\right\}\left\{6\right\}$\\
    3&$\left\{7\right\}\left\{1,2,4,5\right\}\left\{3\right\}\left\{6\right\}$\\
    4&$\left\{7\right\}\left\{1,2,4,5\right\}\left\{3\right\}\left\{6\right\}$
  \end{tabular}
\end{center}
\par\bigskip
\noindent Hur bildas nivå $n+1$ efter att nivå $n$ är bildad?\par
\begin{itemize}
  \item Om två tillstånd tillhör olika delar på nivå $n$, så gör de även det på nivå $n+1$
  \item Antag att $p$ och $q$ är tillstånd som tillhör samma del på nivå $n$, $p$ och $q$ ska särskiljas, dvs placeras i olika delar på nivå $n+1$ \textbf{om} det finns ett $\sigma\in\Sigma$ som driver DFA:n från $p$ till en annan del på nivå $n$ än $\sigma$ som driver DFA:n från $q$ 
\end{itemize}
\par\bigskip
\noindent Efter att vi kom till nivå 4 såg vi att vi inte fick några ändringar, nu kan vi konstruera en DFA $M^{\prime}$ som är minimal.\par
\noindent $M^{\prime}$:s tillstånd är delarna på den sista nivån (4). (\textbf{KOPIERA FRÅN ANTECKNINGAR})\par
\noindent Starttillståndet hos $M^{\prime}$ är den del/mängd som innehåller $M$:s starttillstånd. Samma gäller för accepterande tillstånd.\par
\noindent Givet en del/mängd $P$ och tecken $\sigma$ så lägg till en övergång $P\to Q$ via $\sigma$ där $Q$ är den unika del sp att det finns $p\in P$ och $q\in Q$ så att övergången från $p\to q$ via $\sigma$ finns i $M$ 
\par\bigskip
\noindent\textbf{Övning:}\par
\noindent Betrakta följande tillståndsövergångstabell:
\par\bigskip
\begin{center}
  \begin{tabular}{c|c|c|c|c|c|c|c|c|}
    &1&2&3&4&5&6&7&8\\
    \hline\\
    $a$&1&1&2&3&5&7&8&2\\
    $b$&1&4&6&5&5&5&6&6
  \end{tabular}
\end{center}
\par\bigskip
\noindent Tillstånd $4,6,7$ är accepterande och tillstånd $4$ är starttillståndet. Vi utför algoritmen:
\par\bigskip
\begin{center}
  \begin{tabular}{c|c}
    Nivå&Sönderdelningar\\
    \hline\\
  1&$\left\{1,2,3,5,8\right\}\left\{4,6,7\right\}$\\
  2&$\left\{1,4,5\right\}\left\{2,3,8\right\}$
  \end{tabular}
\end{center}
