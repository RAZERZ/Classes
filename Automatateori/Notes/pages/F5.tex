\section{Analys av reguljära uttryck}\par
\noindent Om $L$ är ett reguljärt uttryck $\Lrarr$ $L$ accpeteras av någon NFA $\Lrarr$ $L$ accpeteras av någon DFA. Detta väcker frågan, vad är den minsta DFA:n vi kan skapa givet ett reguljärt uttryck?
\par\bigskip
\subsection{Minimering av DFA}\hfill\\\par
\begin{theo}
  EEn DFA $M$ är \textit{minimal} om det inte finns en DFA som accepterar samma som $M$ och har färre tillstånd än $M$
\end{theo}
\par\bigskip
\begin{theo}
  LLåt $M$ vara en DFA och $w$ en sträng ($w$ är en sträng över $M$:s alfabet).\par
  \noindent Vi säger att $w$ \textit{driver} DFA:n $M$ från ett tillstnd $p$ till ett annat $q$ om när $M$ startas i tillstånd $P$ med $w$ på tapen så stannar $M$ i tillstånd $q$ 
\end{theo}
\par\bigskip
\noindent\textbf{Exempel:}\par
\begin{figure}[ht!]
    \centering
    \begin{tikzpicture}
      \node[state, initial above, initial text=](p0){$p_0$};
      \node[state, below left of=p0, xshift=-1cm, yshift=-1cm](p1){$p_1$};
      \node[state, right of=p1, xshift=1cm](p2){$p_2$};
      \path[-stealth] (p0) edge[above] node{$a$} (p1);
      \path[-stealth] (p1) edge (p0);
      \path[-stealth] (p1) edge[below] node{$b$} (p2);
      \path[-stealth] (p2) edge (p1);
      \path[-stealth] (p0) edge[left, bend right] node{$b$} (p2);
      \path[-stealth] (p2) edge[right, bend right] node{$a$} (p0);
    \end{tikzpicture}
    \caption{}
\end{figure}\par
\noindent Här driver strängen $aaa$ $M$ från $p_2$ till $p_0$. Strängen $ab$ driver $M$ från $p_2$ till $p_2$
\par\bigskip
\noindent Om ingen sträng driver $M$ från starttillstånd till $q$ så kallas $q$ för \textit{isolerat} tillstånd.
\par\bigskip
\noindent\textbf{Exempel:}
\begin{figure}[ht!]
    \centering
    \begin{tikzpicture}
      \node[state, initial above, initial text=](p0){$p_0$};
      \node[state, below left of=p0, yshift=-1cm, xshift=-1cm](p1){$p_1$};
      \node[state, right of=p0, xshift=1cm](p2){$p_2$};
      \node[state, below of=p2, yshift=-1cm](p3){$p_3$};
      \path[-stealth] (p0) edge[left] node{$a$} (p1);
      \path[-stealth] (p1) edge (p0);
      \path[-stealth] (p1) edge[loop left] node{$b$} (p1);
      \path[-stealth] (p0) edge[loop below] node{$b$} (p0);
      \path[-stealth] (p2) edge[above, bend right] node{$b$} (p0);
      \path[-stealth] (p2) edge[right] node{$a$} (p3);
      \path[-stealth] (p3) edge (p2);
      \path[-stealth] (p3) edge[loop below] node{$b$} (p3);
    \end{tikzpicture}
    \caption{}
\end{figure}\par
\noindent Notera här att $p_2$ och $p_3$ är isolerade tillstånd! Då kan vi lika gärna plocka bort de utan att det påverkar vilka strängar som accepteras.
\newpage
\begin{theo}[Reduktion av ett tillstånd (om möjligt)]{thm:reduct}
  Låt $M$ vara en DFA (utan isolerade tillstånd) och antag att $p$ och $q$ är olika tillstånd i $M$.\par
  \noindent Antag att följande villkor gäller för varje sträng $w$ (även $\varepsilon$):
  \begin{itemize}
    \item $w$ driver $M$ från $p$ till ett accepterande tillstånd om och endast om $w$ driver $M$ från $q$ till ett accepterande tillstånd
  \end{itemize}\par
  \noindent Låt nu $M^{\prime}$ vara som $M$ utom att $q$ tas bort, och varje övergång i Figure 52 ersätts med Figure 53.\par
  \noindent Då gäller $L(M^{\prime}) = L(M)$ och $M^{\prime}$ har färre tillstånd än $M$
\end{theo}\par
\begin{figure}[ht!]
    \centering
    \begin{tikzpicture}
      \node[state](p0){$r$};
      \node[state, right of=p0, xshift=1cm](p1){$q$};
      \path[-stealth] (p0) edge[above] node{$\sigma$} (p1);
    \end{tikzpicture}
    \caption{}
\end{figure}
\begin{figure}[ht!]
    \centering
    \begin{tikzpicture}
      \node[state](p0){$r$};
      \node[state, right of=p0, xshift=1cm](p1){$p$};
      \path[-stealth] (p0) edge[above] node{$\sigma$} (p1);
    \end{tikzpicture}
    \caption{}
\end{figure}\par
\noindent Detta påminner lite om tillståndeliminering, men vi vill inte skapa reguljära uttryck utan det är ok att bara koppla bort pilarna från en nod till en annan.
\par\bigskip
\begin{theo}[Särskiljandealgoritmen för minimering av en DFA]{thm:sfmaedj}
\end{theo}
\par\bigskip
\noindent\textbf{Exempel:}\par
\noindent Låt $M$ vara följande DFA: 
\begin{figure}[ht!]
    \centering
    \begin{tikzpicture}
      \node[state, accepting above](p0){$5$};
      \node[state, accepting below, xshift=-1cm, left of=p0](p1){$4$};
      \node[state, accepting below, right of=p0, xshift=1cm](p2){$6$};
      \node[state, above right of=p2, xshift=.5cm, yshift=1cm](p3){$7$};
      \node[state, accepting below, below of=p0, yshift=-2cm](p4){$3$};
      \node[state, initial left, initial text=, accepting below, left of=p4, xshift=-1cm](p5){$2$};
      \node[state, accepting below, yshift=-1cm, below of=p5, xshift=1cm](p6){$1$};
      \path[-stealth] (p1) edge[loop left] node{$b$} (p1);
      \path[-stealth] (p3) edge[loop right] node{$a,b$} (p3);
      \path[-stealth] (p4) edge[loop right] node{$a$} (p4);
      \path[-stealth] (p6) edge[loop left] node{$b$} (p6);
      \path[-stealth] (p0) edge[above] node{$b$} (p1);
      \path[-stealth] (p2) edge[above] node{$b$} (p0);
      \path[-stealth] (p2) edge[right] node{$a$} (p3);
      \path[-stealth] (p1) edge[right] node{$a$} (p4);
      \path[-stealth] (p0) edge[right] node{$a$} (p4);
      \path[-stealth] (p4) edge[right] node{$b$} (p2);
      \path[-stealth] (p5) edge[above] node{$a$} (p4);
      \path[-stealth] (p5) edge[left] node{$b$} (p6);
      \path[-stealth] (p6) edge[right] node{$a$} (p4);
    \end{tikzpicture}
    \caption{}
\end{figure}

\par\bigskip
\noindent Här är det enklare att skriva en tabell över tillståndsövergångarna:
\par\bigskip
\begin{center}
  \begin{tabular}{c|c|c|c|c|c|c|c}
    &1&2&3&4&5&6&7\\
    \hline\\
    $a$&3&3&3&3&3&3&7\\
    $a$&1&1&6&4&4&5&7
  \end{tabular}
\end{center}
\par\bigskip
\begin{center}
  \begin{tabular}{c|c}
    Nivå&Sönderdelningar\\
    \hline\\
    1&$\left\{7\right\}\left\{1,2,3,4,5,6\right\}$ (initial \& accept. tillst)\\
    2&$\left\{7\right\}\left\{1,2,3,4,5\right\}\left\{6\right\}$\\
    3&$\left\{7\right\}\left\{1,2,4,5\right\}\left\{3\right\}\left\{6\right\}$\\
    4&$\left\{7\right\}\left\{1,2,4,5\right\}\left\{3\right\}\left\{6\right\}$
  \end{tabular}
\end{center}
\par\bigskip
\noindent Hur bildas nivå $n+1$ efter att nivå $n$ är bildad?\par
\begin{itemize}
  \item Om två tillstånd tillhör olika delar på nivå $n$, så gör de även det på nivå $n+1$
  \item Antag att $p$ och $q$ är tillstånd som tillhör samma del på nivå $n$, $p$ och $q$ ska särskiljas, dvs placeras i olika delar på nivå $n+1$ \textbf{om} det finns ett $\sigma\in\Sigma$ som driver DFA:n från $p$ till en annan del på nivå $n$ än $\sigma$ som driver DFA:n från $q$ 
\end{itemize}
\par\bigskip
\noindent Efter att vi kom till nivå 4 såg vi att vi inte fick några ändringar, nu kan vi konstruera en DFA $M^{\prime}$ som är minimal.\par
\noindent $M^{\prime}$:s tillstånd är delarna på den sista nivån (nivå 4).

\begin{figure}[ht!]
    \centering
    \begin{tikzpicture}
      \node[state, accepting below](p0){$6$};
      \node[state, above of=p0, yshift=1cm](p1){$7$};
      \node[state, initial left, accepting below, initial text=, below left of=p0, xshift=-2cm, yshift=-1cm](p2){$1,2,4,5$};
      \node[state, accepting below, right of=p2, xshift=4cm](p3){$3$};
      \path[-stealth] (p1) edge[loop right] node{$a,b$} (p1);
      \path[-stealth] (p2) edge[loop above] node{$b$} (p2);
      \path[-stealth] (p3) edge[loop right] node{$a$} (p3);
      \path[-stealth] (p0) edge[right] node{$a$} (p1);
      \path[-stealth] (p0) edge[above] node{$b$} (p2);
      \path[-stealth] (p2) edge[below] node{$a$} (p3);
      \path[-stealth] (p3) edge[above] node{$b$} (p0);
    \end{tikzpicture}
    \caption{}
\end{figure}

\noindent Starttillståndet hos $M^{\prime}$ är den del/mängd som innehåller $M$:s starttillstånd. Samma gäller för accepterande tillstånd.\par
\noindent Givet en del/mängd $P$ och tecken $\sigma$ så lägg till en övergång $P\to Q$ via $\sigma$ där $Q$ är den unika del sp att det finns $p\in P$ och $q\in Q$ så att övergången från $p\to q$ via $\sigma$ finns i $M$ 
\par\bigskip
\noindent\textbf{Övning:}\par
\noindent Betrakta följande tillståndsövergångstabell:
\par\bigskip
\begin{center}
  \begin{tabular}{c|c|c|c|c|c|c|c|c|}
    &1&2&3&4&5&6&7&8\\
    \hline\\
    $a$&1&1&2&3&5&7&8&2\\
    $b$&1&4&6&5&5&5&6&6
  \end{tabular}
\end{center}
\par\bigskip
\noindent Tillstånd $4,6,7$ är accepterande och tillstånd $4$ är starttillståndet. Vi utför algoritmen:
\par\bigskip
\begin{center}
  \begin{tabular}{c|c}
    Nivå&Sönderdelningar\\
    \hline\\
  1&$\left\{1,2,3,5,8\right\}\left\{4,6,7\right\}$\\
  2&$\left\{1,5\right\}\left\{2,3,8\right\}$\\
3&$\left\{1,5\right\}\left\{2\right\}\left\{3,8\right\}\left\{4\right\}\left\{6\right\}\left\{7\right\}$
  \end{tabular}
\end{center}
\par\bigskip
\noindent DFA:n blir följande:
\par\bigskip
\noindent När man väl är klar med konstruktionen så spelar det ingen roll vad tillstånden kallas (det vill säga, man behöver inte använda namnet som är mängder utan man får lov att säga $p_0$ osv)
\par\bigskip
\begin{theo}
  LLåt $L$ vara ett språk över något alfabet $\Sigma$. Två strängar över alfabet ($x,y\in\Sigma^*$) \textit{särskiljs} av $L$ om det finns $z\in\Sigma^*$ så att precis en av $xz$ och $yz$ tillhör språket $L$.
  \par\bigskip
  \noindent Notera! Alla dessa strängar är tillåtna att vara $\varepsilon$
\end{theo}
\par\bigskip
\begin{theo}
  EEn mängd $A\subseteq\Sigma^*$ särskiljs av $L$ om för varje par $x,y\in A$ så att $x\neq y$ så finns någon sträng $z$ så att Definition 9.5 gäller (det finns $z\in\Sigma^*$ så att precis en av $xz$ och $yz$ tillhör $L$)
\end{theo}
\par\bigskip
\noindent\textbf{Exempel:}\par
\noindent Låt $L_1 = \left\{w\in\left\{a,b\right\}^*:\left|w\right|\equiv_{\text{mod}3} 0\right\}$. Låt $x=aa$ och $y=a$, här är $x\neq a$ och vi påstår att den särskiljs av $L$ eftersom vi kan hitta $z$ så att $xz\in L_1$ men att $yz\notin L_1$.\par
\noindent Välj exempelvis $z=a$, då är $xz=aaa$ och $\left|xz\right|=3\equiv0$, men $yz=aa$ som $\left|yz\right|\equiv2\neq0$
\par\bigskip
\noindent\textbf{Exempel:}\par
\noindent Låt $L_2 = \left\{w\in\left\{a,b\right\}^*: w\text{ innehåller minst 2 } a\text{:n}\right\}$ Låt $x=aab$ och  $y=ab$
\noindent Om $z=\varepsilon$ så särskiljs $x$ och $y$ av $L_2$
\par\bigskip
\noindent\textbf{Exempel:}\par
\noindent Låt $A=\left\{a,aa,aaa\right\}$. Vi påstår att den här mängden särskiljs av $L_1$\par
\noindent För att visa detta måste vi visa att varje par har ett $z$ så att Definition 9.5 gäller.\par
\noindent För att underlätta gör vi en tabell:
\par\bigskip
\begin{center}
  \begin{tabular}{c|c|c|c}
    &$a$&$aa$&$aaa$\\
    \hline
    $a$&X&$a$&$\varepsilon$\\
    $aa$&&X&$\varepsilon$\\
    $aaa$&&&X
  \end{tabular}
\end{center}
\par\bigskip
\noindent Notera att diagonalen inte fylls i. Tabellen visar exempel på $z$ så att exakt en av $xz$ och $yz$ tillhör $L_1$ om $x\neq y$ och $x$ är på lodräta axeln och $y$ på vågräta axeln.
\newpage
\begin{theo}
  AAntag att mängden $A$ särskiljs av språket $L$ och att $A$ innehåller $n$ strängar (de är givetvis över samma alfabet).\par
  \noindent Då måste varje DFA $M$ sådant att $L(M)=L$ ha minst $n$ tillstånd.
  \par\bigskip
  \noindent Det följer att om $L(M)=L$ och $M$ har exakt $n$ tillstånd så är $M$ minimal
\end{theo}
\par\bigskip
\noindent Det följer då från satsen (och de föregående exempel) att varje DFA som accepterar $L_1$ (från föregående exempel) har minst 3 tillstånd.
\par\bigskip
\noindent Eftersom följande DFA accepterar $L_1$ och har 3 tillstånd, så är den minimal. 
\begin{figure}[ht!]
    \centering
    \begin{tikzpicture}
      \node[state, initial left, accepting below, initial text=](p0){$p_0$};
      \node[state, right of=p0, xshift=1cm](p1){$p_1$};
      \node[state, right of=p1, xshift=1cm](p2){$p_2$};
      \path[-stealth] (p0) edge[above] node{$a,b$} (p1);
      \path[-stealth] (p1) edge[above] node{$a,b$} (p2);
      \path[-stealth] (p2) edge[below, bend left] node{$a,b$} (p0);
    \end{tikzpicture}
    \caption{}
\end{figure}
\par\bigskip
\begin{prf}[Sats 9.7]{prf:woribh}
  Antag att $A\subseteq\Sigma^*$ som särskiljs av $L$. Antag även att $A$ innehåller minst $n$ strängar. Låt $M$ vara en DFA som accepterar $L$.\par
  \noindent Låt $A = \left\{x_1,x_2,\cdots,x_n\right\}$. Så om $x_i, x_j\in A$ och $i\neq j$ så finns något $z\in\Sigma^*$ så att precis en av $x_iz$ och $x_jz$ tillhör $L$, dvs accepteras av $M$
  \par\bigskip
  \noindent Det betyder ju att när vi läser av $x_i$ resp. $x_j$ så kommer vi till olika tillstånd på grund av att bara en av de accepteras av $M$. Kommer man till samma tillstånd skulle båda strängar accepteras eller ingen av de. 
  \par\bigskip
  \noindent Alternativt, det betyder att $x_i$ driver $M$ från starttillståndet $s$ till ett annat tillstånd än vad $x_j$ driver $M$ från $s$. Därför måste det finnas minst $n$ tillstånd i $M$.
\end{prf}
\par\bigskip
\noindent Vad händer om det finns en oändligt stor mängd $A$ som särskiljs av ett visst språk?
