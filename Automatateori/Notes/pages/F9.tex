\section{Pushdownautomater [PDA]}\par
\subsection{En liten informell beskrivning:}\hfill\\
\par\bigskip
\begin{theo}[PDA]{thm:pda}
  En PDA har 2 taper, en för inputsträngen och en som kallas för \textit{stacken} (ett slags minne).\par
  \noindent En PDA har även 2 alfabet, 1 input-alfabet (det som används på input tapen) och ett stackalfabet (används på stacken)
  \par\bigskip
  \noindent Är icke-deterministisk, har ändligt många tillstånd, samt ändliga alfabet för både input\& stackalfabet 
\end{theo}
\par\bigskip
\noindent När PDA:n jobbar så konsumerar den en delsträng, likt en NFA. Den avläser alltså inte högst ett tecken, utan kan avläsa flera.\par
\noindent Samtidigt som den avläser från inputtapen så byter den en delsträng överst på stacken med någon sträng.\par
\noindent Samtidigt gör den även en tillståndsövergång.
\par\bigskip
\noindent Om vi struntar i stacken så fungerar en PDA som en NFA.
\par\bigskip
\noindent En PDA har dock enbart 1 starttillstånd. (Vera hävdar att detta är lite onödig konvention).
\par\bigskip
\subsection{Grafisk beskrivning av PDA}\hfill\\\par
\noindent Ganska lik en NFA, men på pilarna måste vi lägga till vad som görs på stacken:
\begin{figure}[ht!]
    \centering
    \begin{tikzpicture}
      \node[state](p0){$p_0$};
      \node[state, right of=p0, xshift=1.5cm](p1){$p_1$};
      \path[-stealth] (p0) edge[above, bend left] node{$u,\beta/\gamma$} (p1);
    \end{tikzpicture}
    \caption{Byte av $\beta$ mot $\gamma$ överst i stacken}
\end{figure}\par
\noindent Såklart så är $u$ sträng över inputalfabet och $\beta$, $\gamma$ sträng över stackalfabet.
\par\bigskip
\noindent PDA:n kan gå från tillstånd $p_0$ till $p_1$ genom att avläsa strängen $u$ på input-tapen ($u$ måste börja där läshuvudet står på input-tapen) och ersätta $\beta$ med $\gamma$ överst på stacken (förutsätter att $\beta$ står överst på stacken).\par
\noindent Det är tillåtet att $u, \beta, \gamma$ är $\varepsilon$
\par\bigskip
\noindent\textbf{Anmärkning:}\par
\noindent Om $\beta$ inte finns överst på stacken, så kan inte den övergången tas.
\newpage
\subsection{Kriterier för acceptans hos PDA}\hfill\\\par
\noindent En PDA $M$ accepterar en sträng $w$ över inputalfabet om $w$ kan delas upp $w = v_1\cdots v_n$ och man kan avläsa delarna $v_1,v_2,\cdots,v_n$ genom att följa övergångar:
\begin{figure}[ht!]
    \centering
    \begin{tikzpicture}
      \node[state, initial left, initial text=](p0){$p_0$};
      \node[state, right of=p0, xshift=1.5cm](p1){$p_1$};
      \node[state, below of=p0, yshift=-1cm](p2){$p_1$};
      \node[state, below of=p1, yshift=-1cm](p3){$p_2$};
      \node[state, below of=p2, yshift=-1cm](p4){$p_n$};
      \node[state, below of=p3, yshift=-1cm, accepting right](p5){$p_{n+1}$};
      \path[-stealth] (p0) edge[above] node{$v_1,\varepsilon/\gamma_1$} (p1);
      \path[-stealth] (p2) edge[above] node{$v_1,\beta_2/\gamma_2$} (p3);
      \path[-stealth] (p4) edge[above] node{$v_n,\beta_n/\varepsilon$} (p5);
    \end{tikzpicture}
    \caption{}
\end{figure}\par
\noindent där $p_1$ är starttillstånd, $p_{n+1}$ är accepterande och stacken är tom i början och när hela $w$ har avlästs. Det är tillåtet att $p_i = p_j$ även om $i\neq j$
\par\bigskip
\subsection{Språk hos PDA}\hfill\\\par
\noindent Om $M$ är en PDA så $L(M) = \left\{w:\text{$w$ är en sträng över $M$:s inputalfabet som accepteras av $M$}\right\}$\par
\noindent Detta är $M$:s språk.
\par\bigskip
\noindent\textbf{Exempel:}\par
\noindent Betrakta följande språk $L=\left\{a^nb^n:n\in\N\right\}$ (icke-sammanhangsfritt språk).\par
\noindent \textit{Konstruera en PDA som accepterar $L$}:\par
\noindent Iden är att vi bara vill läsa av så många $a$ som möjligt, och sedan för varje $a$ lägger vi till något i stacken. Då har vi koll på \textit{hur många} $a$ vi har läst. Låt $M$ vara följande PDA:
\begin{figure}[ht!]
    \centering
    \begin{tikzpicture}
      \node[state, initial left, initial text=](p0){$p_0$};
      \node[state, accepting right, right of=p0, xshift=1.5cm](p1){$p_1$};
      \path[-stealth] (p0) edge[loop above] node{$a,\varepsilon/x$} (p0);
      \path[-stealth] (p0) edge[above] node{$\varepsilon,\varepsilon/\varepsilon$} (p1);
      \path[-stealth] (p1) edge[loop above] node{$b,x/\varepsilon$} (p1);
    \end{tikzpicture}
    \caption{}
\end{figure}\par
\noindent Input-alfabet är $\left\{a,b\right\}$ och stackalfabet är $\left\{x\right\}$
\par\bigskip
\noindent Eftersom PDA:er är icke-deterministiska så finns det flera olika sätt att "köra" PDA:n. Vi ska kolla på hur denna körning ser ut:\par
\begin{center}
  \begin{tabular}{c|c|c}
    Tillstånd&Input-tape&stacken\\
    $p_0$&\textbf{\textit{a}}$abb$&$\varepsilon$\\
    $p_0$&$a$\textbf{\textit{a}}$bb$&$x$\\
    $p_0$&$aa$\textit{\textbf{b}}$b$&$xx$\\
    $q$&$aa$\textit{\textbf{b}}$b$&$xx$\\
    $q$&$aab$\textit{\textbf{b}}&$x$\\
    $q$&$aabb$&$x$\\
    $q$&$aabb$&$\varepsilon$\\
  \end{tabular}
\end{center}
\par\bigskip
\noindent Denna körning vittnar om att $aabb$ accepteras av $M$.\par
\noindent Däremot skulle $aaabb$ \textit{inte} accepteras av följande DFA, detta eftersom det kommer finnas ett för mycket $x$ i stacken men inget $b$ att byta ut mot $\varepsilon$.
\par\bigskip
\begin{theo}
  FFör varje CFG $G$ så finns det en PDA $M$ som har samma språk som $G$, dvs $L(M) = L(G)$
\end{theo}
\par\bigskip
\noindent Detta kan illustreras med en så kallad "top-down" parser metoden:\par
\noindent Låt $G = (\Sigma_n,\Sigma_T,P,S)$ där $P = \left\{A_1\to w_1, A_2\to w_2,\cdots, A_n\to w_n\right\}$ och $\Sigma_T = \left\{\sigma_1,\cdots, \sigma_k\right\}$
\par\bigskip
\noindent För att konstruera $M$ med detta, låter vi $M$:s inputalfabet vara $\Sigma_T$ och $M$:s stackalfabet vara $\Sigma_T\cup\Sigma_N$\par
\noindent Låt $M$ beskrivas med (kom ihåg, $S$ är startsymbol):
\begin{figure}[ht!]
    \centering
    \begin{tikzpicture}
       \node[state, initial left, initial text=](p0){$p_0$};
       \node[state, accepting right, right of=p0, xshift=1.5cm](p1){$p_1$};
       \path[-stealth] (p0) edge[above] node{$\varepsilon,\varepsilon/S$} (p1);
       \path[-stealth] (p1) edge[loop above] node{$\left(\varepsilon,A_1/w_1\right)\cup\left(\varepsilon,A_2/w_2\right)\cup\cdots\cup\left(\varepsilon,A_n/w_n\right)$} (p1);
       \path[-stealth] (p1) edge[loop below] node{$\left(\sigma_1,\sigma_1/\varepsilon\right)\cup\cdots\left(\sigma_k, \sigma_k/\varepsilon\right)$} (p1);
    \end{tikzpicture}
    \caption{}
\end{figure}
\par\bigskip
\noindent Nu gäller att $L(M) = L(G)$
\par\bigskip
\noindent\textbf{Exempel:}\par
\noindent Låt $G$ vara följande CFG:\par
$L(G) = \left\{a^nb^n:n\in\N\right\}$\par
$P = \left\{S\to aSb, S\to\varepsilon\right\}$\par
\noindent Konstruera en PDA med samma språk som grammatiken $G$ m.h.a top-down parser metoden:
\begin{figure}[ht!]
    \centering
    \begin{tikzpicture}
      \node[state, initial left, initial text=](p0){$p_0$};
      \node[state, accepting right, right of=p0, xshift=1.5cm](p1){$p_1$};
      \path[-stealth] (p0) edge[above] node{$\varepsilon,\varepsilon/S$} (p1);
      \path[-stealth] (p1) edge[loop above] node{$\left(\varepsilon,S/aSb\right)\cup\left(\varepsilon,S/\varepsilon\right)$} (p1);
      \path[-stealth] (p1) edge[loop below] node{$\left(a,a/\varepsilon\right)\cup\left(b,b/\varepsilon\right)$} (p1);
    \end{tikzpicture}
    \caption{}
\end{figure}
\par\bigskip
\begin{theo}
  FFör varje PDA $M$ finns en CFG som har samma språk som $M$, dvs $L(M) = L(G)$
\end{theo}
\par\bigskip
\noindent Från Sats 13.3 följer det att ett språk är sammanhangsfritt (CFL) $\Lrarr$ någon PDA accepterar språket.
