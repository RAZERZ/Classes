\section{Turingmaskiner [TM:er]}\par
\noindent Turingmaskinerna påminner mest om "vanliga" datorer och att pogrammera i maskindatorer. Skillnaden mellan turingmaskinerna och finita automater är att de kan skriva på sin texttape, och flytta läshuvudet inte bara på ett håll, men kan hoppa och gå fram och tillbaka.\par
\noindent Detta motsvarar att man har minne, inte bara skriva output och läsa input.
\par\bigskip
\noindent\textbf{Informell beskrivning:}\par
\noindent En TM har en tape som är oändlig i båda riktningar (höger och vänster, likt $\Z$):
\begin{center}
  \begin{tabular}{c|c|c|c|c}
    \hline
    $\cdots$&$a$&$b$&\#&$\cdots$\\
    \hline
  \end{tabular}
\end{center}
\par\bigskip
\noindent Varje ruta innehåller ett tecken varav ett kan vara "blankteknet" \#\par
\noindent TM har en kontrollmekanism som befinner sig i ett ändligt många möjliga tillstånd\par
\noindent Ändligt många tecken får användas på tapen.
\par\bigskip
\noindent Mängden av tecken som får användas för att skriva inputsträngar med kallas TM:s input-alfabet\par
\noindent Tecknet "\#" ingår \textit{inte} i inputalfabetet\par
\noindent Kontrollmekanismen och läshuvudet kan:\par
\begin{itemize}
  \item Byta tillstånt (inkl stanna i samma tillstånd) och skriva tecken i rutan som avläses
  \item Byta tillstånd och flytta läshuvudet en ruta till vänster eller till höger
\end{itemize}\par
\noindent Vad en Turingmaskin gör i en given situation beror bara på vilket tecken den avläses, och vilket tillstånd den befinner sig i 
\par\bigskip
\noindent En TM har (exakt) ett starttilstånd och (exakt) stopptillstånd (på Engelska "Halting state", betecknas oftast med $H$). 
\par\bigskip
\noindent En TM sätts alltid igång i starttilståndet, och fortsätter att arbeta (göra tillståndsövergångar och eventuellt skriva på tapen) tills den hamnar i stopptillståndet, om detta sker.\par
\noindent Det är inte säkert att en TM någonsin hamnar i stopptillstånd, precis med vanliga datorer så behöver inte varje program terminera (exvis en while-true loop)
\par\bigskip
\subsection{Tape-konfigurationer}\hfill\\\par
\noindent Det kommer vara bekvämt att ha notation och veta vad som står på tapen. Vi kommer alltid anta att ändligt många rutor är icke-blanka. I någon mening betyder det att vi inte använt mer än ändlig del av minnet, samma med inputsträng (ändlig)
\par\bigskip
\noindent Vi vill också kunna säga vart läshuvudet står i förhållande till de icke-tomma rutorna på tapen.
\par\bigskip
\noindent Vi antar att TM:s tape bara har ändligt många icke-blanka rutor när den startas (och därmed vid varje senare tillfälle)
\par\bigskip
\noindent Om tapen ser ut så här:
\begin{center}
  \begin{tabular}{c|c|c|c|c|c|c|c|c}
    \hline
    $\cdots$&\#&$\sigma_1$&$\cdots$&$\sigma_k$&$\cdots$&$\sigma_n$&\#&$\cdots$\\
    \hline
  \end{tabular}
\end{center}\par
\noindent och $\sigma_1\neq\#$, $\sigma_n\neq\#$ och alla turot till vänster om $\sigma_1$ och till höger om $\sigma_n$ är blanka, så är tapekonfigugariotnen
\begin{equation*}
  \begin{gathered}
    \sigma_1\cdots\sigma_k\cdots\sigma_n
  \end{gathered}
\end{equation*}\par
\noindent Här är läshuvudet på $\sigma_k$
\par\bigskip
\noindent Om tapen ser ut så här:
\begin{center}
  \begin{tabular}{c|c|c|c|c|c|c}
    \hline
    $\cdots$&\#&$\sigma_1$&$\cdots$&$\sigma_n$&\#&$\cdots$\\
    \hline
  \end{tabular}
\end{center}\par
\noindent och $\sigma_n\neq\#$ och alla rutor till vänster om $\sigma_1$ är blanka och alla rutor till höger om $\sigma_n$ är blanka, så är tapekonfigurationen:
\begin{equation*}
  \begin{gathered}
    \sigma_1\cdots\sigma_n
  \end{gathered}
\end{equation*}\par
\noindent Notera att läshuvudet står på $\sigma_1$. Notera även att $\sigma_1$ kan vara blankteknet
\newpage
\noindent Om tapen ser ut såhär:
\begin{center}
  \begin{tabular}{c|c|c|c|c|c|c}
    \hline
    $\cdots$&\#&$\sigma_1$&$\cdots$&$\sigma_n$&\#&$\cdots$\\
    \hline
  \end{tabular}
\end{center}\par
\noindent och $\sigma_1\neq\#$ och alla rutor till vänster om $\sigma_1$ och till höger om $\sigma_n$ är blanka, så är tapekonfigurationen:
\begin{equation*}
  \begin{gathered}
    \sigma_1\cdots\sigma_n
  \end{gathered}
\end{equation*}\par
\noindent Notera att läshuvudet står på $\sigma_n$ här.
\par\bigskip
\noindent Om TM:en vid ett tillfäle befinner sig i tillstånd $p$ och har tapekonfigurationen $x\sigma y$ så är \textbf{TM:ens konfiguration} vid detta tillfälle helt enkelt paret $(p,x\sigma y)$\par
\noindent ($x$ och $y$ antas vara strängar och $\sigma$ ett tecken/symbol)
\par\bigskip
\noindent Vet man TM:ens konfiguration vet vi hur den kommer arbeta $n$-steg framåt, konfigurationen bestämmer dess fortsatta arbete.
\par\bigskip
\subsection{Grafiska beskrivningar a TM:er}\hfill\\\par
\noindent Vi undersöker grafisk notation vs vad det betyder:
\begin{figure}[ht!]
    \centering
    \begin{tikzpicture}
      \node[state](p0){$p$};
      \node[state, right of=p0, xshift=1cm](p1){$q$};
      \path[-stealth] (p0) edge[above] node{$a/b$} (p1);
    \end{tikzpicture}
    \caption{Betyder att $a$ byts mot $b$ vid tillståndsövergång}
\end{figure}\par
\noindent Betyder helt enkelt att den övergår från tillstånd $p$ till tillstånd $q$ och skriver $b$ i rutan som avläses (om det stod $a$ i den rutan innan)
\par\bigskip
\noindent Vi kan även skriva:
\begin{figure}[ht!]
    \centering
    \begin{tikzpicture}
      \node[state](p0){$p$};
      \node[state, right of=p0, xshift=1cm](p1){$q$};
      \path[-stealth] (p0) edge[above] node{$a/R$} (p1);
    \end{tikzpicture}
    \caption{Flyttar läshuvudet till höger (Right)}
\end{figure}\par
\noindent Övergår från $p$ till $q$ och flyttar läshuvudet en ruta till höger, förutsatt att det stod $a$ i rutan som avlästes innan.
\par\bigskip
\noindent Vi kan även skriva (för vänster):
\begin{figure}[ht!]
    \centering
    \begin{tikzpicture}
      \node[state](p0){$p$};
      \node[state, right of=p0, xshift=1cm](p1){$q$};
      \path[-stealth] (p0) edge[above] node{$a/L$} (p1);
    \end{tikzpicture}
    \caption{Flyttar läshuvudet till vänster (Left)}
\end{figure}
\newpage
\noindent För att det inte ska bli för många pilar i diagrammen, så kan vi använda oss av förkortningar:
\par\bigskip
\noindent Följande:
\begin{figure}[ht!]
  \centering
    \begin{tikzpicture}
      \node[state](p0){$p$};
      \node[state, right of=p0, xshift=1cm](p1){$q$};
      \path[-stealth] (p0) edge[above] node{$a$} (p1);
    \end{tikzpicture}
    \caption{}
\end{figure}\par
\noindent är en förkortning för:
\begin{figure}[ht!]
  \centering
    \begin{tikzpicture}
      \node[state](p0){$p$};
      \node[state, right of=p0, xshift=1cm](p1){$q$};
      \path[-stealth] (p0) edge[above] node{$a/a$} (p1);
    \end{tikzpicture}
    \caption{}
\end{figure}
\par\bigskip
\noindent Följande:
\begin{figure}[ht!]
    \centering
    \begin{tikzpicture}
      \node[state](p0){$p$};
      \node[state, right of=p0, xshift=1cm](p1){$q$};
      \path[-stealth] (p0) edge[above] node{} (p1);
    \end{tikzpicture}
    \caption{}
\end{figure}\par
\noindent är en förkortning för:
\begin{figure}[ht!]
    \centering
    \begin{tikzpicture}
      \node[state](p0){$p$};
      \node[state, right of=p0, xshift=1cm](p1){$q$};
      \path[-stealth] (p0) edge[above, bend left] node{$\sigma_1$} (p1);
      \path[-stealth] (p0) edge[above] node{$\sigma_2$} (p1);
      \path[-stealth] (p0) edge[below, bend right] node{$\sigma_n$} (p1);
    \end{tikzpicture}
    \caption{}
\end{figure}
\par\bigskip
\noindent Följande:
\begin{figure}[ht!]
    \centering
    \begin{tikzpicture}
      \node[state](p0){$p$};
      \node[state, right of=p0, xshift=1cm](p1){$q$};
      \path[-stealth] (p0) edge[above] node{$R$} (p1);
    \end{tikzpicture}
    \caption{}
\end{figure}\par
\noindent är en förkortning för:
\begin{figure}[ht!]
    \centering
    \begin{tikzpicture}
      \node[state](p0){$p$};
      \node[state, right of=p0, xshift=1cm](p1){$q$};
      \path[-stealth] (p0) edge[above, bend left] node{$\sigma_1/R$} (p1);
      \path[-stealth] (p0) edge[above] node{$\sigma_2/R$} (p1);
      \path[-stealth] (p0) edge[below, bend right] node{$\sigma_n/R$} (p1);
    \end{tikzpicture}
    \caption{}
\end{figure}
\par\bigskip
\noindent Följande:
\begin{figure}[ht!]
    \centering
    \begin{tikzpicture}
      \node[state](p0){$p$};
      \node[state, right of=p0, xshift=1cm](p1){$q$};
      \path[-stealth] (p0) edge[above, bend left] node{$\bar{\sigma_1}$} (p1);
    \end{tikzpicture}
    \caption{}
\end{figure}\par
\noindent är en förkortning för:
\begin{figure}[ht!]
    \centering
    \begin{tikzpicture}
      \node[state](p0){$p$};
      \node[state, right of=p0, xshift=1cm](p1){$q$};
      \path[-stealth] (p0) edge[above, bend left] node{$\sigma_2$} (p1);
      \path[-stealth] (p0) edge[above] node{$\sigma_3$} (p1);
      \path[-stealth] (p0) edge[below, bend right] node{$\sigma_n$} (p1);
    \end{tikzpicture}
    \caption{}
\end{figure}
\par\bigskip
\noindent\textbf{Exempel:}\par
\noindent Låt TM $M$ beskrivas av diagrammet/"grafen":
\begin{figure}[ht!]
    \centering
    \begin{tikzpicture}
      \node[state, initial left](p0){$S$};
      \node[state, right of=p0, xshift=1cm](p1){$1$};
      \node[state, right of=p1, xshift=1cm](p2){$2$};
      \node[state, below of=p1, yshift=-1cm, accepting](p3){$H$};
      \path[-stealth] (p0) edge[above] node{$R$} (p1);
      \path[-stealth] (p1) edge[above, bend left] node{$a/b$} (p2);
      \path[-stealth] (p1) edge[above] node{$b/a$} (p2);
      \path[-stealth] (p2) edge[below, bend left] node{$R$} (p1);
      \path[-stealth] (p1) edge[left] node{$\#$} (p3);
    \end{tikzpicture}
    \caption{}
\end{figure}
\par\bigskip
\noindent Vi gör en "körning" på följande inputsträng "$abba$":\par
\noindent Turingmaskinen startar alltid på $S$, och börjar alltid med läshuvudet på det första blanka rutan till vänster om inputsträngen (samma konvention som finita automater):
\begin{itemize}
  \item $(S,\textbf{\#}abba)$
  \item $(1, \textbf{a}bba)$
  \item $(2, \textbf{b}bba)$
  \item $(1, b\textbf{b}ba)$
  \item $(2,b\textbf{a}ba)$
  \item $(1,ba\textbf{b}a)$
  \item $(2,ba\textbf{a}a)$
  \item $(1,baa\textbf{a})$
  \item $(2,baa\textbf{b})$
  \item $(1, baab\textbf{\#})$
  \item $(H, baab\textbf{\#})$
\end{itemize}\par
\noindent När TM har kommit till stopptillståndet så stannar den.\par
\noindent Denna TM har bytt ut $a$ mot $b$.
\par\bigskip
\noindent\textbf{Anmärkning:}\par
\noindent Vi kommer bara titta på deterministiska TM i denna kurs (man måste tala om vad den ska göra i varje tillstånd i kombination med varje tecken i alfabetet).
\newpage
\noindent\textbf{Några standard TM med egna namn:}\par
\begin{figure}[ht!]
    \centering
    \begin{tikzpicture}
      \node[state, initial left, initial text=](p0){$S$};
      \node[state, accepting, right of=p0, xshift=1cm](p1){$H$};
      \path[-stealth] (p0) edge[above] node{$L$} (p1);
    \end{tikzpicture}
    \caption{$L$ - Flyttar läshuvudet ett steg till vänster}
\end{figure}
\par
\begin{figure}[ht!]
    \centering
    \begin{tikzpicture}
      \node[state, initial left, initial text=](p0){$S$};
      \node[state, accepting, right of=p0, xshift=1cm](p1){$H$};
      \path[-stealth] (p0) edge[above] node{$R$} (p1);
    \end{tikzpicture}
    \caption{$R$ - Flyttar läshuvudet ett steg till höger}
\end{figure}\par
\begin{figure}[ht!]
    \centering
    \begin{tikzpicture}
      \node[state, initial left, initial text=](p0){$S$};
      \node[state, accepting, right of=p0, xshift=1cm](p1){$H$};
      \path[-stealth] (p0) edge[above, bend left] node{$\sigma_1/a$} (p1);
      \path[-stealth] (p0) edge[above] node{$\sigma_2/a$} (p1);
      \path[-stealth] (p0) edge[below, bend right] node{$\sigma_n/a$} (p1);
    \end{tikzpicture}
    \caption{$a$ - Skriver $a$ i rutan som avläses}
\end{figure}\par
\begin{figure}[ht!]
    \centering
    \begin{tikzpicture}
      \node[initial left, initial text=](p0){$R$};
      \node[state, accepting, right of=p0, xshift=1cm](p1){$H$};
      \path[-stealth] (p0) edge[above] node{$a$} (p1);
      \path[-stealth] (p0) edge[loop below] node{$\bar{a}$} (p0);
    \end{tikzpicture}
    \caption{$R_a$ - Söker efter första rutan till höger som har ett $a$, varpå termination}
\end{figure}\par
\begin{figure}[ht!]
    \centering
    \begin{tikzpicture}
      \node[initial left, initial text=](p0){$R$};
      \node[state, accepting, right of=p0, xshift=1cm](p1){$H$};
      \path[-stealth] (p0) edge[above] node{$\bar{a}$} (p1);
      \path[-stealth] (p0) edge[loop below] node{$a$} (p0);
    \end{tikzpicture}
    \caption{$R_{\bar{a}}$ - Söker efter första rutan till höger som inte innehåller ett $a$, varpå termination}
\end{figure}\par
\noindent Figure 83 och Figure 84 kallas för \textit{teckenletare}.\par
\noindent $L_a$ och $L_{\bar{a}}$ fungerar på samma sätt, förutom att de söker till vänster istället för till höger.
\par\bigskip
\subsection{Seriekoppling av TM}\hfill\\\par
\noindent Antag att vi har två TM $M_1$ och $M_2$ som beskrivs grafiskt\par
\noindent Seriekopplingen ska börja på $M_1$ starttilstånd och gå till $M_1$ stopptillstånd, men istället för att stoppa så hoppar den över till $M_2$ starttilstånd \textit{utan} att förändra vad som står på tapen.\par
\noindent Detta kallar vi för \textit{seriekoppling} mellan TM:arna, vi får då en ny TM, som vi betecknar $M_1M_2$ 
\par\bigskip
\noindent\textbf{Exempel:}\par
\noindent Betrakta Turingmaskinen $Ra$, som flyttar läshuvudet till höger och skriver $a$, sedan stannar den. $R$ och $a$ är separata TM\par
\noindent Vi kan ha en annan, exvis $RRa$, som flyttar läshuvudet 2 steg till höger, och skriver $a$
\par\bigskip
\noindent\textbf{Exempel:}\par
\noindent Konstruera en TM som terminerar om inputsträngen (inputalfabetet innehåller $a,b$) innehåller \textit{minst} 2st $a$:
\begin{figure}[ht!]
    \centering
    \begin{tikzpicture}
      \node[initial, initial text=](p0){$R$};
      \node[below of=p0, yshift=-1cm](p1){$R_a$};
      \node[state, right of=p0, xshift=1cm](p2){$p_1$};
      \node[state, accepting, right of=p2, xshift=1cm](p3){$p_2$};
      \path[-stealth] (p0) edge[right] node{$\#$} (p1);
      \path[-stealth] (p0) edge[loop above] node{$b$} (p0);
      \path[-stealth] (p0) edge[above] node{$a$} (p2);
      \path[-stealth] (p2) edge[below] node{$\#$} (p1);
      \path[-stealth] (p2) edge[loop above] node{$b/R$} (p2);
      \path[-stealth] (p2) edge[above] node{$a$} (p3);
    \end{tikzpicture}
    \caption{}
\end{figure}
