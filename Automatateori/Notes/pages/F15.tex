\section{Rices Sats}\par
\begin{theo}
  AAntagg att $\Omega$ är en icke-tom mängd av accepterbara språk och att något accepterbart språk inte tillhör $\Omega$ 
  \par\bigskip
  \noindent Då finns ingen TM som avgör för en godtycklig TM $M$ om $L(M)$ tillhör $\Omega$ eller ej
  \par\bigskip
  \noindent Annorlunda uttryckt, språket
  \begin{equation*}
    \begin{gathered}
      L_\Omega = \left\{K_M\:|\:M\text{ är en TM och }L(M)\in\Omega\right\}
    \end{gathered}
  \end{equation*}\par
  \noindent är oavgörbart
\end{theo}
\par\bigskip
\section{Chomskys språkhierarki}\par
\noindent Vi har följande inklusioner $(\subset)$ och likheter $(=)$ mellan de språkklasser som vi har betraktat:
\par\bigskip
Reguljära språk = FA-accpeterbara språk $\subset$ Sammanhangsfria språk = PDA-accepterbara språk $\subset$ Avgörbara språk $\subset$ (TM-) accepterbara språk = Restriktionsfria språk
\par\bigskip
\begin{tst}
  aa
  \begin{tst}
    aa
    \begin{tst}
      aa
      \begin{tst}
        aa
        Reguljära språk = FA-accpeterbara språk
      \end{tst}
      \par\bigskip
      Sammanhangsfria språk = PDA-accpeterbara språk\par
      Ex:
      \begin{equation*}
        \begin{gathered}
          \left\{a^nb^n\:|\: n\in\N\right\}
        \end{gathered}
      \end{equation*}
    \end{tst}
    \par\bigskip
    Avgörbara språk\par
    Ex:
    \begin{equation*}
      \begin{gathered}
        \left\{a^nb^nc^n\:|\: n\in\N\right\}
      \end{gathered}
    \end{equation*}
  \end{tst}
  \par\bigskip
  (TM-) accepterbara språk\par
  Ex: $L_{\text{stopp}}$
\end{tst}
\par\bigskip
Ej (TM-) accepterbara språk\par
Ex: $\bar{L}_{\text{stopp}}$
